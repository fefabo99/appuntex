\documentclass[12pt, a4paper, openany]{book}
\usepackage{fstyle}

\graphicspath{ {./img/} }

\begin{document}
\chapter{Cheatsheet}
\section*{Funzioni}
\paragraph*{Dominio}
\begin{itemize}
	\item Denominatore $\leq 0$
	\item Logaritmo: Argomento $>0$
	\item Radice$^n$: Argomento $\geq 0$ (sse $n$ pari)
	\item $[f(x)]^{g(x)} \to f(x)>0$
\end{itemize}
\paragraph*{Asintoto Obliquo}
$m= \limite{x}{\pm \infty} \frac{f(x)}{x} \rightarrow$ Se $m\in\R$ e $m\neq 0$$ \rightarrow q=\limite{x}{\pm \infty} [f(x)-mx] \rightarrow$ Se $q\in\R \rightarrow y=mx+q$ è asintoto obliquo

\paragraph*{Retta Tangente} al grafico in $x_0$: $y=mx + q$ con
\\$m= f'(x_0)$
\\$q=f(x_0)-f'(x_0)\cdot x_0$


\section*{Serie}
\paragraph*{Serie Armonica Generalizzata}
\begin{equation*}
	\sum \frac{1}{n^\alpha} \begin{cases}
		\text{Diverge}  & \alpha\leq 1 \\
		\text{Converge} & \alpha> 1
	\end{cases}
\end{equation*}
\paragraph*{Serie Geometrica}
\begin{equation*}
	\serie{0}{+\infty} q^n \begin{cases}
		\text{Diverge}    & q\geq 1 \\
		\text{Converge}   & -1<q<1  \\
		\text{Irregolare} & q\leq 1
	\end{cases}
\end{equation*}
\paragraph*{Criteri di Convergenza}
\paragraph*{Condizione Necessaria} ma non sufficiente di convergenza è che il termene generale $a_n$ sia infinitesimo $\limite{n}{+\infty} a_n = 0$

\paragraph*{Criterio del Rapporto}
$a_n>0$ defnitivamente. $limite{n}{+\infty} \frac{a_{n+1}}{a_n}=l$ allora $\sum a_n$:
\\Converge se $l<$, diverge se $l>1$, inconclusivo se $l=1$
\paragraph*{Criterio della Radice}
$a_n>0$ defnitivamente. $limite{n}{+\infty} \sqrt[n]{a_n}=l$ allora $\sum a_n$:
\\Converge se $l<$, diverge se $l>1$, inconclusivo se $l=1$
\paragraph*{Criterio del Confronto}
$0\leq a_n \leq b_n$  definitivamente.
\\Se $b_n$ converge allora $a_n$ converge.
\\Se $a_n$ diverge positivamente, allora anche $b_n$
\paragraph*{Criterio dell'Assoluta Convergenza}
Una serie $\sum a_n$ si dice assolutamente convergente se converge la serie $\sum |a_n|$.
\\Se una serie converge assolutamente, allora converge.

\section*{Limiti}
\paragraph*{Limiti Notevoli}

\subparagraph*{Logaritmo Naturale}
$$\limite{x}{0} \frac{\ln(1+x)}{x} = 1 ; \limite{h(x)}{0} \frac{\ln(1+h(x))}{h(x)}$$ 

\subparagraph*{Funzione Esponenziale}
$$\limite{x}{0} \frac{e^x-1}{x} = 1; \limite{h(x)}{0} \frac{e^{h(x)}-1}{h(x)} = 1$$

\subparagraph*{Costante con Frazione}
$$\limite{x}{0}\frac{ax -1}{x} = \ln(a)$$ 

\subparagraph*{Seno}
$$\limite{x}{0}\frac{\sin(x)}{x} = 1$$

\subparagraph*{Coseno}
$$\limite{x}{0} \frac{1-\cos(x)}{x} = 0 $$

\paragraph*{Equivalenze asintotiche} se $x\to 0$
\begin{itemize}
    \item $\sin x \sim x$
    \item $1-cos x \sim \frac{1}{2}x^2$
    \item $\tan x \sim x$
    \item $\ln(1+x) \sim x$
    \item $(1+x)^\alpha -1 sim \alpha x$
\end{itemize}

\paragraph*{Ordine degli infiniti}
$$ \log_ax\ll x^b\ll x^c\ll d^x\ll g^x\ll x^x $$
\nb{la radice è "più grande" del logaritmo}


\paragraph*{Forme di indecisione}
	$[\frac{0}{0}]$ $[\frac{\infty}{\infty}]$ $[1^\infty]$ $[\infty - \infty]$ $[\infty \cdot 0]$ $[0^0]$ $[\infty^0]$

\section*{Calcolo Differenziale}
\paragraph*{Formule di Taylor e McLaurin} di grado $k$ e centrato in $x_0$:
$$P_k(x)=f(x_0)+f'(x_0)(x-x_0) + \frac{1}{2}f''(x_0)(x-x_0)^2 +... + \frac{1}{k!}f^{(k)}(x_0)(x-x_0)^k$$
Nel caso in cui $x_0=0$ si ottiene la formula di Mclaurin.
\paragraph*{Derivata dell'inversa} Data $g(x) = f^{-1}(x)$, $g'(y_0)=$
\begin{enumerate}
	\item trovo $x_0$ ponendo $y_0=f(x)$
	\item trovo $g'(y_0)=\frac{1}{f'(x_0)}$
\end{enumerate}


\section*{Calcolo Integrale}
\paragraph*{Primitive elementari}
\begin{tabular}{ |c|c| }
	\hline
	Funzione & Primitiva\\
	\hline
	$k$ & $kx$\\
	$x^a$,$a\neq-1$ & $\frac{x^{a+1}}{a+1}$\\
	$\frac{1}{x}$ & $\log|x|$\\
	$\sin x$ & $-\cos x$\\
	$\cos x $&$\sin x$\\
	$a^x$ & $\frac{a^x}{\log a}$\\
	\hline
	$e^{-x}$ & $-e^{-x}$\\
\end{tabular}
\paragraph*{I metodi di risoluzione}
\paragraph*{Proprietà degli integrali}
\begin{itemize}
	\item Somma di integrali: $\int f(x)+g(x) dx = \int f(x) dx + \int g(x) dx$  
	\item Costante moltiplicativa $\int k \cdot f(x) = k \int f(x)$
\end{itemize}

\paragraph*{Rapporto incrementale}
$$\frac{\Delta y}{ \Delta x}\frac{f(x_0+h)-f(x_0)}{h} $$

\end{document}