\documentclass[12pt, a4paper, openany]{article}
\usepackage{fstyle}
\usepackage{enumitem}
\setlist{nolistsep,leftmargin=*}

\graphicspath{ {./img/} }
\def\arraystretch{2} %define table vertical spacing

\begin{document}
\title{CheatSheet di Analisi Matematica}
\author{Fabio Ferrario}
\date{2022}
\maketitle
\section*{Studio di Funzione}
Per lo studio di una funzione bisogna trovare:
\paragraph*{Dominio} della funzione, poni:\\

\begin{tabular}{ l|l } 
	Denominatore & $\leq 0$   \\ 
	Logaritmo & Argomento $>0$\\ 
	Radice$^n$& Argomento $\geq 0$ (\emph{sse $n$ pari})	 \\ 
	$[f(x)]^{g(x)}$ &   $f(x)>0$
\end{tabular}
\paragraph*{Limiti ai punti di frontiera del dominio}
Trovato il dominio, trova \emph{i limiti ai punti di frontiera}, 
quindi porre i limiti ad ogni punto in cui il dominio si interrompe (sia da destra che da sinistra) e eventualmente a $\pm \infty$.
\paragraph*{Asintoti}
Trovati tutti i limiti, se trovi:
\begin{itemize}
	\item $\limite{x}{\alpha^\pm} f(x) = \pm \infty \implies$ Asintoto \emph{Verticale}.
	\item $\limite{x}{\pm \infty} f(x) = l \implies$ Asintoto \emph{Orizzontale} (di equazione $y=l$)
\end{itemize}
Bisogna anche controllare la presenza di \textbf{Asintoti Obliqui}:
\begin{itemize}
	\item $m = \limite{x}{\pm \infty} \frac{f(x)}{x} \implies$ se $m$ \emph{esiste e non è nullo} trovo $q$:
	\item $q = \limite{x}{\pm \infty} [f(x) - mx]\implies$  se $q$ esiste allora $y=mx+q$ è \emph{asintoto obliquo}
\end{itemize}
\paragraph*{Monotonia}
La monotina di una funzione si calcola \emph{ponendo $f'(x)>0$.}
Nei punti in cui la derivata è positiva, la funzione è \textbf{Crescente}, nei punti in cui è negativa la funzione è \textbf{Decrescente}
\paragraph*{Punti di estremo}
Una volta trovato il segno della derivata si possono trovare i punti di estremo.
\\I punti in cui la derivata cambia direzione sono punti di estremo
\paragraph*{Convessità/Concavità}


\paragraph*{Retta Tangente} al grafico in $x_0$:\\
trova $y=mx + q$ ponendo:
\begin{itemize}
	\item $m=f'(x_0)$
	\item $q=f(x_0)-f'(x_0)\cdot x_0$
\end{itemize}


\section*{Serie}
	\subsection*{Serie Notevoli}
	\paragraph*{Serie Armonica Generalizzata}
	\begin{equation*}
		\sum \frac{1}{n^\alpha} \begin{cases}
			\text{Diverge}  & \alpha\leq 1 \\
			\text{Converge} & \alpha> 1
		\end{cases}
	\end{equation*}
	\paragraph*{Serie Geometrica}
	\begin{equation*}
		\serie{0}{+\infty} q^n \begin{cases}
			\text{Diverge}    & q\geq 1 \\
			\text{Converge}   & -1<q<1  \\
			\text{Irregolare} & q\leq 1
		\end{cases}
	\end{equation*}
	\subsection*{Criteri di Convergenza}

	\emph{Condizione necessaria ma non sufficiente} di convergenza è che il termene generale $a_n$ sia infinitesimo $\limite{n}{+\infty} a_n = 0$.
	\\Avendo $a_n>0$ (positiva) \emph{definitivamente} posso usare i seguenti criteri:
	\begin{itemize}
		\item \textbf{Rapporto}: $\limite{n}{+\infty} \frac{a_n+1}{a_n} = l$
		\item \textbf{Radice}: $\limite{n}{+\infty} \sqrt[n]{a_n} = l$
	\end{itemize}
	In entrambi questi casi $\sum a_n$:\\
	\emph{Converge} se $l<1$, \emph{Diverge} se $l>1$ e il criterio è \emph{inconclusivo} se $l=1$

	\begin{itemize}
		\item \textbf{Confronto}: $a_n\leq b_n$ definitivamente $\implies$
		\begin{itemize}
			\item se $b_n$ converge, allora $a_n$ converge.
			\item se $a_n$ diverge positivamente, allora anche $b_n$
		\end{itemize}
	\end{itemize}

	\paragraph*{Criterio dell'Assoluta Convergenza}
	$\sum a_n$ Converge assolutamente se converge $\sum |a_n|$.
	Se una serie converge assolutamente, allora converge.

	\section*{Limiti}
	\begin{multicols}{2}
		\paragraph*{Limiti Notevoli}
		\columnbreak
		\begin{tabularx}{0.8\textwidth}{ |X|X| }
			\hline
			Logaritmo & $\limite{x}{0} \frac{\ln(1+x)}{x} = 1 $\\
			\hline
			$f$ Esponenziale & $\limite{x}{0} \frac{e^x-1}{x} = 1$\\
			\hline
			Costante e Frazione & $\limite{x}{0}\frac{ax -1}{x} = \ln(a)$\\
			\hline
			Seno & $\limite{x}{0}\frac{\sin(x)}{x} = 1$ \\
			\hline
			Coseno & $\limite{x}{0} \frac{1-\cos(x)}{x} = 0 $\\
			\hline
		\end{tabularx}
	\end{multicols}
	\begin{multicols}{2}
	\paragraph*{Equivalenze Asintotiche}
	\columnbreak
	\begin{tabularx}{0.67\textwidth}{|Xcl|}
		\hline
		\multicolumn{3}{|c|}{\textbf{con \emph{x} $\to$ 0}}\\
		\hline
		\hline
		$\sin x$ &$\sim$ & $x$ \\
		\hline
		$1-\cos x$&$\sim$ & $\frac{1}{2}x^2$\\
		\hline
		$\tan x$ & $\sim$ & $x$\\
		\hline
		$\ln(1+x)$ & $\sim$ & $x$ \\
		\hline
		$(1+x)^\alpha -1 $&$\sim$& $\alpha x$ \\
		\hline
	\end{tabularx}
\end{multicols}



	%\begin{itemize}
	%	\item $\sin x \sim x$
	%	\item $1-cos x \sim \frac{1}{2}x^2$
	%	\item $\tan x \sim x$
	%	\item $\ln(1+x) \sim x$
	%	\item $(1+x)^\alpha -1 \sim \alpha x$
	%\end{itemize}

	\paragraph*{Ordine degli infiniti}
	$$ \log_ax\ll x^b\ll x^c\ll d^x\ll g^x\ll x^x $$
	\nb{la radice è "più grande" del logaritmo}


	\paragraph*{Forme di indecisione}
$[\frac{0}{0}]$ $[\frac{\infty}{\infty}]$ $[1^\infty]$ $[\infty - \infty]$ $[\infty \cdot 0]$ $[0^0]$ $[\infty^0]$

	\section*{Calcolo Differenziale}
	\paragraph*{Formule di Taylor e McLaurin} di grado $k$ e centrato in $x_0$:
	$$P_k(x)=f(x_0)+f'(x_0)(x-x_0) + \frac{1}{2}f''(x_0)(x-x_0)^2 +... + \frac{1}{k!}f^{(k)}(x_0)(x-x_0)^k$$
	Nel caso in cui $x_0=0$ si ottiene la formula di Mclaurin.
	\paragraph*{Derivata dell'inversa} Data $g(x) = f^{-1}(x)$, $g'(y_0)=$
\begin{enumerate}
	\item trovo $x_0$ ponendo $y_0=f(x)$
	\item trovo $g'(y_0)=\frac{1}{f'(x_0)}$
\end{enumerate}


\section*{Calcolo Integrale}
\paragraph*{Primitive elementari}
\begin{tabular}{ |c|c| }
	\hline
	Funzione        & Primitiva             \\
	\hline
	$k$             & $kx$                  \\
	$x^a$,$a\neq-1$ & $\frac{x^{a+1}}{a+1}$ \\
	$\frac{1}{x}$   & $\log|x|$             \\
	$\sin x$        & $-\cos x$             \\
	$\cos x $       & $\sin x$              \\
	$a^x$           & $\frac{a^x}{\log a}$  \\
	\hline
	$e^{-x}$        & $-e^{-x}$             \\
	\hline
\end{tabular}
\paragraph*{I metodi di risoluzione}
\paragraph*{Proprietà degli integrali}
\begin{itemize}
	\item Somma di integrali: $\int f(x)+g(x) dx = \int f(x) dx + \int g(x) dx$
	\item Costante moltiplicativa $\int k \cdot f(x) = k \int f(x)$
\end{itemize}

\paragraph*{Rapporto incrementale}
$$\frac{\Delta y}{ \Delta x}\frac{f(x_0+h)-f(x_0)}{h} $$

\end{document}