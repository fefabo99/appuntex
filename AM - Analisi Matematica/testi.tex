\documentclass[12pt, a4paper, openany]{book}
\usepackage{fstyle}

\graphicspath{ {./img/} }

\begin{document}
\title{Testi degli Esami di Analisi Matematica}
\author{Fabio Ferrario}
\date{2022}
\maketitle
\tableofcontents

\section{Febbraio 2021}

\domanda{1}{Sia data la serie $\serie{1}{+\infty} a_n$, con $a_n \geq 0$.
Per la convergenza della serie la condizione $a_n \sim \frac{1}{n^2}$ è}
\rispostechiuse
{sufficiente ma non necessaria}
{necessaria e sufficiente}
{necessaria ma non sufficiente}
{nè necessaria nè sufficiente}

\domanda{1b}{Sia data la serie $\serie{1}{+\infty} a_n$, con $a_n \geq 0$.
Per la convergenza della serie la condizione $a_n \sim \frac{1}{n}$ è}
\rispostechiuse
{sufficiente ma non necessaria}
{necessaria e sufficiente}
{necessaria ma non sufficiente}
{nè necessaria nè sufficiente}

\domanda{2}{La funzione $f_{a,b}(x) = \begin{cases} ax+ x^2 & x\leq 0 \\ be^x + \sin(x) - 1 & x>0 \end{cases}$
è continua in $x=0$ sse:}
\rispostechiuse{$b = 1$ e per ogni $a$}{$a=0$,$b=1$}{per ogni $a,b \in \R$}{per nessun valore di $a,b$}

\domanda{2b}{La funzione $f_{a,b}(x) = \begin{cases} x+ ax^2 & x\leq 0 \\ e^x + \sin(x) - b & x>0 \end{cases}$
è continua in $x=0$ sse:}
\rispostechiuse{$b = 1$ e per ogni $a$}{$a=0$,$b=1$}{per ogni $a,b \in \R$}{per nessun valore di $a,b$}

\section{Luglio 2021}
\subsection{Domande Chiuse}
\domanda{O1}{La funzione $f(x)=\begin{cases} \sin x^2 + a & x\leq 0 \\ \frac{\ln(1+x)}{2x} + \frac{3}{2} &x>0\end{cases}$ è continua se:}
\rispostechiuse{$a = 3/2$}{$a = 2$}{$a = 5/2$}{$a=0$}

\domanda{O2}{Sia $f(x) = x^2+2x+2$. allora $\frac{d}{dx} \ln(f(x))$ per $x=1$ è }
\rispostechiuse{1}{4}{$2/5$}{$4/5$}

\domanda{O3}{La funzione $f(x) = x^5+x^3-1$ ha quanti flessi?}
\rispostechiuse{Ha 5 flessi}{Ha 1 flesso}{non ha flessi}{ha 3 flessi}

\domanda{O4}{$\int_{0}^{1} xe^x dx =$}
\rispostechiuse{0}{-1}{1}{$e$}


\domanda{O5}{La funzione $f(x)=\begin{cases} -|x+3| & -6 < x <-1 \\ -2x^2 & -1 \leq x < 1 \end{cases}$ }
\rispostechiuse{non è limitata}{ha minimo}{ha un unico punto di massimo}{ha come immagine un intervallo}

\domanda{O6}{Sia $f(x) = x\ln(x+1) - x^2$, il rapporto incrementale di $f$ relativo all'intervallo $[0,e-1]$ vale)}
\rispostechiuse{$(e-2)(e-1)$}{$(2-e)(e-1)$}{$e-2$}{$2-e$}

\domanda{O7}{La serie $\serie{1}{+\infty} \frac{n^2}{n\ln n + 2n^{\alpha+1}}$}
\rispostechiuse{converge per ogni $\alpha>0$}{diverge per ogni $\alpha>0$}{converge se e solo se $\alpha > 2$}{converge se $0<\alpha<1$}

\subsection{Domande Aperte}
\domandaaperta{1}{Data la funzione $f(x)=\ln x - \ln^2x$, si studi:
	\begin{enumerate}
		\item Dominio
		\item Limiti ai punti di frontiera del dominio
		\item Eventuali asintoti
		\item Estremanti (specificando se relativi o assoluti)
		\item Monotonia
		\item Punti di flesso
		\item Tangente di flesso
	\end{enumerate}
}

\domandaaperta{2}{data la funzione $f(x)=x \sin x$
	\begin{enumerate}
		\item Si scrivano tutte le primitive
		\item Si determini,se esiste, la primitiva $\phi$ tale che $\phi(\pi) = 2\phi(0)$
		\item si calcoli $\int_{0}^{\pi} f(x) dx$
	\end{enumerate}
}
\domandaaperta{3}{Sia $\serie{1}{+\infty} \cos(\pi n)\sin\frac{1}{n}$.
    \begin{enumerate}
        \item Per studiare la serie uso il critedio:
        \item La successione $\sin \frac{1}{n}$ è strettamente:
        \item La serie data:
        \item E la serie $\serie{1}{+\infty}\sin \frac{1}{n}$:
    \end{enumerate}
}
\end{document}