\documentclass[12pt, a4paper]{article}
\usepackage{fstyle}

\graphicspath{ {./img/} }

\begin{document}
\title{Testi degli Esami di Analisi Matematica}
\author{Fabio Ferrario}
\date{2022}
\maketitle
\tableofcontents

\section{Gennaio 2020}

\subsection{Domande Chiuse}

\domanda{1}{La funzione $f(x) = \begin{cases} \frac{ln(1-x)}{2x} & x<0 \\ x^2 + \frac{1}{2} &x\geq 0 \end{cases}$ ha in $x=0$:}

\rispostechiuse{Una discontinuità di prima specie}{una discontinuità eliminabile}{un punto di continuità}{una discontinuità di seconda specie}

\domanda{1b}{la funzione $f(x) = \begin{cases} \frac{ln(1+x)}{2x} & x<0 \\ x^2 + \frac{1}{2} &x\geq 0 \end{cases}$ ha in $x=0$:}
\rispostechiuse{Una discontinuità di prima specie}{una discontinuità eliminabile}{un punto di continuità}{una discontinuità di seconda specie}

\domanda{2}{Sia $f:\R \to \R$ derivabile e tale che $f(2) = 3$ e $f'(2) = 4$. Se $g(x) = \sqrt{f^2(x)+7}$ allora $g'(2)$ vale:}
\rispostechiuse{1}{2}{3}{4}

\domanda{2b}{Sia $f:\R \to \R$ derivabile e tale che $f(2) = 2$ e $f'(2) = 3$. Se $g(x) = \sqrt{f^2(x)+5}$ allora $g'(2)$ vale:}
\rispostechiuse{1}{2}{3}{4}

\domanda{3}{La funzione $f(x) = x^2 + 2x + k \ln x$ è strettamente convessa in $(0, +\infty)$ se}
\rispostechiuse{$k=-3$}{$k=-1$}{$k=1$}{$k=-2$}

\domanda{3b}{La funzione $f(x) = -x^2 + 2x + k \ln x$ è strettamente convessa in $(0, +\infty)$ se}
\rispostechiuse{$k=3$}{$k=1$}{$k=-1$}{$k=2$}

\domanda{4}{Sia $f(x)= x+e^x + \cos x$. Il polinomio di Mc Laurin del secondo ordine di $f$ è:}
\rispostechiuse{$2+2x+\frac{x^2}{2}$}{$2+2x+x^2$}{$2+2x$}{$2+2x-\frac{x^2}{2}$}

\domanda{4b}{Sia $f(x)= -\frac{x^2}{2} + e^x + \sin x$. Il polinomio di Mc Laurin del secondo ordine di $f$ è:}

\domanda{5}{L'integrale definito $\int_1^2 \frac{2e^x}{e^x+2} dx$ vale:}

\domanda{5b}{l'integrale definito $\int_{\frac{1}{2}}^1 \frac{4e^{2x}}{e^{2x}+2} dx$ vale}

\domanda{6}{L'insieme $A=\{\frac{2+2^{-n}}{3-3^{-n}}, n = 1,2,...\}$}
\rispostechiuse{Ha massimo 15/16}{Ha minimo 2/3}{Non è superiormente limitato}{non è inferiormente limitato}

\domanda{6b}{L'insieme $A=\{\frac{2-2^{-n}}{3+3^{-n}}, n = 1,2,...\}$}
\rispostechiuse{Ha minimo 9/20}{Ha massimo 2/3}{Non è superiormente limitato}{non è inferiormente limitato}

\domanda{7}{lim$_{n\to +\infty} \frac{n\ln^3n-\sqrt{n} + n^{3/2}}{2n+3\sqrt[3]{n}-n\ln^4n}$ vale}
\rispostechiuse{$-\frac{1}{3}$}{0}{$+\infty$}{$-\infty$}

\domanda{7b}{lim$_{n\to +\infty} \frac{n\ln^3n - \sqrt{n} - n^{1/3}}{2n^2 + 3\sqrt[3]{n}- n \ln^4 n}$ vale}
\rispostechiuse{$-\frac{1}{3}$}{0}{$+\infty$}{$-\infty$}

\domanda{8}{La somma della serie $\serie{0}{+\infty}\frac{3}{4^{n+2}}$ è:}


\domanda{8b}{La somma della serie $\serie{0}{+\infty}\frac{4}{3^{n+2}}$ è:}

\subsection{Domande Aperte}
\domandaaperta{1}{
	Data la funzione:
	\[
		f(x)= \frac{e^x}{x^2-1}	
	\]
	Se ne tracci un grafico qualitativo (in particolaresi determinino: dominio, limiti agli estremi, eventuali asintoti,
	monotonia, estremanti relativi e assoluti. 
	Non è richiesto lo studio della derivata seconda). Qual è il più grande
	intervallo del tipo $(-\infty, a)$ su cui $f$ è monotona crescente
}
\domandaaperta{2}{Si dia la definizione di primitiva di una funzione $f:I \to \R$, con $I$ intervallo.
Si determini, se eseite, una primitiva $\phi: \R \to \R$ della funzione $f:( -1, +\infty) \to \R$, $f(x) = 2x+ln(x+1)$ tale che $\phi(1)=2\phi(0)$}

\domandaaperta{3}{
	Data la successione definita per ricorrenza:
	\[
	\begin{cases}
		a_1=3 \\
		a_{n+1} = \sqrt{a_n+2}
	\end{cases}	
	\]
	\begin{enumerate}
		\item Si provi per induzione che $a_n \geq 2$ per ogni $n\in \N$;
		\item si provi senza usare l'induzione che $\{a_n\}$ è monotona decrescente;
		\item si calcoli $\lim_{n\to +\infty} a_n$
	\end{enumerate}
}

\section{Gennaio 2021}
\subsection{Domande Chiuse}

\domanda{1}{La somma della serie $\serie{1}{+\infty} 2^{1-3n}$ vale:}
\rispostechiuse{1/7}{16/7}{2/7}{8/7}

\domanda{2}{$\lim_{n\to+\infty} \frac{n^2\ln^6n-n^3\ln^2n + \sin n}{n^3\ln^2n-n^2\ln^4n-3^{-n^2}}$ vale}
\rispostechiuse{$-\infty$}{$+\infty$}{non esiste}{$-1$}

\domanda{3}{La funzione $f(x)=\begin{cases} \frac{x^2-1}{x-1} & x\neq 1 \\ 2 & x=1\end{cases}$ }
\rispostechiuse{Ha una discontinuità eliminabile}{è continua su $\R$}{ha una discontinuità di prima specie}{ha una discontinuità di seconda specie}

\domanda{4}{Sia $f(x)=x-2e^x+\sin(x^2)$. Il polinomio di Mclaurin del secondo ordine di $f$ è:}
\rispostechiuse{$-2-x-x^2$}{$-x-x^2$}{$-2-x$}{$-2-x^2$}

\domanda{5}{Tra le primitive di $e^x \sin x$ c'è:}
\rispostechiuse{$\frac{1}{2}e^x(\sin x + \cos x)$}{$e^x(\sin x - \cos x)$}{$\frac{1}{2}e^x(\sin x - cos x)$}{$e^{2x}(\sin x - \cos x)$}

\domanda{6}{La funzione $f(x) = \sqrt{x-4} - \frac{x}{2}$ è crescente sse}
\rispostechiuse{$x\in [5,+\infty)$}{$x\in [4,8]$}{$c\in (-\infty,5]$}{$x\in[4,5]$}

\domanda{7}{la derivata di $f(x)= \frac{x\ln x -1}{x^2}$ in $x=1$ è:}
\rispostechiuse{-1}{3}{0}{1}

\domanda{8}{$\lim_{x\to\pm0}$}
\rispostechiuse{0}{1}{non esiste}{$\pm\infty$}

\subsection{Domande Aperte}
\domandaaperta{1}{Studia la funzione 
\[
	f(x)=\ln x - \arctan(x-1)
\]
In particolare: Dominio, limiti, asintoti, punti di massimo/minimo (stazionari).
\\Qual'è l'equazione della retta tangente al grafico nel punto di ascissa $x=1$?

}
\domandaaperta{2}{Data la funzione $f(x)=2x\ln x : (0,+\infty)\to \R$,
Si scrivano tutte le primitive.
Si determini la primitiva $\phi$ tale che $\phi(e)=2\phi(1)$.
Si calcoli $\int_1^2 f(x) dx$.
}
\domandaaperta{3}{Sia $\serie{1}{+\infty}a_n$ una serie numerica.
Si enunci una condizione necessaria per la convergenza.
La condizione enunciata è sufficiente? si motivi la risposta
}

\section{Febbraio 2021}
\subsection{Domande Chiuse}

\domanda{1}{Sia data la serie $\serie{1}{+\infty} a_n$, con $a_n \geq 0$.
Per la convergenza della serie la condizione $a_n \sim \frac{1}{n^2}$ è}
\rispostechiuse
{sufficiente ma non necessaria}
{necessaria e sufficiente}
{necessaria ma non sufficiente}
{nè necessaria nè sufficiente}

\domanda{1b}{Sia data la serie $\serie{1}{+\infty} a_n$, con $a_n \geq 0$.
Per la convergenza della serie la condizione $a_n \sim \frac{1}{n}$ è}
\rispostechiuse
{sufficiente ma non necessaria}
{necessaria e sufficiente}
{necessaria ma non sufficiente}
{nè necessaria nè sufficiente}

\domanda{2}{La funzione $f_{a,b}(x) = \begin{cases} ax+ x^2 & x\leq 0 \\ be^x + \sin(x) - 1 & x>0 \end{cases}$
è continua in $x=0$ sse:}
\rispostechiuse{$b = 1$ e per ogni $a$}{$a=0$,$b=1$}{per ogni $a,b \in \R$}{per nessun valore di $a,b$}

\domanda{2b}{La funzione $f_{a,b}(x) = \begin{cases} x+ ax^2 & x\leq 0 \\ e^x + \sin(x) - b & x>0 \end{cases}$
è continua in $x=0$ sse:}
\rispostechiuse{$b = 1$ e per ogni $a$}{$a=0$,$b=1$}{per ogni $a,b \in \R$}{per nessun valore di $a,b$}

\section{Luglio 2021}
\subsection{Domande Chiuse}
\domanda{O1}{La funzione $f(x)=\begin{cases} \sin x^2 + a & x\leq 0 \\ \frac{\ln(1+x)}{2x} + \frac{3}{2} &x>0\end{cases}$ è continua se:}
\rispostechiuse{$a = 3/2$}{$a = 2$}{$a = 5/2$}{$a=0$}

\domanda{O2}{Sia $f(x) = x^2+2x+2$. allora $\frac{d}{dx} \ln(f(x))$ per $x=1$ è }
\rispostechiuse{1}{4}{$2/5$}{$4/5$}

\domanda{O3}{La funzione $f(x) = x^5+x^3-1$ ha quanti flessi?}
\rispostechiuse{Ha 5 flessi}{Ha 1 flesso}{non ha flessi}{ha 3 flessi}

\domanda{O4}{$\int_{0}^{1} xe^x dx =$}
\rispostechiuse{0}{-1}{1}{$e$}


\domanda{O5}{La funzione $f(x)=\begin{cases} -|x+3| & -6 < x <-1 \\ -2x^2 & -1 \leq x < 1 \end{cases}$ }
\rispostechiuse{non è limitata}{ha minimo}{ha un unico punto di massimo}{ha come immagine un intervallo}

\domanda{O6}{Sia $f(x) = x\ln(x+1) - x^2$, il rapporto incrementale di $f$ relativo all'intervallo $[0,e-1]$ vale)}
\rispostechiuse{$(e-2)(e-1)$}{$(2-e)(e-1)$}{$e-2$}{$2-e$}

\domanda{O7}{La serie $\serie{1}{+\infty} \frac{n^2}{n\ln n + 2n^{\alpha+1}}$}
\rispostechiuse{converge per ogni $\alpha>0$}{diverge per ogni $\alpha>0$}{converge se e solo se $\alpha > 2$}{converge se $0<\alpha<1$}

\subsection{Domande Aperte}
\domandaaperta{1}{Data la funzione $f(x)=\ln x - \ln^2x$, si studi:
	\begin{enumerate}
		\item Dominio
		\item Limiti ai punti di frontiera del dominio
		\item Eventuali asintoti
		\item Estremanti (specificando se relativi o assoluti)
		\item Monotonia
		\item Punti di flesso
		\item Tangente di flesso
	\end{enumerate}
}

\domandaaperta{2}{data la funzione $f(x)=x \sin x$
	\begin{enumerate}
		\item Si scrivano tutte le primitive
		\item Si determini,se esiste, la primitiva $\phi$ tale che $\phi(\pi) = 2\phi(0)$
		\item si calcoli $\int_{0}^{\pi} f(x) dx$
	\end{enumerate}
}
\domandaaperta{3}{Sia $\serie{1}{+\infty} \cos(\pi n)\sin\frac{1}{n}$.
    \begin{enumerate}
        \item Per studiare la serie uso il critedio:
        \item La successione $\sin \frac{1}{n}$ è strettamente:
        \item La serie data:
        \item E la serie $\serie{1}{+\infty}\sin \frac{1}{n}$:
    \end{enumerate}
}

\section{Luglio 2022}
\subsection{Domande Chiuse}
\domanda[convergenza di una serie]{1}{La serie $\serie{1}{+\infty} \frac{1}{n^{(\alpha+1)/2}ln^2n}$}
\rispostechiuse{Converge sse $\alpha \geq 1$}{Converge sse $\alpha > 1$}{converge $\forall \alpha \in \R$}{diverge sse $\alpha \leq 1$}
\domanda[derivabilità]{2}{La funzione $f(x) = \begin{cases} a \sin x - b^2 & -2\leq x \leq 0 \\ 1-e^x 0 < x \leq 3 \end{cases} $ è derivabile in $x=0$ sse}
\rispostechiuse{$a=-1,b=1$}{$a=-1,b=0$}{$a=-1,\forall b\in \R$}{$\forall a \in \R, b=0$}
\domanda[composizione di funzioni]{3}{Date le funzioni $f(x)= \ln(x),g(x)=x^3,h(x)=2-x$, la funzione composta $(h \circ g \circ f)(x)$ è:}
\rispostechiuse{$2-\ln (x^3)$}{$2-x^3-\ln x$}{$(2-\ln x)^3$}{$2-(\ln x)^3$}
\domanda[Intervalli]{4}{Quali dei seguenti insiemi è un intervallo?}
\rispostechiuse{$\{ x \in \R: 3|x| \geq 1 \}$}{$\{ x \in \R: |x^2-1| < 1 \}$}{$\{ x \in \R: 2|x| \geq x^2 \}$}{$\{ x \in \R: |x^2-1| \geq 1 \}$}
\domanda[Limiti di serie]{5}{$\limite{n}{+\infty} n^2 \sin(\frac{1}{n+n^2})$ vale}
\rispostechiuse{1}{non esiste}{$+\infty$}{0}
\domanda[Integrali]{6}{Una primitiva della funzione $f(x) = \frac{e^{2x}}{e^{2x}+1}$ è:}
\rispostechiuse{$2\ln(e^x+1)+3$}{$2\ln(e^x+1)+1$}{$\ln(e^{2x}+1)-4$}{$\frac{ln(e^{2x}+1)}{2}+7$}
\domanda[Massimo/minimo]{7}{La funzione $e^{-x^2}$ ha in $x=0$:}
\rispostechiuse{Un punto di massimo}{Un punto di minimo}{Un punto di flesso}{Un punto di discontinuità}
\domanda{8}{La funzione $f(x)=e^{3x-x^3}$ è monotona decrescente sse:}
\rispostechiuse{$x\in [-1,1]$}{$x\in(-\infty,1]$}{$x\in (-\infty,-1] \vee [1,+\infty)$}{$x\in [-1,+\infty)$}

\subsection{Domande Aperte}
\domandaaperta{1}{Sia $f: \R \to \R$ definita da $f(x)=(x^2 - 2x)e^{-x}$. Allora:
\begin{itemize}
	\item Dominio
	\item Limiti
	\item Asintoti
	\item Massimi/Minimi
	\item Più grosso intervallo di convessità del tipo $(k,+\infty)$
	\item Polinomio di Mclaurin del secondo ordine:
	\item La funzione $g(x)=f(x)+\sqrt{x^2-x}$ per $x\to +\infty$ ha asintoto obliquo di equazione:
\end{itemize}
}
\domandaaperta{2}{Data la funzione $f(x) = \frac{1}{x\ln^x}: (1,+\infty) \to \R$,
\begin{itemize}
	\item Si scrivano tutte le primitive e il loro dominio di definizione
	\item Si determini la primitiva che assume in $x=e$ lo stesso valore della funzione $g(x)=\frac{e}{x}$
	\item La media integrale di $f(x)$ sull'intervallo $[e,e^3]$ vale
\end{itemize}
}

\section{Settembre 2019}
\subsection{Domande Chiuse}

\section{Settembre 2019}
\subsection{Domande Chiuse}
\domanda{1}{La serie $\serie{1}{+\infty}(-1)^n \frac{1}{n^3}$}
\rispostechiuse{converge asolutamente}{converge, ma non assolutamente}{diverge}{è irregolare}
\domanda{2}{$\lim_{n\to +\infty} \frac{n^3+5ln^2n-n^2\sqrt{n^3+1}}{-n^3+e^{1/n}-n^2\sqrt{n}}$ è}
\rispostechiuse{$-\infty$}{$+\infty$}{1}{0}
\domanda{3}{La funzione $f(x) = x^2 + 2\ln x$ è convessa se e solo se}
\rispostechiuse{$x\in (-1,1)$}{$x\in(0,1)$}{$x\in(1,+\infty)$}{$x\in(0,+\infty)$}
\domanda{4}{La funzione $f(x) = \begin{cases}\frac{ln(1+x^2)}{x} & x>0 \\ 1+ k\cos x & x\leq 0 \end{cases}$ è continua in $x=0$ se e solo se}
\rispostechiuse{k=0}{k=1}{k=-1}{per nessun valore di k}
\domanda{5}{L'insieme delle soluzioni della disequazione $\sqrt{4-x^2}\leq \sqrt{3}$ è}
\rispostechiuse{$[-2,-1]\cup [1,2]$}{$(-\infty,-1]\cup[1,+\infty]$}{$[-1,1]$}{$(-2,-1]\cup[1,2)$}
\domanda{6}{la funzione $f(x) = xe^x -3e^x$ ha}
\rispostechiuse{un punto di massimo globale}{un punto di minimo globale}{un punto di minimo locale ma non globale}{un punto di massimo locale ma non globale}
\domanda{7}{Sia $a_n = \frac{1}{n^2+n}$ e $b_n = \frac{1}{n}$. Allora}
\rispostechiuse{$a_n \sim b_n$}{$a_n=o(b_n)$}{$b_n=o(a_n)$}{nessuna delle alternative proposte}
\domanda{8}{L'integrale $\int_{-2}^{5} \sqrt[3]{x+3}dx$ vale}
\rispostechiuse{3}{315/4}{45/4}{7/8}
\subsection{Domande Aperte}
\domandaaperta{1}{
Data la funzione
$$f(x)=\frac{ln x}{4x^2}$$
\begin{enumerate}
	\item Si studi $f$ e se ne tracci un grafico qualitativo (dominio, limiti ai punti di frontiera del dominio, eventuali asintoti, monotonia, punti di estremo relativo e/o assoluto, convessità/concavità);
	\item si scriva l'equazione della retta tangente al grafico di $f$ nel upnto di ascissa $x=e$;
	\item si calcoli $\int_1^4 f(x)dx$
\end{enumerate}
}

\domandaaperta{2}{Data la serie
$$\serie{2}{+\infty}(\frac{1}{x-1})^n$$

\begin{enumerate}
	\item Si determinino i valori di $x\in \R\{1\}$  per cui la serie converge;
	\item per i valori determinati al punto 1, si calcoli la somma della serie. 
\end{enumerate}
}

\section{Settembre 2020}
\subsection{Domande Chiuse}
\domanda{1}{Dato l'insieme $A=\{\frac{(-1)^n2n}{n+1}, n\geq 1 \}$, allora}
\rispostechiuse{inf $A = -2$}{sup $A = 4/3$}{max $A=2$}{inf $A=-1$}
\domanda{2}{$\lim_{n\to +\infty} \cos \frac{1}{n}\cdot \frac{\ln(1+\frac{1}{n})}{\frac{2}{n}+ \frac{1}{n^3}}=$}
\rispostechiuse{1/2}{1}{$+\infty$}{0}
\domanda{3}{La somma della serie $\serie{2}{+\infty} \frac{4}{3^n}$ vale}
\rispostechiuse{2/3}{6}{2}{-3}
\domanda{4}{sia $(x)=\frac{1}{x}+\sqrt{x}$. Allora $\frac{d}{dx}\ln(f(x))$ per $x=4$ è}
\rispostechiuse{1/12}{7/36}{5/36}{1/36}
\domanda{5}{sia $f(x)\begin{cases}x^2-x & x\leq1 \\ \frac{e^x-e}{3(x-1)^2} & x > 1\end{cases}$ Allora in $x=1$ la funzione $f$:}
\rispostechiuse{Ha discontinuità di seconda specie}{Ha discontinuità di prima specie}{Ha discontinuità eliminabile}{Ha punto di continuità}
\domanda{6}{Siano $f(x)=e^x-2$ e $g(x)= e^{|x|}$. Allora $g \circ f(x) = $}
\rispostechiuse{$e^{|e^x-2|}$}{$e^{|x|-2}$}{$e^{e^{|x|}}-2$}{$(e^x-2)\cdot e^{|x|}$}
\domanda{7}{Sia $f(x) = x^2\ln x$. Allora $f$ è crescente in:}
\rispostechiuse{$(0,e^{-1/2})$}{$(e^{-1/2},+\infty)$}{nessun intervallo}{$(0,+infty)$}
\domanda{8}{$\int_0^1 \frac{3x}{x^2+1}dx=$}
\rispostechiuse{$\frac{3}{2}\ln 2$}{$3\ln 2$}{$\frac{\pi}{12}$}{$\frac{\pi}{4}$}

\subsection{Domande Aperte}
\domandaaperta{1}{data la funzione $f(x)=(1-x)e^{\frac{1}{x}}$,

\begin{enumerate}
\item il suo dominio è:
\item i limiti ai punti di frontiera del dominio sono (4):
\item GLi eventuali asintoti verticali sono
\item Gli eventuali asintoti obliqui sono
\item il più ampio intervallo di monotonia del tipo $(-\infty,k)$ si ha per $k=...$ (la monotonia è del tipo?)
\end{enumerate}
}
\domandaaperta{2}{Data la funzione $f(x) = \frac{\ln x}{x} : (0,+\infty) \to \R$
\begin{enumerate}
	\item Si scivano le primitive $\Phi$:
	\item si determini la primitiva $\Phi$ tale che $\Phi(e^2)=2\Phi(e)$
	\item si calcoli $\int_e^{e^2} \frac{\ln x}{x}dx =$
\end{enumerate}
}
\domandaaperta{3}{Sia $\serie{1}{+\infty}$ una serie numerica
\begin{enumerate}
	\item La serie si dice convergente se:
	\item se $a_n = \ln n - \ln(n+1)$, si calcoli la somma parziale $s_n$:
	\item Usando la definizione di serie convergente, si verifichi se la serie $\serie{1}{+\infty}(\ln n - ln(n+1))$ converge oppure no:
\end{enumerate}
}

\section{Settembre 2021}
\subsection{Domande Chiuse}
\domanda{1}{La serie $\serie{1}{+\infty} (-1)^n \frac{1}{2n^4}$}
\rispostechiuse{converge assolutamente}{converge, ma non assolutamente}{diverge}{è irregolare}
\domanda{2}{$\lim_{n+\to +\infty} \frac{n^3+5ne^{-n^2}-n^2\sqrt{n^3+2}}{-n^3+\cos n -n^2 \sqrt{n}}$ è:}
\rispostechiuse{$-\infty$}{$+\infty$}{1}{0}
\domanda{3}{La funzione $f(x) = \ln x + \frac{x^4}{12}$ è convessa se e solo se}
\rispostechiuse{$x\in(-1,1)$}{$x\in(0,1)$}{$x\in(1,+\infty)$}{$x\in(0,+\infty)$}
\domanda{4}{la funzione $f(x)=\begin{cases}\frac{\ln(1-x^2)}{x^2} & x>0 \\ 1+ k \cos x & x\leq 0\end{cases}$ è continua in $x=0$ se e solo se:}
\rispostechiuse{$k=0$}{$k=-1$}{$k=-2$}{per nessun valore di $k$}
\domanda{5}{L'insieme delle soluzioni della disequazione $x(e^{2x} - 3)<0$ è:}
\rispostechiuse{$(0, \frac{\ln3}{2}$)}{$(-\infty,\frac{\ln3}{2})$}{$(-\infty,0)\cup (\frac{\ln3}{2},+\infty)$}{$(\frac{\ln3}{2},+\infty)$}
\domanda{6}{La funzione $f(x)= e^x - xe^x$ ha:}
\rispostechiuse{un punto di minimo globale}{un punto di massimo globale}{un punto di massimo locale ma non globale}{un punto di minimo locale ma non globale}
\domanda{7}{Sia $a_n=\frac{1}{3n^2-n}$ e $b_n=\frac{1}{n}$. Allora}
\rispostechiuse{$a_n\sim b_n$}{$a_n = o(b_n)$}{$b_n=0(a_n)$}{nessuna delle alternative proposte}
\domanda{8}{L'integrale $\int_{-2}^5 \sqrt[3]{x+3}dx$ vale:}
\rispostechiuse{3}{315/4}{45/4}{7/8}

\subsection{Domande Aperte}

\domandaaperta{1}{Data la funzione 
$$f(x) = \ln x + \frac{2}{x}$$

\begin{enumerate}
	\item Il dominio è:
	\item I limiti agli estremi del dominio sono:
	\item Ha asintoti? Se sì quali?
	\item Quali sono gli intervalli di monotonia?
	\item Ci sono estremanti? se si quali? Assoluti o relativi?
	\item Si determinino gli intervalli di concavità/convessità
	\item La retta tangente al graico di $f$ nel punto di ascissa $x=1$ ha equazione:
	\item $\int_1^e f(x) dx$ vale
\end{enumerate}
}
\domandaaperta{2}{Data la serie $\serie{2}{+\infty} (\frac{1}{x-4})^n$,}
\begin{enumerate}
	\item Si determinino i valori di $x\in \R \backslash \{4\}$ per cui la serie converge:
	\item Per i valori determinati al punto precedente si calcoli la somma della serie:
\end{enumerate}


\end{document}