\chapter{Funzioni}

\section{Generalità}
\definizione{
	Siano $X$ e $Y$ insiemi. Si dice che è data una funzione di $X$ in $Y$, se è data una regola che a ogni elemento di $X$ associa \emph{uno e uno solo} elemento di $Y$.
}
\paragraph{In altre parole} assegnare una funzione $f$ di $X$ in $Y$ significa dare un procedimento che consenta di assegnare a ogni $x \in X$ un ben determinato $y \in Y$. Tale $y$, corrispondente di $x$ tramite la funzione $f$, si indica con $f(x)$, cioè $y = f(x)$ e $y$ si chiama \emph{immagine di $x$ secondo $f$}. per indicare che $f$ è funzione di $X$ in $y$ si scrive $f: X \rightarrow Y$.

\definizione{
	se $f: X \rightarrow Y$ è una funzione, gli insiemi \emph{$X$ e $Y$} sono rispettivamente il \textbf{dominio} e il \textbf{codominio} di $f$.
}
\paragraph{Assegnare una funzione} significa assegnare:
\begin{itemize}
	\item Un dominio $X$
	\item Un codominio $Y$
	\item Una regola che a ogni $x$ del dominio associ una $y$ del codominio
\end{itemize}
Pertanto: due funzioni $f$ e $g$ sono \textbf{uguali} se e solo se hanno lo \emph{stesso dominio}, lo\emph{ stesso codominio}, e inoltre si ha \emph{$f(x) = g(x)$ per ogni x del dominio}.\\
Non basta dunque dire che la regola sia la stessa: occorre anche che il dominio e il codominio dati siano gli stessi.

\subsection*{Insieme immagine}
Se $f: X \rightarrow Y$ è una funzione, ed $S$ è sottoinsieme del dominio $X$ (cioè $S \subseteq X$) si indica con $f(s)$ l'insieme degli elementi $y$ che sono immagini secondo $f$ di qualche elemento di $S$.
\\In linguaggio matematico è $f(S) = \{ f(x) : x \in S\}$ oppure si può anche descrivere così: $f(S) = \{y \in Y : \exists x \in S \text{ tc } y = f(x)\}$.
\\L'insieme $f(S)$ è detto \emph{immagine di $S$ tramite $f$}; per $S = X$ (quindi se S corrisponde col dominio), $f(X)$ è detto \emph{immagine di f}.
\\Vale sempre $f(X) \subseteq Y$, quindi \textbf{l'immagine è sempre conenuta nel codomino}, ma in generale è $f(X) \subset Y$, cioè immagine e codominio sono diversi.

\paragraph{In parole povere}
L'immagine di una funzione è l'insieme dei valori assunti da una funzione sul proprio dominio,
ed è quindi contenuta nel codominio (o insieme di arrivo) della funzione, con il quale può al più coincidere.
\paragraph{Come si calcola?}
Per trovare l'insieme immagine, il modo più facile è guardare il grafico della funzione

\subsubsection{Immagine Inversa}
Dualmente al concetto di immagine c'è quello di antiimmagine.
\paragraph{Definizione. }Sia $\function $ una funzione, sia $F \subseteq Y$.
Si chiama immagine inversa $f^\leftarrow(T)$ (o controimmagine, o antiimmagine) di $T$ mediante $f$ l'insieme degli $x \in X$ la cui immagine sta in $T$.
in simboli: $f^\leftarrow(T) = \{ x \in X : f(x) \in T\}$.
\paragraph{In parole povere} L'antiimmagine mediante un insieme $T$ è l'insime dei valori del dominio la cui immagine è contenuta in $T$


\section{Composizione di Funzioni}
Siano $\function$ e $g: Y \rightarrow Z$ funzioni.
Si ottiene una funzione $g \circ f : X \rightarrow Z $ (che si legge "g tondo f" o "g cerchietto f") ponendo $g \circ f = g(f(x))$ per ogni $x \in X$;
$g \circ f$ è detta \emph{funzione composta} di f e g; è la funzione ottenuta applicando $f$ e $g$ nell'ordine.
\\Per poter fare la composizione, \textbf{occorre che il codominio di $f$ coincide con il dominio di $g$}.


\esempio{
	Siano $f: \mathbf{R} \rightarrow \mathbf{R}$ e $g: \mathbf{R} \rightarrow \mathbf{R}$ due funzioni così definite:
	\\ $f(x) = x^2$ e $g(x) = x - 2$ allora $f \circ g = f(g(x)) = (x-2)^2$
}
\section{Iniettività e Suriettività}

\subsection*{Funzioni Suriettive}
Quando l'\emph{insieme immagine e il codomino di una funzione coincidono}, essa si dice che è \emph{suriettiva}
\definizione{Una funzione $\function$ si dice \textbf{suriettiva} se è $f(X) = Y$.
	Attenzione che su alcuni testi di Analisi il termine codominio è inteso nel senso di immagine (e non è propriamente corretto).}
\paragraph{Graficamente} Per determinare graficamente se una funzione è suriettiva è necessario verificare che la proiezione dell'intero
grafico sull'asse delle y copra interamente l'asse stesso, se così fosse possiamo affermare che la funzione è suriettiva.
\subsection*{Funzioni Iniettive}
La funzione $\function$ è detta \textbf{iniettiva} se trasforma elementi distinti in elementi distinti, ovvero:
\definizione{$f: X \rightarrow Y$ si dice iniettiva se per ogni $x_1, x_2 \in X$ e $x_1 \neq x_2$ implicano $f(x_1) \neq f(x_2)$.}
\paragraph{}Per vedere che una $\function$ \emph{non} è iniettiva, basta esibire anche una sola coppia $x_1, x_2$ di elementi distinti ($x_1 \neq x_2$) del dominio per cui sia $f(x_1) = f(x_2)$.
\\Per provare invece che è iniettiva, occorre dimostrare che per ogni coppia di elementi distinti $x_1, x_2 \in X, x_1 \neq x_2$ si ha $f(x_1) \neq f(x_2)$.
\paragraph{Graficamente} Per determinare graficamente se una funziona è iniettiva è necessario verificare che tracciando
 un'ipotetica linea orizzontale la linea non intersechi il grafico in più di un punto.
\\ Per esempio tracciando una linea orizzontale nel grafico di una parabola la linea interseca il grafico in 2 punti e
questo ci indica che la funzione NON è iniettiva.

\subsubsection{Biiezioni}
Una funzione $\function$ è detta biiettiva se è sia iniettiva che suriettiva.
Quindi $\function$ è biiettive se e solo se per ogni $y \in Y$ esiste uno e un solo $x \in X$ tale che sia $y = f(x)$.
Che esista almeno un tale $x$ dice che $f$ è suriettiva, che sia unico dice che $f$ è iniettiva.
Una funzione biiettiva viene detta anche biiezione e corrispondenza biunivoca.

\section{Funzione Inversa}
Se $\function$ è biiettiva, si può definire una funzione inversa di $f$,$f^{-1} : Y \rightarrow X$, nel modo seguente:
dato $y \in Y$, $f^{-1}(y)$ è quell'unico $x \in X$ tale che sia $y = f(x)$.
se $\function$ non è biiettiva, la funzione inversa di f non può essere definita.
\paragraph*{Graficamente} se $f$ è funzione reale di variabile reale biiettiva (ovvero è invertibile),
 il grafico dell'inversa $f^{-1}$ si ottiene facendo il simmetrico del grafico di $f$ rispetto alla retta di equazione $y=x$.

\subsection*{Come si calcola}
In linea generale, per calcolare l'inversa di una funzione bisogna:
\begin{itemize}
	\item mettere la funzione nella forma $y=...$
 \item Isolare la $x$
 \item Una volta isolata la $x$, si sotituisce $y$ con $x$, e $x$ con $f(x)$ per trovare la funzione
\end{itemize}
\esempio{$f(x) = x^3$, allora $f^{-1}(x)$:
\\$y=x^3 \to x^3 = y \to x=\sqrt[3]{y} \to f(x) = \sqrt[3]{x}$
}

\subsection*{Derivata della Funzione Inversa}
In molti casi l'inversa di una funzione non è facilmente calcolabile,
esiste però un modo per \emph{calcolare la derivata di una funzione inversa senza saperne l'espressione analitica}.
\definizione{
	Consideriamo una funzione $y=f(x)$ biunivoca e derivabile in un punto $x_0$ e supponiamo inoltre che $f'(x_0)\neq 0$.
	\\Allora la funzione inversa $x = f^{-1}(y)$ è derivabile nel punto $y_0 = f(x_0)$, e la sua derivata in tale punto è:
	$$ (f^{-1})'(y_0) = \frac{1}{f'(x_0)} = \frac{1}{f'(f^{-1}(y_0))}$$
}
\paragraph*{Ovvero} Questo teorema ci serve all'esame per quelle funzioni di cui dobbiamo trovare la derivata dell'inversa che però sono pressochè impossibili da derivare.
\\Sia $f(x)$ una funzione (ovviamente invertibile e derivabile) e sia $g$ la sua inversa. 
$g'(y_0)$ (con un $y_0$ fornito) è trovabile in questo modo:
\\Dal teorema sappiamo che $g'(y_0) = \frac{1}{f'(x_0)}$, e sappiamo (tramite la formula della funzione inversa) che $y=f(x)$. 
\\Se $y=y_0$ (che è il valore di $y$ in cui dobbiamo trovare $g'$), allora $y_0= f(x)$ ci da un certo valore di $x=x_0$.
\\Avendo adesso il valore di $y_0$ e di $x_0$ basta riempire la formula per trovare $g'(y_0)$.
\\Si può riassumere questo processo nei seguenti passaggi: 
Data $g(x) = f^{-1}(x)$ e avendo $y_0$, sappiamo che $g'(y_0) = \frac{1}{f'(x_0)}$.
Quindi:
\begin{enumerate}
	\item trovo $x_0$ ponendo $y_0=f(x)$
	\item trovo $g'(y_0)=\frac{1}{f'(x_0)}$
\end{enumerate}
\esempio{detta $g$ la funzione inversa di $f(x) = x^{\frac{1}{3}}$, allora $g'(1) =$
\\Dal teorema ho $g'(1) = \frac{1}{f'(x_0)}$, e so che $y=x^{\frac{1}{3}}$.
Ponendoli a sistema per trovare $x$ ottengo:
$$
\begin{cases}
	y= 1 \\
	y=x^{\frac{1}{3}}
\end{cases}
\begin{cases}
	y = 1 \\
	x = 1
\end{cases}
$$
Quindi, $g'(1) = \frac{1}{f'(1)}$.
$f'(x) = \frac{1}{3x^{\frac{2}{3}}}$,
$f'(1) = \frac{1}{3}$.
\\Applicando quindi la formula ottengo $g'(1) = 3$.
}
\section*{Funzioni Pari/Dispari}
Sia $\function$ una funzione reale di variabile reale (quindi $X, Y in \R$).
\paragraph*{Pari}
Si dice che $f$ è pari se per ogni $x \in X$ anche $-x \in X$ ed è $f(x) = f(-x)$;
\\Le funzioni pari hanno grafico simmetrico rispetto all'asse delle ordinate.
\paragraph*{Dispari}
Si dice che $f$ è dispari se per ogni $x \in X$ anche $-x \in X$ ed è $f(x) = -f(x)$;
\\Le funzioni dispari hanno grafico simmetrico rispetto all'origine

\section{Monotonia di una Funzione}
Sia $\function$ una funzione reale di variabile reale (quindi $X, Y in \R$).

\subsection*{Funzione Crescente}
Si dice che $f$ è \emph{crescente} se da $x_1, x_2 \in X$ e $x_1 < x_2$ segue $f(x_1) \leq f(x_2)$;
\\Se invece $x_1, x_2 \in X$ e $x_1 < x_2$ implicano $f(x_1) < f(x_2)$ allora $f$ si dice \emph{strettamente crescente};

\subsection*{Funzione Decrescente}
Si dice che $f$ è \emph{decrescente} se da $x_1, x_2 \in X$ e $x_1 > x_2$ segue $f(x_1) \geq f(x_2)$;
\\Se invece $x_1, x_2 \in X$ e $x_1 < x_2$ implicano $f(x_1) > f(x_2)$ allora $f$ si dice \emph{strettamente decrescente};

\paragraph{Come si calcola la monotonia?}
Per calcolare la monotonia di una funzione, bisogna porre la derivata (che è il "rateo" di crescita della funzione) \emph{Maggiore (o minore o maggiore uguale o minore uguale) di 0}.
Nei punti in cui la derivata rimane maggiore (minore,...) di 0, la funzione è \emph{monotona crescente (decrescente,...)}

\paragraph*{Tip per l'esame (Crocette)} Può capitare che venga indicato in una delle
possibili risposte un intervallo dove $f(x)$ non è definita, per esempio se ho $\log x$ e una
delle possibili risposta è "$log x$ decresce per $(-\infty, 0]$", posso subito
escluderla dato che il dominio di $log x$ è $x > 0$. 

\subsection{Teorema di Weierstrass}
\definizione{
	Sia $f: [a,b] \rightarrow \mathbb{R}$ una funzione continua.
	\\Allora f assume massimo e minimo in $[a,b]$, ovvero:
	\\ esistono $x_m, \, x_M \in [a,b]$ tali che $f(x_m) \leq f(x) \leq f(x_M)$ per ogni $x \in [a,b]$.
	\\ Si dice che $x_m$ è punto di minimo per f, e $m=f(x_m)$ è il minimo di f;
	\\ analogamente $x_M$ è punto di massimo, e $M = f(x_M)$ è il massimo di f.
}

\section{Studio di Funzione}
Una parte importantissima dell'Analisi Matematica (che fa parte di un intero esercizio d'esame) è lo \emph{Studio di Funzione},
che mette insieme quasi tutti gli argomenti del corso.

Lo studio di funzione è composto di 5 passaggi:
\begin{enumerate}
	\item Definizone del Dominio
 \item Segno e intersezione con gli assi
 \item Limiti e Asintoti
 \item Segno della derivata
\end{enumerate}

\subsection*{Definizione del Dominio}
Una funzione $f:\R \to \R$ non sempre è definita in tutto $\R$, spesso ci sono degli intervalli (o dei punti) in cui non è definibile.
Per trovare questi punti in cui non è definibile bisogna:
\begin{itemize}
	\item porre tutti i \textbf{Denominatori $\neq 0$}
 \item porre tutti gli \textbf{Argomenti delle Radici pari $\geq 0$}
 \item porre tutti gli \textbf{Argomenti dei Logaritmi $>0$}
 \item ogni funzione del tipo $f(x)^{g(x)}$, $f(x)$ va posto maggiore di 0
 \item gli argomenti di $\arcsin$ e $\arccos$ vanno posti $-1 < x <1$
\end{itemize}
Ovviamente, una volta definito il dominio della funzione vado a eliminare dal piano cartesiano le zone in cui $f(x)$ non è definita.
\subsection*{Simmetrie/Periodicità}
Questa fase è facoltativa ma ogni tanto ci aiuta.
\begin{equation}
	f(-x) = \begin{cases}
		f(x) \implies \text{Funzione Pari}\\
		-f(x) \implies \text{Funzione Dispari}\\
		NIL \implies \text{Funzione ne Pari ne Dispari}
	\end{cases}
\end{equation}
Le \emph{Funzioni Periodiche} sono generalmente quelle goniometriche.

\subsection*{Segno e Intersezione con gli Assi}
Il segno della funzione ci permette di vedere dove la funzione è positiva.
L'intersezione con gli assi invece ci aiuta a disegnare la funzione.
\begin{itemize}
	\item $f(x)\geq 0$ Per vedere dove la funzione è positiva
 \item $f(x)=0$ è l'intesezione con l'asse delle ascisse
 \item $f(0)$ è l'intersezione con l'asse delle ordinate
\end{itemize}

\subsection*{Limiti e asintoti}
I limiti ci servono a capire come si comporta la funzione ai limiti del dominio e nei punti di discontinuità.
Bisogna calcolare i limiti di x ai punti di accumulazione del dominio che non fanno parte del dominio e agli estremi di esso.

\begin{itemize}
	\item Se $x_0 \notin DOM$ e $\limite{x}{x_0} f(x) = c \implies$ il grafico ha un "buco" in $(x_0,c)$.
 \item Se $x_0 \notin DOM$ e $\limite{x}{x_{0}^\pm} f(x) = \pm \infty \implies$ $x=x_0$ è asintoto VERTICALE.
 \item Se $\limite{x}{\pm \infty} = l$ l è asintoto ORIZZONTALE
 \item Se $\limite{x}{\pm \infty} = \pm \infty$ potrebbe eistere Asintoto Obliquo
\end{itemize}
\subsection*{Monotonia: Crescenza/Decrescenza}
La monotonia di una funzione è \emph{l'intervallo in cui una funzione è crescente o decrscente}.
Per determinare la monotonia di una funzione, bisogna trovarne la derivata e determinare il suo segno, quindi porla $>0$.
Nei punti in cui la derivata di una funzione è \emph{maggiore di 0} la funzione è \emph{crescente}, nei punti in cui invece è \emph{mionre di 0} la funzione è \emph{decrescente}.

\subsection*{Massimi e Minimi}
Tramite lo studio della monotonia si possono anche trovare i punti di massimo e minimo di una funzione %TODO

\subsection*{Concavità e Convessità}
\paragraph*{Cosa sono} Una funzione convessa è tale se il segmento che congiunge due punti qualsiasi del suo grafico giace 
sopra il grafico stesso o coincide con una sua parte. Il contrario vale per le funzioni concave.
\begin{itemize}
	\item CONC\textbf{A}VA $\cap$
	\item CON\textbf{V}ESSA $\cup$
\end{itemize}
\paragraph*{Come si calcola}
Per trovare la concavità di una funzione, bisogna fare la derivata seconda e porla $>0$ per trovarne il segno.
Nei punti in cui la derivata seconda è positiva la funzione è CONVESSA, nei punti in cui è negativa la funzione è CONCAVA.

\begin{itemize}
	\item - CONC\textbf{A}VA $\cap \implies f''(x)$ positiva 
	\item + CON\textbf{V}ESSA $\cup \implies f''(x)$ negativa
\end{itemize}

\subsection*{Retta tangente al grafico}
Ogni tanto ci viene chiesto di determinare l'\emph{equazione della retta tangente al grafico di una funzione in un punto}.
\\Una retta è tangente al grafico nel punto $x_0$ se hanno uno e un solo punto in comune, che sarà $x_0$ appunto.

\paragraph{Come si calcola}
Bisogna innanzitutto avere la derivata della funzione. Poi, tenendo conto dell'equazione generica della retta $y=mx+q$ bisogna:
\begin{itemize}
	\item Trovare $m= f'(x_0)$
 \item Trovare $q = f(x_0)-f'(x_0) \cdot x_0$
 \item Infine, "monti" l'equazione della retta $y=mx+q$
\end{itemize}
