
\chapter{Calcolo Differenziale}
\section{Definizione di Derivata}
Partiamo dal concetto di Rapporto Incrementale, che sta alla base delle derivate:
\paragraph*{Il rapporto incrementale}
Di una funzione in un punto è il rapporto tra la variazione di ordinate e la variazione di ascisse definite a partire da un incremento $h$.
\\Consideriamo una funzione $f:\R\to\R$, $y=f(x)$.
Il rapporto incrementale della funzione f nel punto generico punto $x_0$ è:
\[
	\frac{\Delta y}{\Delta x} := \frac{f(x_0+h)-f(x_0)}{h}	
\]

\definizione{
La \textbf{Derivata} di una funzione $f(x)$ in un punto $x_0$ è definita come
il limite del rapporto incrementale della funzione nel punto al tendere dell'incremento a zero:
\[
	f'(x_0) = \lim_{h \rightarrow 0} \frac{f(x_0 + h) - f(x_0)}{h}
\]
Se questo limite \emph{esiste}, allora $f(x)$ è derivabile nel punto $x_0$.
}
Per gli esercizi ci sono dei trucchi più rapidi per verificare la derivabilità di un punto.


\subsection{Criterio di Derivabilità}
Una funzione è derivabile in un punto $x_0$ \emph{se e solo se}:
Essa è continua nel punto $x_0$ e i limiti sinistro e destro del rapporto incrementale in quel punto esistono finiti e coincidono.

\paragraph*{tips and tricks}
Essendo il limite del rapporto incrementale nel punto la derivata stessa, 
per controllare la derivabilità di una funzione in un punto basta:
\begin{itemize}
	\item Controllare se essa è CONTINUA.
	\item Controllare se i limiti $x \to x_0$ \emph{dx} e {sx} della derivata stessa coincidano
\end{itemize}



\section{Derivate note}
Non esistono veramente delle derivate note, però alcune funzioni hanno delle derivate che vale la pena ricordare per velocizzare il processo di derivazione:
\begin{itemize}
	\item[\textbf{Seno}] \derivata{\sin(x)}{\cos(x)}
	\item[\textbf{Coseno}] \derivata{\cos(x)}{-\sin(x)}
	\item[\textbf{Arcotangente}] \derivata{\arctan(x)}{\frac{1}{1+x^2}}
	\item[\textbf{Logaritmo}] \derivata{\ln(x)}{\frac{1}{x}}
	\item[\textbf{Radice}] Meglio fare il calcolo a mano con le potenze%\derivata{\sqrt[\alpha]{x}}{\frac{1}{\alpha \sqrt[\alpha]{x}}}
	\item[\textbf{e$^x$}] \derivata{e^x}{e^x}
	\item[\textbf{e$^{-x}$}] \derivata{e^{-x}}{-e^{-x}}
	\item[\textbf{1/x}] \derivata{\frac{1}{x^2}}{-\frac{2}{x^3}}
	\item[\textbf{x$^\alpha$}] \derivata{x^\alpha}{\alpha x^{\alpha - 1}}
\end{itemize}
\subsection*{Derivate composte}
\begin{itemize}
	\item[Composizione] $f(g(x)) \implies f'(g(x)) \cdot g'(x)$
	\item[Prodotto] $f(x) \cdot g(x) \implies f'(x) \cdot g(x) + f(x) \cdot g'(x)$
	\item[Divisione] $h(x)=\frac{f(x)}{g(x)} \implies h'(x) = \frac{f'(x)g(x) - f(x)g'(x)}{[g(x)]2}$
\end{itemize}
\subsection*{Punti di non derivabilità}
Una funzione è derivabile in tutto il suo intervallo solamente se è continua, la continuità è quinda una condizione
\textbf{necessaria} per la derivabilità. Non è detto però che se una funziona è continua sia derivabile,
ci sono dei casi in cui la funzione risulta continua, ma non derivabile, in questo caso parliamo di
\textbf{punti di non derivabilità}. Esistono 3 particoli tipi:
\begin{itemize}
	\item Punto Angoloso
	\item Cuspide
	\item Flesso a Tangente Verticale
\end{itemize}
\paragraph*{Verifica della derivabilità di un punto} Per verificare che un punto sia derivabile 
(dopo aver verificato la sua continuità) è necessario calcolare il limite destro e sinistro della derivata della funzione per il punto in questione.
Se i limiti sono finiti e assumono lo stesso valore il punto sarà derivabile, altrimenti il punto sarà non derivabile, in particolare potrò avere
degli specifici punti di non derivabilità:
\begin{itemize}
	\item I limiti destro e sinistro esistono e sono finiti, ma assumono valori diversi - \textbf{Punto Angolso}
	\item I limiti destro e sinistro sono infiniti di segno opposto - \textbf{Punto di Cuspide}
	\item I limiti desto e sinistro sono infiniti dello stesso segno - \textbf{Flesso a Tangente Verticale}
\end{itemize}
Possono esserci anche altri casi di non derivabilità senza nome, ma a noi non interessa, ci basterà dire che si tratta di un punto non derivabile.

\section{Formula di Taylor}
La formula di Taylor consente di \emph{Approssimare, almeno localmente}, tutte le funzioni sufficientemente regolari con dei \textbf{Polinomi}
rendendone quindi più agevole lo studio. Più derivate utilizzo, (quindi più è alto l'ordine del polinomio) e più sarà precisa questa approssimazione.
\\Ovviamente perchè questa formula sia fattibile la funzione deve essere derivabile in $x_0$ almeno $k$ volte.

\paragraph*{Il Polinomio di Taylor}
Una funzione $f(x)$, che passi per un punto $x_0$ e che abbia in quel punto tutte le derivate necessarie,
si può approssimare nel punto $x_0$ mediante un polinomio così definito:
$$P_k(x)=f(x_0)+\frac{1}{1!}f'(x_0)(x-x_0) + \frac{1}{2!}f''(x_0)(x-x_0)^2 +... + \frac{1}{k!}f^{(k)}(x_0)(x-x_0)^k$$
L'errore che si commette in questa approssimazione non è maggiore della prima derivata che si trascura
\subsection*{Polinomio di McLaurin}
Nel caso in cui il punto $x_0$ sia l'origine ($x_0=0$) si ottiene la formula di McLaurin:
$$T_f(x) = f(0) + f'(0)x + \frac{f''(0)}{2!}x^2 +...+ \frac{f^{(k)}(0)}{k!}x^k$$
Nota che quando si chiede il "polinomio di McLaurin del secondo ordine" ci si ferma alla derivata seconda, del terzo ordine alla terza e così via.

\section{Teorema di Rolle}
Sia $f[a,b]\rightarrow \mathbb{R}$ una funzione continua in $[a,b]$ e
derivabile in $[a,b]$. Se la funzione assume lo stesso valore agli estremi
dell'intervallo, ossia:
\[
	f(a)=f(b)
\]
Allora esiste almeno un punto $x_0 \in (a, b)$ tale che:
\[
	f'(x_0) = 0
\]
\paragraph*{Le 3 condizioni da controllare negli esercizi}
Quando in un esercizio ci viene richiesto che la funzione soddisfi le ipotesi del
teorema di Rolle devo verificare 3 condizioni:
\begin{enumerate}
	\item Che la funzione assuma lo stesso valore agli estremi dell'intervallo, quindi che $f(a) = f(b)$
	\item Che $f:[a,b]$ sia continua in tutto l'intervallo $[a,b]$ (quindi estremi compresi)
	\item Che $f:[a,b]$ sia derivabile in tutto l'intervallo $(a,b)$ (quindi estremi esclusi)
\end{enumerate}
Se viene richiesto di trovare il punto c dove $f'(c) = 0$, allora devo
sostituire c nella funzione derivata e porla $= 0$. Infine ricavo c stessa.

\section{Teorema di Lagrange}
Sia $f[a,b]\rightarrow \mathbb{R}$ una funzione continua in $[a,b]$ e derivabile 
in $(a,b)$. Allora esiste almeno un punto $x_0$ interno all'intervallo $(a,b)$ tale che:
\[
	f(b) - f(a) = f'(x_0)(b-a)
\]
\paragraph*{Le 2 condizioni da controllare}
\begin{enumerate}
	\item Funziona continua in $[a,b]$
	\item Funzione derivabile in $(a,b)$
\end{enumerate}
In poche parole chiedere se una funzione rispetti le ipotesi di Lagrande è come
chiedere di verificare se una funzione è continua e derivabile.

