\chapter{Calcolo Differenziale}
\section{Criterio di Derivabilità}
\quickDefinizione{Una funzione è derivabile in un punto $x_0$ se essa è continua in quel punto e se il limite sinistro e destro della derivata in quel punto coincidono.}
\subsection*{Definizione formale - Rapporto incrementale}
\definizione{La derivata di una funzione $y = f(x)$ 
in un punto $x_0$ è definita come il limite del rapporto incrementale della funzione nel punto:
\begin{equation*}
	f'(x_0) = \lim_{\Delta x \rightarrow 0} \frac{\Delta y}{\Delta x} = \lim_{h \rightarrow 0} \frac{f(x_0 + h) - f(x_0)}{h}
\end{equation*}
Se questo limite esiste il punto è derivabile.
}
Per gli esercizi ci sono dei trucchi più rapidi per verificare la derivabilità di un punto.%
% Inserire esempi trucchi per verificare la derivabilità
\section{Derivate note}
Non esistono veramente delle derivate note, però alcune funzioni hanno delle derivate che vale la pena ricordare per velocizzare il processo di derivazione:
\begin{itemize}
	\item[\textbf{Seno}] \derivata{\sin(x)}{\cos(x)}
	\item[\textbf{Coseno}] \derivata{\cos(x)}{-\sin(x)}
	\item[\textbf{Arcotangente}] \derivata{\arctan(x)}{\frac{1}{1+x^2}}
	\item[\textbf{Logaritmo}] \derivata{\ln(x)}{\frac{1}{x}}
	\item[\textbf{Radice}] Meglio fare il calcolo a mano con le potenze%\derivata{\sqrt[\alpha]{x}}{\frac{1}{\alpha \sqrt[\alpha]{x}}}
	\item[\textbf{e$^x$}] \derivata{e^x}{e^x}
	\item[\textbf{e$^{-x}$}] \derivata{e^{-x}}{-e^{-x}}
	\item[\textbf{1/x}] \derivata{\frac{1}{x^2}}{-\frac{2}{x^3}}
	\item[\textbf{x$^\alpha$}] \derivata{x^\alpha}{\alpha x^{\alpha - 1}}
\end{itemize}
\subsection*{Derivate composte}
\begin{itemize}
	\item[Composizione] $f(g(x)) \implies f'(g(x)) \cdot g'(x)$
	\item[Prodotto] $f(x) \cdot g(x) \implies f'(x) \cdot g(x) + f(x) \cdot g'(x)$
	\item[Divisione] $h(x)=\frac{f(x)}{g(x)} \implies h'(x) = \frac{f'(x)g(x) - f(x)g'(x)}{[g(x)]2}$
\end{itemize}
\subsection*{Punti di non derivabilità}
Una funzione è derivabile in tutto il suo intervallo solamente se è continua, la continuità è quinda una condizione
\textbf{necessaria} per la derivabilità. Non è detto però che se una funziona è continua sia derivabile,
ci sono dei casi in cui la funzione risulta continua, ma non derivabile, in questo caso parliamo di
\textbf{punti di non derivabilità}. Esistono 3 particoli tipi:
\begin{itemize}
	\item Punto Angoloso
	\item Cuspide
	\item Flesso a Tangente Verticale
\end{itemize}
\paragraph*{Verifica della derivabilità di un punto} Per verificare che un punto sia derivabile 
(dopo aver verificato la sua continuità) è necessario calcolare il limite destro e sinistro della derivata della funzione per il punto in questione.
Se i limiti sono finiti e assumono lo stesso valore il punto sarà derivabile, altrimenti il punto sarà non derivabile, in particolare potrò avere
degli specifici punti di non derivabilità:
\begin{itemize}
	\item I limiti destro e sinistro esistono e sono finiti, ma assumono valori diversi - \textbf{Punto Angolso}
	\item I limiti destro e sinistro sono infiniti di segno opposto - \textbf{Punto di Cuspide}
	\item I limiti desto e sinistro sono infiniti dello stesso segno - \textbf{Flesso a Tangente Verticale}
\end{itemize}
Possono esserci anche altri casi di non derivabilità senza nome, ma a noi non interessa, ci basterà dire che si tratta di un punto non derivabile.

\section{Formule di Taylor e di Mclaurin}
Lo scopo delle formule di Taylor e di McLaurin è di approssimare una funzione con un polinomio di grado $k$ arbitrario centrato in $x_0$ nel caso della formula di Taylor,
e in 0 nel caso di quella di McLaurin.
\subsection*{Polinomio di Taylor}
Una funzione $f(x)$, che passi per un punto $x_0$ e che abbia in quel punto tutte le derivate necessarie,
si può approssimare nel punto $x_0$ mediante un polinomio così definito:
$$P_k(x)=f(x_0)+\frac{1}{1!}f'(x_0)(x-x_0) + \frac{1}{2!}f''(x_0)(x-x_0)^2 +... + \frac{1}{k!}f^{(k)}(x_0)(x-x_0)^k$$
L'errore che si commette in questa approssimazione non è maggiore della prima derivata che si trascura
\subsection*{Polinomio di McLaurin}
Nel caso in cui il punto $x_0$ sia l'origine ($x_0=0$) si ottiene la formula di McLaurin:
$$T_f(x) = f(0) + f'(0)x + \frac{f''(0)}{2!}x^2 + \frac{f'''(0)}{3!}x^3+...+ \frac{f^{(k)}(0)}{k!}x^k$$
Nota che quando si chiede il "polinomio di McLaurin del secondo ordine" ci si ferma alla derivata seconda, del terzo ordine alla terza e così via.
