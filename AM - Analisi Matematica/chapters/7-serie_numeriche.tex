\chapter{Serie Numeriche}
\section{Introduzione}
Da sempre ci si è posti il problema di sommare infiniti numeri e ovviamente pensare di ottenere un numero finito dalla somma di infiniti numeri potrebbe sembrare qualcosa di assurdo.
In realtà non è cosi, infatti spesso è possibile che la somma di infiniti numeri può darci un numero finito.
\paragraph*{Richiamiamo le successioni}
Come si dovrebbe già sapere, chiamiamo successione ogni funzione il cui dominio sia l'insieme $\mathbb{N}$ dei numeri naturali.
Le successioni a cui siamo particolarmente interessati sono le \emph{successioni reali}, ovvero le funzioni $f : \mathbb{N} \rightarrow \R$.
\definizione{
	Si chiama \textbf{serie numerica reale} una coppia ordinata di successioni \\
	($(x_j)_{j\in \mathbf{N}}, (s_m)_{m\in \mathbf(N)}$) di numeri reali legate dalla seguente relazione
	\begin{equation*}
		s_m = x_0 + ... + x_m
	\end{equation*}
	Che equivale alla notazione:
	\begin{equation*}
		s_m = \sum_{n=0}^m x_n
	\end{equation*}
}
\paragraph*{}La successione $(x_j)_{j\in \mathbf{N}}$ è il \emph{termine generale} della serie, la successione $(s_m)_{m\in \mathbf(N)}$ è la successione delle somme parziali, o \emph{ridotte} della serie data.\\
Si dice che una serie \textbf{converge} se converge la successione $(s_m)_{m\in \mathbf(N)}$ delle sue ridotte, in tal caso il limite finito $s$ di $(s_m)_{m\in \mathbf(N)}$ si chiama \emph{somma della serie}
\paragraph*{}Se la successione delle ridotte ha limite $+ \infty$ o $- \infty$, si usa dire che la serie è \emph{divergente} (a $+ \infty$, $- \infty$), mentre se la successione delle ridotte non ha limite, né finito né infinito, si dice che la serie è \emph{indeterminata}.
\\Questi appellativi descrivono il \emph{carattere} della serie: è  la terminologia delle successioni applicata alla successione delle somme parziali.
\subsection*{In parole povere...}
Una serie numerica è il modo di indicare una somma di infiniti numeri.
Come dice la definizione, una serie è formata da una coppia ordinata di successioni, una ($s$) che indica la successione delle somme parziali, e l'altra ($x$, spesso indicata con $a$) che corrisponde al termine generale della serie.
\\Una serie può convergere, quindi ha un limite finito che si chiama somma della serie. Può anche divergere a piò o meno infinito oppure essere indeterminata
\section{Serie Convergenti}
\subsection{Serie Geometrica}
La serie $\sum_{n=0}^{+\infty} q^n$, con $q \in \R$ è detta \emph{Serie geometrica}
Questo è uno dei (pochi) casi in cui si riescono a calcolare esplicitamente, al variare di $q$, le somme parziali $s_n$.
Infatti possiamo facilmente calcolare il limite della successione $(s_n)_{n\in \mathbb{N}}$ e determinare il carattere della serie geometrica.
risulta:
\begin{equation}
	\limite{n}{+\infty} s_n \begin{cases}
		\text{non esiste} & q \leq -1  \\
		= \frac{1}{1-q}   & -1 < q < 1 \\
		= +\infty         & q \geq 1
	\end{cases}
\end{equation}
Per cui
	\begin{equation}
		\sum_{n=0}^{+\infty} q^n \text{è } \begin{cases}
			indeterminata      & q \leq -1  \\
			convergente        & -1 < q < 1 \\
			divergente +\infty & q \geq 1
		\end{cases}
	\end{equation}

\esempio{
	$\sum_{n=0}^{+\infty} (\frac{1}{2})^n$ Converge (essendo $\frac{1}{2} < 1$) e la somma è $\frac{1}{1-\frac{1}{2}} = 2$
}

\paragraph{In parole povere}
Se la serie è geometrica, quindi del tipo $x^n$ e il suo termine ($x$) è comporeso tra -1 e 1, allora per calcolare il carattere della serie basta fare: $\frac{1}{1-x}$
\\\nb{La formula risolutiva risolve le serie che partono da $n=0$, se parte da $n=1$ bisogna togliere dal risultato $x^0$, se parte da $n=2$ bisogna togliere $x^1$ e $x^0$ e così via.
\\Nota anche che se la serie presenta una costante moltiplicativa, questa va poi moltiplicata anche alle somme che vai a rimuovere!
\\$k\cdot \sum_{n=1} q^n = k(\frac{1}{1-q} - q^0) = \frac{k}{1-q}-k(q^0)$
}
\subsection{Serie Telescopica}
Sia $\sum_{n=1}^{+\infty} a_n$ Una serie tale che esiste una successione infinitesima ($b_n$) per cui $a_n = b_n - b_{n+1} \forall n\in \mathbb{N}$.
Allora risulta:
\begin{equation*}
	s_n = \sum_{k=1}^{n}a_k = \sum_{k=1}^n (b_k - b_{k+1}) = b_1 - b_{n+1}
\end{equation*}
Da cui
\begin{equation*}
	\sum_{k=1}^{n}a_k = \limite{n}{+\infty} s_n = b_ - \limite{n}{+\infty} b_{n+1} = b_1
\end{equation*}
\subsubsection*{Serie di Mengoli}
La serie $sum_{n=1}^\infty \frac{1}{n(n+1)}$ è detta \emph{Serie di Mengoli}.
La caratteristica di questa serie è che il suo termine generale $a_n$ si può semplificare come:
\begin{equation*}
	a_n = \frac{1}{n(n+1)} = \frac{1}{n} - \frac{1}{n+1}
\end{equation*}
quindi, la sua somma parizale è:
\begin{equation*}
	s_n = 1 - \frac{1}{2} + \frac{1}{2} - \frac{1}{3} +  \frac{1}{3} - \frac{1}{4} + ... +  \frac{1}{n-1} - \frac{1}{n} +  \frac{1}{n} - \frac{1}{n+1}
\end{equation*}
questa somma parziale è "speciale" perchè \emph{ogni termine tranne il primo e l'ultimo si annulla}. per cui $s_n = 1 - \frac{1}{n+1}$.
\\Da qui è possibile calcolarne il $\limite{n}{+\infty} s_n = 1$, quindi la serie \emph{converge a 1}.
Questo è l'esempio più semplice di \textbf{serie telescopica}
\section{Criteri di Convergenza}
Se ci troviamo una serie di cui non è banale studiare il carattere, possiamo utilizzare dei metodi per verificarne la convergenza.
Innanzitutto diamo la condizione \textbf{necessaria} per la convergenza:
\definizione{
	Condizione \emph{necessaria (ma non sufficiente)} per la convergenza è che il \emph{termine generale $a_n$ sia infinitesimo}.
	\begin{equation*}
		\limite{n}{+\infty} a_n = 0 \text{ Necessario per la convergenza}
	\end{equation*}
	}
Questa condizione, purchè necessaria, non è sufficiente per verificare la convergenza.
Ci sono quindi dei criteri che forniscono \emph{condizioni sufficienti} per la convergenza di una serie:

\begin{itemize}
	\item Criterio del Rapporto
	\item Criterio della Radice
	\item Criterio del Confronto
	\item Criterio del Confronto asintotico
\end{itemize}
Questi vengono utilizzati per serie a termini generali \emph{non negativi}
\begin{itemize}
	\item Criterio dell'assoluta convergenza
	\item Criterio di Leibniz
\end{itemize}
Questi invece sono usate per serie a termini generali di \emph{segno variabile.}
\subsection{Alcune proprietà utili}
Ecco alcune proprietà delle serie che ci saranno utili a verificare la convergenza.
\paragraph*{1}Siano $\sum_{n=0}^{+\infty}a_n$ e $\sum_{n=0}^{+\infty}b_n$ due serie \emph{convergenti} e sia $k\in\R$.
Allora:
\begin{enumerate}
	\item $\sum_{n=0}^{+\infty}(a_n + b_n) = \sum_{n=0}^{+\infty}a_n + \sum_{n=0}^{+\infty}b_n$
	\item $\sum_{n=0}^{+\infty} k a_n = k \sum_{n=0}^{+\infty}a_n$
\end{enumerate}
\nb{che una costante moltiplicativa non modifica il carattere di una serie}
\paragraph*{2} Una serie $\sum_{n=0}^{+\infty}a_n$ a termini \textbf{definitivamente non negativi} non può essere \emph{indeterminata},
Quindi converge o diverge a $+\infty$
\esempio{
	$\sum_{n=0}^{+\infty}\frac{n^2+3^{-n}}{n^2-5}$
	\begin{itemize}
		\item[-] $a_n \geq 0$ Definitivamente, quindi la serie CONVERGE o DIVERGE a $+\infty$
		\item[-] il termine generale tende a 1 ($\limite{n}{+\infty} \frac{n^2+3^{-n}}{n^2-5} = 1)$ quindi la serie NON CONVERGE (il termine generale non è \emph{infinitesimo}!)
	\end{itemize}
	Per cui questa serie \textbf{diverge a $+\infty$}
	}
\subsection{Criterio del Rapporto}
	Il primo criterio è quello del rapporto, che può essere usato se il termine generale è definitivamente positivo.
	\definizione{
		Sia $a_n > 0$ \emph{definitivamente} e supponiamo che $\limite{n}{+\infty} \frac{a_{n+1}}{a_n} = l$, allora:
		\begin{itemize}
			\item[-] se $l<1$ allora $\sum a_n$ \textbf{Converge}
			\item[-] se $l>1$ allora $\sum a_n$ \textbf{Diverge}
			\item[-] se $l=1$ allora tutto è possibile e il criterio è \textbf{inconclusivo}
		\end{itemize}
	}
	\nb{che siccome $a_n > 0$ allora $l\in [0,+\infty)$ oppure $l = +\infty$}

\esempio{
	Studiare il carattere di $\serie{0}{+\infty} \frac{n^{2015}}{3^n}$:
	\begin{equation*}
		\limite{n}{+\infty} \frac{a_{n+1}}{a_n} = \frac{\frac{(n+1)^{2015}}{3^{n+1}}}{\frac{n^{2015}}{3^n}} =
		\frac{(n+1)^{2015}}{3 \cdot \cancel{3^n}} \cdot \frac{\cancel{3^n}}{n^{2015}} =
		\frac{(n+1)^{2015}}{3 \cdot n^{2015}} = \frac{1}{3} \frac{(n+1)^{2015}}{n^{2015}}
	\end{equation*}
	Siccome $\limite{n}{+\infty} \frac{(n+1)^{2015}}{n^{2015}} = 1$ allora
	\begin{equation*}
		\limite{n}{+\infty} \frac{a_{n+1}}{a_n} = \frac{1}{3}
	\end{equation*}
	$\frac{1}{3} < 1$ quindi la serie \emph{Converge}
}

\subsection{Criterio della Radice}
Questo criterio, molto simile a quello del Rapporto, dice che:
\definizione{
	Sia $a_n \geq 0$ \emph{definitivamente} e supponiamo che $\limite{n}{+\infty} \sqrt[n]{a_n} = l$, allora:
	\begin{itemize}
		\item[-] se $l<1$ allora $\sum a_n$ \textbf{Converge}
		\item[-] se $l>1$ allora $\sum a_n$ \textbf{Diverge}
		\item[-] se $l=1$ allora tutto è possibile e il criterio è \textbf{inconclusivo}
	\end{itemize}
}
\nb{Usando questo criterio capiterà spesso di trovare $\limite{n}{+\infty} \sqrt[n]{n}$ o $\limite{n}{+\infty} (\sqrt[n]{n^\alpha})$
Tali limiti valgono semre 1}
\esempio{
	Studiare il carattere di $\serie{2}{+\infty} \frac{1}{(\log n)^{\frac{n}{2}}}$:
	\begin{equation*}
		\limite{n}{+\infty} \sqrt[n]{\frac{1}{(\log n)^{\frac{n}{2}}}} =
		(\frac{1}{(\log n)^{\frac{n}{2}}})^{\frac{1}{n}} =
		\frac{1}{(\log n)^{\frac{n}{2n}}} =
		\frac{1}{\sqrt{(\log n)}}
	\end{equation*}
	siccome $\limite{n}{+\infty} \log n = +\infty$
	\begin{equation*}
		\limite{n}{+\infty} \frac{1}{\sqrt{(\log n)}} = 0
	\end{equation*}
	$0 < 1$, per cui la serie \emph{Converge}
}

\subsection{Criterio del Confronto}
\definizione{
	Supponiamo che  $0 \leq a_n \leq b_n$ definitivamente.
	Allora valgono le seguenti implicazioni:
	\begin{enumerate}
		\item $\sum b_n$ converge $\implies \sum a_n$ converge
		\item $\sum a_n$ diverge a $+\infty \implies \sum b_n$ diverge a $+\infty$
	\end{enumerate}
	\nb{L'inverso di queste implicazioni generalmente non valgono}
}
\paragraph*{In pratica quindi} Se riusciamo a trovare una serie che sia definitivamente maggiore (o minore) della nostra serie e di cui conosciamo il carattere
possiamo conoscere anche il carattere della nostra serie.
\\Il trucco quindi è scoprire tale serie, per farlo vengono spesso utilizzate le \emph{Serie Armoniche Generalizzate}

\subsubsection{Serie Armoniche Generalizzate}
Riporto le due formule delle serie Armoniche Generalizzate:
\begin{equation*}
	\serie{1}{+\infty} \frac{1}{n^\alpha} \begin{cases}
		\text{CONVERGE} & \text{Se } \alpha > 1    \\
		\text{DIVERGE}  & \text{Se } \alpha \leq 1
	\end{cases}
\end{equation*}
\begin{equation*}
	\serie{1}{+\infty} \frac{1}{n^\alpha (\log n)^\beta} \begin{cases}
		\text{CONVERGE} & \text{Se } \alpha > 1 \vee \alpha = 1 \wedge \beta > 1    \\
		\text{DIVERGE}  & \text{Se } \alpha < 1 \vee \alpha = 1 \wedge \beta \leq 1
	\end{cases}
\end{equation*}

\esempio{
	Studiare il carattere di $\serie{1}{+\infty} (\frac{\cos n}{n})^2$:
	\\$(\cos n)^2$ Varia tra 0 e 1, quindi è $\leq 1$.
		\\$0 \leq (\frac{\cos n}{n})^2 \leq \frac{1}{n^2}$ per ogni $n\geq 1$.
	\\$\serie{1}{+\infty} \frac{1}{n^2}$ converge, quindi anche nostra serie \emph{Converge}

}

\subsection{Criterio del Confronto Asintotico}
\definizione{
	Date 2 successioni $a_n$ e $b_n$ a termini \emph{definitivamente positivi}, se
	\begin{center}
		$a_n \sim b_n$ (ovvero se $\limite{n}{+\inf} \frac{a_n}{b_n} = 1$)
	\end{center}
	Allora le corrispontenti serie $\Sigma a_n$ e $\Sigma b_n$ hanno lo stesso carattere, quindi sono o \emph{entrambe convergenti o entrambe divergenti}
}
\esempio{
	Determinare il carattere della serie $\serie{1}{+\inf} \frac{n+ \cos n}{n^3 - 3n}$.\\
	I termini della serie sono \emph{definitivamente positivi}, quindi possiamo usare il confronto asintotico.
	Al numeratore, essendo $\cos n$ un valore limitato diventa irrilevante in confronto ad $n$.
	Al denominatore, essendo $3n$ molto più piccolo di $n^3$ diventa irrilevante.
	Quindi:
	\begin{equation*}
		\frac{n+\cos n}{n^3 - 3n} \sim \frac{n}{n^3} = \frac{1}{n^2}
	\end{equation*}
	$\Sigma \frac{1}{n}$ converge (serie armonica generalizzata con $\alpha > 1$), quindi anche $\frac{n+\cos n}{n^3 - 3n}$ \textbf{converge}
}
\paragraph{Ma non è sempre così semplice}In quest'ultimo esempio per trovare una successione asintotica alla nostra è stato sufficiente focalizzarci sui \emph{termini preponderanti} al numeratore e al denominatore.
Spesso però le cose sono più complicate, conviene quindi affidarci ai \emph{limiti notevoli} come nel prossimo esempio:
\esempio{
	Determinare il carattere della serie $\serie{1}{\inf} \frac{e^{\frac{1}{n^2}} -1}{4n}$.\\
	Anche qui la serie è a termini positivi (il numeratore è sempre positivo, visto che $e^\frac{1}{n^2}$ sarà sempre $ > 1$).
	In questo caso però è più difficile trovare una serie asintotica, visto che il numeratore tende a 0 quando $n$ tende a $\inf$ non abbiamo un termine prenponderante sull'altro.
	Il numeratore però ricorda il \emph{limite notevole legato all'esponenziale} che diceva:
	\begin{equation*}
		\limite{\epsilon(x)}{0} \frac{e^{\epsilon(x)}-1}{\epsilon(x)} = 1
	\end{equation*}
	Il che, data la definizione di confronto asintotico, implica che $e^{\epsilon(x)}-1 \sim \epsilon(x)$ per $\epsilon(x) \to 0$.
	Nel nostro caso, $\frac{1}{n^2}$ tende a $0$, quindi $e^{\frac{1}{n^2}} -1 \sim \frac{1}{n^2}$.\\
	\subparagraph*{Di conseguenza} $\frac{e^{\frac{1}{n^2}} -1}{4n} \sim \frac{\frac{1}{n^2}}{4n} = \frac{1}{4n^3}$.\\
	Come prima, $\frac{1}{4n^3}$ è una serie armonica generalizzata con termine $\alpha > 1$, quindi \textbf{Converge}
}
\nb{Esiste un enunciato più generale di questo criterio}

\section{Criterio dell'Assoluta Convergenza}
Questo criterio è il primo che vediamo che vale anche per serie a \emph{termine di segno variabile}
\definizione{
	Una serie $\sum a_n$ si dice \emph{assolutamente convergente} se converge la serie $\sum |a_n|$.
	\\ Se la serie $\sum a_n$ converge assolutamente, allora \emph{Connverge}

	\nb{Il viceversa, in generale, non è vero}
}

\esempio{
	Studiare il carattere della serie $\serie{n=1}{+1\infty} \frac{\sin (n!)}{n^4}$
	\\ in questa serie abbiamo il numeratore che oscilla tra $-1$ e $1$, quindi questa serie non è a termini definitivamente positivi.
	Usando quindi il criterio della convergenza assoluta, e poi il criterio del confronto abbiamo che:
	$$ 0 \leq \frac{|\sin (n!)|}{n^4} \leq \frac{1}{n^4} \implies \text{Converge Assolutamente}$$
	Visto che la serie converge assolutamente, allora \emph{converge}
}
\esempio{
	Studiare il carattere della serie $\serie{n=1}{+1\infty} (-1)^n \frac{n^2 + 3}{n^4+2n}$
	\\Anche questa serie ha il segno variabile (per il $(-1)^n$) quindi usiamo il criterio dell'assoluta convergenza
	$$ |(-1)^n \frac{n^2 + 3}{n^4+2n}| = \frac{n^2 + 3}{n^4+2n} \sim \frac{1}{n^2} $$
	sappiamo che $\frac{1}{n^2}$ è una serie armonica generalizzata che converge, quindi la serie converge assolutamente e di conseguenza converge
}
\section*{Criterio di Leibniz}
Il criterio di Leibniz è il secondo criterio utilizzabile per le serie a \emph{Segno variabile}
\definizione{
	Sia $\{a_n\}$ una successione:
	\begin{itemize}
		\item $a_n \geq 0$ Definitivamente
		\item $a_n \to 0$ per $n \to \infty$
		\item $a_{n+1} \leq a_n$ Definitivamente (monotona decrescente)
	\end{itemize}
	Allora la serie $\serie{0}{+\infty}(-1)^na_n$ è convergente
}
In parole povere, se ${a_n}$ è positiva ($\geq0$), infinitesima e monotona decrescente, allora la serie
$\sum (-1)^na_n$ Converge.

\esempio{
	Studiare il carattere della serie $\serie{0}{\infty} \frac{(-1)^n}{n!}$
	\\$a_n = \frac{1}{n!}$, ed è \emph{positiva}, \emph{infinitesima} e \emph{decrescente}
	\\Di conseguenza, \emph{converge per Leibniz}
}
