\documentclass[12pt, a4paper, openany]{book}
\usepackage[italian]{babel}
\usepackage{listings}
\usepackage{amsmath}
\usepackage{amsfonts}

% comando per le liste numerate come 1 - 1.1 , 2 - 2.2
\renewcommand{\labelenumii}{\arabic{enumi}.\arabic{enumii}}

\newcommand{\defaultrule}{\par\noindent\rule{\textwidth}{0.1pt}}

%comandi matematici
\newcommand{\serie}[2]{\sum_{n=#1}^{#2} x^n}
\newcommand{\limite}[2][n]{lim_{#1 \rightarrow #2}}

%comandi per le domande, risposte e spiegazioni
\newcommand{\domanda}[2]{\paragraph*{#1}#2:}
\newcommand{\risposta}[1]{\begin{center}#1 \end{center}}
\newcommand{\spiegazione}[1]{\defaultrule \par #1 \defaultrule}

\begin{document}
\title{Esami di Analisi Matematica}
\author{Fabio Ferrario}
\date{2022}
\maketitle

\tableofcontents
\chapter{Compitini}
\section{Compitino 2019}
\subsection{Domande Chiuse}

\domanda{1a}{La serie $\serie{1}{+\infty} \frac{ln^4n}{n^{\alpha - 2} + 2n}$ è}
\risposta{Convergente sse $\alpha > 3$}
\spiegazione{
    $\frac{ln^4n}{n^{\alpha - 2} + 2n} = \frac{1}{ln^{-4}n\cdot n^{\alpha-2}+2n} $,  se $\alpha >3$ l'esponente di $n$ è $>1$ rendendo la serie convergente
}

\domanda{1b}{La serie $\serie{1}{+\infty} \frac{ln^3}{n^{\alpha - 1} + 3n}$ è}
\risposta{Convergente sse $\alpha > 2$}
\spiegazione{La serie diventa divergente se $\alpha = 2 \lor \alpha \leq 2$}

\domanda{2a}{La somma della serie $\serie{0}{+\infty} 3\cdot (-\frac{1}{2})^{n+1}$ è uguale a}
\risposta{$-1$}
\spiegazione{
    Una serie di tipo $\sum q^n$ con $q^n < 1$ è una serie geometrica, risolvibile quindi come $\frac{1}{1-q}$
}
\domanda{2b}{La somma della serie $\serie{0}{+\infty} 3\cdot (-\frac{1}{2})^{n+2}$ è uguale a}
\risposta{$\frac{1}{2}$}

\domanda{3a}{Il $\limite{n}{+\infty} \frac{3n^2-n+n ln n +(-1)^n}{2n+4n ln n - n^ - \frac 2/n^2}$ vale}
\risposta{3}
 

\end{document}