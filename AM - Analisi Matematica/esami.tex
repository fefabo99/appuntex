\documentclass[12pt, a4paper, openany]{book}
\usepackage[italian]{babel}
\usepackage{listings}
\usepackage{amsmath}
\usepackage[makeroom]{cancel}
\usepackage{amsfonts}
\usepackage[framemethod=tikz]{mdframed} 

\begin{document}
\definecolor{giallochiaro}{rgb}{1,1,0.95}
\newmdenv[
    backgroundcolor=giallochiaro,
    linewidth=0pt,
    innertopmargin=14pt,
    innerbottommargin=14pt,
]{box_spiegazione}
\newcommand{\spiegazione}[1]{\begin{box_spiegazione} \small{ \emph{Spiegazione: }#1}\end{box_spiegazione}}


%File di variabili
%%%%%%%%%%% Variabili di Stile %%%%%%%%%%%%%%%%

\newcommand{\defaultrule}{\par\noindent\rule{\textwidth}{0.1pt}} %Rule di default
\renewcommand{\labelenumii}{\arabic{enumi}.\arabic{enumii}} % Liste numerate

%%%%%%%%%%% Variabili Matematiche %%%%%%%%%%%%%%%%

\newcommand{\R}{\mathbb{R}} % Spazio dei numeri Reali
\newcommand{\N}{\mathbb{N}} % Spazio dei numeri Naturali
\newcommand{\function}{f: X \rightarrow Y} % Funzione standard f: X -> Y
\newcommand{\limite}[2][n]{lim_{#1 \rightarrow #2}} % Limite
\newcommand{\serie}[2]{\sum_{n=#1}^{#2} x^n} % Serie
\newcommand{\derivata}[2]{$f(x) = #1 \rightarrow f'(x) = #2 $}

%%%%%%%%%%% Variabili Strutturali %%%%%%%%%%%%%%%%

\newcommand{\domanda}[2]{\paragraph*{#1}#2:}
\newcommand{\risposta}[1]{\begin{center} \emph{#1} \end{center}}
\newcommand{\spiegazione}[1]{\defaultrule \par #1 \defaultrule}
\newcommand{\esempio}[1]{\defaultrule \paragraph*{Esempio} #1 \defaultrule}

\title{Esami di Analisi Matematica}
\author{Fabio Ferrario}
\date{2022}
\maketitle

\tableofcontents
\chapter{Compitini}
\section{Compitino 2019}
\subsection{Domande Chiuse}

\domanda{1a}{La serie $\serie{1}{+\infty} \frac{ln^4n}{n^{\alpha - 2} + 2n}$ è}
\risposta{Convergente sse $\alpha > 3$}
\spiegazione{
    $\frac{ln^4n}{n^{\alpha - 2} + 2n} = \frac{1}{ln^{-4}n\cdot n^{\alpha-2}+2n} $,  se $\alpha >3$ l'esponente di $n$ è $>1$ rendendo la serie convergente
}

\domanda{1b}{La serie $\serie{1}{+\infty} \frac{ln^3}{n^{\alpha - 1} + 3n}$ è}
\risposta{Convergente sse $\alpha > 2$}
\spiegazione{La serie diventa divergente se $\alpha = 2 \lor \alpha \leq 2$}

\domanda{2a}{La somma della serie $\serie{0}{+\infty} 3\cdot (-\frac{1}{2})^{n+1}$ è uguale a}
\risposta{$-1$}
\spiegazione{
    Una serie di tipo $\sum q^n$ con $q^n < 1$ è una serie geometrica, risolvibile quindi come $\frac{1}{1-q}$
}
\domanda{2b}{La somma della serie $\serie{0}{+\infty} 3\cdot (-\frac{1}{2})^{n+2}$ è uguale a}
\risposta{$\frac{1}{2}$}

\domanda{3a}{Il $\limite{n}{+\infty} \frac{3n^2-n+n ln n +(-1)^n}{2n+4n ln n - n^ - \frac 2/n^2}$ vale}
\risposta{3}
 

\chapter{Esami}
\section{Febbraio 2021}

\domanda{1}{Sia $a_n = \frac{n \ln (1- \frac{2}{n^3})}{n \sqrt[3]{n} - n^3}$. Allore, per $n \to +\infty$, 
}
\risposta{$a_n \sim \frac{2}{n^5}$}
\spiegazione{
    \begin{equation*}
        a_n = \frac{n \ln(1-\frac{2}{n^3})}{n \sqrt[3]{n}-n^3} \rightarrow \frac{\frac{2}{n^2}}{n^3} \rightarrow \frac{2}{n^2} \cdot \frac{1}{n^3} = \frac{2}{n^5}
    \end{equation*}
    Bisogna trovare una successione asintoticamente equivalente sia per il numeratore, che per il denominatore.
    \\
}

\section{Giugno 2019}

\domanda{1}{La serie $\serie{1}{+\infty}(-1)^n sin(\frac{3}{n^2})$}
\risposta{Converge assolutamente}
\spiegazione{
Siccome abbiamo sia $(-1)^n$, che una successione $\sin(a_n)$, sappiamo che questa serie è a \emph{segno variabile}.
\\Usiamo quindi il criterio dell'assoluta convergenza.
\begin{equation}
    |\serie{1}{+\infty}(-1)^n sin(\frac{3}{n^2})| = |sin(\frac{3}{n^2})| \sim \frac{3}{n^2} 
\end{equation}
    è da capire bene
}
\domanda{2}{La Successione $a_n = \frac{\ln(2+n^3)-5\sqrt[]{n^2-n}+2^{-n^4+5n}}{5n+3\ln n - n \ln n}}$ per $n \to +\infty$ ha limite:
\risposta{$\limite{n}{+\infty} = 0$}
\spiegazione{
    Si trova una successione asintoticamente equivalente sia per numeratore, che per denominatore.
    $$ a_n \sim \frac{2^{-n^4}}{5n}$$
    Si noti che nel numeratore, $2^{-n^4 +5n}$ il $+5n$ è "sovrastato" da $-n^4$, quindi è ignorabile.
    \\Siccome al numeratore abbiamo un valore infinitesimo,($ = (\frac{1}{2})^{n^4}$), la successione tende a $0$
}

\domanda{3}{La funzione $f(x) = \frac{2x^3+4x}{2-x^2} + e^{-\frac{1}{x}}$, per $x\to +\infty$, ha asintoto obliquo di equazione:}
\risposta{$$y=-2x+1$$}
\spiegazione{
    Per trovare un asintoto obliquo bisogna trovare $m$ e $q$ che compongono la retta $y=mx +q$:
    \\$
    m=\limite{x}{+\infty}\frac{f(x)}{x} 
    = \frac{\frac{2x^3+4x}{2-x^2} + e^{-\frac{1}{x}}}{x}
    = (\frac{2x^3+4x}{2-x^2} + e^{-\frac{1}{x}})\cdot \frac{1}{x}
    = \frac{2x^3+4x}{2x-x^3} + \frac{e^{-\frac{1}{x}}}{x}
    = \frac{\cancel{x}(2x^2+4)}{\cancel{x}(2-x^2)} + \frac{e^{-\frac{1}{x}}}{x}
    \sim \frac{2\cancel{x^2}}{-\cancel{x^2}} + \frac{e^{-\frac{1}{x}}}{x}
    = -2 + \frac{e^{-\frac{1}{x}}}{x}
    $. Siccome $\frac{e^{-\frac{1}{x}}}{x}$ tende a $0$, allora $m$ equivale a $-2$
    \\Adesso dobbiamo trovare $q$
    \\$
    q = \limite{x}{+\infty} [f(x) -mx]
    = \frac{2x^3+4x}{2-x^2} + e^{-\frac{1}{x}} +2x
    \sim \frac{2\cancel{x^3}}{-\cancel{x^2}} + e^{-\frac{1}{x}} +2x
    =\cancel{-2x} + e^{-\frac{1}{x}} \cancel{+2x} = e^{-\frac{1}{x}} = 1
    $
    \\Quindi, l'asintoto obliquo esiste e ha equazione $y=-2x+1$
}

\domanda{4}{Sia $f(x) = \sqrt{x^2 +2x +3}$. allora $f'(1)$ vale:}
\risposta{$\frac{2}{\sqrt{6}}$}
\spiegazione{
    Basti ricordarsi che:
    \\La derivata di una radice è: \derivata{\sqrt[\alpha]{x}}{\frac{1}{\alpha \sqrt[\alpha]{x}}}
    \\Questa funzione è una funzione composta ($f(x) = \sqrt{g(x)}$), quindi bisogna derivarla come tale: $f(g(x)) = f'(g(x)) \cdot g'(x)$.
    \\Una volta calcolata la derivata e semplificata fino a un punto "comodo", basta sostituire $x$ con 1. 
}

\domanda{5}{L'insieme delle soluzioni della disequazione $e^x \sqrt[3]{x-1} \geq 1$ è del tipo}
\risposta{$(\alpha, +\infty)$ con $\alpha>1$}
\spiegazione{
Ci si chiede l'intervallo dei valori di $x$ per cui la disequazione è sostanzialmente "corretta", quindi quando il termine sinistro è maggiore di 1.
\\Se si prova un po per esclusione, si vede che per $x=0$ è "falsa" e rimane così anche per valori minori di 0.
\\$x=1$ ci da 0, quindi deve essere per forza un valore maggiore di 1
}

\domanda{6}{la funzione $f(x) = \ln(x^2+2x+3)$ è monotona crescente se e solo se}
\risposta{$x\in (-1,+\infty)$}
\spiegazione{
    Per trovare se una funzione è \emph{monotona crescente} bisogna porre la derivata della funzione $\geq 0$.
    Il risultato è $x\geq -1$, che equivale a $x\in (-1,+\infty)$.
}
\domanda{7}{L'estremo inferiore della successione $\{a_n\}_{n\geq0}$, dove $a_n = 3^{n+(-1)^nn}$ è:}
\risposta{
    1
}
\spiegazione{
    Per vedere l'estremo inferiore di una successione bisognerebbe provare qualche valore di $n$ a partire dal più piccolo (in questo caso 0).
    Se il termine è infinitesimale, l'estremo inferiore si trova con il limite, altrimenti vedi il valore minore che trovi.
}
\domanda{8}{$\int_{\pi/2}^{\pi} x \sin x dx$ vale:}
\risposta{
    $\pi -1$
}
\spiegazione{
    Integro per parti, quindi abbiamo:
    \\$f(x) = x$ e quindi $f'(x) = 1$
    \\$g'(x)= \sin(x)$ e quindi $g(x) = -\cos(x)$
    \\Risolvo con la formula dell'\emph{integrazione per parti}: 
    $$x (-\cos(x)) - \int 1(-\cos(x)) = -x\cos(x)+\sin(x) = \sin(x)-x\cos(x)$$
    Risolvo l'integrale per gli estremi dati:
    $F(\pi) = \pi$, $F(\pi/2)=1$. 
    Quindi l'integrale dato vale $\pi -1$
}
\end{document}