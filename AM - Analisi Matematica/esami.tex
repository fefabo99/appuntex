\documentclass[12pt, a4paper, openany]{book}
\usepackage[italian]{babel}
\usepackage{listings}
\usepackage{amsmath}
\usepackage{amsfonts}

% comando per le liste numerate come 1 - 1.1 , 2 - 2.2
\renewcommand{\labelenumii}{\arabic{enumi}.\arabic{enumii}}


\begin{document}
\title{Esami di Analisi Matematica}
\author{Fabio Ferrario}
\date{2022}
\maketitle

\tableofcontents
\chapter{Esame}
\section{compitino 2019}
\paragraph{1a. La serie $\sum^{+\inf}_{n=1} \frac{ln^4n}{n^{\alpha - 2} + 2n}$ è} CONVERGENTE SE E SOLO SE $\alpha > 3$
\subparagraph{}
$\frac{ln^4n}{n^{\alpha - 2} + 2n} = \frac{1}{ln^{-4}n\cdot n^{\alpha-2}+2n} $,  se $\alpha >3$ l'esponente di $n$ è $>1$ rendendo la serie convergente

\paragraph{1b. La serie $\sum^{+\inf}_{n=1} \frac{ln^3}{n^{\alpha - 1} + 3n}$ è} CONVERGENTE SE E SOLO SE $\alpha > 2$
\subparagraph{} La serie diventa divergente se $\alpha = 2 \lor \alpha \leq 2$

\paragraph{2a. La somma della serie $\sum^{+\inf}_{n=0} 3\cdot (-\frac{1}{2})^{n+1}$ è uguale a} $-1$
\subparagraph{}
Una serie di tipo $\sum q^n$ con $q^n < 1$ è una serie geometrica, risolvibile quindi come $\frac{1}{1-q}$
\paragraph{2b. La somma della serie $\sum^{+\inf}_{n=0} 3\cdot (-\frac{1}{2})^{n+2}$ è uguale a} $\frac{1}{2}$

\end{document}