\documentclass[12pt, a4paper, openany]{book}
\usepackage[italian]{babel}
\usepackage{listings}
\usepackage{amsmath}
\usepackage{amsfonts}
\usepackage[framemethod=tikz]{mdframed} 

\begin{document}
\definecolor{giallochiaro}{rgb}{1,1,0.95}
\newmdenv[
    backgroundcolor=giallochiaro,
    linewidth=0pt,
    innertopmargin=14pt,
    innerbottommargin=14pt,
]{box_spiegazione}
\newcommand{\spiegazione}[1]{\begin{box_spiegazione} \small{ \emph{Spiegazione: }#1}\end{box_spiegazione}}


%File di variabili
%%%%%%%%%%% Variabili di Stile %%%%%%%%%%%%%%%%

\newcommand{\defaultrule}{\par\noindent\rule{\textwidth}{0.1pt}} %Rule di default
\renewcommand{\labelenumii}{\arabic{enumi}.\arabic{enumii}} % Liste numerate

%%%%%%%%%%% Variabili Matematiche %%%%%%%%%%%%%%%%

\newcommand{\R}{\mathbb{R}} % Spazio dei numeri Reali
\newcommand{\N}{\mathbb{N}} % Spazio dei numeri Naturali
\newcommand{\function}{f: X \rightarrow Y} % Funzione standard f: X -> Y
\newcommand{\limite}[2][n]{lim_{#1 \rightarrow #2}} % Limite
\newcommand{\serie}[2]{\sum_{n=#1}^{#2} x^n} % Serie
\newcommand{\derivata}[2]{$f(x) = #1 \rightarrow f'(x) = #2 $}

%%%%%%%%%%% Variabili Strutturali %%%%%%%%%%%%%%%%

\newcommand{\domanda}[2]{\paragraph*{#1}#2:}
\newcommand{\risposta}[1]{\begin{center} \emph{#1} \end{center}}
\newcommand{\spiegazione}[1]{\defaultrule \par #1 \defaultrule}
\newcommand{\esempio}[1]{\defaultrule \paragraph*{Esempio} #1 \defaultrule}

\title{Esami di Analisi Matematica}
\author{Fabio Ferrario}
\date{2022}
\maketitle

\tableofcontents
\chapter{Compitini}
\section{Compitino 2019}
\subsection{Domande Chiuse}

\domanda{1a}{La serie $\serie{1}{+\infty} \frac{ln^4n}{n^{\alpha - 2} + 2n}$ è}
\risposta{Convergente sse $\alpha > 3$}
\spiegazione{
    $\frac{ln^4n}{n^{\alpha - 2} + 2n} = \frac{1}{ln^{-4}n\cdot n^{\alpha-2}+2n} $,  se $\alpha >3$ l'esponente di $n$ è $>1$ rendendo la serie convergente
}

\domanda{1b}{La serie $\serie{1}{+\infty} \frac{ln^3}{n^{\alpha - 1} + 3n}$ è}
\risposta{Convergente sse $\alpha > 2$}
\spiegazione{La serie diventa divergente se $\alpha = 2 \lor \alpha \leq 2$}

\domanda{2a}{La somma della serie $\serie{0}{+\infty} 3\cdot (-\frac{1}{2})^{n+1}$ è uguale a}
\risposta{$-1$}
\spiegazione{
    Una serie di tipo $\sum q^n$ con $q^n < 1$ è una serie geometrica, risolvibile quindi come $\frac{1}{1-q}$
}
\domanda{2b}{La somma della serie $\serie{0}{+\infty} 3\cdot (-\frac{1}{2})^{n+2}$ è uguale a}
\risposta{$\frac{1}{2}$}

\domanda{3a}{Il $\limite{n}{+\infty} \frac{3n^2-n+n ln n +(-1)^n}{2n+4n ln n - n^ - \frac 2/n^2}$ vale}
\risposta{3}
 

\chapter{Esami}
\section{Febbraio 2021}

\domanda{1}{Sia $a_n = \frac{n \ln (1- \frac{2}{n^3})}{n \sqrt[3]{n} - n^3}$. Allore, per $n \to +\infty$, 
}
\risposta{$a_n \sim \frac{2}{n^5}$}
\spiegazione{
    \begin{equation*}
        a_n = \frac{n \ln(1-\frac{2}{n^3})}{n \sqrt[3]{n}-n^3} \rightarrow \frac{\frac{2}{n^2}}{n^3} \rightarrow \frac{2}{n^2} \cdot \frac{1}{n^3} = \frac{2}{n^5}
    \end{equation*}
    Bisogna trovare una successione asintoticamente equivalente sia per il numeratore, che per il denominatore.
    \\
}

\section{Giugno 2019}

\domanda{1}{La serie $\serie{1}{+\infty}(-1)^n sin(\frac{3}{n^2})$}
\risposta{Converge assolutamente}
\spiegazione{
Siccome abbiamo sia $(-1)^n$, che una successione $\sin(a_n)$, sappiamo che questa serie è a \emph{segno variabile}.
\\Usiamo quindi il criterio dell'assoluta convergenza.
\begin{equation}
    |\serie{1}{+\infty}(-1)^n sin(\frac{3}{n^2})| = |sin(\frac{3}{n^2}| \sim \frac{3}{n^2} 
\end{equation}
    è da capire bene
}
\domanda{2}{La Successione $a_n = \frac{\ln(2+n^3)-5\sqrt[]{n^2-n}+2^{-n^4+5n}}{5n+3\ln n - n \ln n}}$ per $n \to +\infty$ ha limite:
\risposta{}

\end{document}