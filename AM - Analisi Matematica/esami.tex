\documentclass[12pt, a4paper, openany]{book}
\usepackage{fstyle}
\begin{document}
\title{Esami di Analisi Matematica}
\author{Fabio Ferrario}
\date{2022}
\maketitle
\tableofcontents
\chapter{Compitini}
\section{Compitino 2019}
\subsection{Domande Chiuse}

\domanda{1a}{La serie $\serie{1}{+\infty} \frac{ln^4n}{n^{\alpha - 2} + 2n}$ è}
\risposta{Convergente sse $\alpha > 3$}
\spiegazione{
    $\frac{ln^4n}{n^{\alpha - 2} + 2n} = \frac{1}{ln^{-4}n\cdot n^{\alpha-2}+2n} $,  se $\alpha >3$ l'esponente di $n$ è $>1$ rendendo la serie convergente
}

\domanda{1b}{La serie $\serie{1}{+\infty} \frac{ln^3}{n^{\alpha - 1} + 3n}$ è}
\risposta{Convergente sse $\alpha > 2$}
\spiegazione{La serie diventa divergente se $\alpha = 2 \lor \alpha \leq 2$}

\domanda{2a}{La somma della serie $\serie{0}{+\infty} 3\cdot (-\frac{1}{2})^{n+1}$ è uguale a}
\risposta{$-1$}
\spiegazione{
    Una serie di tipo $\sum q^n$ con $q^n < 1$ è una serie geometrica, risolvibile quindi come $\frac{1}{1-q}$
}
\domanda{2b}{La somma della serie $\serie{0}{+\infty} 3\cdot (-\frac{1}{2})^{n+2}$ è uguale a}
\risposta{$\frac{1}{2}$}

\domanda{3a}{Il $\limite{n}{+\infty} \frac{3n^2-n+n ln n +(-1)^n}{2n+4n ln n - n^ - \frac 2/n^2}$ vale}
\risposta{3}
 

\chapter{Esami}
\section{Febbraio 2021}

\domanda{1}{Sia $a_n = \frac{n \ln (1- \frac{2}{n^3})}{n \sqrt[3]{n} - n^3}$. Allore, per $n \to +\infty$, 
}
\risposta{$a_n \sim \frac{2}{n^5}$}
\spiegazione{
$$
        a_n = \frac{n \ln(1-\frac{2}{n^3})}{n \sqrt[3]{n}-n^3} \rightarrow \frac{\frac{2}{n^2}}{n^3} \rightarrow \frac{2}{n^2} \cdot \frac{1}{n^3} = \frac{2}{n^5}
$$
    Bisogna trovare una successione asintoticamente equivalente sia per il numeratore, che per il denominatore.
}
\domanda{2}{Sia $f:\R \to \R$ con $f'(0)=0,f''(x)=ln(e+x)$.Allora $f$ ha in $x=0$}
\risposta{Un punto di minimo Relativo}
\irrisolta

\domanda{3}{Sia $f:\R \to \R$ una funzione continua e dispari. Allora, $\int_{-3}{4} f(x) dx$ è uguale a}
\risposta{$\int_{3}^{4} f(x) dx$}
\spiegazione{Una funzione dispari è una funzione che ha il grafico simmetrico rispetto all'origine, quindi ha $f(-a) = -f(a)$.
L'integrale è l'area sottesa della funzione \emph{con il segno}, quindi in una $f(x)$ dispari $\int_{-a}^{a} f(x) dx = 0$ (Spiegazione negli appunti).
Di conseguenza è intuibile che in questa funzione l'integrale da -3 a 3 si annulla, e rimane solo l'area da 3 a 4.}
 

\domanda{4}{La derivata della funzione $f(x) = \sqrt[3]{\frac{x^3}{2} +1}$ è}
\risposta{$\frac{x^2}{2\sqrt[3]{(\frac{x^3}{2}+1)^2}}$}
\spiegazione{
    La derivata di una funzione composta è: $f(g(x))' = f'(g(x))\cdot g'(x)$.
    Per quanto riguarda la radice, è meglio farla a mano trasformandola ($\sqrt[\alpha]{x^\beta} = x^\frac{\beta}{\alpha}$).
    Il resto sono calcoli Algebrici
    }

\domanda{5}{la serie $\serie{1}{+\infty} \frac{1}{n^{(\alpha+1)/2}\ln^2n}$ }
\risposta{Converge sse $\alpha \geq 1$}
\spiegazione{Siccome $n^{...}$ si moltiplica al logaritmo, se fosse infinitesimo annullerebbe il denominatore rendendo la serie divergente a infinito.
Quindi, l'esponente di alpha deve essere positivo, di conseguenza $\alpha \geq 1$}


\domanda{6}{La funzione $f(x) = \begin{cases}a \sin x - b^2  & -2 \leq x \leq 0 \\ 1-e^x &0<x\leq 3\end{cases}$ è derivabile in $x=0$ se e solo se:}
\risposta{$a=-1,b=0$}
\spiegazione{
    Una funzione è derivabile in un punto se è continua e se i limiti destro e sinistro della derivata in quel punto coincidono.
    \\Verificando la continuità, è banale che $b^2=0 \to b=0$.
    \\Verficiando i limiti della derivata invece:
    $ f(x) = a sinx \implies f'(x) = a cos(x)$, $\limite{x}{o^-} a cos(x) = a$.
    \\ $f(x) = 1 -e^x \implies f'(x) = -e^x$, $\limite{x}{o^+} -e^x = -1$.
    \\Di conseguenza, $a = -1$ per la derivabilità.
}


\domanda{7}{Quali tra questi insiemi è un intervallo?}
\risposta{$\{x \in \R: 2|x| \geq x^2\}$}
\spiegazione{}

\domanda{8}{Date le funzioni $f(x) = \ln(x), g(x)= x^3, h(x) = 2-x$, la funzione composta $(h \circ g \circ f)(x)$ è:}
\risposta{$2-\ln(x)$}
\spiegazione{$(h \circ g \circ f)(x) = h(g(f(x)))$}


\section{Giugno 2019}

\domanda{1}{La serie $\serie{1}{+\infty}(-1)^n sin(\frac{3}{n^2})$}
\risposta{Converge assolutamente}
\spiegazione{
Siccome abbiamo sia $(-1)^n$, che una successione $\sin(a_n)$, sappiamo che questa serie è a \emph{segno variabile}.
\\Usiamo quindi il criterio dell'assoluta convergenza.
$$
    |\serie{1}{+\infty}(-1)^n sin(\frac{3}{n^2})| = |sin(\frac{3}{n^2})| \sim \frac{3}{n^2} 
$$
la corrispondenza asintotica vale perchè l'argomento del seno è infinitesimale. 
La successione risultante è una serie armonica di grado $> 1$, quindi converge.
}
\domanda{2}{La Successione $a_n = \frac{\ln(2+n^3)-5\sqrt[]{n^2-n}+2^{-n^4+5n}}{5n+3\ln n - n \ln n}}$ per $n \to +\infty$ ha limite:
\risposta{$\limite{n}{+\infty} = 0$}
\spiegazione{
    Si trova una successione asintoticamente equivalente sia per numeratore, che per denominatore.
    $$ a_n \sim \frac{2^{-n^4}}{5n}$$
    Si noti che nel numeratore, $2^{-n^4 +5n}$ il $+5n$ è "sovrastato" da $-n^4$, quindi è ignorabile.
    \\Siccome al numeratore abbiamo un valore infinitesimo,($ = (\frac{1}{2})^{n^4}$), la successione tende a $0$
}

\domanda{3}{La funzione $f(x) = \frac{2x^3+4x}{2-x^2} + e^{-\frac{1}{x}}$, per $x\to +\infty$, ha asintoto obliquo di equazione:}
\risposta{$$y=-2x+1$$}
\spiegazione{
    Per trovare un asintoto obliquo bisogna trovare $m$ e $q$ che compongono la retta $y=mx +q$:
    \\$
    m=\limite{x}{+\infty}\frac{f(x)}{x} 
    = \frac{\frac{2x^3+4x}{2-x^2} + e^{-\frac{1}{x}}}{x}
    = (\frac{2x^3+4x}{2-x^2} + e^{-\frac{1}{x}})\cdot \frac{1}{x}
    = \frac{2x^3+4x}{2x-x^3} + \frac{e^{-\frac{1}{x}}}{x}
    = \frac{\cancel{x}(2x^2+4)}{\cancel{x}(2-x^2)} + \frac{e^{-\frac{1}{x}}}{x}
    \sim \frac{2\cancel{x^2}}{-\cancel{x^2}} + \frac{e^{-\frac{1}{x}}}{x}
    = -2 + \frac{e^{-\frac{1}{x}}}{x}
    $. Siccome $\frac{e^{-\frac{1}{x}}}{x}$ tende a $0$, allora $m$ equivale a $-2$
    \\Adesso dobbiamo trovare $q$
    \\$
    q = \limite{x}{+\infty} [f(x) -mx]
    = \frac{2x^3+4x}{2-x^2} + e^{-\frac{1}{x}} +2x
    \sim \frac{2\cancel{x^3}}{-\cancel{x^2}} + e^{-\frac{1}{x}} +2x
    =\cancel{-2x} + e^{-\frac{1}{x}} \cancel{+2x} = e^{-\frac{1}{x}} = 1
    $
    \\Quindi, l'asintoto obliquo esiste e ha equazione $y=-2x+1$
}

\domanda{4}{Sia $f(x) = \sqrt{x^2 +2x +3}$. allora $f'(1)$ vale:}
\risposta{$\frac{2}{\sqrt{6}}$}
\spiegazione{
    Basti ricordarsi che:
    \\La derivata di una radice è: \derivata{\sqrt[\alpha]{x}}{\frac{1}{\alpha \sqrt[\alpha]{x}}}
    \\Questa funzione è una funzione composta ($f(x) = \sqrt{g(x)}$), quindi bisogna derivarla come tale: $f(g(x)) = f'(g(x)) \cdot g'(x)$.
    \\Una volta calcolata la derivata e semplificata fino a un punto "comodo", basta sostituire $x$ con 1. 
}

\domanda{5}{L'insieme delle soluzioni della disequazione $e^x \sqrt[3]{x-1} \geq 1$ è del tipo}
\risposta{$(\alpha, +\infty)$ con $\alpha>1$}
\spiegazione{
Ci si chiede l'intervallo dei valori di $x$ per cui la disequazione è sostanzialmente "corretta", quindi quando il termine sinistro è maggiore di 1.
\\Se si prova un po per esclusione, si vede che per $x=0$ è "falsa" e rimane così anche per valori minori di 0.
\\$x=1$ ci da 0, quindi deve essere per forza un valore maggiore di 1
}

\domanda{6}{la funzione $f(x) = \ln(x^2+2x+3)$ è monotona crescente se e solo se}
\risposta{$x\in (-1,+\infty)$}
\spiegazione{
    Per trovare se una funzione è \emph{monotona crescente} bisogna porre la derivata della funzione $\geq 0$.
    Il risultato è $x\geq -1$, che equivale a $x\in (-1,+\infty)$.
}
\domanda{7}{L'estremo inferiore della successione $\{a_n\}_{n\geq0}$, dove $a_n = 3^{n+(-1)^nn}$ è:}
\risposta{
    1
}
\spiegazione{
    Per vedere l'estremo inferiore di una successione bisognerebbe provare qualche valore di $n$ a partire dal più piccolo (in questo caso 0).
    Se il termine è infinitesimale, l'estremo inferiore si trova con il limite, altrimenti vedi il valore minore che trovi.
}
\domanda{8}{$\int_{\pi/2}^{\pi} x \sin x dx$ vale:}
\risposta{
    $\pi -1$
}
\spiegazione{
    Integro per parti, quindi abbiamo:
    \\$f(x) = x$ e quindi $f'(x) = 1$
    \\$g'(x)= \sin(x)$ e quindi $g(x) = -\cos(x)$
    \\Risolvo con la formula dell'\emph{integrazione per parti}: 
    $$x (-\cos(x)) - \int 1(-\cos(x)) = -x\cos(x)+\sin(x) = \sin(x)-x\cos(x)$$
    Risolvo l'integrale per gli estremi dati:
    $F(\pi) = \pi$, $F(\pi/2)=1$. 
    Quindi l'integrale dato vale $\pi -1$
}

\section{Giugno 2020}
\domanda{1}{La funzione $f:\R \to \R$ definita da $f(x)=\begin{cases}e^x & x>0 \\ x+2 & x\leq 0\end{cases}$ è}
\risposta{Suriettiva ma non Iniettiva}
\spiegazione{
    La funzione è \emph{suriettiva} perchè se riempie tutto $\R$. per controllarlo basta vedere i grafici dei due pezzi.
    \\Non è iniettiva invece perchè ci sono dei punti del codomino "raggiungibili" con due x diverse. Anche qua per controllarlo si possono usare i grafici oppure verificare provando.
    Per esempio
    sappiamo che con $f(0) = 2$ e $f(0^+)=1$, quindi rendendo la funzione non monotona. 
}

\domanda{2}{Siano $f(x) = e^{\sqrt{x}}$ e $g(x)= \ln(1+x)$. Allora $(g \circ f)$}
\risposta{è definita per ogni $x\geq 0$}
\spiegazione{$g\circ f = g(f(x)) = ln(1+e^{\sqrt{x}})$, quindi bisogna vedere dove è definita quella funzione.
\\Sappiamo che l'argomento di una radice va posto maggiore o uguale a zero.
\\Forse anche il logaritmo (VEDERE)}
\parzialmenterisolta

\domanda{3}{La somma della serie $\serie{1}{+\infty} e^{1-2n}$ vale}
\risposta{$\frac{e}{e^2-1}$}
\spiegazione{Il risultato mi viene $\frac{e^3}{e^2-1}$, quindi bisogna CAPIRE cosa ho sbagliato}
\parzialmenterisolta

\domanda{4}{Sia $f:\R \to \R$ la funzione definita da $f(x) = \int_{1}^{x} t^2 e^{\sqrt[3]{t}} dt$. Allora:}
\risposta{$f$ è Crescente}
\irrisolta

\domanda{5}{$\limite{n}{+\infty} \frac{n^2 \ln^3 n - n^3 \ln^2 n +4 \arctan n}{n^3 \ln^2 n - n^2 \ln^5 n- e^{-n^2}}$ vale:}
\risposta{-1}
\spiegazione{$\frac{n^2 \ln^3 n - n^3 \ln^2 n +4 \arctan n}{n^3 \ln^2 n - n^2 \ln^5 n- e^{-n^2}} \sim \frac{-n^3 \ln^2 n}{n^3 \ln^2 n} = -1$}

\domanda{6}{Il massimo dell'insime $A = {\frac{2+(-1)^n}{2^n+(-1)^{n+1}}}$ è}
\risposta{1}
\spiegazione{per $n=2$ l'insieme vale 1, mentre per tutti gli altri valori di n è minore.}


\domanda{7}{sia $f(x) = \arctan(\sqrt{x})$. Allora $f'(1)=$}
\risposta{Nessuna delle precedenti ($\frac{1}{4}$)}
\spiegazione{Trovo la derivata della funzione:
\\$f'(x) = \frac{1}{1+x} \cdot \frac{1}{2\sqrt{x}} = \frac{1}{2\sqrt{x}(1+x)}$. La derivata vale in $x=1$:
$f'(1) = \frac{1}{4}$}

\domanda{8}{Sia $f$ definita e continua su [a,b]. Allora}
\risposta{Assume il valore $\frac{f(a)+f(b)}{2}$}
\spiegazione{Non ho trovato nessuna definizione che giustifichi questa cosa, ma da Telegram mi dicono:
Se una funzione è continua in un intervallo ed ho $y_1 = f(a)$ e $y_2 = f(b)$ allora la funzione può assumere tutti i valori tra $y_1$ e $y_2$.
$\frac{f(a)+f(b)}{2}$ se lo guardi nell'asse delle y è un valore compreso tra $f(a)$ e $f(b)$.
\\Un'altra risposta corretta potrebbe essere: Assume tutti i valori tra $f(a)$ e $f(b)$.
}

\section{Giugno 2021}
\domanda{O1}{Sia $f:\R \to \R$ definita da $f(x) = x^3 + 4x$. sull'intervallo $I = (-1,1)$ la funzione $f$ è:}
\spiegazione{Studiare CONCAVITA' e CONVESSITA'}
\irrisolta
\domanda{O2}{La somma della serie $\serie{0}{+\infty} (\frac{2}{3})^{2n}$ vale}
\risposta{$\frac{9}{5}$}
\spiegazione{La serie è simile a una geometrica, però c'è quel $2n$ da mandare via, quindi:
$$ (\frac{2}{3})^{2n}= [(\frac{2}{3})^{2}]^n = (\frac{4}{9})^{n}$$
Ora è una normale serie geometrica con argomento $< |1|$, che risolta con la formula $\sum (q)^n = \frac{1}{1-q}$ ci da $\frac{9}{5}$
}
\domanda{O3}{Siano $f(x) = \ln(x^2+1)-1$ e $g(x) = |x+1|$. Allora $(g \circ f)(x) = $}
\risposta{$ln(x^2 + 1)$}
\spiegazione{
Sappiamo che $(g \circ f)(x) = g(f(x))$, quindi lo svoglimento è banale. Segnalo però che $|ln(x^2+1)| = ln(x^2+1)$, perchè il logaritmo ha dominio maggiore di 0, e in questo caso il suo argomento è sempre maggiore di 0, quindi le due funzioni si equivalgono.
}
\domanda{O4}{L'estremo superiore dell'insieme $A={\frac{1}{n+1} +2^{1-2n}, n= 1,2,...}$ è}
\risposta{1}

\domanda{O5}{La funzione $f:\R \to \R$ definita da $f(x) = \begin{cases} e^{-x} & x>0 \\ x^2-1 &x\leq 0 \end{cases}$ è}
\risposta{Nè suriettiva nè iniettiva}
\spiegazione{
    $e^{-x} = \frac{1}{e^x}$, che parte da vicino a uno e tende a 0.
    \\$x^2-1$ è invece una funzione che parte da -1 e va verso l'alto.
    \\Essendo la funzione definita in $\R$, ed essendo il minimo -1, la funzione non è suriettiva.
    Visto che la seconda parte da -1 e va verso l'alto, mentre la prima parte da 1 e va verso 0, si incrociano sicuramente, rendendo la funzione non iniettiva.
}
\domanda{O6}{L'integrale definito $\int_{0}^{2} \frac{x}{x^2+1} dx$ Vale:}
\risposta{$\frac{ln(5)}{2}$}
\spiegazione{Questa è una funzione razionale che può essere semplicemente trasformata in una del tipo $\frac{f'(x)}{f(x)}$ (che può essere risolta ed equivale a $\ln|f(x)|$) moltiplicando e dividendo per due, quindi:
\\$\int  \frac{x}{x^2+1} = \frac{1}{2} \int \frac{2x}{x^2+1} = \frac{1}{2} \ln(x^2+1)$
\\$[\frac{1}{2} \ln(x^2+1)]_{0}^{2} = \frac{ln(5)}{2}$
}

\domanda{O7}{$\limite{n}{+\infty}(\ln n - \sqrt{n}+3e^{-n} +\sin n^2)$ Vale:}
\irrisolta
\domanda{O8}{Siano $f,g : \R \to \R$ derivabili. se $f(2) = -2, f'(2)=4,g'(-2) = -4}$, allora $(g\circ f)'(2)$ Vale:
\irrisolta
\end{document}