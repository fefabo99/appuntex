\documentclass[12pt, a4paper, openany]{book}
\usepackage[italian]{babel}
\usepackage{listings}
\usepackage{amsmath}
\usepackage{amsfonts}

% comando per le liste numerate come 1 - 1.1 , 2 - 2.2
\renewcommand{\labelenumii}{\arabic{enumi}.\arabic{enumii}}


\begin{document}
\title{Analisi Matematica}
\author{Fabio Ferrario}
\date{2022}
\maketitle

\tableofcontents

\chapter{Il Corso}
\section{Programma del corso}
\begin{enumerate}
\item Numeri Reali
\begin{enumerate}
\item Funzioni elementari
\item Generalità sulle funzioni
\item Funzioni reali di una variabile
\end{enumerate}
\item Successioni
\begin{enumerate}
    \item Limiti di successioni reali
    \item Principio di Introduzione
    \item Limiti notevoli
\end{enumerate}
\item Limiti e continuità
\begin{enumerate}
    \item Limiti di Funzioni
    \item Limiti notevoli
    \item Funzioni continue
    \item Proprietà globali delle funzioni continue
\end{enumerate}
\item Calcoli differenziale
\begin{enumerate}
    \item Derivate di una funzione
    \item Proprietà delle funzioni derivabili
    \item Funzioni convesse e concave
    \item Formula di Taylor
    \item Grafici di funzioni
\end{enumerate}
\item Calcolo integrale
\begin{enumerate}
    \item Funzioni integrabili secondo Rienmann
    \item Teorema fondamentale del calcolo e integrali indefiniti
    \item Metodi d'integrazione
\end{enumerate}
\item Serie numeriche
\begin{enumerate}
    \item Serie, convergenza, convergenza assoluta
    \item Serie a termini positivi
    \item Serie a termini di segno variabile
\end{enumerate}
\end{enumerate}

\section{Prerequisiti}
\emph{Algebra elementare}: Calcolo letterale, equazioni e disequazioni di primo e secondo grado
\\\emph{Trigonometria elementare}
\\\emph{Esponenziali e logaritmi}

\chapter{Analisi 0}
\section{introduzione}
Questo capitolo è dedicato ad alcuni argomenti contenuti sia nel precorso di matematica fornito dall'università, che del capitolo omonimo del libro Analisi Uno di Giuseppe De Marco.
\\Qui verrano segnati soltanto gli argomenti che ritengo siano propedeutici al corso di Analisi 1 oppure che ritengo sufficientemente interessanti da essere ricordati.

% \section{Programma del precorso su E-Learning}
% Questo è l'intero programma del precorso:
% \begin{enumerate}
%     \item Preliminari
%     \begin{itemize}
%         \item elemtnti di logica, proposizioni e predicati, connettivi e quantificatori
%         \item Insiemi relazione di appartenenza, di inclusione, relazioni fra Insiemi
%     \end{itemize}
%     \item Numeri
%     \begin{itemize}
%         \item I numeri interi, razionali, Reali
%         \item Relazioni fra insiemi di numeri, piccola combinatorica
%         \item Potenze ed esponenziali, radici
%         \item Fattorizzazione di numeri
%     \end{itemize}
%     \item Polinomi
%     \begin{itemize}
%         \item Prodotti notevoli e scomposizioni
%         \item Divisione tra Polinomi
%         \item Zeri dei Polinomi
%     \end{itemize}
%     \item Equazioni e disequazioni algebriche
%     \begin{itemize}
%         \item Equazioni algebriche di primo e secondo grado
%         \item Disequazioni Razionali intere
%         \item Disequazioni Razionali fratte
%         \item Disequazioni irrazionali
%         \item Disequazioni con valore assoluto
%     \end{itemize}
%     \item Elementi di geometria analitica
%     \begin{itemize}
%         \item Il Piano cartesiano
%         \item La retta
%         \item Le coniche fondamentali (circonferenza, parabola, ellisse, iperbole)
%     \end{itemize}
%     \item Funzioni Reali
%     \begin{itemize}
%         \item Generalità: definizione, dominio, immagine
%         \item Proprietà delle funzioni: crescenti e decrescenti, monotone, massimi e minimi
%         \item Funzioni iniettive, suriettive, biiettive
%         \item Studio del grafico delle principali funzioni elementari: Potenze intere, radici, potenze con esponente razionale e reale, valore assoluto
%     \end{itemize}
%     \item Esponenziali e logaritmi
%     \begin{itemize}
%         \item Esponenziali e proprietà, la funzione esponenziale
%         \item Logaritmi e proprietò, la funzione logaritmo
%         \item Equazioni e disequazioni logatitmiche
%     \end{itemize}
%     \item Trigonometria
%     \begin{itemize}
%         \item Angoli e loro misura
%         \item Le funzioni seno, coseno, tangente, cotangente
%         \item Le formule trigonometriche fondamentali
%         \item Equazioni e disequazioni trigonometriche
%     \end{itemize}
% \end{enumerate}
% Di questi argomenti segnerò gli appunti solo di quelli in cui potrei avere una carenza oppure sono di interesse per la materia Analisi Matematica.

\section{Linguaggio della matematica }
La matematica è fatta di:
\begin{itemize}
    \item Definizioni $\leftarrow$ Circoscrive un campo, in modo da definire se un oggetto vi appartiene o no
    \item Assiomi/Postulati $\leftarrow$  Affermazione che viene accettata come vera senza che essa venga dimostrata
    \item Preposizioni/Teoremi $\leftarrow$  un teorema è struttrato così: data questa premessa (ipotesi) è possibile dedurre delle conseguenze (tesi)
\end{itemize}
\paragraph*{Affermazioni} In matematica se affermo "Se \emph{un} numero è divisibile per 6, allora lo è anche per due", quel "un" in matematica si traduce in \emph{"esiste un numero"} oppure in \emph{"per ogni numero"} 
\begin{itemize}
    \item $\forall \leftarrow$ Per ogni
    \item $\exists \leftarrow$ Esiste
\end{itemize}

\paragraph*{Proposizioni}
Le proposizioni in matematica sono fatte da soggetto + predicato nominale e possono risultare Vere o False.
esempio: 
\\Ogni giorno piove (x = giorno e P(x) = piove) $\leftarrow \forall x P(x)$
\\Non ($\forall x P(x)$) $\leftarrow \exists x : \not P(x)$

\paragraph*{Implicazioni}
A e B due proposizioni. Cosa significa A -> B? (a implica b)
\begin{itemize}
    \item A è sufficiente per B
    \item B è necessario per A
\end{itemize}
Ogni triangolo èquilatero è isoscele
\begin{itemize}
    \item Affinchè un triangolo abbia 2 lati uguali è sufficiente che ne abbia tre uguali
    \item Affinchè un triangolo abbia 3 lati uguali è necessario che ne abbia due uguali tra loro
\end{itemize}
Quindi A è ipotesi, B è la tesi.

\section{Valore assoluto (modulo) di un numero Reale}
\subsection{Definizione} Per ogni $x \in R$ si pone
\begin{equation}
    |x| = \begin{cases}
        x & \text{se $x \geq 0$}\\
        -x & \text{se $x < 0$}
    \end{cases}
\end{equation}
$|x|$ si dice valore assoluto, o modulo, di x\\
\paragraph{ATTENZIONE} dire che "il modulo di un numero è il numero senza il segno" non ha senso in questo contesto, quindi non va usata perchè vale solo se x è "semplice"
\paragraph{In parole povere} il modulo di $x$ è $x$ se x è positivo, il suo opposto se è negativo\\
quindi $|3 - \pi| = \pi - 3$ perchè $3-\pi$ è negativo ($3 - 3.14$) quindi il suo modulo è il suo opposto.
\\Volendo, $|x|$ si può anche interpretare come il massimo tra $x$ e $-x$.
\subsection{Modulo e moltiplicazione}
Per ogni $x, y \in R$ si ha $|xy| = |x||y|$.\\Se $y\neq0$, si ha anche $|x/y|=|x|/|y|$
\subsection{Modulo e addizione}
Il modulo della somma NON coincide con la somma dei moduli. Però vale la seguente importantissima disuguaglianza:
\paragraph{Disuguaglianza triangolare} siano x,y numeri reali. Allora:
\begin{equation}
    |x+y| \leq |x| + |y|
\end{equation} 
(cioè il modulo della somma è minore o uguale della somma dei due moduli)

\section{Intervalli in $\mathbb{R}$}
\paragraph{Definizione.} Diremo intervallo di $\mathbb{R}$ un sottoinsieme $I$ di $\mathbb{R}$ che sia convesso rispetto all'ordine, cioè che soddisfi la seguente condizione: se $a,b \in I$, e $a \leq b$, ogni $x\in R$ tale che $a \leq x \leq b$ appartiene a $I$. 
\\In parole povere: $I$ è intervallo di $\mathbb{R}$ se, contenendo due numeri reali, contiene anche \emph{tutti i numeri reali che stanno fra questi due}.

\paragraph{Intervalli limitati.}
Fissati $a, b \in R$, con $a<b$, si riconosce facilmente che sono intervalli i seguenti sottoinsiemi di $\mathbb{R}$:
\begin{itemize}
    \item[] $[a,b] = \{x \in R : a \leq x \leq b\}$ CHIUSO
    \item[] $[a,b) = \{x \in R : a \leq x < b\}$ SUPERIORMENTE APERTO
    \item[] $(a,b] = \{x \in R : a > x \leq b\}$ INFERIORMENTE APERTO
    \item[] $(a,b) = \{x \in R : a > x < b\}$ APERTO
\end{itemize}
{\tiny(le parentesi tonde a volte vengono sostituite con delle parentesi quadre nel senso opposto)}
\\Tutti questi intervalli sono detti intervalli \emph{limitati}
\paragraph{Intervalli illimitati.} sono invece intervalli illimitati, per ogni $a \in R$, gli insiemi:
\begin{itemize}
    \item[] $\{x \in R: x \geq a\}$ (semiretta \emph{chiusa} inferiormente limitata)
    \item[] $\{x \in R: x > a\}$ (semiretta \emph{aperta} inferiormente limitata)
    \item[] $\{x \in R: x \leq a\}$ (semiretta \emph{chiusa} superiormente limitata)
    \item[] $\{x \in R: x < a\}$ (semiretta \emph{aperta} superiormente limitata)
\end{itemize}

\subsection{Massimi e Minimi dei sottoinsieme di $\mathbb{R}$}
\paragraph{Definizione.} Sia $S$ un sottoinsieme di $\mathbb{R}$. Un elemento $m\in R$ si dice \emph{massimo} di S se appartiene ad S ed è maggiore o uguale di ogni elemento di $S$.
\\Dualmente, $\mu \in \mathbb{R} $ è detto \emph{minimo} di $S$ se appartiene ad $S$ ed è minore o uguale di ogni elemento di $S$
\\Un sottoinsieme può \emph{avere o non avere massimo}, \emph{avere o non avere minimo}, ma se esistono sono UNICI

\subsection{Maggioranti, minoranti. Insiemi superiormente/inferiormente limitati e limitati}
\paragraph{Definizioni.} Sia $S$ sottoinsieme non vuoto di $\mathbb{R}$. si dice che $b\in \mathbb{R}$ è un \emph{maggiorante} per $S$ se risulta $s\leq b$ per ogni $s \in S$. Si dice che $S$ è \emph{Superiormente limitato} (in $\mathbb{R}$) se ammette maggioranti in $\mathbb{R}$.
\\l'insieme dei maggioranti in $\mathbb{R}$ di $S\subseteq \mathbb{R}$ è indicato con $S^*$.
\\dualmente per i minoranti, ovviamente i minoranti sono tutti minori di $S$ e l'insieme dei minoranti si indica con $S_*$.
\\Un insieme non vuoto $S \subseteq \mathbb{R}$ si dice \emph{limitato} se è tanto superiormente limitato quanto inferiormente limitato. 
\end{document}