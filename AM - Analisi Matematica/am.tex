\documentclass[12pt, a4paper, openany]{book}
\usepackage[italian]{babel}
\usepackage{listings}
\usepackage{amsmath}
\usepackage{amsfonts}

% comando per le liste numerate come 1 - 1.1 , 2 - 2.2
\renewcommand{\labelenumii}{\arabic{enumi}.\arabic{enumii}}

% variabile per funzione standard f: X -> Y
\newcommand{\function}{f: X \rightarrow Y}

\begin{document}
\title{Analisi Matematica}
\author{Fabio Ferrario}
\date{2022}
\maketitle

\tableofcontents

\chapter*{Prefazione}
\section{Introduzione}
Questi appunti di Analisi Matematica sono stati fatti con l'obiettivo di riassumere tutti (o quasi) gli argomenti utili per l'esame di Analisi Matematica del corso di Informatica dell'Università degli Studi di Milano Bicocca.
\\Come fonte ho utilizzato:
\begin{itemize}
    \item Appunti di altri studenti
    \item Libro "Analisi Uno, teoria ed esercizi" di Giuseppe De Marco (terza edizione) 
\end{itemize}
\section{Il Corso}
Come già detto, questi appunti sono in funzione del corso di Analisi Matematica di UNIMIB, a.a. 2021/22, insegnato dalla Professoressa Pini.
\subsection{Programma del corso}
\begin{enumerate}
\item Numeri Reali
\begin{enumerate}
\item Funzioni elementari
\item Generalità sulle funzioni
\item Funzioni reali di una variabile
\end{enumerate}
\item Successioni
\begin{enumerate}
    \item Limiti di successioni reali
    \item Principio di Introduzione
    \item Limiti notevoli
\end{enumerate}
\item Limiti e continuità
\begin{enumerate}
    \item Limiti di Funzioni
    \item Limiti notevoli
    \item Funzioni continue
    \item Proprietà globali delle funzioni continue
\end{enumerate}
\item Calcoli differenziale
\begin{enumerate}
    \item Derivate di una funzione
    \item Proprietà delle funzioni derivabili
    \item Funzioni convesse e concave
    \item Formula di Taylor
    \item Grafici di funzioni
\end{enumerate}
\item Calcolo integrale
\begin{enumerate}
    \item Funzioni integrabili secondo Rienmann
    \item Teorema fondamentale del calcolo e integrali indefiniti
    \item Metodi d'integrazione
\end{enumerate}
\item Serie numeriche
\begin{enumerate}
    \item Serie, convergenza, convergenza assoluta
    \item Serie a termini positivi
    \item Serie a termini di segno variabile
\end{enumerate}
\end{enumerate}

\subsection{Prerequisiti}
\emph{Algebra elementare}: Calcolo letterale, equazioni e disequazioni di primo e secondo grado
\\\emph{Trigonometria elementare}
\\\emph{Esponenziali e logaritmi}

\chapter{Analisi 0}
\section{introduzione}
Questo capitolo è dedicato ad alcuni argomenti contenuti sia nel precorso di matematica fornito dall'università, che del capitolo omonimo del libro Analisi Uno di Giuseppe De Marco.
\\Qui verrano segnati soltanto gli argomenti che ritengo siano propedeutici al corso di Analisi 1 oppure che ritengo sufficientemente interessanti da essere ricordati.
\section{Linguaggio della matematica }
La matematica è fatta di:
\begin{itemize}
    \item Definizioni $\leftarrow$ Circoscrive un campo, in modo da definire se un oggetto vi appartiene o no
    \item Assiomi/Postulati $\leftarrow$  Affermazione che viene accettata come vera senza che essa venga dimostrata
    \item Preposizioni/Teoremi $\leftarrow$  un teorema è struttrato così: data questa premessa (ipotesi) è possibile dedurre delle conseguenze (tesi)
\end{itemize}
\paragraph*{Affermazioni} In matematica se affermo "Se \emph{un} numero è divisibile per 6, allora lo è anche per due", quel "un" in matematica si traduce in \emph{"esiste un numero"} oppure in \emph{"per ogni numero"} 
\begin{itemize}
    \item $\forall \leftarrow$ Per ogni
    \item $\exists \leftarrow$ Esiste
\end{itemize}

\paragraph*{Proposizioni}
Le proposizioni in matematica sono fatte da soggetto + predicato nominale e possono risultare Vere o False.
esempio: 
\\Ogni giorno piove (x = giorno e P(x) = piove) $\leftarrow \forall x P(x)$
\\Non ($\forall x P(x)$) $\leftarrow \exists x : \not P(x)$

\paragraph*{Implicazioni}
A e B due proposizioni. Cosa significa A -> B? (a implica b)
\begin{itemize}
    \item A è sufficiente per B
    \item B è necessario per A
\end{itemize}
Ogni triangolo èquilatero è isoscele
\begin{itemize}
    \item Affinchè un triangolo abbia 2 lati uguali è sufficiente che ne abbia tre uguali
    \item Affinchè un triangolo abbia 3 lati uguali è necessario che ne abbia due uguali tra loro
\end{itemize}
Quindi A è ipotesi, B è la tesi.

\section{Valore assoluto (modulo) di un numero Reale}
\subsection{Definizione} Per ogni $x \in R$ si pone
\begin{equation}
    |x| = \begin{cases}
        x & \text{se $x \geq 0$}\\
        -x & \text{se $x < 0$}
    \end{cases}
\end{equation}
$|x|$ si dice valore assoluto, o modulo, di x\\
\paragraph{ATTENZIONE} dire che "il modulo di un numero è il numero senza il segno" non ha senso in questo contesto, quindi non va usata perchè vale solo se x è "semplice"
\paragraph{In parole povere} il modulo di $x$ è $x$ se x è positivo, il suo opposto se è negativo\\
quindi $|3 - \pi| = \pi - 3$ perchè $3-\pi$ è negativo ($3 - 3.14$) quindi il suo modulo è il suo opposto.
\\Volendo, $|x|$ si può anche interpretare come il massimo tra $x$ e $-x$.
\subsection{Modulo e moltiplicazione}
Per ogni $x, y \in R$ si ha $|xy| = |x||y|$.\\Se $y\neq0$, si ha anche $|x/y|=|x|/|y|$
\subsection{Modulo e addizione}
Il modulo della somma NON coincide con la somma dei moduli. Però vale la seguente importantissima disuguaglianza:
\paragraph{Disuguaglianza triangolare} siano x,y numeri reali. Allora:
\begin{equation}
    |x+y| \leq |x| + |y|
\end{equation} 
(cioè il modulo della somma è minore o uguale della somma dei due moduli)

\section{Intervalli in $\mathbb{R}$}
\paragraph{Definizione.} Diremo intervallo di $\mathbb{R}$ un sottoinsieme $I$ di $\mathbb{R}$ che sia convesso rispetto all'ordine, cioè che soddisfi la seguente condizione: se $a,b \in I$, e $a \leq b$, ogni $x\in R$ tale che $a \leq x \leq b$ appartiene a $I$. 
\\In parole povere: $I$ è intervallo di $\mathbb{R}$ se, contenendo due numeri reali, contiene anche \emph{tutti i numeri reali che stanno fra questi due}.

\paragraph{Intervalli limitati.}
Fissati $a, b \in R$, con $a<b$, si riconosce facilmente che sono intervalli i seguenti sottoinsiemi di $\mathbb{R}$:
\begin{itemize}
    \item[] $[a,b] = \{x \in R : a \leq x \leq b\}$ CHIUSO
    \item[] $[a,b) = \{x \in R : a \leq x < b\}$ SUPERIORMENTE APERTO
    \item[] $(a,b] = \{x \in R : a > x \leq b\}$ INFERIORMENTE APERTO
    \item[] $(a,b) = \{x \in R : a > x < b\}$ APERTO
\end{itemize}
{\tiny(le parentesi tonde a volte vengono sostituite con delle parentesi quadre nel senso opposto)}
\\Tutti questi intervalli sono detti intervalli \emph{limitati}
\paragraph{Intervalli illimitati.} sono invece intervalli illimitati, per ogni $a \in R$, gli insiemi:
\begin{itemize}
    \item[] $\{x \in R: x \geq a\}$ (semiretta \emph{chiusa} inferiormente limitata)
    \item[] $\{x \in R: x > a\}$ (semiretta \emph{aperta} inferiormente limitata)
    \item[] $\{x \in R: x \leq a\}$ (semiretta \emph{chiusa} superiormente limitata)
    \item[] $\{x \in R: x < a\}$ (semiretta \emph{aperta} superiormente limitata)
\end{itemize}

\subsection{Massimi e Minimi dei sottoinsieme di $\mathbb{R}$}
\paragraph{Definizione.} Sia $S$ un sottoinsieme di $\mathbb{R}$. Un elemento $m\in R$ si dice \emph{massimo} di S se appartiene ad S ed è maggiore o uguale di ogni elemento di $S$.
\\Dualmente, $\mu \in \mathbb{R} $ è detto \emph{minimo} di $S$ se appartiene ad $S$ ed è minore o uguale di ogni elemento di $S$
\\Un sottoinsieme può \emph{avere o non avere massimo}, \emph{avere o non avere minimo}, ma se esistono sono UNICI

\subsection{Maggioranti, minoranti. Insiemi superiormente/inferiormente limitati e limitati}
\paragraph{Definizioni.} Sia $S$ sottoinsieme non vuoto di $\mathbb{R}$. si dice che $b\in \mathbb{R}$ è un \emph{maggiorante} per $S$ se risulta $s\leq b$ per ogni $s \in S$. Si dice che $S$ è \emph{Superiormente limitato} (in $\mathbb{R}$) se ammette maggioranti in $\mathbb{R}$.
\\l'insieme dei maggioranti in $\mathbb{R}$ di $S\subseteq \mathbb{R}$ è indicato con $S^*$.
\\dualmente per i minoranti, ovviamente i minoranti sono tutti minori di $S$ e l'insieme dei minoranti si indica con $S_*$.
\\Un insieme non vuoto $S \subseteq \mathbb{R}$ si dice \emph{limitato} se è tanto superiormente limitato quanto inferiormente limitato. 
\subsection{Prodotto Cartesiano}
\paragraph{Definizione.}Dati due insiemi $X$ e $Y$, il loro prodotto cartesiano è per definizione l'insieme $X \times Y$ formato da tutte le coppie ordinate $(x,y)$ che hanno la prima componente $x \in X$, la seconda $y \in Y$.\\
Se $X = Y$, il prodotto $X \times X$ si chiama \emph{quadrato cartesiano} di $X$ e si indica anche con $X^2$
\chapter{Funzioni}
\section{Definizione e Proprietà}
\paragraph{Definizione di Funzione.} Siano $X$ e $Y$ insiemi. Si dice che è data una funzione di $X$ in $Y$, se è data una regola che a ogni elemento di $X$ associa \emph{uno e uno solo} elemento di $Y$.
\paragraph{In altre parole} assegnare una funzione $f$ di $X$ in $Y$ significa dare un procedimento che consenta di assegnare a ogni $x \in X$ un ben determinato $y \in Y$. Tale $y$, corrispondente di $x$ tramite la funzione $f$, si indica con $f(x)$, cioè $y = f(x)$ e $y$ si chiama \emph{immagine di $x$ secondo $f$}. per indicare che $f$ è funzione di $X$ in $y$ si scrive $f: X \rightarrow Y$.
\subsection{Dominio e Codominio}
se $f: X \rightarrow Y$ è una funzione, gli insiemi \emph{$X$ e $Y$} sono rispettivamente il \textbf{dominio} e il \textbf{codominio} di $f$.
\paragraph{Assegnare una funzione} significa assegnare:
\begin{itemize}
    \item Un dominio $X$
    \item Un codominio $Y$
    \item Una regola che a ogni $x$ del dominio associ una $y$ del codominio
\end{itemize}
Pertanto: due funzioni $f$ e $g$ sono \textbf{uguali} se e solo se hanno lo \emph{stesso dominio}, lo\emph{ stesso codominio}, e inoltre si ha \emph{$f(x) = g(x)$ per ogni x del dominio}.\\
Non basta dunque dire che la regola sia la stessa: occorre anche che il dominio e il codominio dati siano gli stessi.
\subsection{Insieme immagine}
Se $f: X \rightarrow Y$ è una funzione, ed $S$ è sottoinsieme del dominio $X$ (cioè $S \subseteq X$) si indica con $f(s)$ l'insieme degli elementi $y$ che sono immagini secondo $f$ di qualche elemento di $S$.
In linguaggio matematico è $f(S) = \{ f(x) : x \in S\}$ oppure si può anche descrivere così: $f(S) = \{y \in Y : \exists x \in S \text{ tc } y = f(x)\}$.
L'insieme $f(S)$ è detto \emph{immagine di $S$ tramite $f$}; per $S = X$ (quindi se S corrisponde col dominio), $f(X)$ è detto \emph{immagine di f}. 
Vale sempre $f(X) \subseteq Y$, quindi l'immagine è \emph{sempre} conenuta nel codomino, ma in generale è $f(X) \subset Y$, cioè immagine e codominio sono diversi.

\paragraph{In parole povere} L'insieme immagine di una funzione è l'insieme di tutti i valori della funzione per ogni elemento del dominio.
L'insieme immagine è \emph{sempre} contenuta nel dominio, ma non sempre vi coincide.

\subsection{Funzioni Suriettive}
Quando l\emph{'\textbf{insieme immagine} e il \textbf{codomino} di una funzione coincidono}, essa si dice che è \emph{suriettiva}
\paragraph{Definizione. } Una funzione $\function$ si dice \emph{suriettiva} se è $f(X) = Y$.
Attenzione che su alcuni testi di Analisi il termine codominio è inteso nel senso di immagine.
\subsection{Immagine Inversa}
Dualmente al concetto di immagine c'è quello di antiimmagine.
\paragraph{Definizione. }Sia $\function $ una funzione, sia $F \subseteq Y$.
Si chiama immagine inversa $f^\leftarrow(T)$ (o controimmagine, o antiimmagine) di $T$ mediante $f$ l'insieme degli $x \in X$ la cui imamgine sta in $T$.
in simboli: $f^\leftarrow(T) = \{ x \in X : f(x) \in T\}$.
\paragraph{In parole povere} L'antiimmagine mediante un insieme $T$ è l'insime dei valori del dominio la cui immagine è contenuta in $T$
\subsection{Funzioni Iniettive}
La funzione $\function$ è detta iniettiva se trasforma elementi distinti in elementi distinti, ovvero:
\paragraph{Definizione. } $f: X \rightarrow Y$ si dice iniettiva se per ogni $x_1, x_2 \in X$ e $x_1 \neq x_2$ implicano $f(x_1) \neq f(x_2)$.
\paragraph{}Per vedere che una $\function$ \emph{non} è iniettiva, basta esibire anche una sola coppia $x_1, x_2$ di elementi distinti ($x_1 \neq x_2$) del dominio per cui sia $f(x_1) = f(x_2)$.
\\Per provare invece che è iniettiva, occorre dimostrare che per ogni coppia di elementi distinti $x_1, x_2 \in X, x_1 \neq x_2$ si ha $f(x_1) \neq f(x_2)$
\subsection{Biiezioni}
Una funzione $\function$ è detta biiettiva se è sia iniettiva che suriettiva.
Quindi $\function$ è biiettive se e solo se per ogni $y \in Y$ esiste uno e un solo $x \in X$ tale che sia $y = f(x)$.
Che esista almeno un tale $x$ dice che $f$ è suriettiva, che sia unico dice che $f$ è iniettiva.
Una funzione biiettiva viene detta anche biiezione e corrispondenza biunivoca. 
\subsection{Funzione Inversa}
Se $\function$ è biiettiva, si può definire una funzione inversa di $f$,$f^{-1} : Y \rightarrow X$, nel modo seguente:
dato $y \in Y$, $f^{-1}(y)$ è quell'unico $x \in X$ tale che sia $y = f(x)$.
se $\function$ non è biiettiva, la funzione inversa di f non può essere definita.
\paragraph*{} se $f$ è funzione reale di variabile reale biiettiva, il grafico dell'inversa $f^{-1}$ si ottiene facendo il simmetrico del grafico di $f$ rispetto alla retta di equazione $x=y$.

\section{Funzioni Elementari}


\end{document}