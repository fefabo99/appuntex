\documentclass[12pt, a4paper, openany]{book}
\usepackage[italian]{babel}
\usepackage{listings}
\usepackage{amsmath}

% comando per le liste numerate come 1 - 1.1 , 2 - 2.2
\renewcommand{\labelenumii}{\arabic{enumi}.\arabic{enumii}}


\begin{document}
\title{Analisi Matematica}
\author{Fabio Ferrario}
\date{2022}
\maketitle
\chapter{Il Corso}
\section{Programma del corso}
\begin{enumerate}
\item Numeri Reali
\begin{enumerate}
\item Funzioni elementari
\item Generalità sulle funzioni
\item Funzioni reali di una variabile
\end{enumerate}
\item Successioni
\begin{enumerate}
    \item Limiti di successioni reali
    \item Principio di Introduzione
    \item Limiti notevoli
\end{enumerate}
\item Limiti e continuità
\begin{enumerate}
    \item Limiti di Funzioni
    \item Limiti notevoli
    \item Funzioni continue
    \item Proprietà globali delle funzioni continue
\end{enumerate}
\item Calcoli differenziale
\begin{enumerate}
    \item Derivate di una funzione
    \item Proprietà delle funzioni derivabili
    \item Funzioni convesse e concave
    \item Formula di Taylor
    \item Grafici di funzioni
\end{enumerate}
\item Calcolo integrale
\begin{enumerate}
    \item Funzioni integrabili secondo Rienmann
    \item Teorema fondamentale del calcolo e integrali indefiniti
    \item Metodi d'integrazione
\end{enumerate}
\item Serie numeriche
\begin{enumerate}
    \item Serie, convergenza, convergenza assoluta
    \item Serie a termini positivi
    \item Serie a termini di segno variabile
\end{enumerate}
\end{enumerate}

\section{Prerequisiti}
\emph{Algebra elementare}: Calcolo letterale, equazioni e disequazioni di primo e secondo grado
\\\emph{Trigonometria elementare}
\\\emph{Esponenziali e logaritmi}
\chapter{Esame}
\section{compitino 2019}
\paragraph{1a. La serie $\sum^{+\inf}_{n=1} \frac{ln^4n}{n^{\alpha - 2} + 2n}$ è} CONVERGENTE SE E SOLO SE $\alpha > 3$
\subparagraph{}
$\frac{ln^4n}{n^{\alpha - 2} + 2n} = \frac{1}{ln^{-4}n\cdot n^{\alpha-2}+2n} $,  se $\alpha >3$ l'esponente di $n$ è $>1$ rendendo la serie convergente

\paragraph{1b. La serie $\sum^{+\inf}_{n=1} \frac{ln^3}{n^{\alpha - 1} + 3n}$ è} CONVERGENTE SE E SOLO SE $\alpha > 2$
\subparagraph{} La serie diventa divergente se $\alpha = 2 \lor \alpha \leq 2$

\paragraph{2a. La somma della serie $\sum^{+\inf}_{n=0} 3\cdot (-\frac{1}{2})^{n+1}$ è uguale a} $-1$
\subparagraph{}
Una serie di tipo $\sum q^n$ con $q^n < 1$ è una serie geometrica, risolvibile quindi come $\frac{1}{1-q}$
\paragraph{2b. La somma della serie $\sum^{+\inf}_{n=0} 3\cdot (-\frac{1}{2})^{n+2}$ è uguale a} $\frac{1}{2}$

\end{document}