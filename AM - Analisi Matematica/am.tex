\documentclass[12pt, a4paper, openany]{book}
\usepackage[italian]{babel}
\usepackage{listings}
\usepackage{amsmath}

% comando per le liste numerate come 1 - 1.1 , 2 - 2.2
\renewcommand{\labelenumii}{\arabic{enumi}.\arabic{enumii}}


\begin{document}
\title{Analisi Matematica}
\author{Fabio Ferrario}
\date{2022}
\maketitle
\chapter{Il Corso}
\section{Programma del corso}
\begin{enumerate}
\item Numeri Reali
\begin{enumerate}
\item Funzioni elementari
\item Generalità sulle funzioni
\item Funzioni reali di una variabile
\end{enumerate}
\item Successioni
\begin{enumerate}
    \item Limiti di successioni reali
    \item Principio di Introduzione
    \item Limiti notevoli
\end{enumerate}
\item Limiti e continuità
\begin{enumerate}
    \item Limiti di Funzioni
    \item Limiti notevoli
    \item Funzioni continue
    \item Proprietà globali delle funzioni continue
\end{enumerate}
\item Calcoli differenziale
\begin{enumerate}
    \item Derivate di una funzione
    \item Proprietà delle funzioni derivabili
    \item Funzioni convesse e concave
    \item Formula di Taylor
    \item Grafici di funzioni
\end{enumerate}
\item Calcolo integrale
\begin{enumerate}
    \item Funzioni integrabili secondo Rienmann
    \item Teorema fondamentale del calcolo e integrali indefiniti
    \item Metodi d'integrazione
\end{enumerate}
\item Serie numeriche
\begin{enumerate}
    \item Serie, convergenza, convergenza assoluta
    \item Serie a termini positivi
    \item Serie a termini di segno variabile
\end{enumerate}
\end{enumerate}

\section{Prerequisiti}
\emph{Algebra elementare}: Calcolo letterale, equazioni e disequazioni di primo e secondo grado
\\\emph{Trigonometria elementare}
\\\emph{Esponenziali e logaritmi}

\chapter{Richiami di Matematica}
Aggiungo in questa sezione una parte di richiami di matematica presa dal corso omonimo su e-learning
\section{Programma del precorso}
Questo è l'intero programma del precorso:
\begin{enumerate}
    \item Preliminari
    \begin{itemize}
        \item elemtnti di logica, proposizioni e predicati, connettivi e quantificatori
        \item Insiemi relazione di appartenenza, di inclusione, relazioni fra Insiemi
    \end{itemize}
    \item Numeri
    \begin{itemize}
        \item I numeri interi, razionali, Reali
        \item Relazioni fra insiemi di numeri, piccola combinatorica
        \item Potenze ed esponenziali, radici
        \item Fattorizzazione di numeri
    \end{itemize}
    \item Polinomi
    \begin{itemize}
        \item Prodotti notevoli e scomposizioni
        \item Divisione tra Polinomi
        \item Zeri dei Polinomi
    \end{itemize}
    \item Equazioni e disequazioni algebriche
    \begin{itemize}
        \item Equazioni algebriche di primo e secondo grado
        \item Disequazioni Razionali intere
        \item Disequazioni Razionali fratte
        \item Disequazioni irrazionali
        \item Disequazioni con valore assoluto
    \end{itemize}
    \item Elementi di geometria analitica
    \begin{itemize}
        \item Il Piano cartesiano
        \item La retta
        \item Le coniche fondamentali (circonferenza, parabola, ellisse, iperbole)
    \end{itemize}
    \item Funzioni Reali
    \begin{itemize}
        \item Generalità: definizione, dominio, immagine
        \item Proprietà delle funzioni: crescenti e decrescenti, monotone, massimi e minimi
        \item Funzioni iniettive, suriettive, biiettive
        \item Studio del grafico delle principali funzioni elementari: Potenze intere, radici, potenze con esponente razionale e reale, valore assoluto
    \end{itemize}
    \item Esponenziali e logaritmi
    \begin{itemize}
        \item Esponenziali e proprietà, la funzione esponenziale
        \item Logaritmi e proprietò, la funzione logaritmo
        \item Equazioni e disequazioni logatitmiche
    \end{itemize}
    \item Trigonometria
    \begin{itemize}
        \item Angoli e loro misura
        \item Le funzioni seno, coseno, tangente, cotangente
        \item Le formule trigonometriche fondamentali
        \item Equazioni e disequazioni trigonometriche
    \end{itemize}
\end{enumerate}
Di questi argomenti segnerò gli appunti solo di quelli in cui potrei avere una carenza oppure sono di interesse per la materia Analisi Matematica.
\section{Lezione 1: Linguaggio della matematica}
La matematica è fatta di:
\begin{itemize}
    \item Definizioni $\leftarrow$ Circoscrive un campo, in modo da definire se un oggetto vi appartiene o no
    \item Assiomi/Postulati $\leftarrow$  Affermazione che viene accettata come vera senza che essa venga dimostrata
    \item Preposizioni/Teoremi $\leftarrow$  un teorema è struttrato così: data questa premessa (ipotesi) è possibile dedurre delle conseguenze (tesi)
\end{itemize}
\paragraph*{Affermazioni} In matematica se affermo "Se \emph{un} numero è divisibile per 6, allora lo è anche per due", quel "un" in matematica si traduce in \emph{"esiste un numero"} oppure in \emph{"per ogni numero"} 
\begin{itemize}
    \item $\forall \leftarrow$ Per ogni
    \item $\exists \leftarrow$ Esiste
\end{itemize}

\paragraph*{Proposizioni}
Le proposizioni in matematica sono fatte da soggetto + predicato nominale e possono risultare Vere o False.
esempio: 
\\Ogni giorno piove (x = giorno e P(x) = piove) $\leftarrow \forall x P(x)$
\\Non ($\forall x P(x)$) $\leftarrow \exists x : \not P(x)$

\paragraph*{Implicazioni}
A e B due proposizioni. Cosa significa A -> B? (a implica b)
\begin{itemize}
    \item A è sufficiente per B
    \item B è necessario per A
\end{itemize}
Ogni triangolo èquilatero è isoscele
\begin{itemize}
    \item Affinchè un triangolo abbia 2 lati uguali è sufficiente che ne abbia tre uguali
    \item Affinchè un triangolo abbia 3 lati uguali è necessario che ne abbia due uguali tra loro
\end{itemize}
Quindi A è ipotesi, B è la tesi.
\chapter{Esame}
\section{compitino 2019}
\paragraph{1a. La serie $\sum^{+\inf}_{n=1} \frac{ln^4n}{n^{\alpha - 2} + 2n}$ è} CONVERGENTE SE E SOLO SE $\alpha > 3$
\subparagraph{}
$\frac{ln^4n}{n^{\alpha - 2} + 2n} = \frac{1}{ln^{-4}n\cdot n^{\alpha-2}+2n} $,  se $\alpha >3$ l'esponente di $n$ è $>1$ rendendo la serie convergente

\paragraph{1b. La serie $\sum^{+\inf}_{n=1} \frac{ln^3}{n^{\alpha - 1} + 3n}$ è} CONVERGENTE SE E SOLO SE $\alpha > 2$
\subparagraph{} La serie diventa divergente se $\alpha = 2 \lor \alpha \leq 2$

\paragraph{2a. La somma della serie $\sum^{+\inf}_{n=0} 3\cdot (-\frac{1}{2})^{n+1}$ è uguale a} $-1$
\subparagraph{}
Una serie di tipo $\sum q^n$ con $q^n < 1$ è una serie geometrica, risolvibile quindi come $\frac{1}{1-q}$
\paragraph{2b. La somma della serie $\sum^{+\inf}_{n=0} 3\cdot (-\frac{1}{2})^{n+2}$ è uguale a} $\frac{1}{2}$

\end{document}