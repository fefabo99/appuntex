\documentclass[12pt, a4paper, openany]{book}
\usepackage{../generalStyle}

\graphicspath{ {./images/} }

\begin{document}

\title{Cheatsheet Probabilità e Statistica}

\author{
	Fabio Ferrario\\
	\small{\href{https://t.me/fefabo}{@fefabo}}
}

\date{2023/2024}

\maketitle

\tableofcontents

\chapter{Statistica Descrittiva}
\section{Indici di posizione}
\paragraph*{Media Campionaria}
\formula{Media Campionaria}{$\overline{x} = \frac{\sum \text{Val}_i \cdot F_i}{N}$}
\paragraph*{Mediana Campionaria}
Dati in ordine Crescente
\formula{Mediana Campionaria}
{
    $m = 
    \begin{cases}
         x_{(\frac{N+1}{2})} & \text{Se N dispari}\\
         \frac{x_{(\frac{N}{2})} + x_{(\frac{N+1}{2})}}{2}  & \text{Se N Pari}
    \end{cases}$
}
\paragraph*{Moda}
\formula{Moda}{Valore con $F_i$ maggiore}

\paragraph*{Percentili}
In generale il $k$-esimo percentile $t$ si calcola:
\begin{itemize}
    \item Scrivo $k = 100p \implies p=\frac{k}{100}$.
    \item Ordino i dati in modo Crescente.
    % \item Se:
    % \begin{itemize}
    %     \item $N\cdot p$ NON Intero $\to t= x_i$ con $i=\lceil N\cdot p \rceil $
    %     \item $N\cdot p$ Intero $\to t= \frac{x_{(N\cdot p)} + x_{(N\cdot p +1)}}{2}$
    % \end{itemize}
\end{itemize}
\formula{Percentile}
{$t = 
    \begin{cases}
        x_i \text{ con } i=\lceil N\cdot p \rceil  & N\cdot p\text{ NON Intero}\\
        \frac{x_{(N\cdot p)} + x_{(N\cdot p +1)}}{2}  & N\cdot p \text{Intero}
    \end{cases}$
}
\subparagraph*{Quartili}
I quartili sono: $q_1\to k=25$, $q_2\to k=50$, $q_3 \to k=75$

\end{document}