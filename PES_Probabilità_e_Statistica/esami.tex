\documentclass[12pt, a4paper, openany]{book}
\usepackage{../generalStyle}

\graphicspath{ {./images/} }

\begin{document}

\title{Esami di Probabilità e Statistica}

\author{
	Fabio Ferrario\\
	\small{\href{https://t.me/fefabo}{@fefabo}}
}

\date{2021/2022}

\maketitle

\tableofcontents

\chapter{Esame di Settembre 2023}
\section{Domande chiuse}
\domanda[Quartili]{1}{Il terzo quartile dell'insieme di dati $\{5,-1,4,0,\frac{\sqrt{3}}{2}\}$ è:}
\risposta{$=4$}
\spiegazione{Per calcolare il quartile di un insieme lo devo ordinare (crescente) e poi usare la formula del formulario per il calcolo dei quartili. }

\domanda[Combinazioni]{2}{Ho un'associazione con 50 soci. Devo scegliere 5 membri che compongano il
comitato direttivo. Quante sono le possibili scelte?}
\risposta{$ {50 \choose 5} = \frac{50!}{45!5!}$}
\spiegazione{Questo è il perfetto esempio di \textbf{combinazione}: Una collezione \emph{non ordinata} di $k$ elementi distinti scelti tra $n$ possibili.
Il numero di combinazioni semplici è sul formulario.
}
\domanda[P. Eventi Indipendenti]{3}{Siano A, B eventi indipendenti tali che $P(A) = \frac{1}{2}$ e $P(B) = \frac{1}{2}$.
. Quale delle seguenti affermazioni è corretta?
}
\rispostechiuse{$P(A|B)= \frac{1}{4}$}{$P(A\cup B) = \frac{2}{3}$}{$P(A\cap B) = \frac{1}{2}$}{$P(A \cup B)= \frac{3}{4}$}
\risposta{d: $P(A \cup B)= \frac{3}{4}$}
\spiegazione{Essendo indipendenti, sappiamo che $P(A|B) = P(A) = \frac{1}{2}$ e viceversa, quindi $P(B|A) = P(B) = \frac{1}{2}$.
\\ Dalla \textbf{Regola del Prodotto}, sappiamo che $P(A\cap B) = P(A) \cdot P(B|A) = \frac{1}{4}$.
\\ Dalle proprietà della probabilità sappiamo che $P(A\cup B)= P(A) + P(B) - P(A\cap B) = \frac{3}{4}$, quindi la risposta è la \textbf{d}.
}
\domanda[V.A.]{4}{Sia $X$ una v.a. uniforme continua su ($-z,z)$, dove $z$ è un numero reale. Quanto vale $E[X]$?}
\risposta{$E[X] = 0$}
\spiegazione{ Essendo X una v.a. uniforme continua su ($-z,z)$, dal formulario sappiamo che $E[X] = \frac{a+b}{2} = \frac{-z+z}{2} = 0$.}
\domanda[]{5}{Siano $X_1, X_2, ..., X_{100}$ v.a. i.i.d. Bernoulliane di parametro $\frac{1}{2}$. Quale delle seguenti è la migliore approssimazione di 
\[ P(\frac{X_1 + X_2 + ... + X_{100} - 50}{5} < 1.12)\]
\small(Usare il Teorema Limite Centrale e le tavole delle distribuzioni notevoli)
}
\rispostechiuse{0.5000}{0.5886}{0.8686}{0.9998}
\risposta{$0.8686$}
\spiegazione{ Da capire. dalle tavole,$\phi(1.12) = 0.8686$.}
\domanda[]{6}{In un test statistico per la verifica dell'ipotesi nulla $H_0:\mu \geq 4$ contro l'alternativa
$H_1:\mu<3$, si rifiuta $H_0$ a livello di significativià $5\%$. Allora posso sicuramente concludere che:
}
\rispostechiuse{Rifiuto l'ipotesi $H_0$ a livello di significatività del $10\%$}{Il $p-$value del test è pari a 0.05}{Rifiuto l'ipotesi $H_0$ a livello di significatività del $3\%$}{L'errore di seconda specie del test è pari a $0.05$.}
\risposta{a: Rifiuto l'ipotesi $H_0$ a livello di significatività del $10\%$}
\spiegazione{ Non so}
\domanda[]{7}{Quanto vale l'\emph{ampiezza} di un intervallo bilatero di confidenza al $98\%$ per la media $\mu$ di un campione $x_1,x_2,...,x_n$ estratto da una popolazione normale con varianza campionaria $s_n$ pari a $1$?
\small(Si ricordi che $P(t(n) > t_{n,\beta} = \beta$, dove $t(n)$ è una v.a. t-student a $n$ gradi di libertà.))
}
\risposta{$\frac{2}{\sqrt{n}} t_{n-1,0.01}$}
\spiegazione{
	In questo caso stiamo stimando la media di una popolazione normale con la varianza incognita, perchè quella fornita è $s_n$.
	Se l'IC è al 98\%, allora $\alpha = 0.02$. Dal formulario devo prendere i due estremi dell'intervallo di confidenza e per trovarne l'ampiezza devo sottrarre uno all'altro.
}
\domanda[]{8}{La statistica del test chi-quadrato di indipendenza per due variabili aleatorie $X$ (che assume $r$ valori) e $Y$ (che assume $s$ valori) ha legge:}
\risposta{ Chi-quadrato a (r-1)(s-1) gradi di libertà}
\spiegazione{ Guardando dal formulario il test chi-quadrato di indipendenza, la regione critica avrà una legge: $\chi^2_{(r-1)(s-1),\alpha}$.}

\section{Esercizi}
\domandaaperta{1}{Un'urna contiene 20 palline colorate: 17 rosse e 3 bianche. Vengono estratte a caso tre palline, \emph{con reimmissione}. Determinare la probabilità che:
\begin{enumerate}
	\item La prima pallina estratta sia bianca;
	\item Le prime due palline estratte siano entrambe bianche;
	\item Almeno una delle tre palline estratte sia bianca.
\end{enumerate}
}
\rispostaaperta{
	Definisco prima gli eventi: per $i=1,2,3$, $B_i = \{$ Pallina bianca all'estrazione $i\}$.
	quindi:
	\begin{enumerate}
		\item Visto che ci sono 20 palline in totale, di cui 3 bianche e la probabilità è Uniforme:
		\[ P(B_1) = \frac{3}{20} = 15\%\]
		\item Vogliamo calcolare l'intersezione ("and") di $B_1$ e $B_2$ che, siccome c'è reimmissione, sono indipendenti.
		\\ Quando due eventi sono indipendenti sappiamo che la probabilità dell'intersezione è il prodotto delle loro probabilità. quindi:
		\[ P(B_1\cap B_2) = P(B_1)P(B_2) = (\frac{3}{20})^2 = 2.23\%\]
		\item Siccome vogliamo calcolare un "or" vogliamo calcolare l'unione degli eventi 1,2,3.
		che è uguale a $1-$ la probabilità dell'intersezione dei complementari.
		Per indipendenza, $P(B_1^c \cap B_2^c \cap B_3^c ) = $ Prodotti etc.
	\end{enumerate}
}

\domandaaperta{2}{Una moneta truccata restituisce testa con probabilità $\frac{1}{3}$. Si Lancia un dado equilibrato a 6 facce,
se esce un numero pari si lancia la moneta 1 volta, altrimenti si lancia la moneta per 2 volte consecutive. Sia $X$ la variabile aleatoria che indica il numero di teste osservate. (Si osservi che $X$ può assumere i valori $0,1,2$.)
\begin{enumerate}
	\item Se lanciando il dado esce un numero pari, qual'è la probabilità (condizionale) che non si osservi nessuna testa? Se invece lanciando il dado esce un numero dispari, qual'è la probabilità (condizionale) che non si osservi nessuna testa?
	\item Calcolare $P(X=0)$,
	\item Determinare la densità discreta di $X$.
\end{enumerate}

}

\end{document}