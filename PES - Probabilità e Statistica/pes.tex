\documentclass[12pt, a4paper, openany]{book}
\usepackage[italian]{babel}
\usepackage{listings}
\usepackage{graphicx}
\usepackage{fancyvrb}
\usepackage{amssymb}
\usepackage{amsmath}
\graphicspath{ {./images/} }

\begin{document}
\title{PES - Probabilità e Statistica per l'informatica}
\author{Elia Ronchetti}
\date{Marzo 2022}

\maketitle
\tableofcontents
\chapter{Introduzione}
Il corso di probabilità e statistica per l'informatica è diviso in 2 parti
\begin{enumerate}
    \item Stastica Descrittiva - Descrivere e riassumere i dati
    \begin{enumerate}
        \item Probabilità - Descrivere matematicamente i fenomeni casuali
    \end{enumerate}
    \item Statistica inferenziale - Trarre conclusioni dai dati
\end{enumerate}

%Modalità d'esame
\chapter{Esame}
L'esame sarà strutturato nella seguente maniera
\paragraph{Parte 1 - Teoria}
8 Domande a risposta multipla - Punteggio 10/30
\paragraph{Parte 2 - Pratica}
4 Esercizi a risposta aperta - Punteggio 20/30
\paragraph{Progetto (facoltativo)}
Progetto R, da consegnare prima dell'esame, può fornire un massimo di 2/30

\chapter{Analisi Descrittiva}
\section{Descrivere i dati}
Per descrivere una raccolta dati in maniera chiara e immediata è utile utilizzare una \textbf{tabella delle frequenze}
all'interno della quale sono contenuti:
\begin{itemize}
    \item Valori
    \item Frequenze Assolute - Numero di volte in cui compare "i" nell'insieme di dati
    \item Frequenze Relative - Frazione di volte in cui compare i nell'insieme di dati
    \item Percentuali - (Frequenza relativa x 100)
\end{itemize}

Il dato che compare con frequenza più alta è detto \textbf{moda}.

I dati possono essere
\begin{itemize}
    \item Qualitativi
    \item Quantitativi 
\end{itemize}
Noi useremo i dati \textbf{quantitativi}

\subsection{Rappresentazione dei dati}
Per rappresentare le frequenze (assolute o relative) risulta efficace e immediato l'utilizzo di un grafico a barre detto istogramma,
esso rappresenta in graficamente la tabella, chiaramamente da esso è possibile risalire alla tabella stessa.
Capita di avere degli insiemi di dati che assumono un valore elevato di valori distini, per questo conviene suddividerli in classi e
determinare la frequenza di ciascuna classe. In questo modo c'è una perdita d'informazioni (sui valori specifici), ma così facendo possiamo
calcolare le frequenze delle classi e avere un'idea migliore della distribuzione dei dati.

\subsection{Dati Bivariati}
Quando per ciascun individuo vengono misurate due variabili ci troviamo un insieme di N dati a coppie detti \textbf{dati bivariati}.
Anche in questo caso è possibile calcolare le frequenze, in questo caso detto \textbf{frequenze congiunte}.

è possibile, inoltre, misurare la correlazione tra le due variabili attraverso per esempio un diagramma di dispersione (detto anche scatterplot).

\paragraph{Correlazione non significa causalità!} Non è detto che l'aumento di una variabile causi la diminuzione dell'altra o viceversa, potrebbe esserci una causa comune. 

\section{Riassumere i dati}
Dopo aver rappresentato i dati vogliamo ora riassumerli mediante quantità numeriche, dette \textbf{Statistiche Campionarie}, al fine di sintetizzare le proprietà
salienti dei dati.

\subsection{Indici di posizione}
Per definire il centro dell'insieme dei dati definiamo la 
\paragraph{\textbf{Media Campionaria}} \scalebox{2}{$\frac{x_1 + x_2 + \dots + x_n}{N}$} %Aggiungere sommatoria

Per misurare il valore in posizione centrale (considerando l'insieme di dati ordinato), utilizziamo la
\paragraph{\textbf{Mediana}}
\begin{itemize}
    \item Se N dispari - {$X_\frac{N+1}{2}$}
    \item Se N pari - $m = \frac{x_\frac{N}{2}+x_(\frac{N}{2}+1)}{2}$
\end{itemize}

La mediana è insesibile alle code

\section{Coefficiente di correlazione lineare}
Posso misurare il grado di correlazione tra una coppia di dati attraverso il coefficiente di correlazione lineare. 

\begin{equation}
    r = \frac{\sum_{k=1}^N (x_i - x)(y_i - y)}{(N -1)S_x S_y}
\end{equation}

Si può mostrare che:
\begin{equation}
    -1<=r<=1
\end{equation}

In generale $r > 0$ indica una correlazione positiva
\\ $r < 0$ indica una correlazione negativa 

\subsection{Correlazioni significative}
$|r| > 0.7$ Correlazione significativa
\\$|r| < 0.3$ Correlazione debole

\section{Percentili e quantili}
Per analizzare la distribuzione dei dati è utile fissare un numero k che rappresenta la posizione all'interno dato all'interno dell'insieme
questo valore percentuale è detto \textbf{k-esimo Percentile Campionario}, valore t per cui
\begin{itemize}
    \item almeno il k\% dei dati è $ <= t$
    \item almeno il $(100 -k)\%$ dei dati è $<= t$
\end{itemize}

I casi più importanti sono per k = 25, 50, 75
\\ Risulta pratico scrivere $k = 100p$ dove $p = \frac{k}{100}\in [0, 1]$, dove i casi importanti sono per:
\begin{itemize} 
    \item $p = \frac{1}{4}: k = 100p$ = 25-esimo percentile = primo quartile $q_1$
    \item $p = \frac{1}{2}: k = 100p$ = 50-esimo percentile = secondo quartile $q_2$ = mediana m
    \item $p = \frac{3}{4}: k = 100p$ = 75-esimo percentile = terzo quartile $q_3$
\end{itemize}

Per calcolare il k-esimo percentile t è necessario:
\begin{enumerate}
    \item Ordinare l'insieme di dati $x_1 <= x_2 <= \dots <= x_n$
    \item Se $N_p$ non è intera $t=x_i$ è il dato la cui posizione i è l'intero successivo a $N_p$
    \item Se $N_p$ è intera $t = \frac{x_(Np) x_(Np+1)}{2}$ è la media aritmetica fra il dato in posizione N e il successivo 
\end{enumerate}

\paragraph{Nota per R} Esistono diverse definizioni di quantile, R èer esempio ne utilizza una diversa di default

\'E possibile utilizzare i \textbf{Boxplot} per la rappresentazione dei quantili

\chapter{Probabilità}
Il calcolo delle probabilità è la teoria matematica che permette di descrivere e studiare \textbf{esperimenti aleatori}
\paragraph{Esperimento aleatorio} $\to$ Fenomeno il cui esito non è prevedibile con certezza a priori
\section{Introduzione}
La descrizione matematica si articola in tre passi
\begin{enumerate}
    \item Spazio campionario (o spazio degli esiti)$\to$ Insieme $\Omega$ che contiene tutti i possibili esiti dell'esperimento \\ es. Tiro un dato a sei facce $\Omega = {1, 2, 3, 4, 5, 6}$
    \item Eventi $\to$ sono i sottinsiemi dello spazio campionario $A \subseteq \Omega$ \\ es. Tiro un dado a sei facce: esce un numero pari $A = {2, 4, 6}$
    \item Probabilità $\to$ Regola che assegna, in modo coerente, a ogni evento $A \subseteq \Omega$ un "grado di fiducia" $P(A)$, tra 0 e 1, che attribuiamo al verificarsi di A
    O funzione $P: P(\Omega) \to [0, 1]$ che soddisfa opportune proprietà
\end{enumerate}
\section{Proprietà di base} 
\begin{enumerate}
    \item $P(\Omega) = 1$
    \item Se A e B sono eventi disgiunti, cioè $A \cap B \neq \emptyset$, allora \\ $P (A \cup B) = P(A) + P(B)$
\end{enumerate}
\paragraph{La coppia $(\Omega, P)$ è detta \textbf{Spazio di Probabilità}}.
\\ Fissiamo uno spazio di probabilità $(\Omega, P)$.
Da queste proprietà si deducono molte altre proprietà
\begin{itemize}
    \item $P(\emptyset) = 0$
    \item \textbf{Regola del complementare} $\to P(A^c) = 1 - P(A)$ \\ Vale per ogni A
    \item \textbf{Regola della addizione di probabilità} $\to P(A \cup B) = P(A) + P(B) - P(A \cap B)$ 
    \\ Vale per ogni A, B (anche $A \cap B \neq \emptyset$)
    \item Monotonia: se $A \subseteq B$ allora $P(A) <= P(B)$
\end{itemize}
\paragraph{Analogia} c'è un analogia tra probabilità e area
\begin{center}
    \includegraphics[width=120mm,scale=0.5]{analogia_prob_area.png}
\end{center}

\section{Calcolo combinatorio}
Consideriamo uno spazio di probabilità uniforme $(\Omega, P)$
\begin{itemize}
    \item $P(A) = \frac{|A|}{|\Omega|}$ = $\frac{Casi favorevoli}{Casi Possibili}$
    per ogni $A \subseteq \Omega$
    \item $P({w}) = \frac{1}{\Omega} = \frac{1}{n}$ per ogni $w \\in \Omega$
\end{itemize}
Questo è il modello appropriato per descrivere esperimenti aleatori i cui esiti
siano tutti equiprobabili. Quando scegliamo casualmente una persona/oggetto in un 
insieme finito senza ulteriori specifiche, si sottintende che la scelta è effettuata
in modo uniforme.
Affinchè la probabilità uniforme sia ben definita, lo spazio campionario $\Omega$
deve essere finito (se così non fosse la probabilità uniforme su $\Omega$ non esiste)
\\ In uno spazio di probabilità uniforme calcolare una probabilità significa contare gli 
elementi di un insieme
\begin{equation}
    P(A) = \frac{|A|}{|\Omega|}
\end{equation}
Dato che contare non è banale per insiemi grandi sono nate tecniche di conteggio,
esse formano il \textbf{Calcolo Combinatorio}
\paragraph{Principio Fondamentale} Consideriamo un esperimento costituito da due parti:
\begin{enumerate}
    \item n esiti possibili
    \item m esiti possibili
\end{enumerate}
L'esperimento totale può avere $n*m$ esiti possibili.
\paragraph{Esempio} Il lancio dei dadi. Se lancio 2 dadi a sei facce ho $\Omega$ 
esiti possibili
\begin{equation}
    |\Omega| = 6*6 = 36
\end{equation}
\subsection{Disposizioni con ripetizione}
Sequenze ordinate di k elementi (anche ripetuti) scelti tra n possibili. Numero totale è:
\begin{equation}
    n*n \dots n = n^k
\end{equation}
\paragraph{Esempio} Estrazione casuale di 3 persone, calcolare la probabilità che
siano tutte nate in primavera.
In questo caso lo spazio campionario sono i compleanni delle tre persone quindi
\begin{equation}
    \Omega = {(x_1, x_2, x_3): x_1, x_2, x_3 \\in Calendario}
\end{equation}
Questa è una disposizione con ripetizione di 3 elementi estratti dal calendario
\begin{equation}
    |\Omega| = 365*365*365 = 365^3
\end{equation}
Probabilità uniforme $P(A) = \frac{|A|}{|\Omega|}$
\\ In questo caso tutti i nati in primavera vengono considerati 
nati A = tutti  nati in primavera = [20 marzo, 21 giugno) e in totale sono 92 giorni.
\\ Si tratta anche qua di una disposizione ripetuta di 3 elementi.
\begin{equation}
    |A| = 92*92*92 = 92^3
\end{equation}
Per calcolare la probabilità è sufficiente dividere A per $\Omega$
\begin{equation}
    P(A) = \frac{|A|}{|\Omega|} = \frac{92^3}{365^3} = 0,016 = 1,6\%
\end{equation}
Se avessi estratto k persone sarebbe stato sufficiente sostituire l'esponente con k.

\subsection{Disposizioni semplici}
Sequenze ordinate di k elementi distinti scelti tra n possibili (con $k <= n$) 
\textbf{senza ripetizione}
\begin{equation}
    n*(n-1)*(n-2)\dots(n-k+1) = \frac{n!}{(n-k)!}
\end{equation}
\'E preferibile utilizzare la prima formula su R dato che il fattoriale scala molto male,
su carta spesso si semplifica, ma su Computer si calcolerebbe tutto il fattoriale e spesso richiede 
molto tempo.
\\ Se $k = n$ si parla di \textbf{Permutazioni} di n oggetti. In numero sono:
\begin{equation}
    n! = n*(n-1)*(n-2)\dots 2 * 1
\end{equation}
\paragraph{Esempio}. Quanti sono i poissibili ordini di arrivo di 3 squadre?
Si tratta di una permutazione di 3 elementi.
\paragraph{Esempio Paradosso dei compleanni}

\subsection{Combinazioni}
In molti casi non siamo interessati all'ordine. 
Per esempio, se dobbiamo scegliere un comitato di 2 persone non ci interessa l'ordine 
dei candidati.
Si parla in questo caso di combinazioni, esse si possono ottenere dalle disposizioni semplici
"dimenticando" l'ordine degli elementi.
\begin{equation}
    {n \choose x} = \frac{n!}{k!(n-k)!}
\end{equation}
Insiemi = collezioni (non ordinate) di k elementi distinti scelti tra n possibili (con $k <= n$).
\paragraph{Esempio}. Mano di carte a Poker, un giocatore riceve 5 carte estratte da un mazzo che
ne contiene 52. Il numero di possibili mani è
\begin{equation}
    {52 \choose 5} = \frac{52!}{5!47!}
\end{equation} 

\section{Probabilità Condizionata}
Consideriamo un esperimento aleatorio, che descriviamo con uno spazio di probabilità $(\Omega, P)$.
Consideriamo un evento $A \subseteq \Omega$, che ha un probabilità $P(A)$.
Supponiamo di ricevere l'informazione che un altro evento B si è verificato.
Come è ragionevole aggiornare la probabilità di A per tenere conto di questa informazione aggiuntiva?
\\La soluzione è data dalla probabilità Condizionata.
\begin{equation}
    P(A|B) = \frac{P(A \cap B)}{P(B)}
\end{equation}
Si legge probabilità di A dato B (o sapendo B). Sto quindi calcolando la probabilità di A.
\begin{center}
    \includegraphics[width=120mm,scale=0.5]{prob_condizionata_area.png}    
\end{center}
Quando si verifica un evento lo spazio di probabilità si riduce (vedere esempio sui dadi)
\subsection{Regola del prodotto}
\begin{equation}
    P(A \cap B) = P(A)*P(B|A)
\end{equation}
\subsection{Formula di Disintegrazione}
\begin{equation}
    P(A) = P(A \cap B) + P(A \cap B^c)
\end{equation}
\subsection{Formula delle probabilità totali}
\begin{equation}
    P(A) = P(A|B)*P(B) + P(A|B^c)*P(B^c)
\end{equation}
Inoltre $P(*|B)$ + una probabilità, in particolare:
\begin{equation}
    P(A^c|B) = 1 - P(A|B)
\end{equation}
\subsection{Formula di Bayes}
\begin{equation}
    P(B|A) = \frac{P(A|B)*P(B)}{P(A)}
\end{equation}
\paragraph{Esempio}. Per vedere la parte pratica andare a vedere l'esempio sui tamponi
per rilveare la presenza di un virus. Super interessante e utile.
\\Il file è \textbf{Appunti Lezione 3 - In fondo al PDF}

\section{Indipendenza di eventi}
Può capitare che, per un evento A, l'informazione che un altro evento B si è verificato
non ne cambi la probabilità.
\begin{equation}
    P(A|B) = P(A)
\end{equation}
che equivale
\begin{equation}
    P(A \cap B) = P(A)*P(B)
\end{equation}
In questo caso gli eventi A e B si dicono \textbf{Indipendenti}
\paragraph{Esempi}
Lancio di due dadi, i risultati sono eventi indipendenti
\\ Urna contenente 5 palline rosse e 3 palline verdi. Pesco in successione due palline, 
senza reimmissione. La probabilità che la prima pallina sia rossa e che la seconda sia rossa
sono \textbf{dipendenti}!

\subsection{Eventi indipendenti $\neq$ Eventi disgiunti!}
\begin{center}
    \includegraphics[width=120mm, scale=0.5]{differenza_disgiunti_indipendenti.png}
\end{center}
Quindi due eventi indipendenti non possono essere disgiunti,
tranne nel caso "banale" in cui uno dei due abbia probabilità
nulla.

\paragraph{Estensioni}
Tre eventi A, B, C si dicono indipendenti se valgono
\begin{center}
    $P(A \cap B \cap C) = P(A)P(B)P(C)$
    \\$P(A \cap B) = P(A)P(B)$
    \\$P(B \cap C) = P(B)P(C)$
    \\$P(A \cap C) = P(A)P(C)$
\end{center}

% Variabili Aleatorie
\chapter{Variabili aleatorie}
Variabile aleatoria (detta anche casuale o stocastica), è una variabile che può assumere valori
diversi in dipendenza da qualche fenomeno aleatorio. Il termine "aleatorio" deriva dal latino \textit{alea} (gioco di dadi), ed esprime il
concetto di rischio calcolato.
\begin{center}
    \textit{"alea iacta est"} - "il dado è tratto"
\end{center}

Consideriamo un esperimento aleatorio, descritto da uno spazio di probabilità $(\Omega, P)$.
Spesso non siamo interessati a tutti i dettagli del'esito dell'esperimento, ma solo a 
una quantità (tipicamente numerica) determinata dall'esito dell'esperimento.
Una tale quantità è detta \textbf{Variabile aleatoria}.
\\ Possiamo considerare la variabile aleatoria come:
\begin{itemize}
    \item Intero: Quantità che dipende dal "caso"
    \item Funzione matematica: Funzione definita sullo spazio campionario: $X: \Omega \to R$
\end{itemize}
Ricodiamo che un evento è:
\begin{itemize}
    \item Affermazione sull'esito dell'esperimento aleatorio
    \item Sottoinsieme dello spazio campionario: $A \subset \Omega$
\end{itemize}

Se X è una variabile aleatoria, e se x è un suo possibile valore, allora:
\begin{itemize}
    \item o X assume il valore x
    \item oppure ${w \in \Omega: X(w) = x }$
\end{itemize}

Ogni variabile aleatoria X determina molti eventi!
\paragraph{Attenzione}: non confondere variabili aleatorie ed eventi
\\ Le variabili aleatorie rappresentano esiti esprimibili numericamente di esperimenti
ancora da effettuare! Dove per esperimento si intende qualsiasi fenomeno o sotiazuone con sviluppi
imprevedibili a priori. Essendo imprevedibile a priori il valore assunto da una variabile
aleatoria, tutto ciò che si può fare è esprimere delle valutazioni di tipo probabilistico sui valori
che essa assumerà. Per questa ragione ad ogni variabile aleatoria X è associata una funzione che esprime
in modo chiaro tali valutazioni. 
\\Se la variabile è discreta si parlerà di \textbf{Densità discreta}.
\\Se la variabile è continua di parlerà di \textbf{Funzione di ripartizione}.

\section{Variabili aleatorie discrete}
Una variabile aleatoria X (reale) si dice \textbf{discreta} se i valori che può
assumere sono un insieme finito:
\begin{equation}
    X(\Omega) = {x_1, x_2, \dots, x_n} \subseteq  R
\end{equation}
Oppure un insieme infinito numerabile
\begin{equation}
    X(\Omega) = {x_1, X_2, \dots } = {x_i} i \in N \subseteq R
\end{equation}
Ad ogni variabile aleatoria discreta X possiamo associare
\begin{center}
    \textbf{Densità discreta} $P_x(x_i) = P(X = x_i)$
\end{center}
Definita anche come distribuzione di probabilità, è una funzione che assegna
ad ogni valore possibile di X la probabilità dell'evento elementare $(X = x)$
\subsection{Proprietà}
\begin{center}
    \includegraphics[width=120mm, scale=0.5]{prop_densità_discreta.png}
\end{center}
Concettualmente una v.a. (variabile aleatoria) X è rappresentata matematicamente da una
una funzione definita sullo spazio campionario $\Omega$ di un esperimento aleatorio.
\begin{center}
    $X:\Omega \to R$
\end{center}
Allo stesso tempo possiamo pensare a X come a un numero che dipende dal caso.
Se siamo interessati a una v.a. discreta X, spesso non è necessario scrivere lo
spazio campionario $\Omega$ ed esprimere X come funzione, 
ci basta conoscere la densità discreta.

\subsection{Valore medio di X}
Sia X una variabile aleatoria discreta (reale) che assume una quantità finita di valori
$x_1, x_2, \dots, x_n$. Si definisce

\begin{equation}
    E[X]:= \sum_{i = 1}^{n} x_i (p_X)^{x_i} = \sum_{i = 1}^{n} x_i P(X= x_i)
\end{equation}
\paragraph{Valore Medio} è la somma dei valori assunti da X "pesati" con le rispettive probabilità
Si può notare da alcuni esempi che il valore medio E[X] non è necessariamente
uno dei valori $x_i$ assunti da X!
A maggior ragione, E[X] non è un valore tipico di X, nè un valore che necessariamente
ci aspettiamo di osservare.
\subsubsection{Interpretazione frequentista di E[X]}
Supponendo di ripetere l'esperimento aleatorio un numero elevato di volte $N >> 1$
e indicando con $X_1, X_2, \dots, X_n$ le variabili aleatorie che rappresentano X nelle
ripetizioni dell'esperimento si ha con grande probabilità che:
\begin{equation}
    \frac{X_1 + X_2 + \dots + X_n}{N} \backsimeq  E[X]
\end{equation}

\subsection{Proprietà del valore medio}
Per ogni variabile aleatoria (reale) X
\begin{center}
    $E[X + c] = E[X] + c$
    \\$E[cX] = c E[X]$
\end{center}
Questo vale per ogni costante $c \in R$
\\ Se X e Y sono due variabili aleatorie che dipendono entrambe dallo
stesso esperimento aleatorio, allora:
\begin{center}
    $E[X+Y] = E[X] + E[Y]$
\end{center}
Si dice che il valore medio è un operatore lineare.
\paragraph{Altre importanti proprietà} Se $X = c$ (costante) allora
$E[X] = E[c] = c$
\\ Un'altra proprietà importante: Se $X >= 0$ allora $E[X] >= 0$
\paragraph{Formula di trasferimento} 

\begin{equation}
    E[f(x)] = \sum_{i = 1}^{n} f(x_i)p_x^{x_i}
= \sum_{i = 1}^{n}f(x_i)P(X = x_i)
\end{equation}
Valida per ogni funzione $f:R \to R$. In particolare
\begin{equation}
    E[X^2] = \sum_{i = 1}^{n} x_i^2 p_X^{x_i} = \sum_{i = 1}^{n} x_i^2P(X=x_i)
\end{equation}

\section{Varianza e Deviazione standard con valore medio}
\paragraph{Varianza} $VAR[X] := E[(X-u)^2] >= 0$ con $u:=E[X]$
\paragraph{Deviazione Standard} $SD[X] := \sqrt{VAR[X]}$
\paragraph{Formula alternativa} $VAR[X] = E[X^2] - E[X]^2$
\\ La deviazione standard ha la stessa "unità di misura" di X e fornisce 
una misura della larghezza (o dispersione) dei valori $x_i$ assunti da X rispetto
al valore medio E[X].
Valore medio E[X] e varianza VAR[X] sono due numeri reali che riassumono
le caratteristiche salienti di una v.a. X (meglio della sua densità discreta).
Sono importanti anche perchè talvolta possono essere calcolati
senza conoscere in dettaglio la densità descrita $p_x$, ma sfruttando le
proprietà di valore medio e varianza.

\subsection{Proprietà della varianza}
Per ogni variabile aleatoria (reale) X
\begin{itemize}
    \item $VAR[X+c] = VAR[X]$
    \item $VAR[cX] = c^2 VAR[X]$
\end{itemize}
Per ogni costante reale $c \in R$
\paragraph{Osservazione} Diverse dalle proprietà del valore media!
\\ Varianza = $(SD)^2 \backsimeq $ (larghezza della distribuzione)$^2$
\begin{center}
    Inoltre $X = c \Leftrightarrow VAR[X] = 0$
\end{center}

\subsection{Dipendenza e indipendenza variabili aleatorie}
Siano X e Y due variabili aleatorie, che dipendono entrambe dallo 
stesso esperimento aleatorio. Quanto valre $VAR[X+Y] = ?$
\textbf{dipende da come solo legate X e Y!}
\paragraph{Definizione indipendenza} Due v.a. discrete X e Y si dicono indipendenti
se gli eventi {X = x} e {Y = y} sono indipendenti, ossia:
\begin{equation*}
    P(X=x, Y=y) = P(X = x)P(Y = y)
\end{equation*}
Per ogni scelta di x e y.
\paragraph{Teorema} Se X e Y sono v.a. indipendenti, allora:
\begin{equation*}
    VAR[X+Y] = VAR[X] + VAR[Y]
\end{equation*}

\section{Variabili aleatorie continue}
Esperimento aleatorio $\to$ spazio di probabilità $(\Omega, P)$
\\ Variabile aleatoria $\to$ funzione $X:\Omega \to R$
\\Fin'ora abbiamo studiato v.a. discrete, che assumono un insieme finito oppure
infinito numerabile di valori $X(\Omega) = {x_1, x_2, \dots}$, dove la distribuzione
X è determinata dalla densità discreta:
\begin{equation*}
    P_x^{x_i} = P(X=x_i)
\end{equation*}
\begin{equation}
    P(X \in A) = \sum_{x_i \in A}P_x^{x_i}  
\end{equation}
Consideriamo ora una classe "complementare" di v.a., dette \textbf{assolutamente continue}
che assumono un insieme infinito più che numerabile di valore, come ad es. un intervallo
di R: $[0, 1]$ $[0, +\infty)$ $(-\infty, +\infty)$
\\ Una v.a. è \textbf{assolutamente continua} se la sua distribuzione è determinata
da una funzione $f_x(x)$, a valori positivi, detta densità della v.a. X, nel modo
seguente:
\begin{equation*}
    P(X \in A) = \int_{A}f_x (x) \,dx 
\end{equation*}
In particolare:
\begin{equation*}
    P(X\in[s,t]) = \int_{s}^{t} f_x (x)\,dx 
\end{equation*}
con $-\infty \leq s \leq t \leq +\infty$
\\Area sotto il grafico di $f_x$ tra i punti s e t
\begin{center}
    \includegraphics[width=100mm, scale=0.5]{va continue integrale.png}
\end{center}
\subsection*{Analogie tra v.a ass. continua e X discreta}
Ci sono analogie formali tra le 2: \begin{itemize}
    \item X ass. continua $P(X \in [s, t]) = \int_{s}^{t} f_x (x) \,dx$ \textbf{Integrale}
    \item X discreta $P(X \in [s, t]) = \sum_{x_i \in [s,t]}p_x (x_i)$ \textbf{Somma}
\end{itemize}
Ma anche importanti differenze!
\\ Se X è assolutamente continua:
\begin{equation*}
    \forall x \in R: P(X=x) = 0
\end{equation*}
\begin{equation*}
    P(X \in [s,t]) = P(X \in (s, t))
\end{equation*}
\paragraph*{Proprietà} La densità di una v.a. assolutamente continua X è una funzione 
$f_x : R \to R$ (integrabile) tale che: \begin{itemize}
    \item $f_x (x) \geq 0$ $\forall x \in R$
    \item $\int_{-\infty}^{+\infty} f_x (x) \,dx = 1$ Area 
    totale sotto il grafico di $f_x$
\end{itemize}
 \paragraph*{Osservazione} $1 = P(X \in (-\infty, +\infty)) 
 \neq \sum_{x \in (+\infty, -\infty)} P (X = x) = 0$ Dato che $X \in (+\infty, -\infty)$
 è più che numberabile!
%Esempio estrazione v.a. uniforme continua

\end{document}