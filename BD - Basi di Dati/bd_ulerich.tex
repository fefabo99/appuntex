\documentclass[12pt, a4paper, openany]{book}
\usepackage[italian]{babel}
\usepackage{listings}
\usepackage{graphicx}
\usepackage{fancyvrb}
\graphicspath{ {./images/} }

\begin{document}

\title{Basi di dati}
\author{Elia Ronchetti}
\date{Marzo 2022}

\maketitle
\tableofcontents

\chapter{Introduzione}
\paragraph{Che cos'è un Data Base}
Una collezione di dati utilizzati per rappresentare le informazioni di interesse di un sistema informativo
\paragraph{Che cos'è un DBMS?}
Un DBMS (Data Base Management System) è un insieme di programmi che permettono di creare, usare e gestire una base di dati, è quindi un software
general purpose che facilita il processo di definizione, costruzione e manipolazione del database per varie applicazioni.

\section{La creazione di un database}
\paragraph{Transazione} $\to$ Insieme di operazioni da considerare indivisibile (atomico)

\section{Entità}
Definita come sostantivo al singolare (es. studente, classe, docente, ecc.)
A livello estensionale un'entità è costituira da un insieme di oggetti che sono chiamati le sue istanze.

\end{document}