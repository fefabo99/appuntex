\section{Es.}
Oggi faremo degli esercizi, che faccio prima a mano e poi vedo se farli a pc o fare solo riferimento.
\\Convenzioni: entità - sostantivo singolare, relazione - verbo singolare.
\\Slide 82: corsivo entità, grassetto relazioni.
\\Descrivere lo schema concettuale della seguente realtà:
\\Degli \textit{impiegati} interessa il codice fiscale, il nome, il cognome, i dipartimenti ai quali \textbf{afferiscono} (con la data di afferenza), ed i \textit{progetti} ai quali \textbf{partecipano}. Dei progetti interessa il nome, il budget, e la città in cui \textbf{hanno luogo} le corrispondenti attività. Alcuni progetti sono parti di altri progetti, e sono detti loro \textbf{sottoprogetti}. Dei \textit{dipartimenti} interessa il nome, il numero di telefono, gli impiegati che li \textbf{dirigono}, e la \textit{città} dove è \textbf{localizzata} la sede. Delle città interessa il nome e la regione.
\\Entità:
\begin{itemize}
    \item impiegato
    \item progetto
    \item dipartimento
\end{itemize}
Relazioni:
\begin{itemize}
    \item afferire
    \item partecipare
    \item avere luogo
    \item fare parte/sottoprogetto (ruolo)
    \item dirigere
    \item essere localizzato
\end{itemize}

\subsubsection{Ma domanda: quando un attributo viene classificato come entità?}
Come faccio, leggendo un testo, a capire quando un elemento che mi sembra un attributo possa invece essere effettivamente scritto come entità?
\begin{enumerate}
    \item è descritto da più di un attributo
    \item partecipa a più di una relazione
    \item è autonoma
\end{enumerate}
C'è una slide.

\subsubsection{Come invece scegliere se entità o relazione?}
C'è una slide.

Torniamo alla teoria.
\section{Cardinalità di una relazione}
Si indica come $(x,\, y)$ dove $x$ è la cardinalità minima dell'associazione e $y$ è la cardinalità massima dell'associazione.
\\Si legge "Entità1 è in relazione con $x$ Entità2 o al più con $y$ Entità2".
\\Simboli standard:
\begin{itemize}
    \item $0$: cardinalità minima, indica partecipazione facoltativa;
    \item $1$: cardinalità minima, indica partecipazione obbligatoria;
    \item $n$: cardinalità massima.
\end{itemize}

\subsection{Classificazione delle relazioni}
Praticamente lo decide la cardinalità massima il nome.
\begin{itemize}
    \item uno a uno: $(1,\, 1) <-> (0,1)$
    \item va beh
    \item .
\end{itemize}

\subsection{Cardinalità degli attributi}
\`E possibile associare delle cardinalità anche agli attributi, per:
- indicare opzionalità
- indicare attributi multivalore

\section{Identificatori}
\subsubsection{I. interni}
identificatori interni: pallini pieni
identificatori semplici: pallini vuoti

\subsubsection{I. esterni}
Non così immediati. Uso un attributo composto, che mi comprende 2+ identificatori: l'identificatore esterno è valido e univoco se e solo se considero l'identificatore in relazione con una sola entità con cui l'entità di cui sto definendo l'attributo è in relazione.
\\Non posso ovviamente aggiungere attributi a caso per risolvere la situazione perché se il testo non lo prevede, non sono autorizzato.
\\Es.: in un'università uno studente è identificato univocamente dalla matricola, ma se vado a considerare per esempio tutte le università d'Italia posso avere la stessa matricola per più studenti con nomi e cognomi diversi. Come faccio? Considerando 1 e 1 sola entità di università, matricola diventa di nuovo univoca. Non posso assolutamente ad esempio aggiungere un attributo "codice id" univoco in tutte le università d'Italia, perché la consegna non lo prevede.