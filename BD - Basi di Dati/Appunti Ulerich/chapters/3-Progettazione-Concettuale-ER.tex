\chapter{Progettazione Concettuale}
\section{Reificazione}
%Scrivere la parte di reificazione che è super importante
% Ripassare tutta questa parte di progettazione concettuale che è abbastanza tosta
\section{Stretegie di progetto}
\begin{itemize}
    \item Top-Down
    \item Bottom-Up
    \item Inside-Out
\end{itemize}
\subsection*{Reminder - Rappresentazione dei concetti nella specifica}
\begin{itemize}
    \item \textbf{Entità} - Se il concetto ha proprietà significative e descrive oggetti 
    con \textbf{esistenza autonoma}
    \item \textbf{Attributo} - Se il concetto è una proprietà locale di un altro e non ha
    proprietà a sua volta
    \item \textbf{Relazione} - Se il concetto correla due o più concetti
    \item \textbf{Is-a o Generalizzazione} - Se il concetto è caso particolare di un altro
\end{itemize}
\section{Strategia utilizzata in pratica - Mista}

\section{Qualità di uno schema concettuale}
\begin{itemize}
    \item Correttezza
    \item Completezza
    \item Leggibilità
    \item Minimalità
\end{itemize}
\section{Consigli per lo svolgimento di esercizi}
\begin{enumerate}
    \item Leggere attentamente il testo
    \item Suddividere il testo in varie sezioni (in base per esempio alle entità)
    \item Iniziare a stendere lo schema ER secondo la suddivisione del testo,
    scrivendo quindi le entità e le relazioni per ogni sezione
    \item Avere già un'idea generale di dove andranno posizionate le entità (giusto
    per non dover riscrivere tutto dopo)
    \item Prosegui alla sezione successiva e ripeti il punto 3
    \item Inserire attributi e segnare gli identificatori
    \item Aggiungere le \textbf{CARDINALIT\'A} (da non dimenticare assolutamente)
\end{enumerate}