\chapter{Progettazione Concettuale}
\section{Reificazione}
%Scrivere la parte di reificazione che è super importante
% Ripassare tutta questa parte di progettazione concettuale che è abbastanza tosta
\section{Stretegie di progetto}
\begin{itemize}
    \item Top-Down
    \item Bottom-Up
    \item Inside-Out
\end{itemize}
\subsection*{Reminder - Rappresentazione dei concetti nella specifica}
\begin{itemize}
    \item \textbf{Entità} - Se il concetto ha proprietà significative e descrive oggetti 
    con \textbf{esistenza autonoma}
    \item \textbf{Attributo} - Se il concetto è una proprietà locale di un altro e non ha
    proprietà a sua volta
    \item \textbf{Relazione} - Se il concetto correla due o più concetti
    \item \textbf{Is-a o Generalizzazione} - Se il concetto è caso particolare di un altro
\end{itemize}
\section{Strategia utilizzata in pratica - Mista}

\section{Qualità di uno schema concettuale}
\begin{itemize}
    \item Correttezza
    \item Completezza
    \item Leggibilità
    \item Minimalità
\end{itemize}
\section{Consigli per lo svolgimento di esercizi}
\begin{enumerate}
    \item Leggere attentamente il testo
    \item Suddividere il testo in varie sezioni (in base per esempio alle entità)
    \item Iniziare a stendere lo schema ER secondo la suddivisione del testo,
    scrivendo quindi le entità e le relazioni per ogni sezione
    \item Avere già un'idea generale di dove andranno posizionate le entità (giusto
    per non dover riscrivere tutto dopo)
    \item Prosegui alla sezione successiva e ripeti il punto 3
    \item Inserire attributi e segnare gli identificatori
    \item Aggiungere le \textbf{CARDINALIT\'A} (da non dimenticare assolutamente)
\end{enumerate}

\chapter{Preparazione Primo Parziale}
Il primo parziale sarà composto da due esercizi
\begin{itemize}
    \item Esercizio 1 - Schema ER
    \item Esercio 2 - Modello Relazionale
\end{itemize}
\section{Schema ER}
Ci verrà dato un testo e noi dovremmo creare uno schema ER, quindi dovremo
identificare Entità, Relazioni, Attributi...
\\ Sarà possibile disegnare sul testo, per esempio per dividerlo in sezioni a seconda
dell'entità o relazione che rappresenta quella parte di testo.
\\ Sarà possibile stendere prima una brutta e poi ricopiare lo schema in bella.
\paragraph*{Consigli} La prima volta leggere il testo per intero, anche se noterete
che non riuscirete a ricordarlo tutto, la complessità di questi testi non permette
alla nostra mente di memorizzare tutto. Durante la seconda lettura cercare di Suddividere
il testo in sezioni iniziando a identificare entità e relazioni, segnando quelle che sono
dubbie (per esempio fare un elenco). La terza lettura inizare a stendere lo schema
utilizzando la metodologia mista vista poco fa nella progettazione concettuale.
\\ Iniziare già a segnare gli identificatori e se sono palesi perchè il testo ce le
indica, segnare le cardinalità, se invece richiedono un minimo di ragionamento è consigliabile
finire prima lo schema e poi segnarle, così si avrà una visione d'insieme più chiara.
\\ Personalmente io parto dalla prima entità e mi espando a macchia d'olio segnandomi
le cose su cui ho dubbi, poi quando ho finito di stendere lo schema, mi concentro sulle
parti su cui ho dubbi e verifico che abbiano senso.
\subsection*{Last but not least}
Ricordarsi di \textbf{segnare le cardinalità} e verificare che tutto ciò che è
citato nel testo sia presente, capita a volte di dimenticarsi dei pezzi e ciò è male.
\subsection*{Rileggi tutto}
Rileggi schema e testo per verificare di non aver modellato cose che non erano citate, oppure
di non aver modellato qualcosa di scritto. 
\begin{itemize}
    \item Verifica che tutte le identità abbiano identificatore
    \item Se c'è qualche entità collegata a una sola relazione, quindi isolata, controllarla
    due volte perchè potrebbero esserci degli errori, di solito le entità sono collegate a più elementi
    \item Visto che ce l'ho per vizio di dimenticarmi questa cosa devo ripeterla all'infinito:
    RICORDATI \textbf{LE CARDINALIT\'A}
    \item Infine controlla che tutto abbia senso ("Se siamo in un osepdale fatemi arrivare i pazienti
    in sala operatoria" cit Schettini)
\end{itemize}
\section{Modello Relazionale}
Ci verrà dato un testo e ci verrà già dato lo schema con le relazioni, la richiesta
sarà quella di:
\begin{itemize}
    \item Trovare le chiavi primarie per ogni relazione
    \item Segnare i vincoli di integrità o testualmente (Relazione.attr con Relazione2.attr2)
    oppure graficamente facendo i collegamenti (io preferisco quest'ultima)
    \item Trovare almeno un vincolo di dominio (es. CF deve essere lungo 16 caratteri)
    \item Trovare almeno un vincolo di tupla (es. Data assunzione succesiva a Data nascita)
    \item Indicare una superchiave non minimale (basta indicare una cosa del tipo,
    \textit{per la relazione LIBRERIA gli attributi idLib, orariApertura, quantitàLibri costituiscono una 
    superchiave} indicando quindi tutti gli attributi)
    \item Indicare una chiave che non sia stata scelta come chiave primaria (es. se ho CF come chiave
    primaria e ho anche telefono come attributo, esso è una chiave che non è stata scelta come primaria)
    \item Non viene sempre chiesto: Indicare due attributi che possano assumero valore
     NULL - qui in genere si dovranno indicare attributi che magari inizialmente sono
      settati a NULL perchè costituiscono misurazoni nel tempo 
      (es numero-giorni-pioggia)
\end{itemize}

\subsection*{Consigli}
\begin{itemize}
    \item Se nel testo non vengono citati attributi del tipo "id" o "codice" e me li ritrovo
    nello schema, con buona probabilità quelli saranno possibili chiavi primarie.
    \item Se sto segnando i vincoli di integrità verso una relazione che ha 2 attributi
    come chiave primaria, dovrò per forza cerchiare 2 attributi nella relazione d'origine!
    \item Se ho scelto 2 attributi come vincolo di integrità, non posso spezzarli e prenderne 1 solo
    per vincolarlo con un'altra relazione
    \item Occhio al testo, a volte si nascondono delle specifiche diverse dalla realtà
    a cui siamo abituati
    \item Indicare esplicitamente che si effettua la risoluzione dei vincoli di integrità
    graficamente 
    \item Occhio agli attributi cattivi (es. portabandiera indica un atleta)
    \item Verifica che le chiavi primarie scelte siano minimali
\end{itemize}
