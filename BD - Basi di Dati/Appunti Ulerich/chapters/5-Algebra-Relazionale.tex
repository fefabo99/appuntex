\chapter{Algebra Relazionale}
Nell ebasi di dati relazionali esistono 2 tipi di linguaggi di interrogazione
\begin{itemize}
    \item Procedurali - Specificano le modalità di generazione del risultato - Come
    \item Dichiarativi - Specificano le proprietà del risultato - Che cosa
\end{itemize}
L'algebra relazionale è procedurale, mentre SQL è parzialmente dichiarativo.\\
L'algebra relazionale è composta da un insieme di operatori che possono essere
utilizzati su relazioni per produrre relazioni. Possono essere composti dando
luogo a espressioni algebriche di complessità arbitraria.\\
\section{Operatori dell'algebra relazionale}
\paragraph*{Unarie}
\begin{itemize}
    \item Ridenominazione
    \item Selezione
    \item Proiezione
\end{itemize}
\paragraph*{Binarie}
\begin{itemize}
    \item Unione, Intersezione, Differenza (Operatori insiemistici)
    \item Join (Join naturale, Prodotto cartesiano, Theta-join)
\end{itemize}
\section{Operatori insiemistici}
Le relazioni sono insiemi e quindi è possibile applicare gli operatori insiemistici,
è fondamentale sapere che è possibile applicare queste operazioni \textbf{solo a relazioni
    definite sugli stessi attributi}.
\subsection*{Unione}
L'unione di due relazioni $r_1$ e $r_2$ è la relazione che contiene le tuple
che appartengono ad $r_1$ oppure ad $r_2$, oppure ad entrambe.\\
L'unione è commutativa e associativa.
\subsection*{Intersezione}
L'intersezione di due relazioni $r_1$ e $r_2$ è la relazione che contiene le tuple
che appartengono sia a $r_1$ che a $r_2$.\\
L'intersezione è commutativa e associativa ed è inoltre esprimibile per mezzo della
differenza:
\begin{equation*}
    r(X) = r_1(X) \cap r_2(X) = r_1(X) - (r_1(X) - r_2(X))
\end{equation*}
\subsection*{Differenza}
La differenze di due relazioni $r_1(X)$ e $r_2(X)$ definite su un insieme di attributi
X è la relazione $r(X) = r_1(X) - r_2(X)$ che contiene le tuple che appartengono a $r_1(X)$, ma
non a $r_2(X)$.\\
La differenza NON è commutativa.\\
\subsection*{Operatori insiemistici e valori nulli}
Gli operatori insiemistici sono definiti anche per relazioni che contengono valori nulli.\\
\section{Operatorri unari}
\subsection{Operatore di ridenominazione}
Per poter applicare operazioni insiemistiche come unione, intersezione, differenza a relazioni
su attributi in parte diversi è necessario ridenominare attributi, in modo da uniformare
i nomi. Questo viene fatto dall'operatore ridenominazione.\\
Si tratta di un operatore monadico (cioè un solo argomento) che modifica lo schema lasciando
inalterata l'istanza dell'operando. Cambia quindi il nome dell'attributo, ma non il valore.
\paragraph*{Sintassi} Si indica con $\rho_{y \leftarrow x}(r)$ o 
$\text{REN}_{y \leftarrow x}(r)$, dove x è
il nome originale dell'attributo, mentre y è quello nuovo. L'operatore è sempre seguito dal
nome della relazione che stiamo considerando.\\
\'E possibile rinominare più attributi, in questo caso è importante l'ordine degli attributi
dato che la sintassi sarà la seguente:
\begin{equation*}
    \rho_{y1, y_2 \leftarrow x_1, x_2}(r)
\end{equation*}
\paragraph*{Esempio}
\begin{tabular}{|c|c|c|}
    \hline
    \textbf{Padre} & \textbf{Figlio} \\
    \hline
    Adamo          & Abele           \\
    \hline
    Adamo          & Caino           \\
    \hline
\end{tabular}
$\text{REN}_{\text{Genitore}\leftarrow\text{Padre}}(\text{Paternità})$
\begin{tabular}{|c|c|c|}
    \hline
    \textbf{Genitore} & \textbf{Figlio} \\
    \hline
    Adamo          & Abele           \\
    \hline
    Adamo          & Caino           \\
    \hline
\end{tabular}\\
Questa operazione è fondamentale per poter effettuare operazioni insiemistiche tra
relazioni con attributi diversi, in questo modo possiamo uniformare i nomi degli attributi.

