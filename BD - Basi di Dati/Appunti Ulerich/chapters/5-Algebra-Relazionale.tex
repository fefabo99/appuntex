\chapter{Algebra Relazionale}
Nell ebasi di dati relazionali esistono 2 tipi di linguaggi di interrogazione
\begin{itemize}
    \item Procedurali - Specificano le modalità di generazione del risultato - Come
    \item Dichiarativi - Specificano le proprietà del risultato - Che cosa
\end{itemize}
L'algebra relazionale è procedurale, mentre SQL è parzialmente dichiarativo.\\
L'algebra relazionale è composta da un insieme di operatori che possono essere
utilizzati su relazioni per produrre relazioni. Possono essere composti dando
luogo a espressioni algebriche di complessità arbitraria.\\
\section{Operatori dell'algebra relazionale}
\paragraph*{Unarie}
\begin{itemize}
    \item Ridenominazione
    \item Selezione
    \item Proiezione
\end{itemize}
\paragraph*{Binarie}
\begin{itemize}
    \item Unione, Intersezione, Differenza (Operatori insiemistici)
    \item Join (Join naturale, Prodotto cartesiano, Theta-join)
\end{itemize}
\section{Operatori insiemistici}
Le relazioni sono insiemi e quindi è possibile applicare gli operatori insiemistici,
è fondamentale sapere che è possibile applicare queste operazioni \textbf{solo a relazioni
definite sugli stessi attributi}.
\subsection*{Unione}
L'unione di due relazioni $r_1$ e $r_2$ è la relazione che contiene le tuple
che appartengono ad $r_1$ oppure ad $r_2$, oppure ad entrambe.\\
L'unione è commutativa e associativa.
\subsection*{Intersezione}
L'intersezione di due relazioni $r_1$ e $r_2$ è la relazione che contiene le tuple
che appartengono sia a $r_1$ che a $r_2$.\\
L'intersezione è commutativa e associativa ed è inoltre esprimibile per mezzo della
differenza:
\begin{equation*}
    r(X) = r_1(X) \cap r_2(X) = r_1(X) - (r_1(X) - r_2(X))
\end{equation*}
\subsection*{Differenza}

