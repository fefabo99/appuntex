\chapter{Algebra Relazionale}
Nell ebasi di dati relazionali esistono 2 tipi di linguaggi di interrogazione
\begin{itemize}
    \item Procedurali - Specificano le modalità di generazione del risultato - Come
    \item Dichiarativi - Specificano le proprietà del risultato - Che cosa
\end{itemize}
L'algebra relazionale è procedurale, mentre SQL è parzialmente dichiarativo.\\
L'algebra relazionale è composta da un insieme di operatori che possono essere
utilizzati su relazioni per produrre relazioni. Possono essere composti dando
luogo a espressioni algebriche di complessità arbitraria.\\
\section{Operatori dell'algebra relazionale}
\begin{itemize}
    \item Unione, Intersezione, Differenza (Operatori insiemistici)
    \item Ridenominazione
    \item Selezione
    \item Proiezione
    \item Join (Join naturale, Prodotto cartesiano, Theta-join)
\end{itemize}
