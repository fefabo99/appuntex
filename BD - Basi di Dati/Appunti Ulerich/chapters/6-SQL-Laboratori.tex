\chapter{SQL - Structured Query Language}
SQL è un linguaggio per la definizione e la manipolazione dei dati
in database relazionali adottato da molti DBMS. \\
Ci sono diverse versioni, la prima versione ufficiale risale al 1986.
Poi sono state rilasciate altre versioni come SQL-89, SQL-2, SQL-3\dots\\
Noi faremo riferimento principalmente a SQL-2. Questa versione è ricca e complessa,
tanto che nessun sistema commerciale lo implementa in maniera completa.\\
Esistono 3 livelli di conformità:
\begin{itemize}
  \item Entry level: molto simile a SQL-89
  \item Intermediate level: versione che soddisfa le esigenze di mercato
  \item Full level: versione completa anche delle funzioni avanzate che non sono realizzate
  in alcun DBMS
\end{itemize}
La maggior parte dei database è conforme solo all'entry level.\\
Alcune famose implementazioni di SQL sono:
\begin{itemize}
    \item ORACLE
    \item DB2 (IBM)
    \item Access (Microsoft)
    \item MSSQL server (Microsoft)
    \item MySQL
    \item Firebird
\end{itemize}
\section{SQL e Algebra Relazionale}
SQL è relazionalmente completo: ogni espressione logica può essere tradotta in SQL. 
Viene adottata la logica dei 3 valori (T, F, U) dell'Algebra relazionale (U = Unknown).\\
Il modello dati di SQL è basato su tabelle anzichè relazioni (possono essere presenti righe duplicate).\\
SQL è computazionalmente completo, ha istruzioni di controllo.
