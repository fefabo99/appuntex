\chapter{Introduzione - Che cos'è un DB e un DBMS}
\paragraph{Che cos'è un Data Base}
Una collezione di dati utilizzati per rappresentare le informazioni di interesse di un sistema informativo
\paragraph{Che cos'è un DBMS?}
Un DBMS (Data Base Management System) è un insieme di programmi che permettono di creare, usare e gestire una base di dati, è quindi un software
general purpose che facilita il processo di definizione, costruzione e manipolazione del database per varie applicazioni.

\section{Perchè creare un Database}
Un soggetto, come per esempio un'azienda, ha molti dati da manipolare
\begin{itemize}
    \item Persone
    \item Denaro
    \item Materiali
    \item Informazioni
\end{itemize}
Per gestire questi dati è necessario un sistema che li organizzi e li gestisca 
in modo efficiente e sicuro.
Questo sistema è detto \textbf{Sistema Informativo}, cioè un componente
di una organizzazione che gestisce le informazioni di interesse, con i seguenti scopi:
\begin{itemize}
    \item Acquisizione/Memorizzazione
    \item Aggiornamento
    \item Interrogazione
    \item Elaborazione
\end{itemize}
Il \textbf{Sistema Informatico} è invece la porzione automatizzata del Sistema
informativo, la parte quindi che gestisce informazioni tramite tecnologia informatica.
\\ Il Sistema Informatico ha i seguenti obiettivi:
\begin{itemize}
    \item Garantisce che i dati siano conservati in modo permanente sui dispositivi di 
    memorizzazione
    \item Permette un rapido Aggiornamento dei dati
    \item Rende i dati accessibili alle interrogazoni degli utenti
    \item Può essere distribuito sul territorio
\end{itemize}
\paragraph*{Gestione delle informazioni}
Nei sistemi informatici le informazoni vengono rappresentate in modo essenziale
attraverso i dati.
I Dati hanno bisogno di essere interpretati, ma costituiscono una precisa rappresentazione
di forme più ricche di informazioni e conoscenza, inoltre sono più stabili nel tempo rispetto
ad altre componenti (come processi, tecnologie, ruoli umani) e restano gli stessi
nella migrazione da un sistema al successivo.
\section{Base di Dati - DB - Data Base}
\paragraph*{Data Base - DB} Collezione di dati utilizzati per rappresentare le
informazioni di interesse di un sistema informativo
\paragraph*{Altra definizione di DB} Insieme di archivi in cui ogni dato è rappresentato
logicamente una sola volta e può essere utilizzato da un insieme di applicazioni da
diversi utenti secondo opportuni criteri di riservatezza.
\paragraph*{Data Base Management System - DBMS} Sistema software capace di
gestire collezioni di dati che siano grandi, condivise e persistenti, assicurando la loro affidabilità
 e privatezza.
\paragraph*{Elenco caratteristiche DBMS} Sistema che garantisce collezioni di dati:
\begin{itemize}
    \item grandi
    \item persistenti
    \item condivise
\end{itemize}
Garantendo:
\begin{itemize}
    \item Privatezza - Meccanismi di autorizzazione (come ACL)
    \item Affidabilità - Resistenza malfunzionamenti hardware e software
    (tramite tecniche come la gestione delle transazioni)
    \item Efficienza
    \item Efficacia
\end{itemize}
\paragraph{Transazione} $\to$ Insieme di operazioni da considerare indivisibile (atomico),
la sequenza di operazioni sulla base di dati viene eseguita per intero o per niente.
\\L'effetto di transazioni concorrenti deve essere coerente (ad esempio "equivalente"
all'esecuzione separata).
\\I risultati delle transizioni sono permanenti, la conclusione di una transazione
corrisponde a un impegno (in inglese commitment) a mantenere traccia del risultato in modo
definitivo.
\\ I DBMS devono essere efficienti cercando di utilizzare al meglio le risorse di
spazio di memoria e tempo.
\\ Efficacia intesa come resa produttiva delle attività dei loro utilizzatori.
\subsection*{Caratteristiche di un DB}
\begin{itemize}
    \item Ridondanza minima e controllata
    \item Consistenza delle informazioni
    \item Dati disponibili per utenze diverse e concorrenti
    \item Dati controllati e protetti (da malfunzionamenti hardware e software)
    \item Indipendenza dei dati dal programma
\end{itemize}

\subsection{Creazione di un database}
\paragraph*{Le tre fasi}
\begin{itemize}
    \item Definizione
    \item Creazione/Popolazione
    \item Manipolazione
\end{itemize}

\paragraph*{Query}
Fondamentale è poter interrogare un DB, attraverso per esempio delle \textbf{query}.

\section{Descrizioni dei dati nei DBMS}
Per descrivere e rappresentare i dati di un problema è necessario utilizzare 

\section{Entità}
Definita come sostantivo al singolare (es. studente, classe, docente, ecc.)
A livello estensionale un'entità è costituira da un insieme di oggetti che sono chiamati le sue istanze.