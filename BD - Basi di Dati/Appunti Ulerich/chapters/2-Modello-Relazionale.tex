\chapter{Modello Relazionale}
\label{sec:Modello_relazionale}
Un modello dei dati è un insieme di concetti per organizzare i dati e 
descriverne la struttura.
Componente fondamentale di ogni modello sono i meccanismi di strutturazione 
(analogi ai costruttori di tipo).
\\ Ogni modello dei dati prevede alcuni costruttori che permettono di definire nuovi
tipi sull abase di tipi predefiniti (elementari).
\\ Il \textbf{modello relazionale} è il modello di dati più diffuso.
\section{Introduzione al modello relazionale e cenni storici}
Il modello relazionale permette di definire tipi per mezzo del costruttore \textbf{relazione}
 che permette di organizzare i dati in insiemi di recordo a \textbf{struttura fissa}.
 \\ Una \textbf{relazione} è spesso rappresentata da una tabella, dove:
 \begin{itemize}
    \item Le righe rappresentano specifici record (istanze)
    \item Le colonne corrispondono ai campi dei record
 \end{itemize}
 L'ordine di righe e colonne è sostanzialmente irrilevante.
 La tabella è il livello logico di cui parlavamo inizialmente (schema DBMS, divisione
 livello fisico e logico). Ribadiamo che il livello logico è indipendente da quello
 fisico infatti una tabella è utilizzata nello stesso modo qualunque sia la sua realizzazione
 fisica. In questo corso vedremo solo il livello logico.
 \subsection{I modelli logici dei dati}
 Tradizionalmente ci sono tre modelli logici:
 \begin{itemize}
    \item Gerarchico - Organizzazione ad albero
    \item Reticolare - Organizzazione a grafo
    \item Relazionale - Organizzazione a tabella
 \end{itemize}
 Poi ci sono altri modelli più recenti (e meno diffusi)
 \begin{itemize}
    \item a oggetti
    \item XML
    \item NoSQL - Basato su documenti
 \end{itemize}
 \subsection{Il modello relazionale}
 Proposto da E.F. Codd nel 1970 per favorire l'indipendenza dei dati, è
 diventato disponibile in DBMS reali nel 1981.
 \\ \'E basato sul concetto matematico di relazione a livello formale (con una variante),
 mentre concettualmente è basato su tabelle.
 \\ Essendo uno schema logico definisce come sono organizzati i dati e non come
 sono memorizzati e gestiti dal sistema informatico.
 \subsection{Il termine relazione in 3 accezioni}
 \begin{enumerate}
    \item Relazione matematica - come nella teoria degli insiemi
    \item Relazione - secondo il modello relazionale dei dati
    \item Relazione (dall'inglese relationship) che rappresenta una classe di fatti,
    nel modello Entity-Relationship, tradotto anche con associazione e correlazione
 \end{enumerate}
 \section{Modello relazionale - definizione formale}
 