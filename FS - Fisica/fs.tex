\documentclass[12pt, a4paper, openany]{book}
\usepackage[inline]{enumitem}
\usepackage{style}


\begin{document}

\title{Fisica}
\author{Fabio Ferrario}
\date{2022/2023}
\maketitle

\tableofcontents

\chapter{Introduzione}

\section{Il corso}
\subsection*{Turno 1}
Il corso di fisica (turno 1) verrà svolto da:
\begin{itemize}
    \item Davide Gerosa (Responsabile corso)
    \item Costantino Pacilio (Esercitatore)
\end{itemize} 
\paragraph*{Orario} Per il turno 1, il corso coprirà 48 ore di lezione frontale e 20 ore di esercitazioni:
\begin{itemize}
    \item Lunedì 13.30-16.30 U3-08, Lezione.
    \item Martedì 14.30-16.30 U2-02, Esercitazione.
    \item Mercoledì, 8.30-10.30 U1-09, Lezione.  
\end{itemize}
\subsection*{Turno 2}
Il turno 2 verrà erogato da:
\begin{itemize}
    \item Alberto Bravin (Lezioni)
    \item Mario Marini (Esercitazioni)
\end{itemize}
\begin{itemize}
    \item Lunedì 8.30-11.30 U9-01, Lezione.
 
\end{itemize}
Le esercitazioni saranno ogni 8 ore di lezione (2 ore esercitazione)

\paragraph*{testo di riferimento} \emph{D.Halliday, R. Resnick. Fondamenti di Fisica (vol. 1 e 2), Casa Editrice Ambrosiana}.

\section{Il Programma}
\paragraph*{Prerequisiti} Le nozioni acquisite nel corso di Analisi Matematica, fra cui derivate ed integrali.

\paragraph*{Contenuti Sintetici} del programma:
\begin{enumerate}
    \item Meccanica.
    \item Gravitazione.
    \item Fluidodinamica.
    \item Onde.
    \item Termodinamica.
    \item Elettromagnetismo.
\end{enumerate}

\paragraph*{Programma Esteso}

\begin{enumerate}
    \item \textbf{Meccanica}.
    \small{\begin{enumerate*}[label=(\roman*)]
        \item Sistemi di coordinate e vettori.
        \item Moto in una e più dimensioni.
        \item Moto rettilineo uniforme, uniformemente accelerato, parabolico, armonico.
        \item Leggi di Newton.
        \item Energia cinetica, energia potenziale principio di conservazione.
        \item Centro di massa.
        \item Corpo rigido.
        \item Momento lineare.
        \item Moti di rotazione e di rotolamento.
        \item Momento angolare, momento di inerzia, momento torcente.
        \item Moti relativi.
    \end{enumerate*}}
    \item \textbf{Gravitazione}.
    \small{\begin{enumerate*}[label=(\roman*)]
        \item Leggi di Keplero.
        \item Legge di gravitazione universale.
        \item Campo gravitazionale.
        \item Legge di Gauss.
        \item Velocità di fuga.
        \item Potenziale efficace.
    \end{enumerate*}}
    \item \textbf{Fluidodinamica}.
    \small{\begin{enumerate*}[label=(\roman*)]
        \item Fluidi, densità e pressione.
        \item Legge di stevino.
        \item Principio di Pascal.
        \item Forza di Archimede.
        \item Equazione di Continuità.
        \item Equazione di Bernoulli.
    \end{enumerate*}}
    \item \textbf{Onde}.
    \small{\begin{enumerate*}[label=(\roman*)]
        \item Oscillatore armonico.
        \item Pendolo semplice.
        \item Oscillatore smorzato.
        \item Risonanza.
        \item Concetto di onda.
        \item Onda piana.
        \item Periodo, lunghezza d'onda, velocità.
        \item Riflessione e interferenza.
        \item Onde stazionarie.
        \item Onde sonore.
        \item Battimenti.
        \item Effetto Doppler
    \end{enumerate*}}
    \item \textbf{Termodinamica}.
    \small{\begin{enumerate*}[label=(\roman*)]
        \item Temperatura e calore.
        \item Calore specifico, calore latente.
        \item Energia interna.
        \item Primo principio della termodinamica.
        \item Trasformazioni termodinamiche.
        \item Trasmissione del calore (conduzione, convezione, irraggiamento).
        \item Legge dei gas perfetti.
        \item Teoria cinetica dei gas.
        \item Irreversibilità, entropia.
        \item Secondo principio della termodinamica.
        \item Macchine termiche.
        \item Ciclo di Carnot.
        \item Zero assoluto.
    \end{enumerate*}}
    \item \textbf{Elettromagnetismo}.
    \small{\begin{enumerate*}[label=(\roman*)]
        \item Carica elettrica.
        \item Legge di Coulomb.
        \item Campo elettrico.
        \item Legge di Gauss.
        \item Potenziale.
        \item Conduttori.
        \item Condensatori.
        \item Corrente elettrica.
        \item Legge di Ohm.
        \item Legge delle maglie, legge dei nodi.
        \item Circuito RC.
        \item Campo magnetico.
        \item Forza di Lorentz.
        \item Legge di Biot-Savart.
        \item Legge di Ampere.
        \item Induzione elettromagnetica.
        \item Legge di Faraday-Lenz.
        \item Circuito RL.
        \item Oscillazione LC.
        \item Oscillazione Smorzata RLC.
        \item Cenni di magnetismo nei materiali.
        \item Legge di Ampere-Maxwell.
        \item Correnti di spostamento.
        \item Equazioni di Maxwell.
        \item Onde elettromagnetiche.
        \item Velocità della luce.
    \end{enumerate*}}
\end{enumerate}

\section{Esame}
L'esame ha \emph{una prova scritta e una orale facoltativa}.
\paragraph*{Prova scritta}
La prova scritta consite in alcuni esercizi da svolgere e alcune domante teoriche.
Ha una durata di 2 ore, il voto massimo è 30/30 e \emph{non vengono sottratti punti per le risposte sbagliate}. 
Ogni esercizio ha l'indicazione del punteggio e le risposte devono essere complete.
è consentito l'utilizzo della calcolatrice (non grafica) ma \emph{non del formulario}.
\paragraph*{Prova orale}
Totalmente facoltativa, ha un punteggio di $\pm 5$ ed è necessaria per il raggiungimento della \emph{lode}.

\chapter*{Introduzione alla Fisica}

\subsection*{Introduzione alla Fisica}
La fisica studia i fenomeni naturali e le leggi che li governano.
Si basa sulla \emph{Semplificazione} dei concetti, tramite \emph{Modelli e Approssimazioni}.

\paragraph*{Antica $\neq$ Moderna} In antichità la fisica era legata alla filosofia e alla religione
e vigeva il principio di autorità, \emph{"ipse dixit"} (Aristotele).
\\La fisica moderna invece nasce con Galileo, separando ciò che è oggettivo da ciò che è soggettivo.
Si basa sulle cose \textbf{Misurabili}.
\paragraph*{Il metodo scientifico} è un sistema che permette a chiunque abbia i mezzi di ripetere un esperimento.
La scienza non è una religione, vale infatti il principio di Falsicabilità:
\begin{center}
    \emph{Principio di Falsicabilità}: Se un esperimento da risultati contrastanti alle teorie correnti, allora c'è bisogno di una nuova teoria.
    \\Ovvero ogni teoria è valida finchè non ne esiste una migliore.
\end{center}
\section*{Le unità di misura}
Il Sistema Internazionale definisce varie unità di misure standard: lunghezza, massa, mole, etc\dots
\\In una formula, le dimensioni devono essere \underline{Bilanciate}.
\subsection*{Le cifre significative}
Un concetto molto importante è quello delle cifre significative.
\\Ogni Misurazione è affetto da \textbf{incertezze}:
\begin{enumerate}
    \item Indeterminazioni nell'effettuare la Misurazione
    \item Limite di sensibilità dello strumento usato
    \item Capacità dello sperimentatore
    \item \emph{Aleatorietà} della sperimentazione
\end{enumerate}
L'incertezza può essere \emph{determinata} oppure \emph{stimata}, in base al caso in esame.
\paragraph*{Contare gli 0} Quando facciamo una misurazione, il numero di cifre significative è sempre importante, anche quando si tratta di 0,
infatti vale che $39,0 \neq 39,00$ \emph{Perchè} 3 cifre significative $\neq$ 4 cifre significative.
\paragraph*{Proprietà delle cifre significative}
\begin{itemize}
    \item \textbf{Moltiplicando (o dividendo)} quantità affette da incertezza, il numero determinato ha \emph{lo stesso numero di cifre significative della Meno accurata delle quantità}
    \item \textbf{Sommando (o sottraendo)} Vale la stessa proprietà.
\end{itemize}



\paragraph*{Ordini di grandezza} 
è un'approssimazione di un numero e indica la potenza di 10 più vicina al numero dato

\subsection*{Vettori e scalari} Esistono due tipi di grandezze nella fisica:
\begin{itemize}
    \item \textbf{Grandezze Scalari}: determinate da un solo numero (la misura) ed una unità di misura.
    \item \textbf{Grandezze Vettoriali}: determinate da più valori $\to$ \emph{Modulo (grandezza), direzione e verso}
    \begin{itemize}
        \item Quando diventa necessario conoscere un punto specifico di localizzazione del vettore (l'origine) si usa la dizione Vettore Applicato.
    \end{itemize}
\end{itemize}
\paragraph*{I Vettori e le proprietà}
I vettori vengono indicati in grassetto o con freccia sormontata: \textbf{a,AB}, \overrightarrow{AB} o \overrightarrow{v}, con il modulo generalmente scritto in corsivo \emph{v}, oppure AB o $|$\overrightarrow{AB}$|$
\\subparagraph*{Algebra Vettoriale}
\begin{itemize}
    \item a=b : vettori uguali \emph{sse} hanno lo stesso modulo, direzione e verso.
    \item b=-a: vettori opposti \emph{sse} hanno stesso modulo e direzione ma verso opposto
    \item vettore nullo: sse ha modulo nullo
\end{itemize}
Somma e differenza di vettori:

\chapter{Lezioni}
\section*{Lezione 2 05-10-22}
\paragraph*{Recap} Avevamo visto i grafici di velocità e accelerazione, e avevamo visto come ricavare acc dalla velocità.
Si era visto il moto uniformemente accelerato con la velocità che cambia linearmente nel tempo.

\paragraph*{Moto vario 1D} Se la velocità varia linearmente, allora l'accelerazione è \emph{costante}.
\paragraph*{Area sotto una curva} Significato fisico dell'area sotto una curva è l'integrale della curva f(x), l'area è definita positiva se è sopra l'asse delle x e negativa se è sotto l'asse delle x.
\paragraph*{Applicazione della proprietà degli integrali} a velocità e accelerazione, lo spazio percorso dal corpo è calcolato con l'integrale dello spostamento/tempo.
L'integrale sotto la curva accelerazione/tempo corrisponde alla velocità del moto
\subsection*{La caduta di un grave} L'acelerazione di gravità è $\pm$ costante ed uguale a $g\approx 9,81m/s^2$. Per un oggetto che cade la forza $g$ va applicata in negativo (va verso il basso).
Date le formule, espoicitando rispetto a t ed indicando con h l'altezza si ha: $t_c=\sqrt{\frac{2h}{g}}$. NB: La caduta non dipende dalla massa del grave

\paragraph*{le equazioni cinematiche} $a:x = \frac{dv_x}{dt} \to dv_x = a_xdt$ (con d = differenziale).
\\per trovare $v_x$ bisogna fare l'integrale (antiderivata) $to v_x = \int a_xdt+c$.
\\Per risolverlo: Se $a_x = costante \to v_x = a_x \int dt + C = a_xt+C$ (C è la costante di integrazione determinata dalle condizioni di contorno, cioè la velocità iniziale)
\\$v_x = dx/dt \to dx=v_xdt$ con l'integrazione $x = \int v_xdt + C$, anche qui se la velocità è costante l'integrale è banale, altrimenti bisogna risolvere con $v_x(t) = v_i + a_x \cdot t$.

\subsection*{Cinematica del punto materiale in 2D}
Fin'ora abbiamo considerato solo il moto in una dimensione, adesso considereremo quella in 2 dimensioni sia uniformemente accelerato che a velocità costante.
Quando studio il movimento 2D o studio il suo movimento in un piano o studio il cambiamento delle coordinate nel tmepo.
Le stesse formule monodimensionali valgono anche per il moto bidimensionale, però bisogna usare i vettori.
\\Quindi per semplice estensione del caso 1d, la velocità media sarà: $v^_ = \frac{\Delta r}{\Delta t}$, poichè t è uno scalare, $v_m$ ha la stessa direzione e verso di $\Delta r$.
\\La velocità istantanea sarà $v = lim_{\Delta \to 0 } \frac{\Delta r}{\Delta t} = \frac{dr}{dt}$ (NB: r è il vettore posizione)
\\In 2D la velocità è la tangente alla traiettoria, che anche se è a modulo costante cambia in direzione punto per punto.
\\Anche se la velocità è costante di modulo, quando cambia di direzione si verifica una \emph{accelerazione}.
$a^_ = \frac{v_f-v_i}{t_f-t_i}= \frac{\Delta v}{\Delta t}$ è l'acelerazione media è ha la stessa direazione del vettore $\Delta v$.
\\Quando l'acelerazione varia nel tempo, è utile definire l'acelerazione istantanea $a = ... = \frac{d^2 \overrightarrow{r} }{dt^2}$
\paragraph*{definizione acelerazione 2D} L'acelerazione di una particella in moto in uno spazio 2D può dunque corrispondere:
\begin{itemize}
    \item ad una variazione del modulo di \textbf{v}
    \item alla variazione di direzione a modulo di \textbf{v} costante
    \item Entrambe le cose combinate
\end{itemize}
\paragraph*{Moto uniformemente acelerato in 2D} %bisogna capire cos'è il raggio vettore %Segnarsi le formule da imparare a memoria
Per passare da 1D a 2D è semplice: si scompone il moto 2d in 2 moti 1d sugli assi x e y
(RIASSUNTO A SLIDE 16)

\paragraph*{Esempio} L'equazione vettoriale del moto di una particella è $r = (2\alpha t^2)i +\beta(2t + t_0)j + 4\delta k$, con a.b.d e t0 costanti. Trovare le componenti cartesiane della velocità e del suo modulo.
Quindi: abbiamo il raggio vettore del moto di una particella, trovare le componenti cartesiane.
per trovare la velocità bisogna trovare la derivata delle tre copmonenti (ijk) quindi:
$$ \begin{cases}
    x^1 = 4\alpha t \\ y^1 = 2\beta \\ z^1 = 0 
\end{cases}
\to 
\begin{cases}
   v= 4(\alpha t )
\end{cases}
$$ finire con slide 17

\subsection*{Il moto di un proiettile}
Il moto di un proiettile è "facile" determinarlo sotto due condizioni:
\begin{itemize}
    \item L'acelerazione di gravita è costante lungo tutto il percorso del proiettile
    \item La resistenza dell'aria è considerata trascurabile
\end{itemize}
Sotto queste condizioni il moto ha un'accelerazione costante ed è di tipo parabolico come già visto nel caso unidimensionale.
\\Se si lancia un proiettile con vettore velocità iniziale $v_i$ che forma un angolo $\theta_i$ con l'asse delle x:
\\si ha che $a_x = 0$, $a_y = -g$, e $v_{xi} = v_i \cos \theta_i$, $v_{yi} = sin \theta_i$. $x_i$ e $y_i$ sono uguale a 0 (origine).
\\IMMAGINE slide 19 
\\RIASSUMENDO : l'unica forza che agisce sul proiettile è quella di gravità, ed agisce come acelerazione nella direzione verticale.
Il moto di un proiettile può essere considerato come l asovrapposizione di due moti indipendenti: una a velictà costante lungo x e uno a caduta libera lungo la direzione verticale.

\paragraph*{eserizio da reisnick}
Una palla è lanciata con velocità iniziale $v_x = 20 m/s$ e $v_y=40m/s$. Quanto tempo rimarrà in aria e quale sarà la distanza percorsa?
\begin{enumerate}
    \item I moti sono indipendenti, quindi mi occupo delle due componenti x,y separatamente
    \item su $v_x$ la velocità rimane costante ma non su $v_y$
    \item SLIDE 26 
\end{enumerate}

\section*{Lezione 3 06-10-22}


\end{document}