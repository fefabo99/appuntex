\documentclass[12pt, a4paper, openany]{book}
\usepackage[inline]{enumitem}


\begin{document}

\title{Fisica}
\author{Fabio Ferrario}
\date{2022/2023}
\maketitle

\tableofcontents

\chapter{Introduzione}

\section{Il corso}
Il corso di fisica (turno 1) verrà svolto da:
\begin{itemize}
    \item Davide Gerosa (Responsabile corso)
    \item Costantino Pacilio (Esercitatore)
\end{itemize} 
\paragraph*{Orario} Per il turno 1, il corso coprirà 48 ore di lezione frontale e 20 ore di esercitazioni:
\begin{itemize}
    \item Lunedì 13.30-16.30 U3-08, Lezione.
    \item Martedì 14.30-16.30 U2-02, Esercitazione.
    \item Mercoledì, 8.30-10.30 U1-09, Lezione.  
\end{itemize}
Il testo di riferimento è: \emph{D.Halliday, R. Resnick. Fondamenti di Fisica (vol. 1 e 2), Casa Editrice Ambrosiana}.

\section{Il Programma}
\paragraph*{Prerequisiti} Le nozioni acquisite nel corso di Analisi Matematica, fra cui derivate ed integrali.

\paragraph*{Contenuti Sintetici} del programma:
\begin{enumerate}
    \item Meccanica.
    \item Gravitazione.
    \item Fluidodinamica.
    \item Onde.
    \item Termodinamica.
    \item Elettromagnetismo.
\end{enumerate}

\paragraph*{Programma Esteso}

\begin{enumerate}
    \item \textbf{Meccanica}.
    \small{\begin{enumerate*}[label=(\roman*)]
        \item Sistemi di coordinate e vettori.
        \item Moto in una e più dimensioni.
        \item Moto rettilineo uniforme, uniformemente accelerato, parabolico, armonico.
        \item Leggi di Newton.
        \item Energia cinetica, energia potenziale principio di conservazione.
        \item Centro di massa.
        \item Corpo rigido.
        \item Momento lineare.
        \item Moti di rotazione e di rotolamento.
        \item Momento angolare, momento di inerzia, momento torcente.
        \item Moti relativi.
    \end{enumerate*}}
    \item \textbf{Gravitazione}.
    \small{\begin{enumerate*}[label=(\roman*)]
        \item Leggi di Keplero.
        \item Legge di gravitazione universale.
        \item Campo gravitazionale.
        \item Legge di Gauss.
        \item Velocità di fuga.
        \item Potenziale efficace.
    \end{enumerate*}}
    \item \textbf{Fluidodinamica}.
    \small{\begin{enumerate*}[label=(\roman*)]
        \item Fluidi, densità e pressione.
        \item Legge di stevino.
        \item Principio di Pascal.
        \item Forza di Archimede.
        \item Equazione di Continuità.
        \item Equazione di Bernoulli.
    \end{enumerate*}}
    \item \textbf{Onde}.
    \small{\begin{enumerate*}[label=(\roman*)]
        \item Oscillatore armonico.
        \item Pendolo semplice.
        \item Oscillatore smorzato.
        \item Risonanza.
        \item Concetto di onda.
        \item Onda piana.
        \item Periodo, lunghezza d'onda, velocità.
        \item Riflessione e interferenza.
        \item Onde stazionarie.
        \item Onde sonore.
        \item Battimenti.
        \item Effetto Doppler
    \end{enumerate*}}
    \item \textbf{Termodinamica}.
    \small{\begin{enumerate*}[label=(\roman*)]
        \item Temperatura e calore.
        \item Calore specifico, calore latente.
        \item Energia interna.
        \item Primo principio della termodinamica.
        \item Trasformazioni termodinamiche.
        \item Trasmissione del calore (conduzione, convezione, irraggiamento).
        \item Legge dei gas perfetti.
        \item Teoria cinetica dei gas.
        \item Irreversibilità, entropia.
        \item Secondo principio della termodinamica.
        \item Macchine termiche.
        \item Ciclo di Carnot.
        \item Zero assoluto.
    \end{enumerate*}}
    \item \textbf{Elettromagnetismo}.
    \small{\begin{enumerate*}[label=(\roman*)]
        \item Carica elettrica.
        \item Legge di Coulomb.
        \item Campo elettrico.
        \item Legge di Gauss.
        \item Potenziale.
        \item Conduttori.
        \item Condensatori.
        \item Corrente elettrica.
        \item Legge di Ohm.
        \item Legge delle maglie, legge dei nodi.
        \item Circuito RC.
        \item Campo magnetico.
        \item Forza di Lorentz.
        \item Legge di Biot-Savart.
        \item Legge di Ampere.
        \item Induzione elettromagnetica.
        \item Legge di Faraday-Lenz.
        \item Circuito RL.
        \item Oscillazione LC.
        \item Oscillazione Smorzata RLC.
        \item Cenni dimagnetismo nei materiali.
        \item Legge di Ampere-Maxwell.
        \item Correnti di spostamento.
        \item Equazioni di Maxwell.
        \item Onde elettromagnetiche.
        \item Velocità della luce.
    \end{enumerate*}}
\end{enumerate}

\section{Esame}
L'esame ha \emph{una prova scritta e una orale facoltativa}.
\paragraph*{Prova scritta}
La prova scritta consite in alcuni esercizi da svolgere e alcune domante teoriche.
Ha una durata di 2 ore, il voto massimo è 30/30 e \emph{non vengono sottratti punti per le risposte sbagliate}. 
Ogni esercizio ha l'indicazione del punteggio e le risposte devono essere complete.
è consentito l'utilizzo della calcolatrice (non grafica) ma \emph{non del formulario}.
\paragraph*{Prova orale}
Totalmente facoltativa, ha un punteggio di $\pm 5$ ed è necessaria per il raggiungimento della \emph{lode}.


\end{document}