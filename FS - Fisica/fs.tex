\documentclass[12pt, a4paper, openany]{book}
\usepackage[inline]{enumitem}
\usepackage{style}


\begin{document}

\title{Fisica}
\author{Fabio Ferrario}
\date{2022/2023}
\maketitle

\tableofcontents

\chapter{Introduzione}

\section{Il corso}
\subsection*{Turno 1}
Il corso di fisica (turno 1) verrà svolto da:
\begin{itemize}
    \item Davide Gerosa (Responsabile corso)
    \item Costantino Pacilio (Esercitatore)
\end{itemize} 
\paragraph*{Orario} Per il turno 1, il corso coprirà 48 ore di lezione frontale e 20 ore di esercitazioni:
\begin{itemize}
    \item Lunedì 13.30-16.30 U3-08, Lezione.
    \item Martedì 14.30-16.30 U2-02, Esercitazione.
    \item Mercoledì, 8.30-10.30 U1-09, Lezione.  
\end{itemize}
\subsection*{Turno 2}
Il turno 2 verrà erogato da:
\begin{itemize}
    \item Alberto Bravin (Lezioni)
    \item Mario Marini (Esercitazioni)
\end{itemize}
\begin{itemize}
    \item Lunedì 8.30-11.30 U9-01, Lezione.
 
\end{itemize}
Le esercitazioni saranno ogni 8 ore di lezione (2 ore esercitazione)

\paragraph*{testo di riferimento} \emph{D.Halliday, R. Resnick. Fondamenti di Fisica (vol. 1 e 2), Casa Editrice Ambrosiana}.

\section{Il Programma}
\paragraph*{Prerequisiti} Le nozioni acquisite nel corso di Analisi Matematica, fra cui derivate ed integrali.

\paragraph*{Contenuti Sintetici} del programma:
\begin{enumerate}
    \item Meccanica.
    \item Gravitazione.
    \item Fluidodinamica.
    \item Onde.
    \item Termodinamica.
    \item Elettromagnetismo.
\end{enumerate}

\paragraph*{Programma Esteso}

\begin{enumerate}
    \item \textbf{Meccanica}.
    \small{\begin{enumerate*}[label=(\roman*)]
        \item Sistemi di coordinate e vettori.
        \item Moto in una e più dimensioni.
        \item Moto rettilineo uniforme, uniformemente accelerato, parabolico, armonico.
        \item Leggi di Newton.
        \item Energia cinetica, energia potenziale principio di conservazione.
        \item Centro di massa.
        \item Corpo rigido.
        \item Momento lineare.
        \item Moti di rotazione e di rotolamento.
        \item Momento angolare, momento di inerzia, momento torcente.
        \item Moti relativi.
    \end{enumerate*}}
    \item \textbf{Gravitazione}.
    \small{\begin{enumerate*}[label=(\roman*)]
        \item Leggi di Keplero.
        \item Legge di gravitazione universale.
        \item Campo gravitazionale.
        \item Legge di Gauss.
        \item Velocità di fuga.
        \item Potenziale efficace.
    \end{enumerate*}}
    \item \textbf{Fluidodinamica}.
    \small{\begin{enumerate*}[label=(\roman*)]
        \item Fluidi, densità e pressione.
        \item Legge di stevino.
        \item Principio di Pascal.
        \item Forza di Archimede.
        \item Equazione di Continuità.
        \item Equazione di Bernoulli.
    \end{enumerate*}}
    \item \textbf{Onde}.
    \small{\begin{enumerate*}[label=(\roman*)]
        \item Oscillatore armonico.
        \item Pendolo semplice.
        \item Oscillatore smorzato.
        \item Risonanza.
        \item Concetto di onda.
        \item Onda piana.
        \item Periodo, lunghezza d'onda, velocità.
        \item Riflessione e interferenza.
        \item Onde stazionarie.
        \item Onde sonore.
        \item Battimenti.
        \item Effetto Doppler
    \end{enumerate*}}
    \item \textbf{Termodinamica}.
    \small{\begin{enumerate*}[label=(\roman*)]
        \item Temperatura e calore.
        \item Calore specifico, calore latente.
        \item Energia interna.
        \item Primo principio della termodinamica.
        \item Trasformazioni termodinamiche.
        \item Trasmissione del calore (conduzione, convezione, irraggiamento).
        \item Legge dei gas perfetti.
        \item Teoria cinetica dei gas.
        \item Irreversibilità, entropia.
        \item Secondo principio della termodinamica.
        \item Macchine termiche.
        \item Ciclo di Carnot.
        \item Zero assoluto.
    \end{enumerate*}}
    \item \textbf{Elettromagnetismo}.
    \small{\begin{enumerate*}[label=(\roman*)]
        \item Carica elettrica.
        \item Legge di Coulomb.
        \item Campo elettrico.
        \item Legge di Gauss.
        \item Potenziale.
        \item Conduttori.
        \item Condensatori.
        \item Corrente elettrica.
        \item Legge di Ohm.
        \item Legge delle maglie, legge dei nodi.
        \item Circuito RC.
        \item Campo magnetico.
        \item Forza di Lorentz.
        \item Legge di Biot-Savart.
        \item Legge di Ampere.
        \item Induzione elettromagnetica.
        \item Legge di Faraday-Lenz.
        \item Circuito RL.
        \item Oscillazione LC.
        \item Oscillazione Smorzata RLC.
        \item Cenni di magnetismo nei materiali.
        \item Legge di Ampere-Maxwell.
        \item Correnti di spostamento.
        \item Equazioni di Maxwell.
        \item Onde elettromagnetiche.
        \item Velocità della luce.
    \end{enumerate*}}
\end{enumerate}

\section{Esame}
L'esame ha \emph{una prova scritta e una orale facoltativa}.
\paragraph*{Prova scritta}
La prova scritta consite in alcuni esercizi da svolgere e alcune domante teoriche.
Ha una durata di 2 ore, il voto massimo è 30/30 e \emph{non vengono sottratti punti per le risposte sbagliate}. 
Ogni esercizio ha l'indicazione del punteggio e le risposte devono essere complete.
è consentito l'utilizzo della calcolatrice (non grafica) ma \emph{non del formulario}.
\paragraph*{Prova orale}
Totalmente facoltativa, ha un punteggio di $\pm 5$ ed è necessaria per il raggiungimento della \emph{lode}.

\chapter{Introduzione alla Fisica}

\subsection{Introduzione alla Fisica}
La fisica studia i fenomeni naturali e le leggi che li governano.
Si basa sulla \emph{Semplificazione} dei concetti, tramite \emph{Modelli e Approssimazioni}.

\paragraph{Antica $\neq$ Moderna} In antichità la fisica era legata alla filosofia e alla religione
e vigeva il principio di autorità, \emph{"ipse dixit"} (Aristotele).
\\La fisica moderna invece nasce con Galileo, separando ciò che è oggettivo da ciò che è soggettivo.
Si basa sulle cose \textbf{Misurabili}.
\paragraph{Il metodo scientifico} è un sistema che permette a chiunque abbia i mezzi di ripetere un esperimento.
La scienza non è una religione, vale infatti il principio di Falsicabilità:
\begin{center}
    \emph{Principio di Falsicabilità}: Se un esperimento da risultati contrastanti alle teorie correnti, allora c'è bisogno di una nuova teoria.
    \\Ovvero ogni teoria è valida finchè non ne esiste una migliore.
\end{center}
\section{Le unità di misura}
Il Sistema Internazionale definisce varie unità di misure standard: lunghezza, massa, mole, etc\dots
\\In una formula, le dimensioni devono essere \underline{Bilanciate}.
\subsection{Le cifre significative}
Un concetto molto importante è quello delle cifre significative.
\\Ogni Misurazione è affetto da \textbf{incertezze}:
\begin{enumerate}
    \item Indeterminazioni nell'effettuare la Misurazione
    \item Limite di sensibilità dello strumento usato
    \item Capacità dello sperimentatore
    \item \emph{Aleatorietà} della sperimentazione
\end{enumerate}
L'incertezza può essere \emph{determinata} oppure \emph{stimata}, in base al caso in esame.
\paragraph{Contare gli 0} Quando facciamo una misurazione, il numero di cifre significative è sempre importante, anche quando si tratta di 0,
infatti vale che $39,0 \neq 39,00$ \emph{Perchè} 3 cifre significative $\neq$ 4 cifre significative.
\paragraph{Proprietà delle cifre significative}
\begin{itemize}
    \item \textbf{Moltiplicando (o dividendo)} quantità affette da incertezza, il numero determinato ha \emph{lo stesso numero di cifre significative della Meno accurata delle quantità}
    \item \textbf{Sommando (o sottraendo)} Vale la stessa proprietà.
\end{itemize}



\paragraph{Ordini di grandezza} 
è un'approssimazione di un numero e indica la potenza di 10 più vicina al numero dato

\subsection{Vettori e scalari} Esistono due tipi di grandezze nella fisica:
\begin{itemize}
    \item \textbf{Grandezze Scalari}: determinate da un solo numero (la misura) ed una unità di misura.
    \item \textbf{Grandezze Vettoriali}: determinate da più valori $\to$ \emph{Modulo (grandezza), direzione e verso}
    \begin{itemize}
        \item Quando diventa necessario conoscere un punto specifico di localizzazione del vettore (l'origine) si usa la dizione Vettore Applicato.
    \end{itemize}
\end{itemize}
\paragraph{I Vettori e le proprietà}
I vettori vengono indicati in grassetto o con freccia sormontata: \textbf{a,AB}, \overrightarrow{AB} o \overrightarrow{v}, con il modulo generalmente scritto in corsivo \emph{v}, oppure AB o $|$\overrightarrow{AB}$|$
\\subparagraph*{Algebra Vettoriale}
\begin{itemize}
    \item a=b : vettori uguali \emph{sse} hanno lo stesso modulo, direzione e verso.
    \item b=-a: vettori opposti \emph{sse} hanno stesso modulo e direzione ma verso opposto
    \item vettore nullo: sse ha modulo nullo
\end{itemize}
Somma e differenza di vettori:

\chapter{Cinematica di un Punto 1D}
\paragraph{introduzione}
Sia dato un sistema di riferimento orientato x.
Sia dato un punto materiale.
Avremo queste tre formule (dei \emph{moduli}):
\begin{itemize}
    \item \textbf{Spostamento}: $s=\Delta x = x_2-x_1$
    \item \textbf{Velocità Media}: $v_m = \frac{(x_f-x_i)}{(t_f-ti)} = \frac{\Delta x}{\Delta t}$
    \item \textbf{Velocità Istantanea}: $v_x = \lim_{t\to 0} \frac{\Delta x}{\Delta t} = \frac{dx}{dt}$
\end{itemize}
\paragraph{Spostamento $\neq$ Distanza}
Lo \emph{spostamento} è un vettore che si annulla quando il punto torna alla posizione di partenza (in un tempo diverso),
mentre la distanza è uno scalare che indica \emph{tutta la distanza che il punto ha già percorso}.
\paragraph{L'accelerazione} L'acelerazione è la variazione della velocità nel tempo.
Le formule dell'acelerazione sono:
\begin{itemize}
    \item \textbf{Accelerazione Media}: $a_m = \frac{v_f-v_i}{t_f-t_i} = \frac{\Delta v}{\Delta t}$
    \item \textbf{Accelerazione Istantanea}: $a_x = \lim_{\Delta t \to 0} = \frac{\Delta v}{\Delta t} = \frac{dv_x(t)}{dt}= \frac{d}{dt}\frac{dx}{dt} = \frac{d^2x}{dt^2}$
\end{itemize}
\emph{L'accelerazione è la derivata della velocità}.
\subsection{I tipi di Moto}
Esistono tre tipi di moto del punto materiale 1D, ed essi variano in base all'accelerazione e alla velocità:
\paragraph*{Moto Vario} Se l'accelerazione varia continuamente, il moto non è facile da analizzare. Infatti non lo vedremo in questo corso.
\paragraph*{Moto rettilineo uniforme}\emph{Velocità costante, accelerazione: 0}.
\\Questo tiop di moto si verifica quando la velocità è costante e ha le seguenti formule per velocità istantanea e spostamento:
\begin{itemize}
    \item $x(t)= x_i + v_x \cdot t$
    \item $v_x = k$, con $k\in R$ quindi COSTANTE.
\end{itemize}
\paragraph*{Moto Uniformemente accelerato} \emph{Con accelerazione costante}.
\\In questo caso l'accelerazione istantanea = accelerazione media in ogni istante, e la velocità cambia \emph{linearmente} durante il moto.
\begin{itemize}
    \item $x(t) = x_i + v_x\cdot t +\frac{1}{2} a_x \cdot t^2$
    \item $v_x(t) = v_{ix} + a_x \cdot t$
    \item $a_x = k1$ con $k1 \in R$.
\end{itemize}
Notare che se $a_x=0$ si ottiene il moto rettilineo uniforme.

\paragraph*{La Caduta di un Grave}
Un caso particolare del moto rettilineo uniforme è la caduta di un grave, in cui la velocità iniziale è \emph{zero} e
l'accelerazione è quella di gravità, ovvero $g\approx 9,81m/s^2$ ed è $\pm$ costante in tutto il mondo.
Essendo il grave in caduta l'accelerazione è negativa, quindi usando le formule del moto uniformemente accelerato e applicando $a=-g=-9,81m/s^2$ otteniamo:
\begin{center} %box formule da inserire
    $y(t)=-\frac{1}{2} g\cdot t^2$\\$ v(t) = -g \cdot t$
\end{center}
Se esplicitiamo la fromula rispetto a t (per sapere quanto ci mette a cadere un grave) e indichiamo con h l'altezza da cui cade otteniamo:
$$t_c =\sqrt{\frac{2h}{g}}$$
Nota che la massa di un oggetto è irrilevante, quindi (nel vuoto) ci mettono tutti lo stesso tempo ad arrivare a terra.

\paragraph*{le equazioni Cinematiche}
Siccome la \emph{Derivazione è la funzione inversa dell'integrazione e viceversa} e l'accelerazione è la derivata della velocità che è la derivata dello spostamento
possiamo da una ottenere l'altra e viceversa:
\begin{itemize} %\to da sostituire con \implies o la doppia implicazione
    \item $a_x = \frac{dv_x}{dt} \to dv_x = a_xdt \to v_x = \int a_x dt + C$
    \begin{itemize}
        \item se $a_x$ è costante allora $v_x = ... = a_x t + C$ (con C velocità iniziale)
    \end{itemize}
    \item $v_x = \frac{dx}{dt} \to dx = v_xdt \to x = \int v_xdt + C$
    \begin{itemize}
        \item se la velocità non è costante allora $v_x(t) = v_i + a_x \cdot t$ 
    \end{itemize}
\end{itemize}
\section*{La cinematica nel punto materiale 2D}
Fin'ora abbiamo considerato solo il moto in una dimensione, adesso considereremo quella in 2 dimensioni.
Quando studio il movimento 2D posso studiarlo in due modi: o studio il suo movimento in un piano o studio il cambiamento delle sue coordinate nel tempo.
\subparagraph*{Movimento in un piano} Per studiare il movimento nel piano (quindi senza usare le coordinate) posso considerare ogni 
punto come un vettore $r$ avente punto di applicazione in 0.
%immagine lez2slide11
In questo caso posso considerare lo spostamento da A a B nel tempo $\Delta t = t_f-t_i$ come $\Delta r = r_f-r_i$

\paragraph*{Estensione del caso 1D} Le stesse formule monodimenisonali valgono anche per il moto bidimensionale, bisogna solo usare i vettori.
\\Per estensione del caso 1D quindi la velocità media sarà:
\begin{center}
    $\overline{v} = \frac{\Delta r}{\Delta t}$ poichè t è uno scalare, $\overline{v}$ ha la stessa direzione e verso di $\Delta r$
\end{center}
Analogamente la velocità istantanea sarà
\begin{center}
    $v = lim_{\Delta \to 0 } \frac{\Delta r}{\Delta t} = \frac{dr}{dt}$
\end{center}
Il vettore velocità è quindi la derivata del vettore posizione rispetto al tempo e in un punto avrà direzione della tangente
alla curva dello spostamento in quel punto.
\\Quando un punto materiale viaggia su una traiettoria curva in 2D, il vettore velocità \emph{varia di direzione punto per punto anche se il modulo rimane costante} e si verifica quindi una \textbf{accelerazione}.
\\L'accelerazione media si calcola:
\[\overline{a}=\frac{v_f-v_i}{t_f-t_i}= \frac{\Delta v}{\Delta t}\]
Analogamente ai casi simili già visti, $a_m$ avrà la stessa direzione del vettore $\Delta v$
%inserire accelerazione istantanea lez2slide13
\paragraph*{riassumento}L'accelerazione di una particella in moto in uno spazio 2d può dunque corrispondere à:
\begin{enumerate}
    \item una variazione del modulo di v
    \item una variazione di direzione a modulo costante di v
    \item una combinazione delle due
\end{enumerate}
Un utile notazione diversa degli stessi concetti la si ha considerando le componenti cartesiane, in cui l'accelerazione è la somma delle acc di tutte le dimensioni cartesiane.
$$ a = \frac{dv_x}{dt}i + \frac{dv_y}{dt}j + \frac{dv_z}{dt}k $$
\subsection*{Moto uniformemente accelerato in 2D}
Per detereminare le equazioni del moto in 2D si estendono i concetti per il moto 1D.
Questo passaggio logico è molto semplice: si \emph{scompone il moto 2D in 2 moti 1D} sugli assi XY.
\paragraph*{Moto con accelerazione costante}
Analogamente al caso 1D.




\chapter{Lezioni}
\section*{Lezione 2 05-10-22}
\paragraph*{Recap} Avevamo visto i grafici di velocità e accelerazione, e avevamo visto come ricavare acc dalla velocità.
Si era visto il moto uniformemente accelerato con la velocità che cambia linearmente nel tempo.

\paragraph*{Moto vario 1D} Se la velocità varia linearmente, allora l'accelerazione è \emph{costante}.
\paragraph*{Area sotto una curva} Significato fisico dell'area sotto una curva è l'integrale della curva f(x), l'area è definita positiva se è sopra l'asse delle x e negativa se è sotto l'asse delle x.
\paragraph*{Applicazione della proprietà degli integrali} a velocità e accelerazione, lo spazio percorso dal corpo è calcolato con l'integrale dello spostamento/tempo.
L'integrale sotto la curva accelerazione/tempo corrisponde alla velocità del moto
\subsection*{La caduta di un grave} L'acelerazione di gravità è $\pm$ costante ed uguale a $g\approx 9,81m/s^2$. Per un oggetto che cade la forza $g$ va applicata in negativo (va verso il basso).
Date le formule, esplicitando rispetto a t ed indicando con h l'altezza si ha: $t_c=\sqrt{\frac{2h}{g}}$. NB: La caduta non dipende dalla massa del grave

\paragraph*{le equazioni cinematiche} $a:x = \frac{dv_x}{dt} \to dv_x = a_xdt$ (con d = differenziale).
\\per trovare $v_x$ bisogna fare l'integrale (antiderivata) $to v_x = \int a_xdt+c$.
\\Per risolverlo: Se $a_x = costante \to v_x = a_x \int dt + C = a_xt+C$ (C è la costante di integrazione determinata dalle condizioni di contorno, cioè la velocità iniziale)
\\$v_x = dx/dt \to dx=v_xdt$ con l'integrazione $x = \int v_xdt + C$, anche qui se la velocità è costante l'integrale è banale, altrimenti bisogna risolvere con $v_x(t) = v_i + a_x \cdot t$.

\subsection*{Cinematica del punto materiale in 2D}
Fin'ora abbiamo considerato solo il moto in una dimensione, adesso considereremo quella in 2 dimensioni sia uniformemente accelerato che a velocità costante.
Quando studio il movimento 2D o studio il suo movimento in un piano o studio il cambiamento delle coordinate nel tmepo.
Le stesse formule monodimensionali valgono anche per il moto bidimensionale, però bisogna usare i vettori.
\\Quindi per semplice estensione del caso 1d, la velocità media sarà: $v^- = \frac{\Delta r}{\Delta t}$, poichè t è uno scalare, $v_m$ ha la stessa direzione e verso di $\Delta r$.
\\La velocità istantanea sarà $v = lim_{\Delta \to 0 } \frac{\Delta r}{\Delta t} = \frac{dr}{dt}$ (NB: r è il vettore posizione)
\\In 2D la velocità è la tangente alla traiettoria, che anche se è a modulo costante cambia in direzione punto per punto.
\\Anche se la velocità è costante di modulo, quando cambia di direzione si verifica una \emph{accelerazione}.
$a^- = \frac{v_f-v_i}{t_f-t_i}= \frac{\Delta v}{\Delta t}$ è l'acelerazione media è ha la stessa direazione del vettore $\Delta v$.
\\Quando l'acelerazione varia nel tempo, è utile definire l'acelerazione istantanea $a = ... = \frac{d^2 \overrightarrow{r} }{dt^2}$
\paragraph*{definizione acelerazione 2D} L'acelerazione di una particella in moto in uno spazio 2D può dunque corrispondere:
\begin{itemize}
    \item ad una variazione del modulo di \textbf{v}
    \item alla variazione di direzione a modulo di \textbf{v} costante
    \item Entrambe le cose combinate
\end{itemize}
\paragraph*{Moto uniformemente acelerato in 2D} %bisogna capire cos'è il raggio vettore %Segnarsi le formule da imparare a memoria
Per passare da 1D a 2D è semplice: si scompone il moto 2d in 2 moti 1d sugli assi x e y
(RIASSUNTO A SLIDE 16)

\paragraph*{Esempio} L'equazione vettoriale del moto di una particella è $r = (2\alpha t^2)i +\beta(2t + t_0)j + 4\delta k$, con a.b.d e t0 costanti. Trovare le componenti cartesiane della velocità e del suo modulo.
Quindi: abbiamo il raggio vettore del moto di una particella, trovare le componenti cartesiane.
per trovare la velocità bisogna trovare la derivata delle tre copmonenti (ijk) quindi:
$$ \begin{cases}
    x^1 = 4\alpha t \\ y^1 = 2\beta \\ z^1 = 0 
\end{cases}
\to 
\begin{cases}
   v= 4(\alpha t )
\end{cases}
$$ finire con slide 17

\subsection*{Il moto di un proiettile}
Il moto di un proiettile è "facile" determinarlo sotto due condizioni:
\begin{itemize}
    \item L'acelerazione di gravita è costante lungo tutto il percorso del proiettile
    \item La resistenza dell'aria è considerata trascurabile
\end{itemize}
Sotto queste condizioni il moto ha un'accelerazione costante ed è di tipo parabolico come già visto nel caso unidimensionale.
\\Se si lancia un proiettile con vettore velocità iniziale $v_i$ che forma un angolo $\theta_i$ con l'asse delle x:
\\si ha che $a_x = 0$, $a_y = -g$, e $v_{xi} = v_i \cos \theta_i$, $v_{yi} = sin \theta_i$. $x_i$ e $y_i$ sono uguale a 0 (origine).
\\IMMAGINE slide 19 
\\RIASSUMENDO : l'unica forza che agisce sul proiettile è quella di gravità, ed agisce come acelerazione nella direzione verticale.
Il moto di un proiettile può essere considerato come l asovrapposizione di due moti indipendenti: una a velictà costante lungo x e uno a caduta libera lungo la direzione verticale.

\paragraph*{eserizio da reisnick}
Una palla è lanciata con velocità iniziale $v_x = 20 m/s$ e $v_y=40m/s$. Quanto tempo rimarrà in aria e quale sarà la distanza percorsa?
\begin{enumerate}
    \item I moti sono indipendenti, quindi mi occupo delle due componenti x,y separatamente
    \item su $v_x$ la velocità rimane costante ma non su $v_y$
    \item SLIDE 26 
\end{enumerate}

\section*{Lezione 3 06-10-22}
\paragraph*{Secondo esercizio} Spariamo a un bersglio di distanza 30m. il proiettile colpisce il bersaglio 1.9cm sotto il centro. determinare il itempo di volo del proiettile e la velocità alla bocca del fucile

Equazioni del moto parabolico : $x_f = v_{xi}t = (v_i cos \theta_i) t | y_f = ...$ Slide 29.

\subsection{il moto circolare uniforme} è il moto di un punto materiale lungo una circonferenza, con un modulo della veloctià costante.
altra definizione: \emph{Dato un punto P sulla circonferenza, questo punto percorrerà spazi uguali in eguali intervalli di tempo.}
\\Siccome c'è una variazione della direzione del vettore velocità $\to$ c'è in gioco una acelerazione. (infatti è definita acelerazione la variazione nel tempo del vettore velocità).
\begin{itemize}
    \item Consideriamo una circonferenza di raggio r
    \item il punto materiale si muove a elocità di moulo costante lungo la traiettoria circolare
    \item il vettore velocità è tangente alla traiettorie circolare
    \item la variazione del vettore velocità $\Delta v$ nel limite di $\Delta t$ sempre più piccoli ha una direzione sempre più rivolta al centro della circonferenza.
\end{itemize}
\paragraph*{L'acelerazione istantanea} è diretta al centro della circonferenza, e per questo motivo si chiama acelerazione centripeta. il suo modulo è: $a_r = \frac{v^2}{r}$ (il pedice r indice che è diretta lungo il raggio r) con r raggio del cerchio.
\\Si chiama velocità tangenziale la velocità diretta lungo la circonferenza.
\\DEFINIZIONE: In un moto circolare uniforme l'accelerazione è diretta verso il centro del cerchio ed ha un modulo pari a $v^2/4$ dove v è il modulo della velocità tangenziale (costante) e r è il raggio del cerchio
\paragraph*{L'equazione angolare oraria} supponiamo che la traiettoria del punto materiale sia di raggio R, centro 0 e giaccia nel piano xy.
Dal punto di vista angolare, l'equazione oraria del moto ha una forma del tipo $\theta = \theta(t)$ che esprime il fatto che il punto materiale <<spazza>> nel taempo l'angolo $\theta$.
\\Esprimiamo la velocità angolare media come di consueto, cioè in questo caso come la variazione dell'angolo $\theta = \theta(t)$ nel tempo.
$$\omega = \frac{\Delta \theta}{\Delta t} = \frac{\theta_f - \theta_i}{t_f - t_i}$$
Se consideriamo il limite per intervalli piccoli del tempo tendenti a zero otteniamo la derivata rispoetto al tempo e la definizione di velictà angolare istantanea. SLIDE 34.
V angolare media e istantanea coincideranno perchè è moto circolare uniforme.
\\possimao ricavarci le equazioni del moto per l'integrazione dell'equazione differenziale. Queste equazioni sono analoghe a quelle del moto rettilineo
\paragraph*{Frequenza e periodo} \begin{itemize}
    \item Il tempo impiegato dal punto a percorre l'intera circonferenza si chiama periodo (T)
    \item il numero di periodi nell'unità di tempo (al secondo) è detto frequenza
\end{itemize}
$\nu = \frac{1}{T}$ la frequenza è indicata con il carattere greco nu $\nu$ (oppure con f)
\begin{itemize}
    \item La velocità tangenziale è lo spazio percorso dal punto nell'intervallo di tempo . $v= \frac{2\pi r}{T}$ ($2\pi r$ è la circonferenza)
\end{itemize}
la velocità angolare è lo spazio angolare percorso dal punto materiale nel tempo $ \omega_0 = 2 \pi \nu = \frac{2 \pi }{T}$ [rad $\cdot t^{-1}$]
\\nel moto circ uni $\omega_0$ è anche detta pulsazione 
\\relazione della velocità tangenziale v con la velocità angolare omega = $v = \omega \cdot r$. quinid la velocità aumenta con il raggio 

\subsection*{Moto circolare generico} è il moto di un punto materiale lungo una circonferenza con un modulo della velocità che cambia nel tempo (per esempio una ruota che acelera e rallenta).
In questo caso l'accelerazione tangenziale è quella che genera la variazione di velocità (Riassunto SLIDE 39).

\paragraph*{Es 4.8 da resnick sul pendolo}
accelerazione tangenziale è dato da a sin(phi), a lungo r è a per il coseno di phi. a = ar + at. (SLIDE 41)

\paragraph*{Sistema di coorinate polari} (questa parte non verrà chiesta ma la fa per noi, perchè se ci danno un programma in cui ci chideono le coordinate polari dobbiamo saperle.)
SLIDE 45 c'è il raggio vettore

\paragraph*{Equazione generica del moto cirolare uniforme in forma parametrica} Supponiamo che la traiettoria del punto materiale sia di raggio R, centro 0 e giaccia nel piano xy e viaggi a velocità angolare $\omega$ costante.
Dla punto di vista angolare, l'equazione oraria ha una forma del tipo :Le sue equazioni parametriche, intermini dell'angolo $\theta$ (detto anomalia) che il vettore posizione forma con l'asse x sono:
$$\begin{cases}
    x(t) = R \cos \theta(t)\\
    y(t) = R \sin \theta (t)
\end{cases}
$$
cioè, le coordinate x e y sono trovabili con il raggio e l'angolo (theta) nel momento in cui si sta cercando la posizione..
\\Ricordando che l'equazione oraria è $\theta = \theta_0 + \omega t$ sostituiamo e otteniamo $x(t) = R \cos (\theta_0 + \omega t)$ e $y(t)=...$.
quindi $r(s) = R\cos\theta(t) i$ è $R\sin\theta(t) j$

\subsection*{Il moto armonico}
SLIDE 49 (nota che nelle formule $\omega$ salta fuori perchè si deriva una funzione composta $\omega t \to \omega$)
Il moto armonico si chiama così perchè dopo un certo tempo assume lo stesso valore di partenza

\section*{Lezione 4 10-10-22}
\paragraph*{Recap moto armonico} %si comincia con un video di flippin physics che fa ridere
Il moto armoinico è un moto che dopo un certo tempo ritorna al punto di partenza. 
La proiezione del moto circolare uniforme (sia su asse x che su y) sono dei moti 1D che si ritrovano in tanti fenomeni fisici.
Ricordo le equazioni parametriche dell'MCU $x(t) = A \cos(\theta_0 + \omega t)$ e y con il seno, con $\omega$ velocità angolare.
La velocità è la derivata dello spazio rispetto al tempo, e l'accelerazione è la derivata della velocità rispetto al tempo, da qui si ottengono le formule (SLIDE 3).
\\Si noti che la velocità massima (solo sull'asse y) di un moto armonico la si ha verso il centro, mentre l'accelerazione è massima agli estremi.
Quindi agli estremi accelerazione masssima e velocità 0, al centro velocità massima e accelerazione 0.
Se si fa una proiezione di un punto su un solo asse si ottiene un grafico sinusoidale.
\subparagraph{velocità}$v_x(t)?-A \omega \sin (\theta_0+\omega t)$ anche il grafico della velocità è periodico (periodo 2pi).
I valori della velocità sono sfasati rispoetto al grafico della posizione.
anche l'accelerazione è periodica e sfasata rispetto alla velocità. 
%FINE RIASSUNTO
\subsection*{Dinamica. Sistemi di riferimento}%Nuovo argomento, non più cinematica
Abbiamo visto l'importanza della scelta del sistema di riferimento, le equazioni del moto variano al variare dei sistemi scelti.
\emph{Come si passa da un sistema di riferimento a un altro?} 

\end{document}