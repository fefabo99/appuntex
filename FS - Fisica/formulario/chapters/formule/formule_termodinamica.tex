\chapter*{Termodinamica}

    \section*{Principio Zero della Termodinamica} Se due corpi $A$ e $B$ si 
    trovano in equilibrio termico con un terzo corpo $T$, allora sono in 
    reciproco quilibrio termico.

    \section*{Dilatazione Termica}
            \begin{equation*}
                \Delta l = \alpha l \Delta T
            \end{equation*}
        Nella quale $\alpha$ è il \textbf{coefficiente di dilatazione lineare}.

        \subsection*{Dilatazione Volumica}
            \begin{equation*}
                \Delta V = \beta V \Delta T
            \end{equation*}
        dove $\beta = 3\alpha$ è il \textbf{coefficiente di dilatazione 
        volumica} della sostanza.

    \section*{Calore}
        \begin{equation*}
            1 \; cal = 4,1868 \; J
        \end{equation*}

        \subsection*{Capacità Termica}
            \begin{equation*}
                Q = C \Delta T = C (T_2 - T_1)
            \end{equation*}
        dove $C$ è la \textbf{capacità termica} dell'oggetto.

        \subsection*{Calore Specifico}
            \begin{equation*}
                Q = cm \Delta T
                \implies
                c = \frac{Q}{m \Delta T}
             \; \Bigg[\frac{J}{K \cdot Kg}\Bigg]
            \end{equation*}

        \subsection*{Calore Latente}
            \begin{equation*}
                Q = Lm \implies L = \frac{Q}{m}
            \end{equation*}
        Il \textbf{calore latente di evaporazione} $L_v$ è la quantità di 
        energia per unità di massa che deve essere fornita per far evaporare un
        liquido o che deve essere \textbf{sottratta} per liquefare un gas. Il 
        \textbf{calore latente di fusione} $L_f$ è la quantità di energia per 
        unità di massa ched eve essere fornita per fondere un solido o che deve
        essere sottratta per solidifcare un liquido.
        
    \section*{Lavoro associato ad una variazione di Volume} 
        \begin{equation*}
            W = \int_{V_i}^{V_f} dW =  \int_{V_i}^{V_f} p \,dV  
        \end{equation*}
    L'integrazione è necessaria perché la pressione $p$ può cambiare durante la
    varizione del volume.

    \section*{Primo Principio della Termodinamica}
        \begin{align*}
            \Delta E_{int} &= E_{int, f} - E_{int, i} = \Delta Q - \Delta W \\
            d E_{int} &= d Q - d W
        \end{align*}

        \subsection*{Applicazioni del Primo Principio} Il primo principio della 
        termodinamica trova applicazione in numerosi casi particolari, tra cui:
            \begin{itemize}
                \item \textit{Trasformazioni Isocore ($V$ costante)}
                \item \textit{Trasformazioni Isobara ($p$ costante)}
                \item \textit{Trasformazioni Isoterma ($T$ costante)}
                \item \textit{Trasformazioni Cicliche}
                    $\Delta E_{int} = 0 \implies \Delta Q = \Delta W$
                \item \textit{Trasformazioni ad Espansione Libera}
                \item \textit{Trasformazioni Adiabatiche}: Ricodiamo 
                        inoltre che $C_v = $ (per tipo di gas):
                        \begin{itemize}
                            \item Monoatomico: $\frac{3}{2} R$
                            \item Biatomico: $\frac{5}{2} R$
                            \item Poliatomico: $3 R$
                        \end{itemize}
            \end{itemize}

            Di seguito saranno analizzati i casi per i gas perfetti:
            
            \begin{tabular}{ |c|c|c|c|c|c|c| } 
                \hline
                    \textbf{Tipo} & $p$ & $V$ & $T$ & $Q$ & $W$ & $E_{int}$ \\
                \hline
                    Isocora 
                        & - 
                        & cost 
                        & - 
                        & $Q = nC_v\Delta T$ 
                        & $W = 0$
                        & $E_{int} = Q$ \\
                \hline
                    Isobara
                        & cost
                        & - 
                        & - 
                        & $Q = nC_p\Delta T$
                        & $W = p\Delta V$
                        & $E_{int}$ \\
                \hline
                    Isoterma
                        & -
                        & - 
                        & cost
                        & $Q = + W$
                        & $W = nRT\ln(\frac{V_f}{V_i})$
                        & $E_{int} = 0$ \\
                \hline
                    Adiabatica
                        & -
                        & - 
                        & -
                        & $Q = 0$
                        & $W = nC_v\Delta T$
                        & $E_{int} = -W$ \\
                \hline
                    Ciclica
                        & -
                        & - 
                        & -
                        & $Q = W$
                        & $W = Q$
                        & $E_{int} = 0$ \\
                \hline
            \end{tabular}

        \begin{itemize}
            \item $C_v$: Calore specifico molare a volume costante
            \item $C_p$: Calore specifico molare a pressione costante
        \end{itemize}

        \section*{Conduzione, Convezione ed Irraggiamento} 

            \subsection*{Conduzione}
                \begin{equation*}
                    P_c = \frac{Q}{t} = kA\frac{T_1 - T_2}{l}
                \end{equation*}

            \paragraph{Resistenza termina alla conduzione}
                \begin{equation*}
                    R = \frac{l}{k}
                \end{equation*}

            \subsection*{Convezione} La convezione ha luogo quando le differenze
            di temperatura causano il moto che trasferisce calore all'interno 
            di un fluido.

            \subsection*{Irraggiamento}
                \begin{equation*}
                    P_r = \sigma \varepsilon AT^4
                \end{equation*}
            Dove $\sigma = 5,6703 \cdot 10^{-8} \; [\frac{W}{m^2 \cdot K^4}
            ]$ è la costante di Stefan-Boltzmann, $\varepsilon$ è l'emittanza
            caratteristica della superficie, $A$ è l'area irraggiante e $T$ la 
            temperatura superficiale in Kelvin. La potenza $P_a$ che un oggetto
            assorbe per via radiativa dell'ambiente a temperatura uniforme
            $T_{amb}$ (in Kelvin) è:
                \begin{equation*}
                    P_a = \sigma \varepsilon AT^4_{amb}
                \end{equation*}

    \section*{Numero di Avogadro}
        \begin{equation*}
            N_a = 6,022 \cdot 10^{23}
        \end{equation*}

        \subsection*{Massa molare}
            \begin{equation*}
                M = mN_a
            \end{equation*}
        
        \subsection*{Numero di moli}
            \begin{equation*}
                n = \frac{N}{N_A} = \frac{M_{cam}}{M} = \frac{M_{cam}}{mN_A}
            \end{equation*}

    \section*{Gas ideale o Gas perfetti} 
        \begin{equation*}
            pV = nRT
        \end{equation*}
    Con:
        \begin{itemize}
            \item $p$ Pressione del cilindro.
            \item $n$ Numero di moli.
            \item $R$ Costante dei gas $= 8,31 \; \frac{J}{mol \cdot K}$
            \item $T$ Temperatura del cilindro.
        \end{itemize}
    Questa legge si può scrivere anche come:
        \begin{equation*}
            pV = NkT
        \end{equation*}
    Dove $k$ è la costante di Boltzmann e vale:
        \begin{equation*}
            k = \frac{R}{N_A} = 1,38 \cdot 10^{-23} \; \Bigg[\frac{J}{K}\Bigg]
        \end{equation*}

    \section*{Pressione, temperatura e velocità molecolare}
            \begin{equation*}
                p = \frac{nMv^2_{qm}}{3V}
            \end{equation*}
    Dove $v_{qm} = \sqrt[]{v^2}$ è la \textbf{velocità quadratica media} delle
    molecole del gas. Con l'equazione dei gas perfetti si ha:
            \begin{equation*}
                v_{qm} = \sqrt[]{\frac{3RT}{M}}
            \end{equation*}