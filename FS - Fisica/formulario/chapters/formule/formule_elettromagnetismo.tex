\chapter*{Elettromagnetismo}

    \section*{Forza di Lorentz}
        \begin{equation*}
            F_B = qv \times B = qvB\cos\Theta \; [T]
        \end{equation*}

    \section*{Carica in moto circolare uniforme}
        \begin{equation*}
            |q|vB = \frac{mv^2}{r}
        \end{equation*}
    da cui troviamo il raggio:
        \begin{equation*}
            r = \frac{mv}{|q|B}
        \end{equation*}
    La frequenza di rivoluzione $f$, la pulsazione $\omega$ e il periodo $T$
    sono dati da:
        \begin{equation*}
            f = \frac{\omega}{2\pi} = \frac{1}{T} = \frac{|q|B}{2\pi m}
        \end{equation*}

    \section*{Forza agente su un filo percorso da corrente}
        \begin{equation*}
            F_B = iL\times B
        \end{equation*}
    La forza agente su un elemento infinitesimo $idL$ in un campo magnetico è:
        \begin{equation*}
            dF_B = idL\times B
        \end{equation*}

    \section*{Campo Magnetico generato da una corrente elettrica}

        \subsection*{Legge di Biot-Savart}
            \begin{equation*}
                dB = \Bigg(\frac{\mu_o}{4\pi}\Bigg)\frac{ids\times r}{r^3}
            \end{equation*}
        In questo caso r è il vetore diretto dall'elemento di corrente verso il
        punto in questione. La quantità chiamata $\mu_0$, chiamata costante di 
        permeabilità magnetica nel vuoto, ha un valore pari a $4\pi \cdot 
        10{^-7} = 1,26 \cdot 10^{-6} T \frac{m}{A}$.

        \subsection*{Campo magnetico in un filo lungo rettilineo}
            \begin{equation*}
                B = \frac{\mu_0i}{2\pi r}
            \end{equation*}

        \subsection*{Campo magnetico in un filo piegato ad arco}
            \begin{equation*}
                B = \frac{\mu_0i\phi}{4\pi R}
            \end{equation*}
        
        \subsection*{Forza tra due fili conduttori paralleli}
            \begin{equation*}
                F_{ba} = i_bLB_a\sin90 = \frac{\mu_0Li_ai_b}{2\pi d}
            \end{equation*}

    \section*{Legge di Ampere} La legge di Ampere afferma:
        \begin{equation*}
            \oint B \, ds = \mu_0i_{ch}
        \end{equation*}

    \section*{Solenoide}
        \begin{equation*}
            B = \mu_0in
        \end{equation*}
    dove $n$ è il numero delle spire per unità di lunghezza. Perciò il campo 
    magnetico interno è uniforme.