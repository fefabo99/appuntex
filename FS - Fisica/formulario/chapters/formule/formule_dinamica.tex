\chapter*{Dinamica}

    \section*{Principi della Dinamica - Leggi di Newton}

        \subsection*{Prima Legge di Newton} Se la somma delle forze che agiscono 
        su un corpo è nulla, allora il corpo in quiete rimarrà in quiete, 
        mentre se è in moto, continuerà a muoversi di moto rettilineo uniforme.

        \subsection*{Seconda Legge di Newton} La forza agente su un corpo è 
        direttamente proporzionale all'accelerazione e ne condivide la 
        direzione e il verso, ed è direttamente proporzionale alla massa. Di 
        contro l'accelerazione cui è soggetto il corpo è direttamente 
        proporzionale alla forza e inversamente proporzionale rispetto alla 
        massa.

        \begin{equation*}
            \vec{F}_{net} = m \vec{a} \; \Bigg[N = \frac{Kg \cdot m}
            {s^2} \Bigg]
        \end{equation*}

        \subsection*{Terza Legge di Newton} Se un corpo A esercita una forza su 
        un corpo B, allora il corpo B esercita su A una forza uguale e 
        contraria.

        \begin{equation*}
            F_{ab} = - F_{ba}
        \end{equation*}

        \begin{quote}
            Attenzione! Le forze hanno modulo uguale ma con segno vettoriale 
            opposto!
        \end{quote}

    \section*{Forza Elastica}
        \begin{equation*}
            F = -kx
        \end{equation*}

    \section*{Carrucola}
        \begin{equation*}
            \begin{cases}
                F_{y1} = T - m_1g \\
                F_{y2} = m_2g - T
            \end{cases}
        \end{equation*}

    \section*{Attrito Statico e Dinamico}
        \begin{itemize}
            \item \textbf{Attrito statico}: agente quando il corpo è fermo, 
            impedendo lo spostamento iniziale.
            \item \textbf{Attrito dinamico}: agente da quando il corpo ha 
            appena compiuto lo spostamento iniziale ed è in movimento.
        \end{itemize}
    Le formule sono per l'attrito statico:
        \begin{equation*}
            f_{s, max} = \mu_sF_N \;\Bigg[N = \frac{Kg \cdot m} {s^2} \Bigg]
        \end{equation*}
        Mentre per quello dinamico:
        \begin{equation*}
            f_{k} = \mu_kF_N \;\Bigg[N = \frac{Kg \cdot m} {s^2} \Bigg]
        \end{equation*}
    
    \section*{Resistenza di un corpo}
        \begin{equation*}
            D = \frac{1}{2}CA\rho v^2
        \end{equation*}
        Con:
        \begin{itemize}
            \item C : Coefficiente di resistenza aerodinamica.
            \item A : Area efficace della sezione trasversale del corpo.
            \item $\rho$ : densità dell'aria
            \item v : velocità.
        \end{itemize}

    \section*{Lavoro}

        \begin{equation*}
            W = Fd = Fd \cos \Theta \;\Bigg[J = N \cdot m = 
            \frac{Kg \cdot m^2}{s^2}\Bigg]
        \end{equation*}

        \subsection*{Lavoro compiuto dalla Forza Gravitazionale}
            \begin{equation*}
                F = mgd\cos \Theta = mgd \cos 180 = - mgd
            \end{equation*}

        \subsection*{Lavoro compiuto dalla Forza Elastica} 
            \begin{equation*}
                W_m = \frac{1}{2}kx^2_i - \frac{1}{2}kx^2_f \;[J]
            \end{equation*}

    \section*{Energia Cinetica}
        \begin{equation*}
            K = \frac{1}{2}mv^2 \;[J]
        \end{equation*}

        \subsection*{Teorema dell'Energia Cinetica}
            \begin{equation*}
                \Delta K = K_f - K_i = W
            \end{equation*}

        \subsection*{Energia Cinetica del Moto Armonico Semplice}
            \begin{equation*}
                K = \frac{1}{2}mv^2 = \frac{1}{2}m\omega^2A^2\sin^2
                (\Theta + \omega t)
            \end{equation*}

    
    \section*{Potenza}
        \begin{equation*}
            P = \frac{W}{\Delta t} \; \Bigg[W \textsf{  Watt} = \frac{J}{s} 
            = \frac{Kg \cdot m^2}{s^2} \cdot \frac{1}{s} 
            = \frac{Kg \cdot m^2}{s^3} \Bigg]
        \end{equation*}
        \begin{equation*}
            P = \frac{dL}{dt} = \frac{F \cos \Theta dx}{dv} 
            = F \cos \Theta (\frac{dx}{dt})
        \end{equation*}
        Ma sappiamo benissimo che $\frac{dx}{dv}$ non è altro che la definzione 
        di velocità. Possiamo quindi riscriverla in modo più semplice:
        \begin{equation*}
            P = Fv\cos\Theta
        \end{equation*}
        Ovvero il \textbf{prodotto scalare} tra $F$ e $v$ (dove $v$ è la 
        velocità della particella). Possiamo quindi scrivere che la Potenza 
        Istantanea di una paricella a velocità $v$ non è altro che:
        \begin{equation*}
            P = F \cdot v
        \end{equation*}

    \section*{Energia Potenziale}
        \begin{equation*}
            \Delta U = - W \;[J]
        \end{equation*}

        \subsection*{Energia Potenziale Gravitazionale}
            \begin{equation*}
                \Delta U = mg(y_f - y_i) = mg \Delta y
            \end{equation*}

        \subsection*{Energia Potenziale Elastica}
            \begin{equation*}
                U(x)     = \frac{1}{2}kx^2
            \end{equation*}

        \subsection*{Energia Potenziale del Moto Armonico Semplice}
            \begin{equation*}
                U = \frac{1}{2}kx^2 = \frac{1}{2}kA^2\cos^2(\Theta + \omega t)
            \end{equation*}

    \section*{Energia Meccanica}
        \begin{equation*}
            E_m = K + U
        \end{equation*}

        \subsection*{Principio di conservazione dell'Energia Meccanica}
            \begin{equation*}
                \Delta E_m = \Delta K + \Delta U = 0
            \end{equation*}

        \paragraph{Principio esteso}
            \begin{equation*}
                \Delta E_m = W_{nc}
            \end{equation*}
        

    \section*{Moto Armonico e Pendolo} \ref{moto_armonico}
        \begin{equation*}
            x(t) = A \cos(\Theta_0 + \omega t)
        \end{equation*}
        ma con:
        \begin{equation*}
            \omega = \sqrt[]{\frac{k}{m}}
        \end{equation*}
        ricordando l'equazione dei un sistema molla-blocco:
        \begin{equation*}
            F = -kx
        \end{equation*}
        
        \paragraph{Il pendolo} Il pendolo può essere descritto tramite il
        Moto Circolare Uniformemente Accelerato:
        \begin{equation*}
            \omega = \sqrt[]{\frac{g}{L}}
        \end{equation*}
    
\newpage
    
    \section*{Momento Lineare}
        \begin{equation*}
            p = mv \; \Bigg[Kg \cdot \frac{m}{s}\Bigg] 
        \end{equation*}

    \section*{Impulso}
        \begin{equation*}
            I = \Delta p
        \end{equation*}

        \paragraph{Connessione con il Momento Lineare} Sia una forza $F = F(t)$
        agente su una particella. Applicando la II Legge di Newton:

        \begin{equation*}
            F = \frac{dp}{dt} \implies dp = F dt
        \end{equation*}
        \begin{equation*}
            \Delta p = p_f - p_i = \int_{t_f}^{t_i} F \,dt = I 
        \end{equation*}
    
    \section*{Urti}
    \begin{itemize}
        \item \textbf{Urti elastici}: Se nell'urto tra due corpi l'Energia 
        Cinetica totale del sistema non cambia ma si conserva completamente.
        \item \textbf{Urti anaelastici}: L'Energia Cinetica non si conserva ma 
        parte viene dispersa in calore o suono (ad esempio).
    \end{itemize} 

        \subsection*{Urti Anaelastici}
            \begin{equation*}
                p_{1,i} + p_{2,i} = p_{1,f} + p_{2,f}
            \end{equation*}
        ovvero:
            \begin{equation*}
                m_1v_{1,i} + m_2v_{2,i} = m_1v_{1,f} + m_2v_{2,f}
            \end{equation*}
        Nel caso di un \textbf{urto completamente anaelastico} uno dei due 
        corpi sarà inizalmente fermo (prendendo il nome di bersaglio):
        \begin{equation*}
            m_1v_{1,i} = (m_1+m_2)V
        \end{equation*}
        

