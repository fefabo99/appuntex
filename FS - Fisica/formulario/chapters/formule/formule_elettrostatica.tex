\chapter*{Elettrostatica}

    \section*{Legge di Coulomb}
        \begin{equation*}
            F = k\frac{q_1q_2}{r^2}\widehat{r}
        \end{equation*}
    Con:
    \begin{equation*}
        k = \frac{1}{4\pi\varepsilon_0}
    \end{equation*}
    Sostituendo nella Legge di Coulomb otteniamo:
    \begin{equation*}
        F = \frac{1}{4\pi\varepsilon_0}\frac{q_1q_2}{r^2}\widehat{r}
    \end{equation*}
    con $\varepsilon_0 = 8,85\cdot 10^{-12}\;C^2/(N\cdot m^2)$.
        \subsection*{Principio di sovrapposizione}
        Se su una particella agiscono più forze di carica, la $F_{tot}$ non 
        sarà altro che la somma vettoriale di tutte le forze!

        \subsection*{Corrente elettrica}
            \begin{equation*}
                i = \frac{dq}{dt} \; [A \cdot s]
            \end{equation*}
    
        \subsection*{La carica è quantizzata} 
            \begin{equation*}
                q = ne
            \end{equation*}
        dove $e = 1,602\cdot10^{-19} [C]$ è la cosstante elementare, una delle
        costanti fondamentali della materia.
        \begin{center}
            \begin{tabular}{ |c|c| } 
                \hline
                Particella & Carica \\
                \hline
                Elettrone & $- e$ \\
                Positrone & $+ e$ \\
                Protone & $+ e$ \\
                Antiprotone & $- e$ \\
                quark & $\pm \frac{1}{3}e \;\; \pm \frac{2}{3}e$ \\
                Neutrone & $0$ \\
                \hline
            \end{tabular}
        \end{center}

        \section*{Campo Elettrico}
            \begin{equation*}
                E = \frac{F}{q_0}\;\Bigg[\frac{N}{C}\Bigg]
            \end{equation*}

            \subsection{Campo elettrico generato da una carica puntiforme}
                \begin{equation}
                    E = \frac{1}{4\pi\varepsilon_0}\frac{|q|}{r^2}
                \end{equation}

            \subsection*{Principio di sovrapposizione} 
                \begin{equation*}
                    E = \frac{F_0}{q_0} = \frac{F_{01} + F_{02} + \cdots + 
                    F_{0n}}{q_0} = E_1 + E_2 + \cdots + E_n
                \end{equation*}

            \subsection*{Densità del campo elettrico} 
            \begin{itemize}
                \item \textit{Densità di carica lineare} $\lambda$: numero di 
                cariche per unità di lunghezza.\\
                    \begin{equation*}
                        \lambda = \frac{Q}{m}
                    \end{equation*}
                \item \textit{Densità di carica superficiale} $\sigma$: numero 
                di cariche per unità di superifice.\\
                    \begin{equation*}
                        \sigma = \frac{Q}{m^2}
                    \end{equation*}
                \item \textit{Densità di carica volumetrica} $\rho$: numero
                di cariche per unità di volume.\\
                    \begin{equation*}
                        \rho = \frac{Q}{m^3}
                    \end{equation*}
            \end{itemize}

        \section*{Flusso di Campo Elettrico}
            \subsection*{Caso 1: superficie piana e parallela con campo $E$
            uniforme}
                \begin{equation*}
                    \Delta\phi = E \Delta A = E \Delta A \cos\Theta 
                \end{equation*}
            Che sull'interezza della superficie diventa:
                \begin{equation*}
                \phi = \int E\cos\Theta \,dA  = E\cos\Theta\int\,dA = E\cos
                \Theta A 
                \end{equation*}
            \subsection*{Caso 2: superficie chiusa e campo $E$ uniforme}
                \begin{equation*}
                    \phi = \oint E \,dA
                \end{equation*}
        L'unità di misura del flusso è la seguente:
                \begin{equation*}
                    [\phi] = [E][A] = [N] \frac{[m^2]}{[C]}
                \end{equation*}
         
        \section*{Teorema di Gauss}
            \begin{equation*}
                \phi = \oint E \, dA = \frac{Q}{\varepsilon_0}
            \end{equation*}
        Non hanno alcuna importanza la dimensione e al forma della superficie,
        l'importante è che sia chiusa.

            \subsection*{Campo elettrico generato da un filo conduttore 
            infinitamente lungo}
                \begin{equation*}
                    E \cdot 2\pi rh = \frac{\lambda h}{\varepsilon_0}
                    \implies
                    E = \frac{\lambda}{2\pi\varepsilon_0r}
                \end{equation*}

            \subsection*{Campo elettrico esterno generato da un conduttore}
                \begin{equation*}
                    E = \frac{\sigma}{\varepsilon_0}
                \end{equation*}

        \section*{Potenziale Elettrico}
            \begin{equation*}
                V = \frac{-W_\infty}{q_0} = \frac{U}{q_0}
            \end{equation*}

            \subsection*{Energia Potenziale Elettrica}
                \begin{equation*}
                    U = qV
                \end{equation*}
            Se la particella si sposta subendo una variazione di potenziale 
            $\Delta V$, la variazione di energia potenziale elettrica è:
                \begin{equation*}
                    \Delta U = q \Delta V = q (V_f - V_i)
                \end{equation*}
            
        \section*{Energia Meccanica}
            \begin{equation*}
                \Delta K = -q \Delta V
            \end{equation*}
        Se al contrario è presente una forza applicata che compie lavoro 
        $w_{app}$ su di essa, la variazione di energia cinetica diventa:
            \begin{equation*}
                \Delta K = -q\Delta V + W_{app}
            \end{equation*}
        Nel caso particolare in cui $\Delta K = 0$:
            \begin{equation*}
                W_{app} = q \Delta V
            \end{equation*}

        \section*{Superfici Equipotenziali}
            \subsection*{Calcolo di $V$ a partire da $E$}
                \begin{equation*}
                    V_f - V_i = - \int_{i}^{f} E \,ds
                \end{equation*}
            Se il punto iniziale è posto all'infinito e $V_i = 0$ si ha, per il
            potenziale in un punto:
                \begin{equation*}
                    V = -\int_{i}^{f} E \,ds
                \end{equation*}
            Nel caso particolare di un campo uniforme di modulo $E$, la 
            differenza di potenziale tra le due linee equipotenziali adiacenti
            (necessariamente parallele) separate da una distanza $Delta x$ è 
            data da:
                \begin{equation*}
                    \Delta V = -E\Delta x
                \end{equation*}

        \section*{Potenziale generato da cariche puntiformi}
            \begin{equation*}
                V = \frac{1}{4\pi\varepsilon_0}\frac{q}{r}
            \end{equation*}
        dove $q$ ha lo stesso segno di $V$. Il potenziale dovuto a una 
        distribuzione di cariche puntiformi è:
            \begin{equation*}
                V = \sum_{i = 1}^{n}V_i = \frac{1}{4\pi\varepsilon_0}
                \sum_{i = 1}^{n}\frac{q_i}{r_i}
            \end{equation*}

        \section*{Potenziale elettrico generato da una carica continua}
            \begin{equation*}
                V = \frac{1}{4\pi\varepsilon_0}\int \frac{dq}{r}
            \end{equation*}
        dove l'integrale è esteso all'intera distribuzione.

        \section*{Calcolo di $E$ partendo da $V$}
            \begin{equation*}
                E_s = - \frac{\partial V}{\partial s}
            \end{equation*}
        Si possono definire le componenti di $E$ secondo $x$, $y$ e $z$ nel 
        seguente modo:
            \begin{align*}
                E_x &= - \frac{\partial V}{\partial x}\\
                E_y &= - \frac{\partial V}{\partial y}\\
                E_z &= - \frac{\partial V}{\partial z}
            \end{align*}
        Se il campo $E$ è uniforme, la prima equazione si riduce a:
            \begin{equation*}
                E = - \frac{\Delta V}{\Delta s}
            \end{equation*}

        \section*{Energia Potenziale di un sistema di cariche puntiformi} 
            \begin{equation*}
                U = L = \frac{1}{4\pi\varepsilon_0}\frac{q_1q_2}{r}
            \end{equation*}