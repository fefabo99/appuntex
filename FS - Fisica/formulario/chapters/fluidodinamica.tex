\chapter{Fluidodinamica}
Un fluido ideale è un fluido incomprimibile; non ha viscosità e il suo flusso è
laminare e irrotazionale. Una \textit{linea di flusso} è il cammino seguito da 
una singola particella di fluido. Un \textit{tubo di flusso} è un fascio di 
linee di flusso.

    \section{Equazione di Continuità} Il principio di conservazione della massa
    impone che il flusso attraverso ogni tubo di flusso obbedisca all'Equazione
    di Continuità.
        \begin{equation}
            R_v = Av = \textsf{cost} \, \Bigg[\frac{m^3}{s}\Bigg]
        \end{equation}
    $R_v$ prende il nome di \textbf{portata volumica}, con $A$ la sezione del 
    tubo in un qualsiasi punto e $v$ la velocità del fluido.
    Sappiamo che la massa di un fluido è data dalla seguente equazione:
        \begin{equation*}
            m = \rho V
        \end{equation*}
    Quindi moltiplicando la Portata Volumica per la densità del fluido 
    otterremo la \textbf{portata massica}, definita come:
        \begin{equation}
            R_m = \rho R_v = \rho A v = \textsf{cost}
        \end{equation}

    \section{Equazione di Bernoulli} Essa ci permette di connettere la 
    pressione, la velocità e l'altezza di un liquido. Supponiamo che in un 
    intervallo di tempo $\Delta t$ una certa quantità di liquido sia entrata da
    sinistra (in un tubo di flusso) e la stessa quantità sia uscita a destra 
    (ricordiamoci che il liquido è incompressibile). Sappiamo inoltre dal 
    Principio di Pascal (pressione!) che la $F$ esercitata dal fluido a 
    sinistra sarà:
        \subsection{Sviluppo dell'Equazione}
            \begin{equation*}
                F_1 = p_1 A_1
            \end{equation*}
        Mentre a destra sarà:
            \begin{equation*}
                F_2 = p_2 A_2
            \end{equation*}
        Il Lavoro $W$ fatto quindi sarà:
            \begin{align*}
                W_1 &= F_1 \Delta x_1 = p_1 A_1 \Delta x_1 \cos 0\\
                W_2 &= F_2 \Delta x_2 = p_2 A_2 \Delta x_2 \cos 180
            \end{align*}
        Il secondo lavoro ha segno negativo siccome la forza gravitazionale ha 
        verso opposto alla massa del fludio in salita! \\
        Il lavoro totale quindi sarà:
            \begin{align*}
                W &= W_1 + W_2 \\
                &= p_1 A_1 \Delta x_1 - p_2 A_2 \Delta x_2 \\
                &= (p_1 - p_2) V \\
                W &= (p_1 - p_2)V 
            \end{align*}

        \subsection{Applicazione Teorema del Lavoro}
        Ricordiamo inoltre il teorema del lavoro: il lavoro di una forza 
        esterna su un sistema è uguale a $\Delta E_{mecc}$. Una parte del 
        lavoro effettuato fa cambiare l'energia cinetica del liquido mentre 
        l'altra fa cambiare l'energia potenziale.
            \begin{align*}
                W &= \Delta K + \Delta U \\
                (p_1 - p_2)V &= \frac{1}{2}mv^2_2 - \frac{1}{2}mv^2_1 
                                + mgy_2 - mgy_1 \\
                &= \frac{1}{2}\rho Vv^2_2 - \frac{1}{2}\rho Vv^2_1 
                                + \rho V g y_2 - \rho V g y_1 \\
                (p_1 - p_2) &= \frac{1}{2}\rho v^2_2 
                                - \frac{1}{2}\rho v^2_1 
                                + \rho g y_2 - \rho g y_1 \\
                &= p_1 + \frac{1}{2}\rho v^2_1 + \rho g y_1 =
                    p_2 + \frac{1}{2}\rho v^2_2 + \rho g y_2 = \textsf{cost}
            \end{align*}
        Quindi:
            \begin{equation}
                p_1 + \frac{1}{2}\rho v^2_1 + \rho g y_1 
                =
                p_2 + \frac{1}{2}\rho v^2_2 + \rho g y_2 
                = 
                \textsf{cost}
            \end{equation}
            \begin{quote}
                In un fluido a flusso laminare, la somma della pressione, 
                dell'Energia Cinetica per unitò di Volume e dell'Energia 
                Potenziale gravitazionale per unità di Volume è 
                \textbf{costante}.
            \end{quote}
