\chapter{Termodinamica} La temperatura è una grandezza fondamentale legata alla
nostra sensazione di caldo-freddo. Nel SI la temperatura si misura in Kelvin 
$K$. La temperatura $T$ quindi assume il seguente valore:
    \begin{equation*}
        T = 273,16 \; [K]
    \end{equation*}

    \paragraph{Conversione da Fahrenheit a Celsius} La quantità corrispondente
    a 1 grado della scala Celsius è equivalente a quella della scala Kelvin, ma
    lo zero della prima è spostato a un valore più comodo. Se $T_c$ rappresenta
    una data temperatura Celsius, allora:
        \begin{equation}
            T_c = T - 273,15^\circ 
        \end{equation}

    \section{Principio Zero della Termodinamica} Se due corpi $A$ e $B$ si 
    trovano in equilibrio termico con un terzo corpo $T$, allora sono in 
    reciproco quilibrio termico.




    \section{Dilatazione Termica}
        \subsection{Dilatazione Lineare} Tutti gli oggetti cambiano dimensioni
        al vaariare della temperatura. La variazione $\Delta l$  di qualsiasi 
        dimensione lineare $l$ è data da:
            \begin{equation}
                \Delta l = \alpha l \Delta T
            \end{equation}
        Nella quale $\alpha$ è il \textbf{coefficiente di dilatazione lineare}.

        \subsection{Dilatazione Volumica} La varaizione $\Delta V$ nel volume 
        di un solido o di un liquido è:
            \begin{equation}
                \Delta V = \beta V \Delta T
            \end{equation}
        dove $\beta = 3\alpha$ è il \textbf{coefficiente di dilatazione 
        volumica} della sostanza.

    \section{Calore} Il calore $Q$ è l'energia che viene trasferita tra un 
    sistema e il suo ambiente a causa della differenza di temperatura tra di 
    essi. Può essere misurato in \textbf{calorie} ($cal$) o in \textbf{Joule}
    ($J$), dove:
        \begin{equation*}
            1 \; cal = 4,1868 \; J
        \end{equation*}

        \subsection{Capacità Termica} È la costante di proporzionalità tra il 
        calore somministrato e la variazione di temperatura indotta:
            \begin{equation}
                Q = C \Delta T = C (T_2 - T_1)
            \end{equation}
        dove $C$ è la \textbf{capacità termica} dell'oggetto.

        \subsection{Calore Specifico} Due oggetti dello stesso materiale hanno 
        capacità termiche proporzionali alla massa. È utile definire quindi una 
        capacità terminca $c$ per unità di massa, che si riferisce non 
        all'oggetto specifico, ma alla massa unitaria di cui il materiale è 
        fatto:
            \begin{equation}
                Q = cm \Delta T
                \implies
                c = \frac{Q}{m \Delta T}
             \; \Bigg[\frac{J}{K \cdot Kg}\Bigg]
            \end{equation}

        \subsection{Calore Latente} La materia si presenta in tre diversi stati
        fisici: solida, liquida e gassosa. Il calore fornito a una sostanza può
        cambiare lo stato fisico della sostanza stessa, per esempio, da solida 
        a liquida o da liquida a gassosa. La quantità di calore richiesta per 
        unità di massa di una determinata sostanza per cambiare il suo stato
        è chiamato \textbf{calore 
        latente} $L$, perciò:
            \begin{equation}
                Q = Lm \implies L = \frac{Q}{m}
            \end{equation}
        Il \textbf{calore latente di evaporazione} $L_v$ è la quantità di  
        energia per unità di massa che deve essere fornita per far evaporare un
        liquido o che deve essere \textbf{sottratta} per liquefare un gas. Il 
        \textbf{calore latente di fusione} $L_f$ è la quantità di energia per 
        unità di massa che deve essere fornita per fondere un solido o che deve
        essere sottratta per solidifcare un liquido.
        
    \section{Lavoro associato ad una variazione di Volume} Un sistema può anche
    scambiare energia con il suo ambiente attraverso il lavoro. La quantità di 
    lavoro $W$ compiuto \textit{da} un sistema quando si espande o quando si 
    riduce da un volume iniziale $V_i$ a un volume finale $V_f$ può essere 
    calcolata con:
        \begin{equation}
            W = \int_{V_i}^{V_f} dW =  \int_{V_i}^{V_f} p \,dV  
        \end{equation}
    L'integrazione è necessaria perché la pressione $p$ può cambiare durante la
    varizione del volume.
        \subsection{Lavoro a temperatura costante e a pressione costante} 
        Applicando la legge dei gas perfetti \ref{gasp} possiamo riscrivere 
        la pressione nel seguente modo:
            \begin{equation*}
                W =  \int_{V_i}^{V_f} p \,dV =  \int_{V_i}^{V_f} 
                \frac{nRT}{V}\,dV 
            \end{equation*}
        Considerando una espansione \textit{isoterma}, la temperatura deve
        rimanere costante, quindi:
            \begin{align*}
                W &= \int_{V_i}^{V_f}\frac{nRT}{V}\,dV \\
                &= nRT \int_{V_i}^{V_f}\frac{dV}{V} 
                = nRT \Bigg[ \ln V \Bigg]_{V_i}^{V_f}
            \end{align*}
        da cui:
            \begin{equation}
                W = nRT \ln \frac{V_f}{V_i}
            \end{equation}
        Se considerassimo invece una espansione \textit{isobara}, ovvero dove 
        la pressione rimane costante, il lavoro risultante sarà:
            \begin{equation}
                W = p(V_f - V_i)
            \end{equation}
        Osservando bene questa formula, affiancata ad un grafico e
        Pressione(y)-Volume(x), notiamo come la definzione di lavoro come 
        integrale (area sottesa) si riduce all'area di un rettangolo 
        ($b\cdot h$).

    \section{Primo Principio della Termodinamica} Il principio di conservazione
    dell'energia per una campione di sostanza che scambia energia con 
    l'ambiente circostante per mezzo di lavoro e calore è espresso nel \textbf{
    primo principio della termodinamica}, che può assumere le due forme:
        \begin{align}
            \Delta E_{int} &= E_{int, f} - E_{int, i} = \Delta Q - \Delta W \\
            d E_{int} &= d Q - d W
        \end{align}
    $E_{int}$ rappresente l'energia interna della sostanza, che dipende solo 
    dal suo stato (temperatura, pressione e volume). $Q$ rappresenta il calore
    scambiato dal sistema con l'ambiente; $Q$ è positivo se il sistema acquista
    calore e negativo se il sistema perde calore. $W$ è il lavoro compiuto
    \textit{dal} sistema; $L$ è positivo se il sistema si espande contro una 
    qualche forza esterna esercitata dall'ambiente, e negativo se il sistema si
    contrae a causa di una forza esterna. Sia $Q$ sia $W$ \textit{dipendono dal
    percorso seguito} $\Delta E_{int}$ \textit{no!}

        \subsection{Applicazioni del Primo Principio} Il primo principio della 
        termodinamica trova applicazione in numerosi casi particolari, tra cui:
            \begin{itemize}
                \item \textit{Trasformazioni Isocore ($V$ costante)}
                \item \textit{Trasformazioni Isobara ($p$ costante)}

                \item \textit{Trasformazioni Isoterma ($T$ costante)}
                \item \textit{Trasformazioni Cicliche}
                \item \textit{Trasformazioni ad Espansione Libera}: si tratta 
                        di trasformazioni adiabatiche nelle quali non viene 
                        compiuto alcun lavoro sul sistema o da parte di esso.
                        Ad esempio se abbiamo due contenitori collegati isolati 
                        dall'esterno. Nel primo c'è un gas mentre il secondo è
                        vuoto. Nel momento in cui apriamo il rubinetto che 
                        collega i due, il gas inizia ad occupare lo spazio 
                        vuoto ma non cambiando calore non compie lavoro 
                        (siccome ci troviamo in un sistema isolato), inoltre,
                        la sua espansione non è contratasta da alcuna pressione
                        . Siamo quindi nella seguente situazione:
                        \begin{align*}
                            \Delta Q &= \Delta W = 0 \\
                            \Delta E_{int} &= 0
                        \end{align*}
                        
                \item \textit{Trasformazioni Adiabatiche}:
                        Ricodiamo inoltre che $C_v = $ (per tipo di gas):
                        \begin{itemize}
                            \item Monoatomico: $\frac{3}{2} R$
                            \item Biatomico: $\frac{5}{2} R$
                            \item Poliatomico: $3 R$
                        \end{itemize}
            \end{itemize}

            Di seguito saranno analizzati i casi per i gas perfetti:
            
            \begin{tabular}{ |c|c|c|c|c|c|c| } 
                \hline
                    \textbf{Tipo} & $p$ & $V$ & $T$ & $Q$ & $W$ & $E_{int}$ \\
                \hline
                    Isocora 
                        & - 
                        & cost 
                        & - 
                        & $Q = nC_v\Delta T$ 
                        & $W = 0$
                        & $E_{int} = Q$ \\
                \hline
                    Isobara
                        & cost
                        & - 
                        & - 
                        & $Q = nC_p\Delta T$
                        & $W = p\Delta V$
                        & $E_{int}$ \\
                \hline
                    Isoterma
                        & -
                        & - 
                        & cost
                        & $Q = + W$
                        & $W = nRT\ln(\frac{V_f}{V_i})$
                        & $E_{int} = 0$ \\
                \hline
                    Adiabatica
                        & -
                        & - 
                        & -
                        & $Q = 0$
                        & $W = nC_v\Delta T$
                        & $E_{int} = -W$ \\
                \hline
                    Ciclica
                        & -
                        & - 
                        & -
                        & $Q = W$
                        & $W = Q$
                        & $E_{int} = 0$ \\
                \hline
            \end{tabular}
             
        \begin{itemize}
            \item $C_v$: Calore specifico molare a volume costante
            \item $C_p$: Calore specifico molare a pressione costante
        \end{itemize}

        \section{Conduzione, Convezione ed Irraggiamento} Sono tre tipi di 
        trasmissione del calore, ognuno funzionante in modo differente 
        dall'altro. Analizziamoli nel dettaglio.

            \subsection{Conduzione} La conduzione avviene mediante il contatto
            tra due superfici. La conduzione di calore $P_c$ su unità di tempo,
            attraverso una lastra le cui superifici sono mantenute alle 
            temperature $T_1$ e $T_2$ è:
                \begin{equation}
                    P_c = \frac{Q}{t} = kA\frac{T_1 - T_2}{l}
                \end{equation}
            Dove $A$ e $l$ sono l'area e la lunghezza della lastra e $k$ è la 
            conducibilità terminca del materiale. Grandi valori di $k$ indicano
            ottimi conduttori termici.

            \paragraph{Resistenza termina alla conduzione} La resistenza 
            termica $R$ per una lastra di spessore $l$ è definita come:
                \begin{equation}
                    R = \frac{l}{k}
                \end{equation}

            \subsection{Convezione} La convezione ha luogo quando le differenze
            di temperatura causano il moto che trasferisce calore all'interno 
            di un fluido.

            \subsection{Irraggiamento} L'irraggiamento è il trasferimento di 
            calore attraverso l'emissione e l'assorbimento di energia 
            elettromagnetica. La potenza $P_r$ irraggiata da un oggetto è:
                \begin{equation}
                    P_r = \sigma \varepsilon AT^4
                \end{equation}
            Dove $\sigma = 5,6703 \cdot 10^{-8} \; [\frac{W}{m^2 \cdot K^4}
            ]$ è la costante di Stefan-Boltzmann, $\varepsilon$ è l'emittanza
            caratteristica della superficie, $A$ è l'area irraggiante e $T$ la 
            temperatura superficiale in Kelvin. La potenza $P_a$ che un oggetto
            assorbe per via radiativa dell'ambiente a temperatura uniforme
            $T_{amb}$ (in Kelvin) è:
                \begin{equation}
                    P_a = \sigma \varepsilon AT^4_{amb}
                \end{equation}

    \section{Numero di Avogadro} Una mole di una sostanza contiene $N_a$
    (\textit{numero di Avogadro}) unità elementari (generalmente atomi o 
    molecole) dove $N_a$ risulta sperimentalmente:
        \begin{equation*}
            N_a = 6,022 \cdot 10^{23}
        \end{equation*}

        \subsection{Massa molare} Una massa molare $M$ di qualsiasi sostanza è
        la massa di una mole della sostanza. È correlata con la massa $m$ di 
        una singola molecola della sostanza:
            \begin{equation}
                M = mN_a
            \end{equation}
        
        \subsection{Numero di moli} Il numero di moli $n$ contenute in un 
        campione di massa $M_{cam}$ costituito da $N$ molecole è dato da:
            \begin{equation}
                n = \frac{N}{N_A} = \frac{M_{cam}}{M} = \frac{M_{cam}}{mN_A}
            \end{equation}

    \section{Gas ideale o Gas perfetti} \label{gasp} Un gas \textit{ideale} o 
        \textit{ perfetto} è quello per il quale la pressione $p$, il volume 
        $V$ e la temperatura $T$ sono correlati da:
        \begin{equation}
            pV = nRT
        \end{equation}
    Con:
        \begin{itemize}
            \item $p$ Pressione del cilindro.
            \item $n$ Numero di moli.
            \item $R$ Costante dei gas $= 8,31 \; \frac{J}{mol \cdot K}$
            \item $T$ Temperatura del cilindro.
        \end{itemize}
    Questa legge si può scrivere anche come:
        \begin{equation}
            pV = NkT
        \end{equation}
    Dove $k$ è la costante di Boltzmann e vale:
        \begin{equation*}
            k = \frac{R}{N_A} = 1,38 \cdot 10^{-23} \; \Bigg[\frac{J}{K}\Bigg]
        \end{equation*}

    \section{Pressione, temperatura e velocità molecolare} La pressione 
    esercitata da $n$ moli di un gas ideale, in funzione della velocità delle
    sue molecole, è:
            \begin{equation}
                p = \frac{nMv^2_{qm}}{3V}
            \end{equation}
    Dove $v_{qm} = \sqrt[]{v^2}$ è la \textbf{velocità quadratica media} delle
    molecole del gas. Con l'equazione dei gas perfetti si ha:
            \begin{equation}
                v_{qm} = \sqrt[]{\frac{3RT}{M}}
            \end{equation}