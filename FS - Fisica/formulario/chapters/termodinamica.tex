\chapter{Termodinamica}

    \section{Dilatazione Termica}
        \subsection{Dilatazione Lineare}
            \begin{equation}
                \Delta L = \alpha L \Delta T
            \end{equation}
        \subsection{Dilatazione Volumica}
            \begin{equation}
                \Delta V = \beta V \Delta T
            \end{equation}
    \section{Numero di Avogadro}
        \begin{equation*}
            N_a = 6,022 \cdot 10^{23}
        \end{equation*}

        \subsection{Massa del campione}
            \begin{equation}
                M = mN_a
            \end{equation}
    
    \section{Gas ideale o Gas perfetti}
        \begin{equation}
            pV = nRT
        \end{equation}
    Con:
        \begin{itemize}
            \item $p$ Pressione del cilindro.
            \item $n$ Numero di moli.
            \item $R$ Costante dei gas $= 8,31 \; \frac{J \cdot mol}{K}$
            \item $T$ Temperatura del cilindro.
        \end{itemize}
        \subsection{Equazione di stato per i gas ideali}
            \begin{equation}
                (p + \frac{an^2}{v^2})(V - nb) = nRT
            \end{equation}

    \section{Capacità Termica} È la costsante di proporzionalità tra il calore
    somministrato e la variazione di temperatura indotta:
            \begin{equation}
                Q = C \Delta T = C (T_2 - T_1)
            \end{equation}
    
    \section{Calore Specifico} Due oggetti dello stesso materiale hanno 
    capacità termiche proporzionali alla massa. È utile definire quindi una 
    capacità terminca $c$ per unità di massa, che si riferisce non all'oggetto
    specifico, ma alla massa unitaria di cui il materiale è fatto:
            \begin{equation}
                c = \frac{C}{m} \; \Bigg[\frac{J}{K \cdot Kg}\Bigg]
            \end{equation}

    \section{Calore Latente}