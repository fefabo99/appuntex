\chapter{Termodinamica} La temperatura è una grandezza fondamentale legata alla
nostra sensazione di caldo-freddo. Nel SI la temperatura si misura in Kelvin 
$K$. La temperatura $T$ quindi assume il seguente valore:
    \begin{equation*}
        T = 273,16 \; [K]
    \end{equation*}

    \paragraph{Conversione da Fahrenheit a Celsius} La quantità corrispondente
    a 1 grado della scala Celsius è equivalente a quella della scala Kelvin, ma
    lo zero della prima è spostato a un valore più comodo. Se $t_c$ rappresenta
    una data temperatura Celsius, allora:
        \begin{equation}
            T_c = T - 273,15^\circ 
        \end{equation}

    \section{Principio Zero della Termodinamica} Se due corpi $A$ e $B$ si 
    trovano in equilibrio termico con un terzo corpo $T$, allora sono in 
    reciproco quilibrio termico.




    \section{Dilatazione Termica}
        \subsection{Dilatazione Lineare} Tutti gli oggetti cambiano dimensioni
        al vaariare della temperatura. La variazione $\Delta$ l di qualsiasi 
        dimensione lineare $l$ è data da:
            \begin{equation}
                \Delta l = \alpha l \Delta T
            \end{equation}
        Nella quale $\alpha$ è il \textbf{coefficiente di dilatazione lineare}.

        \subsection{Dilatazione Volumica} La varaizione $\Delta V$ nel volume 
        di un solido o di un liquido è:
            \begin{equation}
                \Delta V = \beta V \Delta T
            \end{equation}
        dove $\beta = 3\alpha$ è il \textbf{coefficiente di dilatazione 
        volumica} della sostanza.

    \section{Calore} Il calore $Q$ è l'energia che viene trasferita tra un 
    sistema e il suo ambiente a causa della differenza di temperatura tra di 
    essi. Può essere misurato in \textbf{calorie} ($cal$) o in \textbf{Joule}
    ($J$), dove:
        \begin{equation*}
            1 \; cal = 4,1868 \; J
        \end{equation*}

        \subsection{Capacità Termica} È la costsante di proporzionalità tra il 
        calore somministrato e la variazione di temperatura indotta:
            \begin{equation}
                Q = C \Delta T = C (T_2 - T_1)
            \end{equation}
        dove $C$ è la \textbf{capacità termica} dell'oggetto.

        \subsection{Calore Specifico} Due oggetti dello stesso materiale hanno 
        capacità termiche proporzionali alla massa. È utile definire quindi una 
        capacità terminca $c$ per unità di massa, che si riferisce non 
        all'oggetto specifico, ma alla massa unitaria di cui il materiale è 
        fatto:
            \begin{equation}
                Q = cm \Delta T
                \implies
                c = \frac{Q}{m \Delta T}
             \; \Bigg[\frac{J}{K \cdot Kg}\Bigg]
            \end{equation}

        \subsection{Calore Latente} La materia si presenta in tre diversi stati
        fisici: solida, liquida e gassosa. Il calore fornito a una sostanza può
        cambiare lo stato fisico della sostanza stessa, per esempio, da solida 
        a liquida o da liquida a gassosa. La quantità di calore richiesta per 
        unità di massa di una determinata sostanza è il \textbf{calore 
        latente} $L$, perciò:
            \begin{equation}
                Q = Lm \implies L = \frac{Q}{m}
            \end{equation}
        Il \textbf{calore latente di evaporazione} $L_v$ è la quantità di 
        energia per unità di massa che deve essere fornita per far evaporare un
        liquido o che deve essere \textbf{sottratta} per liquefare un gas. Il 
        \textbf{calore latente di fusione} $L_f$ è la quantità di energia per 
        unità di massa ched eve essere fornita per fondere un solido o che deve
        essere sottratta per solidifcare un liquido.
        
    \section{Lavoro associato ad una variazione di Volume} Un sistema può anche
    scambiare energia con il suo ambiente attraverso il lavoro. La quantità di 
    lavoro $W$ compiuto \textit{da} un sistema quando si espande o quando si 
    riduce da un volume iniziale $V_i$ a un volume finale $V_f$ può essere 
    calcolata con:
        \begin{equation}
            L = \int_{V_i}^{V_f} dL =  \int_{V_i}^{V_f} p \,dV  
        \end{equation}
    L'integrazione è necessaria perché la pressione $p$ può cambiare durante la
    varizione del volume.

    \section{Primo Principio della Termodinamica} Il principio di conservazione
    dell'energia per una campione di sostanza che scambia energia con 
    l'ambiente circostante per mezzo di lavoro e calore è espresso nel \textbf{
    primo principio della termodinamica}, che può assumere le due forme:
        \begin{align}
            \Delta E_{int} &= E_{int, f} - E_{int, i} = \Delta Q - \Delta W \\
            d E_{int} &= d Q - d W
        \end{align}
    $E_{int}$ rappresente l'energia interna della sostanza, che dipedne solo 
    dal suo stato (temperatura, pressione e volume). $Q$ rappresenta il calore
    scambiato dal sistema con l'ambiente; $Q$ è positivo se il sistema acquista
    calore e negativo se il sistema perde calore. $W$ è il lavoro compiuto
    \textit{dal} sistema; $L$ è positivo se il sistema si espande contro una 
    qualche forza esterna esercitata dall'ambiente, e negativo se il sistema si
    contrae a causa di una forza esterna. Sia $Q$ sia $W$ \textit{dipendono dal
    percorso seguito} $\Delta E_{int}$ \textit{no!}



    \section{Numero di Avogadro}
        \begin{equation*}
            N_a = 6,022 \cdot 10^{23}
        \end{equation*}

        \subsection{Massa del campione}
            \begin{equation}
                M = mN_a
            \end{equation}
    
    \section{Gas ideale o Gas perfetti}
        \begin{equation}
            pV = nRT
        \end{equation}
    Con:
        \begin{itemize}
            \item $p$ Pressione del cilindro.
            \item $n$ Numero di moli.
            \item $R$ Costante dei gas $= 8,31 \; \frac{J \cdot mol}{K}$
            \item $T$ Temperatura del cilindro.
        \end{itemize}
        \subsection{Equazione di stato per i gas ideali}
            \begin{equation}
                (p + \frac{an^2}{v^2})(V - nb) = nRT
            \end{equation}