\chapter{Elettrostatica}

    \section{Legge di Coulomb}
        \begin{equation}
            F = k\frac{q_1q_2}{r^2}\widehat{r}
        \end{equation}
    Con:
    \begin{equation}
        k = \frac{1}{4\pi\varepsilon_0}
    \end{equation}
    Sostituendo nella Legge di Coulomb otteniamo:
    \begin{equation}
        F = \frac{1}{4\pi\varepsilon_0}\frac{q_1q_2}{r^2}\widehat{r}
    \end{equation}
        \subsection{Principio di sovrapposizione}
        Se su una particella agiscono più forze di carica, la $F_{tot}$ non 
        sarà altro che la somma vettoriale di tutte le forze!

        \subsection{Corrente elettrica}
            \begin{equation}
                i = \frac{dq}{dt} \; [A \cdot s]
            \end{equation}
    
        \subsection{La carica è quantizzata} Qualunque carica $q$ può essere
        scritta come:
            \begin{equation}
                q = ne
            \end{equation}
        dove $e = 1,602\cdot10^{-19} [C]$ è la cosstante elementare, una delle
        costanti fondamentali della materia.
        \begin{center}
            \begin{tabular}{ |c|c| } 
                \hline
                Particella & Carica \\
                \hline
                Elettrone & $- e$ \\
                Positrone & $+ e$ \\
                Protone & $+ e$ \\
                Antiprotone & $- e$ \\
                quark & $\pm \frac{1}{3}e \;\; \pm \frac{2}{3}e$ \\
                Neutrone & $0$ \\
                \hline
            \end{tabular}
        \end{center}
        Quando una grandezza fisica assume solo valori discreti, allora si dice
        \textbf{quatizzata}. Anche la massa è quantizzata, non è quantizzata ad
        esempio la velocità, il flusso, il calore, la temperatura etc\dots

        \section{Campo Elettrico}
            \begin{equation}
                E = \frac{F}{q_0}
            \end{equation}

            \subsection{Principio di sovrapposizione} Anche per il campo 
            elettrico vale il pricipio di sovrapposizione. Infatti se una 
            particella è in interazione con più campi, il campo $E$ risultante
            non sarà altro che la somma dei campi:
                \begin{equation}
                    E = \frac{F_0}{q_0} = \frac{F_{01} + F_{02} + \cdots + 
                    F_{0n}}{q_0} = E_1 + E_2 + \cdots + E_n
                \end{equation}

            \subsection{Campo elettrico generato da una carica lineare} Diamo 
            ora delle definizioni utili per dei calcoli e delle considerazioni
            successive:
            \begin{itemize}
                \item \textit{Densità di carica lineare} $\lambda$: numero di 
                cariche per unità di lunghezza.\\
                    \begin{equation}
                        \lambda = \frac{Q}{m}
                    \end{equation}
                \item \textit{Densità di carica superficiale} $\sigma$: numero 
                di cariche per unità di superifice.\\
                    \begin{equation}
                        \lambda = \frac{Q}{m^2}
                    \end{equation}
                \item \textit{Densità di carica volumetrica} $\rho$: numero
                di cariche per unità di volume.\\
                    \begin{equation}
                        \lambda = \frac{Q}{m^3}
                    \end{equation}
            \end{itemize}

        \section{Flusso di Campo Elettrico} Il flusso di campo elettrico $E$ 
        attraverso una superficie infinitesima $\Delta A$ è il prodotto scalare
        del campo elettrico e il vettore-area $\Delta A$ in quel punto 
        (perfetta analogia con la fluidodinamica). Il vettore area $\Delta A$ è
        il vwttore che ha per modulo l'area $\Delta A$ e direzione quella 
        perperndicolare all'area stessa.
            \subsection{Caso 1: superficie piana e parallela con campo $E$
            uniforme} Scompondendo il vettore del campo nelle sue componenti 
            cartesiane, noteremo che parte del campo viene scomposto lungo $y$,
            di conseguenza, il passaggio massimo si ha quando la normale della 
            superficie $A$ è parallela al campo, avendo così $\cos0$. Definiamo
            quindi la formula per il flusso del campo elettrico:
                \begin{equation}
                    \Delta\phi = E \Delta A = E \Delta A \cos\Theta 
                \end{equation}
            Che sull'interezza della superficie diventa:
                \begin{equation}
                \phi = \int E\cos\Theta \,dA  = E\cos\Theta\int\,dA = E\cos
                \Theta A 
                \end{equation}
            \subsection{Caso 2: superficie chiusa e campo $E$ uniforme} Nel 
            caso di una superficie chiusa dobbiamo valutare la somma dei flussi
            attraverso tutte le superfici, facendo l'intergrale su ogni piccola
            area:
                \begin{equation}
                    \phi = \oint E \,dA
                \end{equation}
        L'unità di misura del flusso è la seguente:
                \begin{equation*}
                    [\phi] = [E][A] = [N] \frac{[m^2]}{[C]}
                \end{equation*}
         
        \section{Teorema di Gauss} Il Teorema di Gauss afferma che il flusso 
        $\phi$ del campo elettrico attraverso una superficie chiusa è eguale
        alla carica totale Q racchiusa nella superficie, diviso $\varepsilon_0$:
            \begin{equation}
                \phi = \oint E \, dA = \frac{Q}{\varepsilon_0}
            \end{equation}
        Non hanno alcuna importanza la dimensione e al forma della superficie,
        l'importante è che sia chiusa.

        \section{Energia Potenziale Elettrica}
        