\chapter{Elettrostatica}

    \section{Legge di Coulomb}
        \begin{equation}
            F = k\frac{q_1q_2}{r^2}\widehat{r}
        \end{equation}
    Con:
        \begin{equation}
            k = \frac{1}{4\pi\varepsilon_0}
        \end{equation}
    Sostituendo nella Legge di Coulomb otteniamo:
        \begin{equation}
            F = \frac{1}{4\pi\varepsilon_0}\frac{q_1q_2}{r^2}\widehat{r}
        \end{equation}
    con $\varepsilon_0 = 8,85\cdot 10^{-12}\;C^2/(N\cdot m^2)$.
        \subsection{Principio di sovrapposizione}
        Se su una particella agiscono più forze di carica, la $F_{tot}$ non 
        sarà altro che la somma vettoriale di tutte le forze!

        \subsection{Corrente elettrica}
            \begin{equation}
                i = \frac{dq}{dt} \; [A \cdot s]
            \end{equation}
    
        \subsection{La carica è quantizzata} Qualunque carica $q$ può essere
        scritta come:
            \begin{equation}
                q = ne
            \end{equation}
        dove $e = 1,602\cdot10^{-19} [C]$ è la costante elementare, una delle
        costanti fondamentali della materia.
        \begin{center}
            \begin{tabular}{ |c|c| } 
                \hline
                Particella & Carica \\
                \hline
                Elettrone & $- e$ \\
                Positrone & $+ e$ \\
                Protone & $+ e$ \\
                Antiprotone & $- e$ \\
                quark & $\pm \frac{1}{3}e \;\; \pm \frac{2}{3}e$ \\
                Neutrone & $0$ \\
                \hline
            \end{tabular}
        \end{center}
        Quando una grandezza fisica assume solo valori discreti, allora si dice
        \textbf{quantizzata}. Anche la massa è quantizzata, non è quantizzata ad
        esempio la velocità, il flusso, il calore, la temperatura etc\dots

        \section{Campo Elettrico}
            \begin{equation}
                E = \frac{F}{q_0}\;\Bigg[\frac{N}{C}\Bigg]
            \end{equation}

            \subsection{Campo elettrico generato da una carica puntiforme} Il 
            modulo del campo elettrico $E$ stabilito da una carica puntiforme 
            $q$ alla distanza $r$ dalla particella è:
                \begin{equation}
                    E = \frac{1}{4\pi\varepsilon_0}\frac{|q|}{r^2}
                \end{equation}

            \subsection{Principio di sovrapposizione} Anche per il campo 
            elettrico vale il principio di sovrapposizione. Infatti se una 
            particella è in interazione con più campi, il campo $E$ risultante
            non sarà altro che la somma dei campi:
                \begin{equation}
                    E = \frac{F_0}{q_0} = \frac{F_{01} + F_{02} + \cdots + 
                    F_{0n}}{q_0} = E_1 + E_2 + \cdots + E_n
                \end{equation}

            \subsection{Densità del campo elettrico} Diamo 
            ora delle definizioni utili per dei calcoli e delle considerazioni
            successive:
            \begin{itemize}
                \item \textit{Densità di carica lineare} $\lambda$: numero di 
                cariche per unità di lunghezza.\\
                    \begin{equation}
                        \lambda = \frac{Q}{m}
                    \end{equation}
                \item \textit{Densità di carica superficiale} $\sigma$: numero 
                di cariche per unità di superifice.\\
                    \begin{equation}
                        \sigma = \frac{Q}{m^2}
                    \end{equation}
                \item \textit{Densità di carica volumetrica} $\rho$: numero
                di cariche per unità di volume.\\
                    \begin{equation}
                        \rho = \frac{Q}{m^3}
                    \end{equation}
            \end{itemize}

        \section{Flusso di Campo Elettrico} Il flusso di campo elettrico $E$ 
        attraverso una superficie infinitesima $\Delta A$ è il prodotto scalare
        del campo elettrico e il vettore-area $\Delta A$ in quel punto 
        (perfetta analogia con la fluidodinamica). Il vettore area $\Delta A$ è
        il vettore che ha per modulo l'area $\Delta A$ e direzione quella 
        perperndicolare all'area stessa.
            \subsection{Caso 1: superficie piana e parallela con campo $E$
            uniforme} Scompondendo il vettore del campo nelle sue componenti 
            cartesiane, noteremo che parte del campo viene scomposto lungo $y$,
            di conseguenza, il passaggio massimo si ha quando la normale della 
            superficie $A$ è parallela al campo, avendo così $\cos0$. Definiamo
            quindi la formula per il flusso del campo elettrico:
                \begin{equation}
                    \Delta\phi = E \Delta A = E \Delta A \cos\Theta 
                \end{equation}
            Che sull'interezza della superficie diventa:
                \begin{equation}
                \phi = \int E\cos\Theta \,dA  = E\cos\Theta\int\,dA = E\cos
                \Theta A 
                \end{equation}
            \subsection{Caso 2: superficie chiusa e campo $E$ uniforme} Nel 
            caso di una superficie chiusa dobbiamo valutare la somma dei flussi
            attraverso tutte le superfici, facendo l'intergrale su ogni piccola
            area:
                \begin{equation}
                    \phi = \oint E \,dA
                \end{equation}
        L'unità di misura del flusso è la seguente:
                \begin{equation*}
                    [\phi] = [E][A] = [N] \frac{[m^2]}{[C]}
                \end{equation*}
         
        \section{Teorema di Gauss} Il Teorema di Gauss afferma che il flusso 
        $\phi$ del campo elettrico attraverso una superficie chiusa è eguale
        alla carica totale Q racchiusa nella superficie, diviso $\varepsilon_0$:
            \begin{equation}
                \phi = \oint E \, dA = \frac{Q}{\varepsilon_0}
            \end{equation}
        Non hanno alcuna importanza la dimensione e al forma della superficie,
        l'importante è che sia chiusa.

            \subsection{Campo elettrico generato da un filo conduttore 
            infinitamente lungo} Tramite la \textit{densità lineare di carica}
            è possibile calcolarsi il campo elettrico generato da un filo 
            conduttore. Dobbiamo considerare il filo come una superficie 
            gaussiana cilindrica:
                \begin{equation}
                    \phi = \oint E \, dA = \frac{Q}{\varepsilon_0} = 
                    \frac{\lambda h}{\varepsilon_0}
                \end{equation}
            scompondendo l'integrale come somma di area alta (cima del cilindro
            ), area bassa (fondo del cilindro) e area laterale:
                \begin{equation*}
                    \int_{sup-alta} E \, dA + \int_{sup-bassa} E \, dA
                    + \int_{sup-lat} E \, dA
                \end{equation*}
            Ma nella superficie alta e quella bassa $E$ e $dA$ sono 
            perpendicolari, quindi:
                \begin{align*}
                    \int_{sup-alta} E \, dA &= 0 \\
                    \int_{sup-bassa} E \, dA &= 0
                \end{align*}
            quindi:
                \begin{equation*}
                    \phi = \oint E \, dA = \frac{\lambda h}{\varepsilon_0}
                    = \int_{sup-lat} E \, dA = E \cdot 2\pi rh
                \end{equation*}
            da cui:
                \begin{equation}
                    E \cdot 2\pi rh = \frac{\lambda h}{\varepsilon_0}
                    \implies
                    E = \frac{\lambda}{2\pi\varepsilon_0r}
                \end{equation}

            \subsection{Campo elettrico esterno generato da un conduttore}
                \begin{equation*}
                    \phi = \oint E \, dA = \frac{Q}{\varepsilon_0} = 
                    \frac{\sigma A}{\varepsilon_0} = \int_{sup_est} E \, dA
                \end{equation*}
            quindi:
                \begin{equation*}
                    \phi = \oint E \, dA = \frac{Q}{\varepsilon_0} = 
                    \frac{\sigma A}{\varepsilon_0} = EA
                \end{equation*}
            da cui:
                \begin{equation}
                    E = \frac{\sigma}{\varepsilon_0}
                \end{equation}

        \section{Potenziale Elettrico} Il potenziale elettrico $V$ nel punto 
        $P$ di un campo elettrico generato da un corpo carico è (ricordando che
        $U = -W$):
            \begin{equation}
                V = \frac{-W_\infty}{q_0} = \frac{U}{q_0}
            \end{equation}
        dove $W_\infty$ è il lavoro che dovrebbe compiere la forza elettrica su
        una carica esplorativa per portarla in $P$ da una distanza infinita, 
        mentre $U$ è l'energia potenziale che verrebbe così immagazzinata nel
        sistema corpo carico-carica esplorativa.
            \begin{quote}
                \textbf{Attenzione!} La parola \textit{Potenziale} non va 
                confusa con \textit{Energia Potenziale}. Sebbene hanno nomi 
                simili, il loro significato è assolutamente distinto.
                Non potendo spostare una carica elettrica "fino all'infinito" 
                si pone allora l'attenzione sull'energia "potenziale" 
                liberabile da questa durante l'ipotetico movimento. È 
                interessante però notare che il potenziale elettrico è così 
                definito per convenzione (concetto di "limite" per una 
                variabile che "tende" all'infinito) ma poiché si tratta di un 
                "lavoro" compiuto dal campo per spostare la carica da un punto 
                ad un altro, se il punto di arrivo non è all'infinito ma con 
                posizione nota allora il "potenziale" può essere espresso 
                rispetto ad esso (potenziale zero di riferimento): in sostanza 
                il potenziale è sempre riconducibile ad una "differenza" tra 
                due valori.\\
                L'energia potenziale elettrica della carica è il livello di 
                energia che la carica possiede a causa della sua posizione 
                all'interno del campo elettrico, e pertanto il potenziale 
                elettrico $V$ della carica di prova è definito operativamente 
                come il rapporto tra l'energia potenziale $U$ e il valore della
                carica stessa.
            \end{quote}

            \subsection{Energia Potenziale Elettrica} Data una particella di 
            carica $q$ situata in un punto dove il potenziale elettrico 
            prodotto da un corpo carico è $V$, l'energia potenziale elettrica
            $U$ del sistema corpo-particella è:
                \begin{equation}
                    U = qV
                \end{equation}
            Se la particella si sposta subendo una variazione di potenziale 
            $\Delta V$, la variazione di energia potenziale elettrica è:
                \begin{equation}
                    \Delta U = q \Delta V = q (V_f - V_i)
                \end{equation}
            
        \section{Energia Meccanica} Se una particella si sposta subendo una 
        variazione di potenziale $\Delta V$ senza che agiscano su di essa forze
        applicate esterne, il principio di conservazione dell'Energia Meccanica
        impone che la variazione di energia cinetica sia:
            \begin{equation}
                \Delta K = -q \Delta V
            \end{equation}
        Se al contrario è presente una forza applicata che compie lavoro 
        $W_{app}$ su di essa, la variazione di energia cinetica diventa:
            \begin{equation}
                \Delta K = -q\Delta V + W_{app}
            \end{equation}
        Nel caso particolare in cui $\Delta K = 0$, il lavoro svolto dalla 
        forza applicata comporta solo il moto della particella e una variazione
        del suo potenziale pari a:
            \begin{equation}
                W_{app} = q \Delta V
            \end{equation}

        \section{Superfici Equipotenziali} Una superficie equipotenziale è il 
        luogo dei punti che hanno lo stesso potenziale. Il lavoro compiuto per
        portare una carica di prova da una di queste superfici a un'altra non 
        dipende dalla posizione dei punti iniziale e finale su queste 
        superfici, ne dal cammino che li unisce. Il campo elettrico $E$ è 
        sempre \textit{perpendicolare} alle superfici equipotenziali.

            \subsection{Calcolo di $V$ a partire da $E$} La differenza di 
            potenziale tra due punti qualsiasi $i$ e $f$ è data da:
                \begin{equation}
                    V_f - V_i = - \int_{i}^{f} E \,ds
                \end{equation}
            dove l'integrale è calcolato lungo una linea qualsiasi che coniuga 
            i punti. Possiamo scegliere pertanto il percorso che ci renda 
            l'operazione d'integrale più semplice possibile. Se il punto 
            iniziale è posto all'infinito e $V_i = 0$ si ha, per il potenziale
            in un punto:
                \begin{equation}
                    V = -\int_{i}^{f} E \,ds
                \end{equation}
            Nel caso particolare di un campo uniforme di modulo $E$, la 
            differenza di potenziale tra le due linee equipotenziali adiacenti
            (necessariamente parallele) separate da una distanza $\Delta x$ è 
            data da:
                \begin{equation}
                    \Delta V = -E\Delta x
                \end{equation}

        \section{Potenziale generato da cariche puntiformi} Il potenziale 
        generato da una carica puntiforme isolata a una distanza $r$ dalla 
        carica puntiforme è:
            \begin{equation}
                V = \frac{1}{4\pi\varepsilon_0}\frac{q}{r}
            \end{equation}
        dove $q$ ha lo stesso segno di $V$. Il potenziale dovuto a una 
        distribuzione di cariche puntiformi è:
            \begin{equation}
                V = \sum_{i = 1}^{n}V_i = \frac{1}{4\pi\varepsilon_0}
                \sum_{i = 1}^{n}\frac{q_i}{r_i}
            \end{equation}

        \section{Potenziale elettrico generato da una carica continua} Per una 
        distrinuzione continua di cariche, l'equazione precedente diventa:
            \begin{equation}
                V = \frac{1}{4\pi\varepsilon_0}\int \frac{dq}{r}
            \end{equation}
        dove l'integrale è esteso all'intera distribuzione.

        \section{Calcolo di $E$ partendo da $V$} La componente di $E$ in 
        qualsiasi direzione è l'inverso della derivata del potenziale rispetto
        alla distanza in quella direzione:
            \begin{equation}
                E_s = - \frac{\partial V}{\partial s}
            \end{equation}
        Si possono definire le componenti di $E$ secondo $x$, $y$ e $z$ nel 
        seguente modo:
            \begin{align}
                E_x &= - \frac{\partial V}{\partial x}\\
                E_y &= - \frac{\partial V}{\partial y}\\
                E_z &= - \frac{\partial V}{\partial z}
            \end{align}
        Se il campo $E$ è uniforme, la prima equazione si riduce a:
            \begin{equation}
                E = - \frac{\Delta V}{\Delta s}
            \end{equation}
        dove $s$ è perpendicolare alla superfice equipotenziale. Il campo 
        elettrico è nullo in qualsiasi direzione parallela a una superficie 
        equipotenziale.

        \section{Energia Potenziale di un sistema di cariche puntiformi} 
        L'energia potenziale elettrica di un sistema di cariche puntiformi è
        definita come il lavoro necessario per costruire il sistema, partendo
        da una situazione in cui le cariche sono a riposo e infinitamente
        distanti l'una dall'altra. Per due cariche a distanza $r$ abbiamo:
            \begin{equation}
                U = L = \frac{1}{4\pi\varepsilon_0}\frac{q_1q_2}{r}
            \end{equation}