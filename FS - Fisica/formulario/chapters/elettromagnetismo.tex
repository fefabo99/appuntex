\chapter{Elettromagnetismo}

    \section{Forza di Lorentz} Un campo magnetico $B$ è definito in funzione 
    della forza agente su una particella di prova dotata di carica $q$ che si 
    muove in un campo magnetico con velocità $v$:
        \begin{equation}
            F_B = qv \times B = qvB\cos\Theta \; [T]
        \end{equation}

    \section{Carica in moto circolare uniforme} Una particella carica con massa
    $m$ e carica $|q|$ che si muove con velocità $v$ perpendicolare al campo
    uniforme $B$ percorre una circonferenza. Applicando la seconda legge di 
    Newton al moto circolare abbiamo:
        \begin{equation*}
            |q|vB = \frac{mv^2}{r}
        \end{equation*}
    da cui troviamo il raggio:
        \begin{equation*}
            r = \frac{mv}{|q|B}
        \end{equation*}
    La frequenza di rivoluzione $f$, la pulsazione $\omega$ e il periodo $T$
    sono dati da:
        \begin{equation}
            f = \frac{\omega}{2\pi} = \frac{1}{T} = \frac{|q|B}{2\pi m}
        \end{equation}

    \section{Forza agente su un filo percorso da corrente} Un filo rettilineo
    percorso dalla corrente $i$ in un campo magnetico uniforme subisce una 
    forza trasversale:
        \begin{equation}
            F_B = iL\times B
        \end{equation}
    La forza agente su un elemento infinitesimo $idL$ in un campo magnetico è:
        \begin{equation}
            dF_B = idL\times B
        \end{equation}

    \section{Campo Magnetico generato da una corrente elettrica} Il campo 
    magnetico dovuto a un conduttore diu corrente è descritto dalla legge di
    \textit{Biot-Savart}.

        \subsection{Legge di Biot-Savart} Il contributo $dB$ al campo dovuto a
        un elemento infinitesimo di corrente $i\,ds$ nel punto $P$, a una
        distanza $r$ dall'elemento di corrente, vale:
            \begin{equation}
                dB = \Bigg(\frac{\mu_o}{4\pi}\Bigg)\frac{ids\times r}{r^3}
            \end{equation}
        In questo caso r è il vetore diretto dall'elemento di corrente verso il
        punto in questione. La quantità chiamata $\mu_0$, chiamata costante di 
        permeabilità magnetica nel vuoto, ha un valore pari a $4\pi \cdot 
        10{^-7} = 1,26 \cdot 10^{-6} T \frac{m}{A}$.

        \subsection{Campo magnetico in un filo lungo rettilineo} Per un filo 
        lungo rettilinea percorso da una corrente $i$, la legge di Biot-Savart 
        dà, per il campo magnetico a una distanza $r$ dal filo:
            \begin{equation}
                B = \frac{\mu_0i}{2\pi r}
            \end{equation}

        \subsection{Campo magnetico in un filo piegato ad arco} Per un filo 
        piegato ad arco di circonferenza con raggio $R$ e angolo dal centro 
        $\phi$, percorso da una corrente $i$, il campo magnetico è:
            \begin{equation}
                B = \frac{\mu_0i\phi}{4\pi R}
            \end{equation}
        
        \subsection{Forza tra due fili conduttori paralleli} Fili paralleli 
        percorsi da correnti con lo stesso verso si attraggono, mentre fili 
        percorsi da correnti con versi opposti si respingono. L'intensità della
        forza per una lunghezza $L$ di un o l'altro dei fili è:
            \begin{equation}
                F_{ba} = i_bLB_a\sin90 = \frac{\mu_0Li_ai_b}{2\pi d}
            \end{equation}

    \section{Legge di Ampere} La legge di Ampere afferma:
        \begin{equation}
            \oint B \, ds = \mu_0i_{ch}
        \end{equation}
    L'integrale di linea in questa equazione viene calcolato lungo una linea
    chiusa detta \textit{linea amperiana}. La corrente $i$ è la corrente totale
    netta che circola entro la linea chiusa. Per distribuzioni di corrente 
    particolari questa equazione risulta, nel calcolo dei campi magnetici 
    generati da correnti, di più semplice impiego della legge di Biot-Savart.

    \section{Solenoide} All'interno di un lungo solenoide percorso dalla 
    corrente $i$, nei punti vicini al suo centro, l'intensità $B$ del campo 
    magnetico è data da:
        \begin{equation}
            B = \mu_0in
        \end{equation}
    dove $n$ è il numero delle spire per unità di lunghezza. Perciò il campo 
    magnetico interno è uniforme.