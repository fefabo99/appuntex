\chapter{Gravitazione}

Date due masse separate da una distanza $r$, l'ampiezza delle forze è data 
dalla seguente formula:
\begin{equation}
    F_g = G\frac{m_1m_2}{r^2}
\end{equation}
Dove $G$ è la costante di gravitazione universale:
\begin{equation*}
    G = 6,673^{-11} N\frac{m^2}{Kg^2}
\end{equation*}
    
    \paragraph{Energia Potenziale Gravitazionale} Cerchiamo l'Energia 
    Potenziale Gravitazionale generica determinata dalla legge di gravitazione 
    universale.
    \begin{align*}
        \Delta U_g &= U_f - U_i = -W_g \\
        &= - \int_{r_f}^{r_i} F(r) \,dr \\
        &= \int_{r_f}^{r_i} GMm \, \frac{1}{r^2} \,dr \\
        &= GMm \bigg[-\frac{1}{r}\bigg]^{r_f}_{r_i} \\
    \end{align*}
    \begin{equation}
        \Delta U_g = -GMm\biggl(\frac{1}{r_f} - \frac{1}{r}\biggr)
    \end{equation}
    Questa è la formula per l'energia potenziale gravitazionale del sistema
    Terra-Particella per $r \geq R_t$. Non vale per un raggio inferiore a
    quello terrestre.
    L'espressione può essere applicata a qualunque delle masse separate da una
    distanza $r$. L'Energia Potenziale è sempre negativa, perché abbiamo posto
    che sarà $= 0$ a distanza infinita.