\chapter{Gravitazione}

Date due masse separate da una distanza $r$, l'ampiezza delle forze è data 
dalla seguente formula:
\begin{equation}
    F_g = G\frac{m_1m_2}{r^2} \; 
    \Bigg[
        N \cdot  \frac{m^2}{kg^2} \cdot
        \frac{kg^2}{m^2} = N
    \Bigg]
\end{equation}
Dove $G$ è la costante di gravitazione universale:
\begin{equation*}
    G = 6,673 \cdot 10^{-11} \;\Bigg[N \cdot \frac{m^2}{kg^2}\Bigg]
\end{equation*}
    
    \paragraph{Energia Potenziale Gravitazionale} Cerchiamo l'Energia 
    Potenziale Gravitazionale generica determinata dalla legge di gravitazione 
    universale. Precedentemente (\ref{en_po_gr}) abbiamo considerato la 
    particella di massa $m$ vicino alla superficie terrestre, così da rendere 
    costante la forza di gravità. Per particelle che non si trovano sulla 
    superficie terrestre l'energia potenziale gravitazionale decresce col 
    diminuire della distanza tra la particella e la Terra. Qui consideriamo 
    due particelle separate da una distanza $R$. Per rendere $U = 0$ ci 
    mettiamo nella condizione di $ r = \infty$ in modo da semplificare i conti.
    \begin{align*}
        \Delta U_g &= U_f - U_i = -W_g \\
        &= - \int_{R}^{\infty} F(r) \,dr \\
        &= \int_{R}^{\infty} GMm \, \frac{1}{r^2} \,dr \\
        &= GMm \bigg[-\frac{1}{r}\bigg]^{\infty}_{R} \\
        &= GMm\biggl(\frac{1}{\infty} - \frac{1}{R}\biggr) \\
        &= GMm\biggl(0 - \frac{1}{R}\biggr)
    \end{align*}
    \begin{equation}
        \Delta U_g = \frac{-GMm}{R}
    \end{equation}
    Questa è la formula per l'energia potenziale gravitazionale del sistema
    Terra-Particella per $r \geq R_t$. Non vale per un raggio inferiore a
    quello terrestre.
    L'espressione può essere applicata a qualunque delle masse separate da una
    distanza $r$. L'Energia Potenziale è sempre negativa, perché abbiamo posto
    che sarà $= 0$ a distanza infinita.

    \paragraph{Velocità di Fuga} Un oggetto sfuggirà all'attrazione 
    gravitazionale di un corpo astronomico di massa $M$ e raggio $r$ se la sua 
    velocità in vicinanza della superficie del corpo sarà non inferiore alla 
    velocità di fuga data dalla seguente formula:
        \begin{equation}
            v = \sqrt[]{\frac{2GM}{R}}
        \end{equation}