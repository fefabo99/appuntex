\chapter{Fluidostatica}

    \section{Pressione} Le forze esercitate da un fluido su un'oggetto sono 
    sempre \textbf{perpendicolari} alla superficie. Se $F$ è il modulo della 
    forza esercitata e $A$ la superificie sulla quale essa agisce, definiamo
    \textit{pressione} $P$ il seguente rapporto:
        \begin{equation}
            p = \frac{F}{A} \; \Bigg[Pa \textsf{ Pascal} = \frac{Kg}{m \cdot 
            s^2} \Bigg]
        \end{equation}
    È uno scalare, proporizionale al modulo della forza.

    \section{Densità} La densità di una sostanza è uno scalare definito come 
    rapporto tra massa e volume:
        \begin{equation}
            \rho = \frac{m}{V} \; \Bigg[\frac{Kg}{m^3}\Bigg]
        \end{equation}
    Se $\rho$ è costante, il liquido preso in considerazione è incompressibile.

    \section{Legge di Stevino} La pressione di un fluido a riposo in un campo
    gravitazionale uniforme varia con la quota verticale $y$. Assegnando valori
    positivi all'orientamento verso l'alto, si ha:
        \begin{equation*}
            p_2 = p_1 + \rho g(y_1-y_2)
        \end{equation*}
    Chiamiamo $h$ la profondità di un campione in un fluido misurata a partire
    da un livello di riferimento a cui la pressione assume un valore $p_0$, la 
    precedente equazione diventa:
        \begin{equation}
            p = p_0 + \rho gh
        \end{equation}
    In cui $p$ è la pressione alla profondità $h$.
    La pressione è la stessa per tutti i punti allo stesso livello.

    \section{Forza di Bouyant} Un oggetto immerso in un fluido è soggetto ad 
    una forza esercitata dal liquido, diretta dal basso verso l'alto, tale 
    forza è detto di Bouyant o di galleggiamento.
    Il \textit{Principio di Archimede} ci indica l'intensità di questa forza:
    essa è uguale al peso del liquido spostato.
        \begin{align}
            F_{archimede} &= m_{liquido}g \\
            F_a &= (\rho V) g
        \end{align}

        \subsection{Oggetto totalmente sommerso} Se l'oggetto ha massa $M$, e
        densità $\rho_0$, il suo peso è:
            \begin{equation}
                F_g = M_g = (\rho_0 V_0) g
            \end{equation}
        Se la densità dell'oggetto è $<$ di quella del fluido, la forza 
        gravitazionale è inferiore a quella di galleggiamento, di conseguenza
        l'oggetto verrà accelerato verso l'alto. In caso contrario affonderà.

        \subsection{Oggetto galleggiante} Sia l'oggetto di volume $V_0$, in 
        equilibrio nel fluido e galleggiante, cioè parzialmente sommerso. Ciò
        significa che la forza di galleggiamento verso l'alto è 
        \textbf{bilanciata} dalla forza di gravità agente verso il basso.
            \begin{align}
                F_a = F_g :
                \begin{cases}
                    F_a &= (\rho_f V_f)g \\
                    F_g &= M_g = (\rho_0 V_0)g
                \end{cases}
            \end{align}
        $F_a = F_g$ perché l'oggetto è in equilibrio.

    \section{Peso appartente di un oggetto} Si consideri un oggetto con un 
    determinato peso (forza peso!). Ripetiamo l'operazione sott'acqua o più in 
    generale all'interno di un fluido: il peso sarà minore a causa della spinta
    di galleggiamento:
        \begin{equation}
            P_{app} = P - F_a
        \end{equation}
        \begin{quote}
            \textbf{Caso particolare!} Per un corpo galleggiante ricordiamo che
            la forza di galleggiamento $F_a$ è uguale alla forza peso $F_g$, di
            conseguenza il peso apparente dell'oggetto sara \textbf{nullo}!
        \end{quote}
            