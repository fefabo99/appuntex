\chapter{Dinamica}

    \section{Principi della Dinamica - Leggi di Newton}

        \subsection{Prima Legge di Newton} Se la somma delle forze che agiscono 
        su un corpo è nulla, allora il corpo in quiete rimarrà in quiete, 
        mentre se è in moto, continuerà a muoversi di moto rettilineo uniforme.

        \subsection{Seconda Legge di Newton} La forza agente su un corpo è 
        direttamente proporzionale all'accelerazione e ne condivide la 
        direzione e il verso, ed è direttamente proporzionale alla massa. Di 
        contro l'accelerazione cui è soggetto il corpo è direttamente 
        proporzionale alla forza e inversamente proporzionale rispetto alla 
        massa.

        \begin{equation}
            \vec{F}_{net} = m \vec{a} \; \Bigg[N = \frac{Kg \cdot m}
            {s^2} \Bigg]
        \end{equation}

        \subsection{Terza Legge di Newton} Se un corpo A esercita una forza su 
        un corpo B, allora il corpo B esercita su A una forza uguale e 
        contraria.

        \begin{equation}
            F_{ab} = - F_{ba}
        \end{equation}

        \begin{quote}
            Attenzione! Le forze hanno modulo uguale ma con segno vettoriale 
            opposto!
        \end{quote}

    \section{Forza Elastica} La forza elastica di un corpo (o di una molla) è 
    descritta dalla Legge di Hook nel seguente modo:
        \begin{equation}
            F = -kx
        \end{equation}
    Dove $-k$ è chiamata \textbf{costante elastica} ed è una misura della 
    rigidità della molla. Maggiore è $k$, più rigida è la molla: cioè maggiore 
    è $k$, maggiore sarà la forza per uno stesso valore di spostamento.

    \section{Carrucola} Le forze agenti su due corpi collegati in un sistema a
    carrucola (se aventi masse diverse) sono sempre una l'opposta dell'altra.
        \begin{equation}
            \begin{cases}
                F_{y1} = T - m_1g \\
                F_{y2} = m_2g - T
            \end{cases}
        \end{equation}
    In questo caso si considera $m_2 > m_1$ e con un sistema di riferimento 
    verticale. Si considera infatti un sistema a carrucola con forze agenti 
    solo sull'asse $y$ e con forze nulle sull'asse $x$. Nel caso in cui la 
    carrucola non sia orientata unicamente lungo l'asse $y$ basterà scomporre
    la forza lungo gli assi di riferimento!

    \section{Attrito Statico e Dinamico} La forza di attrito è una forza che 
    agisce in direzione opposta allo spostamento (opponendosi al movimento). La 
    forza di attrito può agire in due modi differenti:
        \begin{itemize}
            \item \textbf{Attrito statico}: agente quando il corpo è fermo, 
            impedendo lo spostamento iniziale.
            \item \textbf{Attrito dinamico}: agente da quando il corpo ha 
            appena compiuto lo spostamento iniziale ed è in movimento.
        \end{itemize}
    Le formule sono per l'attrito statico:
        \begin{equation}
            f_{s, max} = \mu_sF_N \;\Bigg[N = \frac{Kg \cdot m} {s^2} \Bigg]
        \end{equation}
        Mentre per quello dinamico:
        \begin{equation}
            f_{k} = \mu_kF_N \;\Bigg[N = \frac{Kg \cdot m} {s^2} \Bigg]
        \end{equation}
    
    \section{Resistenza di un corpo} Quando un corpo solido si muove 
    all'interno di un fludio, ad esso si oppone una forza contraria chiamata
    resistenza $D$ la quale farà raggiungere al corpo una velocità massima:
        \begin{equation}
            D = \frac{1}{2}CA\rho v^2
        \end{equation}
        Con:
        \begin{itemize}
            \item C : Coefficiente di resistenza aerodinamica.
            \item A : Area efficace della sezione trasversale del corpo.
            \item $\rho$ : densità dell'aria
            \item v : velocità.
        \end{itemize}

    \section{Lavoro} Si applichi una forza F ad un oggetto per spostarlo. La 
    Forza sarà tanto efficace ad ottenere uno spostamento \textbf{tanto più
    è applicata nella stessa direzione dello spostamento.}

        \begin{equation}
            W = Fd = Fd \cos \Theta \;\Bigg[J = N \cdot m = 
            \frac{Kg \cdot m^2}{s^2}\Bigg]
        \end{equation}

        \subsection{Lavoro compiuto dalla Forza Gravitazionale} Il Lavoro svolto 
        dalla Forza Gravitazionale ovviamente è descritto come $F d$, per un 
        corpo che sale la $F_g$ è diretta in senso opposto allo spostamento 
        formando un angolo $\Theta$ di $180^{\circ}$.
        
        \begin{equation}
            F = mgd\cos \Theta = mgd \cos 180 = - mgd
        \end{equation}
        Mentre nel momento in cui un corpo cade, la $F_g$ avrà stessa direzione
        dello spostamento (verso il basso), conferendo un segno positivo al
        Lavoro.
        
        \subsection{Lavoro compiuto dalla Forza Elastica} La Forza Elastica non 
        è una forza costante e di conseguenza non possiamo utilizzare la 
        classica equazione del Lavoro (per una forza costante). Possiamo però 
        suddividere lo spostamento della molla in parti infinitesime in modo da
        avere forze infinitesime per ogni spostamento infinitesimo, facendo 
        risultare cosi la forza infinitesima costante su uno spostamento 
        infinitesimo. Integrando questa operazione otterremo così la formula 
        del lavoro per la Forza Elastica 
        (e in generale per una forza non costante!).
        \begin{align*}
            W_{molla} &= \int_{x_i}^{x_f} F \,dx \\
            W &= \int_{x_i}^{x_f} -kx \,dx \\
            W &= -k \int_{x_i}^{x_f} x \,dx \\
            &= (-\frac{1}{2}k)\bigg[x^2 \bigg]_{x_f}^{x_i} \\
            &= (-\frac{1}{2}k) (x^2_f - x^2_i)
        \end{align*}
        \begin{equation}
            W_m = \frac{1}{2}kx^2_i - \frac{1}{2}kx^2_f \;[J]
        \end{equation}
        Il Lavoro $W_m$ è positivo quando il blocco si avvicina alla posizione 
        di riposo $x=0$ ed è negativo quando se ne allontana. Il Lavoro è nullo
        se la distanza finale da $x=0$ non è mutata.

    \section{Energia Cinetica} Rappresenta la quantità di energia associata al 
        moto di una particella (corpo puntiforme) che si muove alla velocità 
        $v$.
        \begin{equation}
            K = \frac{1}{2}mv^2 \;[J]
        \end{equation}

        \begin{quote}
            Il lavoro fatto su una particella è uguale a $\Delta K$. L'energia 
            cinetica (e la velocità) aumentano se il lavoro svolto è positivo, 
            mentre diminuiscono se il lavoro svolto è negativo.
        \end{quote}

        \subsection{Teorema dell'Energia Cinetica} Chiamiamo $\Delta K$ la 
        variazione di Energia Cinetica del corpo e $L$ il Lavoro totale 
        compiuto su di esso. Allora possiamo scrivere:
        \begin{equation}
            \Delta K = K_f - K_i = W
        \end{equation}

        \subsection{Energia Cinetica del Moto Armonico Semplice} Si consideri un
        sistema molla-blocco, nel caso senza attriti, possiamo visualizzare il 
        suo andamento come un'oscillazione armonica e descrivere la sua Energia
        Cinetica come:
        \begin{equation}
            K = \frac{1}{2}mv^2 = \frac{1}{2}m\omega^2A^2\sin^2
            (\Theta + \omega t)
        \end{equation}

    
    \section{Potenza} Se una forza esterna è applicata ad un oggetto e se il 
    Lavoro è fatto in un intervallo di tempo, definiamo \textbf{potenza}:
        
        \begin{equation}
            P = \frac{W}{\Delta t} \; \Bigg[W \textsf{  Watt} = \frac{J}{s} 
            = \frac{Kg \cdot m^2}{s^2} \cdot \frac{1}{s} 
            = \frac{Kg \cdot m^2}{s^3} \Bigg]
        \end{equation}

        \subsection{Un altro sguardo alla Potenza} La Potenza Istantanea può 
        essere espressa derivando la formula della Potenza (ovviamente!). Di 
        conseguenza possiamo scrivere la potenza come:

        \begin{equation*}
            P = \frac{dL}{dt} = \frac{F \cos \Theta dx}{dv} 
            = F \cos \Theta (\frac{dx}{dt})
        \end{equation*}
        Ma sappiamo benissimo che $\frac{dx}{dv}$ non è altro che la definzione 
        di velocità. Possiamo quindi riscriverla in modo più semplice:
        \begin{equation*}
            P = Fv\cos\Theta
        \end{equation*}
        Ovvero il \textbf{prodotto scalare} tra $F$ e $v$ (dove $v$ è la 
        velocità della particella). Possiamo quindi scrivere che la Potenza 
        Istantanea di una paricella a velocità $v$ non è altro che:
        \begin{equation}
            P = F \cdot v
        \end{equation}

    \section{Energia Potenziale} Se la configurazione di un sistema cambia, 
    allora cambierà anche la sua Energia potenziale. Quando un oggetto si trova
    ad una certa distanza dal suolo, il sistema terra-oggetto ha un'energia 
    potenziale che si trasforma in lavoro. \\
    L'Energia Potenziale è associata con la configurazione del sistema nel 
    quale le forze conservative agiscono. Quando una forza conservativa compie 
    lavoro $W$ su una particella (corpo) del sistema, il cambiamento $\Delta U$ 
    dell'energia potenziale del sistema è definito come:
        
        \begin{equation}
            \Delta U = - W \;[J]
        \end{equation}

        \subsection{Energia Potenziale Gravitazionale}
        \label{en_po_gr}
        L'energia potenziale in 
        un sistema composto dalla terra e dalla particella (corpo) è chiamata 
        Energia Potenziale Gravitazionale. Se la particella si muove da 
        un'altezza iniziale $y_i$ ad una finale $y_f$, il cambiamento 
        dell'Energia Potenziale Gravitazionale è definito come:
        \begin{equation}
            \Delta U = mg(y_f - y_i) = mg \Delta y
        \end{equation}
        Se considerassimo come punto di arrivo un altezza $ h = 0$, allora 
        l'Energia Potenziale gravitazionale piò essere riscritta come:
        \begin{equation}
            \Delta U = mgh
        \end{equation}
        Dove $h$ è l'altezza dalla quale il corpo cade.

        \subsection{Energia Potenziale Elastica} L'energia Potenziale Elastica è 
        l'energia associata allo stato di compressione o estensione di un 
        oggetto elastico (molla). Per una molla con una forza definita come 
        $ F = -kx $, l'Energia Potenziale Elastica sarà definita come:
        \begin{equation}
            U(x)     = \frac{1}{2}kx^2
        \end{equation}

        \subsection{Energia Potenziale del Moto Armonico Semplice} L'energia 
        Potenziale Elastica di un oscillatore armonico, immagazzinata dalla 
        molla a seguito di un allungamento $x$ è:
        \begin{equation}
            U = \frac{1}{2}kx^2 = \frac{1}{2}kA^2\cos^2(\Theta + \omega t)
        \end{equation}

    \section{Energia Meccanica} La somma dell'Energia Cinetica e dell'Energia
    Potenziale è detta Energia Meccanica, definita come:

        \begin{equation}
            E_m = K + U
        \end{equation}

        \subsection{Principio di conservazione dell'Energia Meccanica} Quando in 
        un sistema isolato agiscono solo forze conservative, l'Energia Cinetica 
        e l'Energia Potenziale prese singolarmente possono variare, ma la loro
        somma, l'Energia Meccanica $E_m$ del sistema non cambia. Questo 
        risultato è chiamato principio di conservazione dell'Energia Meccanica
        esprimibile nel seguente modo:
        \begin{equation}
            \Delta E_m = \Delta K + \Delta U = 0
        \end{equation}
        Il principio di conservazione dell'Energia Meccanica ci permette di 
        risolvere problemi che sarebbe arduo risolvere usando solo le Leggi di 
        Newton.
        Quando l'Energia Meccanica di un sistema si conserva, possiamo mettere
        in relazione il totale dell'Energia Cinetica e dell'Energia Potenziale
        in un istante con quello di un altro istante, \textit{senza dover
        considerare gli stati intermedi e senza necessità di conoscere il 
        lavoro compiuto dalle forze coinvolte!}

        \paragraph{Principio esteso} La variazione dell'Energia Meccanica è 
        uguale al lavoro svolto dalle Forze non conservative:
        \begin{equation}
            \Delta E_m = W_{nc}
        \end{equation}
        

    \section{Moto Armonico e Pendolo} L'oscillatore armonico può essere 
    rappresentato da un sistema molla-blocco il quale oscillando descrive un 
    moto circolare uniforme. Le equazioni del moto sono state descritte in 
    precedenza (\ref{moto_armonico}) durante il moto circolare uniforme. 
        \begin{equation}
            x(t) = A \cos(\Theta_0 + \omega t)
        \end{equation}
        ma con:
        \begin{equation*}
            \omega = \sqrt[]{\frac{k}{m}}
        \end{equation*}
        ricordando l'equazione dei un sistema molla-blocco:
        \begin{equation*}
            F = -kx
        \end{equation*}
        
        \paragraph{Il pendolo} Il pendolo può essere descritto tramite il
        Moto Circolare Uniformemente Accelerato, considerando la lunghezza del 
        filo inestensibile come il raggio della circonferenza. In questo caso 
        abbiamo:
        \begin{equation*}
            \omega = \sqrt[]{\frac{g}{L}}
        \end{equation*}
    
\newpage
    
    \section{Momento Lineare}
    Per una singola particella definiamo una quantità vettoriale $p$ chiamata
    \textbf{momento lineare} o \textit{quantità di moto}: 

        \begin{equation}
            p = mv \; \Bigg[Kg \cdot \frac{m}{s}\Bigg] 
        \end{equation}

        \paragraph{Connessione con la II Legge di Newton} Deriviamo il 
        Momento Lineare rispetto al tempo. La derivata della quantità di moto 
        di un punto materiale di massa $m$ è uguale alla risultante della forza 
        applicata.

        \begin{equation*}
            \frac{dp}{dt} = \frac{d(mv)}{dt} = ma
        \end{equation*}

        \begin{equation*}
            \frac{dm}{dt} v + m \frac{dv}{dt} = ma
        \end{equation*}

        \begin{quote}
            Ma la massa rimane costante nel tempo, quindi la derivata sarà 0.
        \end{quote}

        \begin{equation*}
            m\frac{dv}{dt} = ma = \sum F
        \end{equation*}

        \begin{equation*}
            \sum F = 0 \implies p = \textsf{cost}\
        \end{equation*}

        \paragraph{Legge di conservazione del momento lineare} Quando due o più
        particelle di un sistema isolato interagiscono, il momento lineare 
        totale del sistema resta \textbf{costante}.


    \section{Impulso}
        Applicando al \textit{momento lineare} la Seconda Legge di Newton ad un
        corpo puntiforme che subisce un urto, si ricava il \textbf{Teorema
        dell'Impulso}:
        \begin{equation}
            I = \Delta p
        \end{equation}

        \paragraph{Significato} L'impulso della forza che agisce su una 
        particella è aguale al $\Delta$ del momento lineare della particella 
        determinato dalla forza. L'impulso non è una caratteristica della 
        particella, bensì una misura della modifica del momento lineare da 
        parte di una forza esterna.
        Se $F$ è l'intensità di una forza e $\Delta t$ la durata della 
        collisione, allora l'Impulso può essere descritto nel seguente modo:
        \begin{equation}
            I = F \Delta t
        \end{equation}

        \paragraph{Connessione con il Momento Lineare} Sia una forza $F = F(t)$
        agente su una particella. Applicando la II Legge di Newton:

        \begin{equation*}
            F = \frac{dp}{dt} \implies dp = F dt
        \end{equation*}
        \begin{equation*}
            \Delta p = p_f - p_i = \int_{t_f}^{t_i} F \,dt = I 
        \end{equation*}
    
    \section{Urti} Gli urti accadono frequentemente nella vita quotidiana e 
    possono essere caratterizzati in due differenti tipi:
    \begin{itemize}
        \item \textbf{Urti elastici}: Se durante l'urto tra due corpi l'Energia 
        Cinetica totale del sistema non cambia ma si conserva completamente.
        \item \textbf{Urti anaelastici}: L'Energia Cinetica non si conserva ma 
        parte viene dispersa in calore o suono (ad esempio).
    \end{itemize} 

        \subsection{Urti Anaelastici} La collisione anaelastica comporta sempre
        una perdita di Energia Cinetica del sistema. La massima perdita si ha 
        quando i corpi si incollano insieme, in questo caso l'urto prenderà il 
        nome di \textbf{urto completamente anaelastico}.
        Un urto anaelastico può essere descritto tramite la seguente formula:
        \begin{equation}
            p_{1,i} + p_{2,i} = p_{1,f} + p_{2,f}
        \end{equation}
        ovvero:
        \begin{equation*}
            m_1v_{1,i} + m_2v_{2,i} = m_1v_{1,f} + m_2v_{2,f}
        \end{equation*}
        Nel caso di un \textbf{urto completamente anaelastico} uno dei due 
        corpi sarà inizalmente fermo (prendendo il nome di bersaglio). Dopo la 
        collisione proseguiranno attaccati con una velocità $V$. Definiamo 
        quindi l'equazione di un urto di questa tipologia:
        \begin{equation}
            m_1v_{1,i} = (m_1+m_2)V
        \end{equation} 
        \begin{quote}
            Abbiamo analizzato gli urti in una singola dimensione. In caso di 
            urti in due dimensioni le considerazioni appena fatte non cambiano.
            L'unica cosa da aggiungere è la scomposizione lungo gli assi della
            velocità!
        \end{quote}
        

