\documentclass[12pt, a4paper, openany]{book}
\usepackage[inline]{enumitem}
\usepackage{style}


\begin{document}

\title{Fisica}
\author{Fabio Ferrario}
\date{2022/2023}
\maketitle


\chapter{Lezioni}
\section*{Lezione 2 05-10-22}
\paragraph*{Recap} Avevamo visto i grafici di velocità e accelerazione, e avevamo visto come ricavare acc dalla velocità.
Si era visto il moto uniformemente accelerato con la velocità che cambia linearmente nel tempo.

\paragraph*{Moto vario 1D} Se la velocità varia linearmente, allora l'accelerazione è \emph{costante}.
\paragraph*{Area sotto una curva} Significato fisico dell'area sotto una curva è l'integrale della curva f(x), l'area è definita positiva se è sopra l'asse delle x e negativa se è sotto l'asse delle x.
\paragraph*{Applicazione della proprietà degli integrali} a velocità e accelerazione, lo spazio percorso dal corpo è calcolato con l'integrale dello spostamento/tempo.
L'integrale sotto la curva accelerazione/tempo corrisponde alla velocità del moto
\subsection*{La caduta di un grave} L'acelerazione di gravità è $\pm$ costante ed uguale a $g\approx 9,81m/s^2$. Per un oggetto che cade la forza $g$ va applicata in negativo (va verso il basso).
Date le formule, esplicitando rispetto a t ed indicando con h l'altezza si ha: $t_c=\sqrt{\frac{2h}{g}}$. NB: La caduta non dipende dalla massa del grave

\paragraph*{le equazioni cinematiche} $a:x = \frac{dv_x}{dt} \to dv_x = a_xdt$ (con d = differenziale).
\\per trovare $v_x$ bisogna fare l'integrale (antiderivata) $to v_x = \int a_xdt+c$.
\\Per risolverlo: Se $a_x = costante \to v_x = a_x \int dt + C = a_xt+C$ (C è la costante di integrazione determinata dalle condizioni di contorno, cioè la velocità iniziale)
\\$v_x = dx/dt \to dx=v_xdt$ con l'integrazione $x = \int v_xdt + C$, anche qui se la velocità è costante l'integrale è banale, altrimenti bisogna risolvere con $v_x(t) = v_i + a_x \cdot t$.

\subsection*{Cinematica del punto materiale in 2D}
Fin'ora abbiamo considerato solo il moto in una dimensione, adesso considereremo quella in 2 dimensioni sia uniformemente accelerato che a velocità costante.
Quando studio il movimento 2D o studio il suo movimento in un piano o studio il cambiamento delle coordinate nel tmepo.
Le stesse formule monodimensionali valgono anche per il moto bidimensionale, però bisogna usare i vettori.
\\Quindi per semplice estensione del caso 1d, la velocità media sarà: $v^- = \frac{\Delta r}{\Delta t}$, poichè t è uno scalare, $v_m$ ha la stessa direzione e verso di $\Delta r$.
\\La velocità istantanea sarà $v = lim_{\Delta \to 0 } \frac{\Delta r}{\Delta t} = \frac{dr}{dt}$ (NB: r è il vettore posizione)
\\In 2D la velocità è la tangente alla traiettoria, che anche se è a modulo costante cambia in direzione punto per punto.
\\Anche se la velocità è costante di modulo, quando cambia di direzione si verifica una \emph{accelerazione}.
$a^- = \frac{v_f-v_i}{t_f-t_i}= \frac{\Delta v}{\Delta t}$ è l'acelerazione media è ha la stessa direazione del vettore $\Delta v$.
\\Quando l'acelerazione varia nel tempo, è utile definire l'acelerazione istantanea $a = ... = \frac{d^2 \overrightarrow{r} }{dt^2}$
\paragraph*{definizione acelerazione 2D} L'acelerazione di una particella in moto in uno spazio 2D può dunque corrispondere:
\begin{itemize}
    \item ad una variazione del modulo di \textbf{v}
    \item alla variazione di direzione a modulo di \textbf{v} costante
    \item Entrambe le cose combinate
\end{itemize}
\paragraph*{Moto uniformemente acelerato in 2D} %bisogna capire cos'è il raggio vettore %Segnarsi le formule da imparare a memoria
Per passare da 1D a 2D è semplice: si scompone il moto 2d in 2 moti 1d sugli assi x e y
(RIASSUNTO A SLIDE 16)

\paragraph*{Esempio} L'equazione vettoriale del moto di una particella è $r = (2\alpha t^2)i +\beta(2t + t_0)j + 4\delta k$, con a.b.d e t0 costanti. Trovare le componenti cartesiane della velocità e del suo modulo.
Quindi: abbiamo il raggio vettore del moto di una particella, trovare le componenti cartesiane.
per trovare la velocità bisogna trovare la derivata delle tre copmonenti (ijk) quindi:
$$ \begin{cases}
    x^1 = 4\alpha t \\ y^1 = 2\beta \\ z^1 = 0 
\end{cases}
\to 
\begin{cases}
   v= 4(\alpha t )
\end{cases}
$$ finire con slide 17

\subsection*{Il moto di un proiettile}
Il moto di un proiettile è "facile" determinarlo sotto due condizioni:
\begin{itemize}
    \item L'acelerazione di gravita è costante lungo tutto il percorso del proiettile
    \item La resistenza dell'aria è considerata trascurabile
\end{itemize}
Sotto queste condizioni il moto ha un'accelerazione costante ed è di tipo parabolico come già visto nel caso unidimensionale.
\\Se si lancia un proiettile con vettore velocità iniziale $v_i$ che forma un angolo $\theta_i$ con l'asse delle x:
\\si ha che $a_x = 0$, $a_y = -g$, e $v_{xi} = v_i \cos \theta_i$, $v_{yi} = sin \theta_i$. $x_i$ e $y_i$ sono uguale a 0 (origine).
\\IMMAGINE slide 19 
\\RIASSUMENDO : l'unica forza che agisce sul proiettile è quella di gravità, ed agisce come acelerazione nella direzione verticale.
Il moto di un proiettile può essere considerato come l asovrapposizione di due moti indipendenti: una a velictà costante lungo x e uno a caduta libera lungo la direzione verticale.

\paragraph*{eserizio da reisnick}
Una palla è lanciata con velocità iniziale $v_x = 20 m/s$ e $v_y=40m/s$. Quanto tempo rimarrà in aria e quale sarà la distanza percorsa?
\begin{enumerate}
    \item I moti sono indipendenti, quindi mi occupo delle due componenti x,y separatamente
    \item su $v_x$ la velocità rimane costante ma non su $v_y$
    \item SLIDE 26 
\end{enumerate}

\section*{Lezione 3 06-10-22}
\paragraph*{Secondo esercizio} Spariamo a un bersglio di distanza 30m. il proiettile colpisce il bersaglio 1.9cm sotto il centro. determinare il itempo di volo del proiettile e la velocità alla bocca del fucile

Equazioni del moto parabolico : $x_f = v_{xi}t = (v_i cos \theta_i) t | y_f = ...$ Slide 29.

\subsection{il moto circolare uniforme} è il moto di un punto materiale lungo una circonferenza, con un modulo della veloctià costante.
altra definizione: \emph{Dato un punto P sulla circonferenza, questo punto percorrerà spazi uguali in eguali intervalli di tempo.}
\\Siccome c'è una variazione della direzione del vettore velocità $\to$ c'è in gioco una acelerazione. (infatti è definita acelerazione la variazione nel tempo del vettore velocità).
\begin{itemize}
    \item Consideriamo una circonferenza di raggio r
    \item il punto materiale si muove a elocità di moulo costante lungo la traiettoria circolare
    \item il vettore velocità è tangente alla traiettorie circolare
    \item la variazione del vettore velocità $\Delta v$ nel limite di $\Delta t$ sempre più piccoli ha una direzione sempre più rivolta al centro della circonferenza.
\end{itemize}
\paragraph*{L'acelerazione istantanea} è diretta al centro della circonferenza, e per questo motivo si chiama acelerazione centripeta. il suo modulo è: $a_r = \frac{v^2}{r}$ (il pedice r indice che è diretta lungo il raggio r) con r raggio del cerchio.
\\Si chiama velocità tangenziale la velocità diretta lungo la circonferenza.
\\DEFINIZIONE: In un moto circolare uniforme l'accelerazione è diretta verso il centro del cerchio ed ha un modulo pari a $v^2/4$ dove v è il modulo della velocità tangenziale (costante) e r è il raggio del cerchio
\paragraph*{L'equazione angolare oraria} supponiamo che la traiettoria del punto materiale sia di raggio R, centro 0 e giaccia nel piano xy.
Dal punto di vista angolare, l'equazione oraria del moto ha una forma del tipo $\theta = \theta(t)$ che esprime il fatto che il punto materiale <<spazza>> nel taempo l'angolo $\theta$.
\\Esprimiamo la velocità angolare media come di consueto, cioè in questo caso come la variazione dell'angolo $\theta = \theta(t)$ nel tempo.
$$\omega = \frac{\Delta \theta}{\Delta t} = \frac{\theta_f - \theta_i}{t_f - t_i}$$
Se consideriamo il limite per intervalli piccoli del tempo tendenti a zero otteniamo la derivata rispoetto al tempo e la definizione di velictà angolare istantanea. SLIDE 34.
V angolare media e istantanea coincideranno perchè è moto circolare uniforme.
\\possimao ricavarci le equazioni del moto per l'integrazione dell'equazione differenziale. Queste equazioni sono analoghe a quelle del moto rettilineo
\paragraph*{Frequenza e periodo} \begin{itemize}
    \item Il tempo impiegato dal punto a percorre l'intera circonferenza si chiama periodo (T)
    \item il numero di periodi nell'unità di tempo (al secondo) è detto frequenza
\end{itemize}
$\nu = \frac{1}{T}$ la frequenza è indicata con il carattere greco nu $\nu$ (oppure con f)
\begin{itemize}
    \item La velocità tangenziale è lo spazio percorso dal punto nell'intervallo di tempo . $v= \frac{2\pi r}{T}$ ($2\pi r$ è la circonferenza)
\end{itemize}
la velocità angolare è lo spazio angolare percorso dal punto materiale nel tempo $ \omega_0 = 2 \pi \nu = \frac{2 \pi }{T}$ [rad $\cdot t^{-1}$]
\\nel moto circ uni $\omega_0$ è anche detta pulsazione 
\\relazione della velocità tangenziale v con la velocità angolare omega = $v = \omega \cdot r$. quinid la velocità aumenta con il raggio 

\subsection*{Moto circolare generico} è il moto di un punto materiale lungo una circonferenza con un modulo della velocità che cambia nel tempo (per esempio una ruota che acelera e rallenta).
In questo caso l'accelerazione tangenziale è quella che genera la variazione di velocità (Riassunto SLIDE 39).

\paragraph*{Es 4.8 da resnick sul pendolo}
accelerazione tangenziale è dato da a sin(phi), a lungo r è a per il coseno di phi. a = ar + at. (SLIDE 41)

\paragraph*{Sistema di coorinate polari} (questa parte non verrà chiesta ma la fa per noi, perchè se ci danno un programma in cui ci chideono le coordinate polari dobbiamo saperle.)
SLIDE 45 c'è il raggio vettore

\paragraph*{Equazione generica del moto cirolare uniforme in forma parametrica} Supponiamo che la traiettoria del punto materiale sia di raggio R, centro 0 e giaccia nel piano xy e viaggi a velocità angolare $\omega$ costante.
Dla punto di vista angolare, l'equazione oraria ha una forma del tipo :Le sue equazioni parametriche, intermini dell'angolo $\theta$ (detto anomalia) che il vettore posizione forma con l'asse x sono:
$$\begin{cases}
    x(t) = R \cos \theta(t)\\
    y(t) = R \sin \theta (t)
\end{cases}
$$
cioè, le coordinate x e y sono trovabili con il raggio e l'angolo (theta) nel momento in cui si sta cercando la posizione..
\\Ricordando che l'equazione oraria è $\theta = \theta_0 + \omega t$ sostituiamo e otteniamo $x(t) = R \cos (\theta_0 + \omega t)$ e $y(t)=...$.
quindi $r(s) = R\cos\theta(t) i$ è $R\sin\theta(t) j$

\subsection*{Il moto armonico}
SLIDE 49 (nota che nelle formule $\omega$ salta fuori perchè si deriva una funzione composta $\omega t \to \omega$)
Il moto armonico si chiama così perchè dopo un certo tempo assume lo stesso valore di partenza

\section*{Lezione 4 10-10-22}
\paragraph*{Recap moto armonico} %si comincia con un video di flippin physics che fa ridere
Il moto armoinico è un moto che dopo un certo tempo ritorna al punto di partenza. 
La proiezione del moto circolare uniforme (sia su asse x che su y) sono dei moti 1D che si ritrovano in tanti fenomeni fisici.
Ricordo le equazioni parametriche dell'MCU $x(t) = A \cos(\theta_0 + \omega t)$ e y con il seno, con $\omega$ velocità angolare.
La velocità è la derivata dello spazio rispetto al tempo, e l'accelerazione è la derivata della velocità rispetto al tempo, da qui si ottengono le formule (SLIDE 3).
\\Si noti che la velocità massima (solo sull'asse y) di un moto armonico la si ha verso il centro, mentre l'accelerazione è massima agli estremi.
Quindi agli estremi accelerazione masssima e velocità 0, al centro velocità massima e accelerazione 0.
Se si fa una proiezione di un punto su un solo asse si ottiene un grafico sinusoidale.
\subparagraph{velocità}$v_x(t)?-A \omega \sin (\theta_0+\omega t)$ anche il grafico della velocità è periodico (periodo 2pi).
I valori della velocità sono sfasati rispoetto al grafico della posizione.
anche l'accelerazione è periodica e sfasata rispetto alla velocità. 
%FINE RIASSUNTO
\subsection*{Dinamica. Sistemi di riferimento}%Nuovo argomento, non più cinematica
Abbiamo visto l'importanza della scelta del sistema di riferimento, le equazioni del moto variano al variare dei sistemi scelti.
\emph{Come si passa da un sistema di riferimento a un altro?} 
\end{document}