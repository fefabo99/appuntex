\documentclass[12pt, a4paper, openany]{book}
\usepackage[inline]{enumitem}
\usepackage{../generalStyle}


\begin{document}

\title{Fisica}
\author{Fabio Ferrario}
\date{2022/2023}
\maketitle


\chapter{Lezioni}
\section*{Lezione 2 05-10-22}
\paragraph*{Recap} Avevamo visto i grafici di velocità e accelerazione, e avevamo visto come ricavare acc dalla velocità.
Si era visto il moto uniformemente accelerato con la velocità che cambia linearmente nel tempo.

\paragraph*{Moto vario 1D} Se la velocità varia linearmente, allora l'accelerazione è \emph{costante}.
\paragraph*{Area sotto una curva} Significato fisico dell'area sotto una curva è l'integrale della curva f(x), l'area è definita positiva se è sopra l'asse delle x e negativa se è sotto l'asse delle x.
\paragraph*{Applicazione della proprietà degli integrali} a velocità e accelerazione, lo spazio percorso dal corpo è calcolato con l'integrale dello spostamento/tempo.
L'integrale sotto la curva accelerazione/tempo corrisponde alla velocità del moto
\subsection*{La caduta di un grave} L'acelerazione di gravità è $\pm$ costante ed uguale a $g\approx 9,81m/s^2$. Per un oggetto che cade la forza $g$ va applicata in negativo (va verso il basso).
Date le formule, esplicitando rispetto a t ed indicando con h l'altezza si ha: $t_c=\sqrt{\frac{2h}{g}}$. NB: La caduta non dipende dalla massa del grave

\paragraph*{le equazioni cinematiche} $a:x = \frac{dv_x}{dt} \to dv_x = a_xdt$ (con d = differenziale).
\\per trovare $v_x$ bisogna fare l'integrale (antiderivata) $to v_x = \int a_xdt+c$.
\\Per risolverlo: Se $a_x = costante \to v_x = a_x \int dt + C = a_xt+C$ (C è la costante di integrazione determinata dalle condizioni di contorno, cioè la velocità iniziale)
\\$v_x = dx/dt \to dx=v_xdt$ con l'integrazione $x = \int v_xdt + C$, anche qui se la velocità è costante l'integrale è banale, altrimenti bisogna risolvere con $v_x(t) = v_i + a_x \cdot t$.

    \subsection*{Cinematica del punto materiale in 2D}
    Fin'ora abbiamo considerato solo il moto in una dimensione, adesso considereremo quella in 2 dimensioni sia uniformemente accelerato che a velocità costante.
    Quando studio il movimento 2D o studio il suo movimento in un piano o studio il cambiamento delle coordinate nel tmepo.
    Le stesse formule monodimensionali valgono anche per il moto bidimensionale, però bisogna usare i vettori.
    \\Quindi per semplice estensione del caso 1d, la velocità media sarà: $v^- = \frac{\Delta r}{\Delta t}$, poichè t è uno scalare, $v_m$ ha la stessa direzione e verso di $\Delta r$.
    \\La velocità istantanea sarà $v = lim_{\Delta \to 0 } \frac{\Delta r}{\Delta t} = \frac{dr}{dt}$ (NB: r è il vettore posizione)
    \\In 2D la velocità è la tangente alla traiettoria, che anche se è a modulo costante cambia in direzione punto per punto.
    \\Anche se la velocità è costante di modulo, quando cambia di direzione si verifica una \emph{accelerazione}.
$a^- = \frac{v_f-v_i}{t_f-t_i}= \frac{\Delta v}{\Delta t}$ è l'acelerazione media è ha la stessa direazione del vettore $\Delta v$.
    \\Quando l'acelerazione varia nel tempo, è utile definire l'acelerazione istantanea $a = ... = \frac{d^2 \overrightarrow{r} }{dt^2}$
    \paragraph*{definizione acelerazione 2D} L'acelerazione di una particella in moto in uno spazio 2D può dunque corrispondere:
    \begin{itemize}
        \item ad una variazione del modulo di \textbf{v}
        \item alla variazione di direzione a modulo di \textbf{v} costante
        \item Entrambe le cose combinate
    \end{itemize}
    \paragraph*{Moto uniformemente acelerato in 2D} %bisogna capire cos'è il raggio vettore %Segnarsi le formule da imparare a memoria
    Per passare da 1D a 2D è semplice: si scompone il moto 2d in 2 moti 1d sugli assi x e y
    (RIASSUNTO A SLIDE 16)

    \paragraph*{Esempio} L'equazione vettoriale del moto di una particella è $r = (2\alpha t^2)i +\beta(2t + t_0)j + 4\delta k$, con a.b.d e t0 costanti. Trovare le componenti cartesiane della velocità e del suo modulo.
Quindi: abbiamo il raggio vettore del moto di una particella, trovare le componenti cartesiane.
per trovare la velocità bisogna trovare la derivata delle tre copmonenti (ijk) quindi:
$$ \begin{cases}
        x^1 = 4\alpha t \\ y^1 = 2\beta \\ z^1 = 0
    \end{cases}
    \to
    \begin{cases}
        v= 4(\alpha t )
    \end{cases}
$$ finire con slide 17

\subsection*{Il moto di un proiettile}
Il moto di un proiettile è "facile" determinarlo sotto due condizioni:
\begin{itemize}
    \item L'acelerazione di gravita è costante lungo tutto il percorso del proiettile
    \item La resistenza dell'aria è considerata trascurabile
\end{itemize}
Sotto queste condizioni il moto ha un'accelerazione costante ed è di tipo parabolico come già visto nel caso unidimensionale.
\\Se si lancia un proiettile con vettore velocità iniziale $v_i$ che forma un angolo $\theta_i$ con l'asse delle x:
\\si ha che $a_x = 0$, $a_y = -g$, e $v_{xi} = v_i \cos \theta_i$, $v_{yi} = sin \theta_i$. $x_i$ e $y_i$ sono uguale a 0 (origine).
\\IMMAGINE slide 19
\\RIASSUMENDO : l'unica forza che agisce sul proiettile è quella di gravità, ed agisce come acelerazione nella direzione verticale.
Il moto di un proiettile può essere considerato come l asovrapposizione di due moti indipendenti: una a velictà costante lungo x e uno a caduta libera lungo la direzione verticale.

\paragraph*{eserizio da reisnick}
Una palla è lanciata con velocità iniziale $v_x = 20 m/s$ e $v_y=40m/s$. Quanto tempo rimarrà in aria e quale sarà la distanza percorsa?
\begin{enumerate}
    \item I moti sono indipendenti, quindi mi occupo delle due componenti x,y separatamente
    \item su $v_x$ la velocità rimane costante ma non su $v_y$
    \item SLIDE 26
\end{enumerate}

\section*{Lezione 3 06-10-22}
\paragraph*{Secondo esercizio} Spariamo a un bersglio di distanza 30m. il proiettile colpisce il bersaglio 1.9cm sotto il centro. determinare il itempo di volo del proiettile e la velocità alla bocca del fucile

Equazioni del moto parabolico : $x_f = v_{xi}t = (v_i cos \theta_i) t | y_f = ...$ Slide 29.

\subsection{il moto circolare uniforme} è il moto di un punto materiale lungo una circonferenza, con un modulo della veloctià costante.
altra definizione: \emph{Dato un punto P sulla circonferenza, questo punto percorrerà spazi uguali in eguali intervalli di tempo.}
\\Siccome c'è una variazione della direzione del vettore velocità $\to$ c'è in gioco una acelerazione. (infatti è definita acelerazione la variazione nel tempo del vettore velocità).
\begin{itemize}
    \item Consideriamo una circonferenza di raggio r
    \item il punto materiale si muove a elocità di moulo costante lungo la traiettoria circolare
    \item il vettore velocità è tangente alla traiettorie circolare
    \item la variazione del vettore velocità $\Delta v$ nel limite di $\Delta t$ sempre più piccoli ha una direzione sempre più rivolta al centro della circonferenza.
\end{itemize}
\paragraph*{L'acelerazione istantanea} è diretta al centro della circonferenza, e per questo motivo si chiama acelerazione centripeta. il suo modulo è: $a_r = \frac{v^2}{r}$ (il pedice r indice che è diretta lungo il raggio r) con r raggio del cerchio.
\\Si chiama velocità tangenziale la velocità diretta lungo la circonferenza.
\\DEFINIZIONE: In un moto circolare uniforme l'accelerazione è diretta verso il centro del cerchio ed ha un modulo pari a $v^2/4$ dove v è il modulo della velocità tangenziale (costante) e r è il raggio del cerchio
\paragraph*{L'equazione angolare oraria} supponiamo che la traiettoria del punto materiale sia di raggio R, centro 0 e giaccia nel piano xy.
Dal punto di vista angolare, l'equazione oraria del moto ha una forma del tipo $\theta = \theta(t)$ che esprime il fatto che il punto materiale <<spazza>> nel taempo l'angolo $\theta$.
\\Esprimiamo la velocità angolare media come di consueto, cioè in questo caso come la variazione dell'angolo $\theta = \theta(t)$ nel tempo.
$$\omega = \frac{\Delta \theta}{\Delta t} = \frac{\theta_f - \theta_i}{t_f - t_i}$$
Se consideriamo il limite per intervalli piccoli del tempo tendenti a zero otteniamo la derivata rispoetto al tempo e la definizione di velictà angolare istantanea. SLIDE 34.
V angolare media e istantanea coincideranno perchè è moto circolare uniforme.
\\possimao ricavarci le equazioni del moto per l'integrazione dell'equazione differenziale. Queste equazioni sono analoghe a quelle del moto rettilineo
\paragraph*{Frequenza e periodo} \begin{itemize}
    \item Il tempo impiegato dal punto a percorre l'intera circonferenza si chiama periodo (T)
    \item il numero di periodi nell'unità di tempo (al secondo) è detto frequenza
\end{itemize}
$\nu = \frac{1}{T}$ la frequenza è indicata con il carattere greco nu $\nu$ (oppure con f)
\begin{itemize}
    \item La velocità tangenziale è lo spazio percorso dal punto nell'intervallo di tempo . $v= \frac{2\pi r}{T}$ ($2\pi r$ è la circonferenza)
\end{itemize}
la velocità angolare è lo spazio angolare percorso dal punto materiale nel tempo $ \omega_0 = 2 \pi \nu = \frac{2 \pi }{T}$ [rad $\cdot t^{-1}$]
\\nel moto circ uni $\omega_0$ è anche detta pulsazione
\\relazione della velocità tangenziale v con la velocità angolare omega = $v = \omega \cdot r$. quindi la velocità aumenta con il raggio

\subsection*{Moto circolare generico} è il moto di un punto materiale lungo una circonferenza con un modulo della velocità che cambia nel tempo (per esempio una ruota che acelera e rallenta).
In questo caso l'accelerazione tangenziale è quella che genera la variazione di velocità (Riassunto SLIDE 39).

\paragraph*{Es 4.8 da resnick sul pendolo}
accelerazione tangenziale è dato da a sin(phi), a lungo r è a per il coseno di phi. a = ar + at. (SLIDE 41)

\paragraph*{Sistema di coorinate polari} (questa parte non verrà chiesta ma la fa per noi, perchè se ci danno un programma in cui ci chideono le coordinate polari dobbiamo saperle.)
SLIDE 45 c'è il raggio vettore

\paragraph*{Equazione generica del moto cirolare uniforme in forma parametrica} Supponiamo che la traiettoria del punto materiale sia di raggio R, centro 0 e giaccia nel piano xy e viaggi a velocità angolare $\omega$ costante.
Dla punto di vista angolare, l'equazione oraria ha una forma del tipo :Le sue equazioni parametriche, intermini dell'angolo $\theta$ (detto anomalia) che il vettore posizione forma con l'asse x sono:
$$\begin{cases}
        x(t) = R \cos \theta(t) \\
        y(t) = R \sin \theta (t)
    \end{cases}
$$
cioè, le coordinate x e y sono trovabili con il raggio e l'angolo (theta) nel momento in cui si sta cercando la posizione..
\\Ricordando che l'equazione oraria è $\theta = \theta_0 + \omega t$ sostituiamo e otteniamo $x(t) = R \cos (\theta_0 + \omega t)$ e $y(t)=...$.
quindi $r(s) = R\cos\theta(t) i$ è $R\sin\theta(t) j$

\subsection*{Il moto armonico}
SLIDE 49 (nota che nelle formule $\omega$ salta fuori perchè si deriva una funzione composta $\omega t \to \omega$)
Il moto armonico si chiama così perchè dopo un certo tempo assume lo stesso valore di partenza

\section*{Lezione 4 10-10-22}
\paragraph*{Recap moto armonico} %si comincia con un video di flippin physics che fa ridere
Il moto armoinico è un moto che dopo un certo tempo ritorna al punto di partenza.
La proiezione del moto circolare uniforme (sia su asse x che su y) sono dei moti 1D che si ritrovano in tanti fenomeni fisici.
Ricordo le equazioni parametriche dell'MCU $x(t) = A \cos(\theta_0 + \omega t)$ e y con il seno, con $\omega$ velocità angolare.
La velocità è la derivata dello spazio rispetto al tempo, e l'accelerazione è la derivata della velocità rispetto al tempo, da qui si ottengono le formule (SLIDE 3).
\\Si noti che la velocità massima (solo sull'asse y) di un moto armonico la si ha verso il centro, mentre l'accelerazione è massima agli estremi.
Quindi agli estremi accelerazione masssima e velocità 0, al centro velocità massima e accelerazione 0.
Se si fa una proiezione di un punto su un solo asse si ottiene un grafico sinusoidale.
\subparagraph{velocità}$v_x(t)?-A \omega \sin (\theta_0+\omega t)$ anche il grafico della velocità è periodico (periodo 2pi).
I valori della velocità sono sfasati rispoetto al grafico della posizione.
anche l'accelerazione è periodica e sfasata rispetto alla velocità.
%FINE RIASSUNTO
\subsection*{Dinamica. Sistemi di riferimento}%Nuovo argomento, non più cinematica
Abbiamo visto l'importanza della scelta del sistema di riferimento, le equazioni del moto variano al variare dei sistemi scelti.
\emph{Come si passa da un sistema di riferimento a un altro?}


\chapter{Lezione 5 12-10-22}
Oggi vedremo le tre leggi della dinamica, e in particolare le forze. ci saranno anche degli esperimenti.
\section{Le Forze}
\paragraph*{Introduzione}
Per determinare le cause di un moto dovremmo considerare si ala massa del corpo, che le forze che agiuscono su questa massa.

In alcuni casi le forze determinano degli spostamenti, un moto ma in altri casi no (stiamo seduti senza cadere al suolo).

Newton disse che le forze determinano i cambiamenti di velocità delle masse, cioè delle
accelerazioni.

\definizione{
    Un oggetto che stia fermo, o sia in moto rettilineo uniforme, accelera se e solo se
    la forza totale, come somma vettoriale delle forze, detta anche la risultante delle
    forze, è diversa da zero
}

Le forze possono agire:
\begin{itemize}
    \item A contatto, come l'urto di un oggetto con un altro
    \item A distanza (forze di campo), come la gravitazione
\end{itemize}

%LE forze fondamentali non saranno all'esame
\paragraph*{Alcune condizioni di studio della dinamica}
\begin{itemize}
    \item Le forze sono grandezze vettoriali, caratterizzate da intensità (modulo), direzione, verso,
          e un punto di applicazione
    \item Le forze possono deformare i corpi su cui agiscono. Nella nostra descrizione, a meno
          non sia espressamente detto, le forze NON modificano i corpi che sono rigidi ed
          indeformabili. Faranno eccezione le molle, e gli urti anelastici.
    \item non verranno considerati gli attriti (a meno di esplicitamente specificato).
    \item tutte le molle sono perfettamente elastiche.
    \item
\end{itemize}

\paragraph*{Paragrafo su newton} Perchè una persona con un CV simile naque in inghilterra? perchè dominava i mari (bisognava essere cartografi, horologi).
Disclaimer: la fisica newtoniana è $\pm$ giusta, vale solo per velocità minori di quella della luce e per dimensioni maggiori di quella di un atomo.

\subsection*{Prima legge di newton - il principio di inerzia}
Si consideri un oggetto fermo su un piano. Lo si spinge orizzontalmente, si muoverà solo quando la forza applicata supererà la forza di attrito del tavolo.
\\Per mantenere il corpo a velocità costante, bisognerà applicare una forza che eguagli la forza di attrito con il piano, quindi la forza di attrito va nella direzione opposta alla nostra.
\\Se applichi una forza maggiore, l'oggetto accellererà

\definizione{
    Quando un punto materiale non è soggetto a forze esterne, oppure la loro risultante è
    nulla, allora il punto materiale ha una velocità costante (o nulla).
}
oppure: Un corpo permane nel suo stato di quiete o di moto rettilineo uniforme a meno che
non intervenga una forza esterna a modificare tale stato.
\\Ogni \textbf{sistema isolato} è in quiete o si muove di moto rettilineo uniforme.

\subsection*{Sistemi inerziali}
\definizione{Si definisce inerzia la resistenza naturale di un oggetto alle modificazioni della sua velocità}
Un sistema di riferimento si dice inerziale \emph{se esso non accelera}, in esso vale il primo principio della dinamica.
\\Se in un sistema inerziale un osservatore nota che l'accelerazione di un oggetto è
nulla, o, in modo equivalente, che la risultante delle forze è nulla, allora esse
saranno nulle in qualunque altro sistema inerziale. Le leggi della meccanica classica sono invarianti solo per osservatori inerziali.
\subparagraph*{La terra non è un sistema inerziale} a causa della rotazione e rivoluzione.
Ma le accelerazioni in gioco sono molto piccole rispetto a quelle dovute alla gravità
(dell’ordine di qualche punto percentuale). Per cui la Terra è \emph{in ottima approssimazione un
    sistema inerziale}.
\section*{La massa}
\definizione{
    La massa di un oggetto specifica quanta inerzia ha l'oggetto all'azione di una forza.
    Più grande è la massa, meno l'oggetto accelera come conseguenza della forza applicata.
    L'unità di misura in SI è il \emph{kg}
}
Due oggetti, di massa $m_1$ e $m_2$, soggetti alla stessa forza essa determina delle
accelerazioni a1 ed a2. si ottiene sperimentalmente che $m1a1=m2a2$.

La massa è una caratteristica dell'oggetto e non dipende dall'ambiente circostante.
È diversa dal peso che invece dipenderà dal luogo dove l'oggetto è misurato.

\subsection*{La seconda legge di Newton}
Cosa succede se la risultante delle forze applicate ad un corpo è diversa da zero?

\definizione{
    L’ accelerazione di un oggetto è direttamente proporzionale alla risultante delle forze agenti
    ed è inversamente proporzionale alla massa.
}
Si possono quindi correlare la massa e le forze agenti in base alla II legge di newton, che si esprime matematicamente come:
\[\sum \overrightarrow{F} = m\overrightarrow{a}\]
La forza è un vettore e quindi l'espressione precedente è valida \emph{per tutte e tre le componenti cartesiane della forza}
\\Consideriamo il modulo della formula scritta in modo compatto: $F=ma$, l'unità di misura è il Newton.
\definizione{
    1 Newton è quella forza che impressa ad una massa di 1 kg, determina una sua accelerazione
    di 1 m/s2
    \[1 [N] = 1[kg \cdot m/s^2]\]
}
Nota che con la stessa forza, all'incrementare della massa l'accelerazione decrementa.

\paragraph*{Esercizio Halliday} Notare che si usando Sin e Cos per la proiezione del vettore (ripassati la trigonometria coglione)

\subsection*{La forza gravitazionale e il peso}
La Terra attrae tutti gli oggetti. La forza esercitata dalla Terra verso un oggetto è detta forza di
gravità Fg
La forza di gravità è diretta verso il centro della Terra e la sua ampiezza è chiamata peso di un
oggetto.
\\Per la seconda legge di Newton: Un corpo di massa m, che è soggetto solo alla forza di gravità, si muove di moto uniformemente
accelerato, con accelerazione costante, indicata con g, verso il centro della Terra.
Tale accelerazione si chiama Accelerazione di gravità il cui modulo è pari a -9.81 $m/s^2$
\\Il peso di un oggetto, che è pari all’ampiezza (modulo) della forza di gravità: Fg = mg.

Il peso dipende da g, e quindi dal luogo geografico. Il peso non è quindi una proprietà intrinseca
dell’oggetto come la massa.
\subsection*{Sistemi di riferimento non inerziali}
\subsection{La terza legge di newton}
\definizione{
    Se due corpi interagiscono tra di loro, la forza F12 esercitata dal corpo 1 sul corpo 2 è eguale in
    ampiezza, ed ha la stessa direzione e verso opposto della forza F21 esercitata dal corpo 2 sul corpo 1.
    \[\overrightarrow{F_{12}} = -\overrightarrow{F_{21}}\]   
}
La terza legge di Newton può essere anche riformulata nel modo seguente.
Ad ogni azione corrisponde una reazione eguale in modulo ed avente la stessa direzione ma
verso opposto dell’azione.
N.B: L’azione e la reazione agiscono su oggetti diversi!

\paragraph*{La forza normale}
Si consideri il caso di un oggetto su un tavolo. Si è definita forza di gravità la forza esercitata dalla Terra su
ogni oggetto. Perché non accelera nella direzione di Fg ? A causa della presenza del tavolo!
Esso esercita una forza nella stessa direzione di Fg e quindi sarà una forza normale (e verso opposto). Si
tratta di una forza di contatto, che può assume il modulo massimo pari al punto di rottura del tavolo. La
forza normale diventa zero quando l'oggetto è tolto dal tavolo.
La forza normale NON è la forza di reazione della terza legge di Newton, anche se lo sembra,
perchè è esercitata sullo stesso corpo, mentre si era indicato che la reazione è su un corpo
diverso!

\subsection*{La tensione}

\subsection{Il piano inclinato} %%LEZIONE 6 ripasso piano inclinato
Abbiamo una scatola di massa m su un piano inclinato di angolo $\theta$. 
Bisogna innanzitutto capire quali sono le forze in gioco.
La cosa importante è che il moto è dovuto esclusivamente alla componente su x %poi capisci in che senso
$a_x = g \sin \theta$

\chapter*{Lezione 6 13-10-22}
\section*{L'ascensore}
Si pesi un oggetto di massa $m$ dentro un ascensore.
La massa sia attaccata ad un dinamometro fisso sul soffitto.
Se l'ascensore accelera o decelera rispetto ad un osservatore a Terra, allora \emph{non è un sistema inerziale}.
Se accelera o decelera, ci sarà un effetto sulla forza peso.
Se è fermo o se va a velocità costante invece non ci sarà effetto (I legge
di Newton).

\paragraph*{Se l'ascensore è fermo, quali sono le forze in gioco?}
Se l'ascensore è fermo o a velocità costante, le forze in gioco sono $F_g$ (forza di gravità) e la tensione T sul filo
$\sum F_u = mg-T=0 \implies T=mg$
\\Se l'ascensore \emph{Accelera verso l'alto} allora la bilancia segna un valore maggiore: $\sum F_u = mg-T=-ma_y \implies T=mg + ma_y$, viceversa verso il basso.
\\Se $a_y = g$ (cioè l'ascensore è in caduta libera) allora il peso della massa sarà $T=0$.
\paragraph*{Esercizio} %Immagine lez6slide3
(a) = la forza normale è uguale a $mg$.
\\(b) = La forza normale è maggiore di $mg$.

\section{La carrucola} (Nella slide 4 si applica la seconda legge di newton per le due masse (caso generale))
\paragraph*{Esempio piano inclinato con la carrucola} La metodologia è: trova le forze e vedi dove vanno, trovando quelle che si annullano e quelle che no.
(Nota che $n$ è la normale)

\section{Forze d'attrito o di Frizione}
La forza di attrito statico $f_s$ è una forza che si oppone al movimento di un oggetto.
La forza di attrito statico aumenta in modulo quando aumenta la forza applicata, a un certo punto la forza di attrito statico raggiunge un massimo e il libro inizia a muoversi.
%NOTA ricordati di fare la trigonometria, quando ti viene dato un angolo e devi scomporre i due assi cartesiani, si fa Vettore x coseno(angolo) per il segmento adiacente all'angolo
% e vettore x sin(angolo) per il segmento che è opposto all'angolo.

\paragraph*{Esempio blocco su piano inclinato} misura l'angolo critico.

\section*{La resistenza del mezzo e la velocità limite}
\subsection*{La resistenza alla caduta di un corpo}

\section*{Dinamica: seconda legge di newton applicata al moto circolare uniforme}

\chapter*{Lezione 7 17-10-22}
Esercizio auto e deportanza interessante.
\section*{Il lavoro di una forza costante in modulo e direzione}
\paragraph*{Definzione di lavoro}
\definizione{Si definisce lavoro $W$ fatto su un qualunque corpo da una forza di modulo costante F come il 
prodotto scalare della forza F per il vettore spostamento d del corpo.
\[W=Fd=fd\cos \theta\]
W è uno scalare con segno.
}
oppure: il prodotto della componente della forza nella direzione dello spostamento e l'ampiezza dello spostamento stesso
\end{document}