\chapter{Introduzione}

Appunti di Sicurezza e Affidabilitá di Fabio Ferrario.

\section*{Il Corso}
Gli appunti fanno riferimento alle lezioni di SEA erogate nel secondo semestre dell'anno accademico 23/24.

\subsection*{Programma del corso}
Il programma si sviluppa come segue:
\begin{enumerate}
	\item Garantire l'Affidabilitá del Software
	\begin{enumerate}
		\item Introduzione al Test e l'Analisi del Software
		\item Test Combinatorio
		\begin{itemize}
			\item Combinazione a Coppie
			\item Metodi di partizione delle Categorie
			\item Cataloghi per il Test
		\end{itemize}
		\item Test Strutturale
		\begin{itemize}
			\item Test basato sugli Statement
			\item Test basato sui Branch
			\item Test basato sulle Condizioni
		\end{itemize}
		\item Esecuzione del Test
		\begin{itemize}
			\item Speicifica e Implementazione del caso di Test
			\item Scaffolding: Driver e Stub
			\item Oracoli
		\end{itemize}
		\item Analisi Statica
	\end{enumerate}
	\item La sicurezza del Software
	\begin{enumerate}
		\item Rischi nell'uso dei sistemi informativi, ruoli e competenze
		\item Tecniche e protocolli per la sicurezza
		\begin{itemize}
			\item Crittografia, errori di implementazione e attacchi
			\item Sicurezza nei sistemi operativi e nelle strutture di rete
		\end{itemize}
		\item Programmazione sicura
		\begin{itemize}
			\item Errori di sicurezza nelle applicazioni
			\item Analisi di noti programmi che presentano vulnerabilitá
		\end{itemize}
		\item Programmi pericolosi: Troiani, Back-door, Bombe logiche, Virus, Worm
		\item Difese: intrusion Detection System, Attacchi di verifica, Firewall.
	\end{enumerate}
\end{enumerate}