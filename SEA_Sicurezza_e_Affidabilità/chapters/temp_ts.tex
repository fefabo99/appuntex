%%Restart da slide 15 perché ho missato un commit.

\paragraph*{Unsatisfiable Test Obligations}
In alcuni programmi potrebbero esistere dei criteri di adeguatezza insoddisfacibili, come ad esempio 
alcuni statement non raggiungibili dai test.
Se non consideriamo questi elementi potremmo trovarci con delle suite che risultano inadeguate solo perché alcune obbligazioni non sono fisicamente soddisfacibili.

Abbiamo due approcci possibili a questo problema:
\begin{itemize}
    \item Rimuovere dai criteri di adeguatezza tutte le obbligazioni di test insoddisfacibili.
    \item Usare la covergae come una misura di quanto ci siamo avvicinati all'adeguatezza.
\end{itemize}

\subsection{Basic Blocks}
Si puó notare che se due statements sono in sequenza, eseguirne uno automaticamente implica eseguire anche l'altro.
Quindi possiamo considerare i \textbf{basic blocks}, dove le obbligazioni sono i blocchi del CFG\footnote{Control Flow Graph} del programma.

\definizione{Un Basic Block è una sequenza massima di istruzioni di programma contigue con un punto di ingresso e un punto di uscita.}