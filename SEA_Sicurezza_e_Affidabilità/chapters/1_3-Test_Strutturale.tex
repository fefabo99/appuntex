\chapter{Test Strutturale}
\section{Adeguatezza e Copertura}
\paragraph{Thoroughness}
Come possiamo grantire la \textbf{scrupolosità} (completezza) dei test?
\\Per farlo dobbiamo rispondere alle seguenti domande:
\begin{itemize}
    \item \textbf{Quali} test dobbiamo generare?
    \item \textbf{Quanti} test dobbiamo generare?
    \item \textbf{Quando dobbiamo fermarci} a generare test?
\end{itemize}
In linea di principio, l'obiettivo dovrebbe essere quello di generare un'\textbf{adeguata} suite di test, vale a dire
una suite di test che, se il software sottoposto a test viene superato con successo, garantisca
una qualche proprietà del software stesso.

L'adeguatezza è quindi in principio una sorta di "assicurazione" sull'abilità della suite di test nel trovare difetti.

\paragraph{Non possiamo avere quello che vogliamo!}
Non possiamo garantire in nessun modo che una suite trovi tutti o alcuni dei difetti, e non possiamo garantire neanche che li trovi con alta probabilità:
\begin{center}
    \emph{«Testing can be used to prove the presence, not the absence, of errors»}
\end{center}
In sostanza nessun metodo di progettazione dei test fornisce alcuna garanzia sulla capacità di
scoprire difetti per le suite di test generate

\paragraph{Cosa Facciamo?}
Quindi come costruiamo una suite di test accettabile?
\\Potremmo continuare a generare test random e continuare a farlo finche non finiamo il tempo o il budget,
ma questa strategia non ci soffisfa euristicamente!

\subsection{Criteri di Adeguatezza}
Bisogna rinunciare all'idea disperata dell'adeguatezza come garanzia sul potere di rilevamento dei difetti,
e definire criteri euristici di adeguatezza simili alle regole di progettazione.

Molte discipline progettuali utilizzano regole di progettazione per valutare non se un progetto è
adeguato, ma se \emph{esso è inadeguato}.
L'idea è che un design che segue queste rule non è necessariamente adeguato, ma uno che non segue queste regole necessariamente sarà inadeguato!

\paragraph{Criteri pratici di (in)adeguatezza per i test}
Molti criteri di (in)adeguatezza per il testing derivano da osservazioni di buon senso su ciò che ci aspetteremmo come minimo da una suite di test.

\textbf{Ricorda} che questi criteri ci aiutano a capire perchè ci piace o non piace una suite di test,
ma soddisgarli (o no) non implica niente sull'effettiva abilità della suite nel trovare difetti!

\paragraph{Definizione di Criteri di Adeguatezza}
Diamo una definizione formale di criteri di adeguatezza:
\definizione{
    Un criterio di adeguatezza è un predicato che assume valore vero o falso
    per una coppia $<P,T>$, dove $P$ è un programma e $T$ è una suite di test.
    Se il criterio è $True$, diciamo che la suite è adeguata per il programma.
}
Un criterio di adeguatezza generalmente è fatto da sottopredicati chiamati \textbf{test obligations}.

Una suite $T$ soddisfa i criteri di adeguatezza per un dato programma $P$ \emph{se e solo se}:
\begin{itemize}
    \item Tutte le esecuzioni dei casi di test in $T$ su $P$ passano.
    \item Tutte le test obligations sono soddisfatte da almeno un test case nella suite.
\end{itemize}

\begin{center}
    \includegraphics[width=.8\textwidth]{images/example_testing_adequacy.jpg}
\end{center}
In questo esempio abbiamo che i singoli test non riescono a soddisfare singolarmente i criteri di adeguatezza perchè non riescono a coprire l'intero codice.

%SLIDE 10

\section{Structural Testing}
A differenza del testing funzionale, in cui la completezza del test è giudicata sui requsiti senza tener conto del programma sotto esame,
nel testing strutturale viene giudicata la completezza del test in base alla struttura del programma.

Il test strutturale è anche chiamato "white/glass box testing", mentre quello funzionale è chiamato "black box testing".

Si noti che il testing strutturale consiste ancora nel testare il prodotto (codice) rispetto alle specifiche: cambia solo la misura della completezza!

\osservazione{I test confrontano \textbf{sempre} un programma con una specifica}

\subsection{Statement Coverage}
Per una suite di test T possiamo definire la \textbf{coverage} come la frazione di enunciati di P eseguiti da almeno un caso di test in T:
\[ C_{stmt} = \frac{\text{\# executed stmts}}{\text{\# stmts}} \]
Il test $T$ soddisfa il criterio di adeguatezza se e solo se $C_{stmt} = 1$.

%slide 15