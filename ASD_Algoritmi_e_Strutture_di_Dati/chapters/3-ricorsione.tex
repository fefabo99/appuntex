\chapter{Ricorsione}
La programmazione ricorsiva è strettamente legata all'induzione matematica, si basa
sul fatto che per risolvere un problema mi riconduco a un problema non ancora risolto, 
ma più semplice, fino ad arrivare a un caso base già risolto.\\
I vantaggi della ricorsione sono due:
\begin{itemize}
    \item Più semplice rispetto agli algoritmi iterativi (solitamente)
    \item La logica ricorsiva è più efficiente rispetto a quella iterativa
\end{itemize}
\section{Fattoriale}
\subsection{Iterativo}
\begin{lstlisting}[language=Java]
    int Fatt(n)
        Ris=1
        For i=n downto 1
            Ris=Ris*i;
        return Ris;
\end{lstlisting}
\subsection{Ricorsivo}
\begin{lstlisting}
    int Fatt(int n)
        if n==0
            return(1);
        else
             Ris=(Fatt(n-1));
             Tot = n*Ris;
             return(Tot);
\end{lstlisting}
                  

