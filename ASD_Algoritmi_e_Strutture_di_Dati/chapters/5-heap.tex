\chapter{Heap - BOZZA}
%Riguardare lezioni e completare%
Utilizzati per esempio per Heap Sort, che come tempi è
confrontabile con Quick e Mergesort, avviene in loco, non è stabile.
\section{Che cos'è un Heap}
\'E un array visto come un albero binario quasi completo.\\
Un array sappiamo che come proprietà ha length(A) per indicare la lunghezza 
dell'array ed è una proprietà statica, mentre negli Heap c'è anche
heapsize(A) $\leq$ length(A) che è una proprietà dinamica dato che gli elementi
appartenenti all'Heap variano nel tempo.
\paragraph*{Proprietà} A[parenti(i)] $\geq$ A[i].\\
Questo oltre a definire la struttura dell'Heap ci dice che la radice sarà il valore
massimo. Sapere il minimo risulta più vago e anche l'ordine è meno preciso dell'albero binario
di ricerca.\\
\subsection{Cosa serve l'Heap}
Serve per implementare code con priorità, perchè con lo Heap riesco a valutare
chi è l'elemento più importante (la radice), dato che qua non mi interessa l'ordine globale,
mi interessa definire la priorità e ridefinirla quando la radice viene tolta dalla coda.
\subsection{Heapify}
\begin{lstlisting}[language=Java]
    Heapify(A, h)
\end{lstlisting}