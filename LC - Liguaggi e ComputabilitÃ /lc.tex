\documentclass[12pt, a4paper, openany]{book}
\usepackage{estyle}

\graphicspath{ {./images/} }
\renewcommand{\labelenumii}{\arabic{enumi}.\arabic{enumii}}

\begin{document}
\title{LC - Linguaggi e Computabilità}
\author{Elia Ronchetti\\
    \small{\href{https://t.me/ulerich}{@ulerich}}}
\date{Settembre 2022}

\maketitle
\tableofcontents

\chapter{Linguaggi formali}
Preso un alfabeto $\Sigma=\{a,b,c\}$ posso formare delle stringhe come
\\ $\{bab, aca, b\}$, questo insieme è un linguaggio sull'alfabeto $\Sigma$.
\\ Con un linguaggio posso fare principalmente 2 cose
\begin{itemize}
    \item Riconoscitori - detti anche \textbf{automi}, sono degli automi che prendono in input
    una stringa e mi dicono se appartiene o no al linguaggio, \textbf{riconoscono} quindi il linguaggio
    \item Generatori - sono le Grammatiche, dalla grammatica posso generare il linguaggio,
    esse mi danno diverse regole attraverso le quali posso generare un linguaggio.
\end{itemize}
\paragraph*{Alfabeto} Un alfabeto è un insieme non vuoto e finito di simboli
\definizione{Ciao}

\end{document}