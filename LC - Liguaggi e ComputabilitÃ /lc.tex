\documentclass[12pt, a4paper, openany]{book}
\usepackage{estyle}

\graphicspath{ {./images/} }
\renewcommand{\labelenumii}{\arabic{enumi}.\arabic{enumii}}

\begin{document}
\title{LC - Linguaggi e Computabilità}
\author{Elia Ronchetti\\
    \small{\href{https://t.me/ulerich}{@ulerich}}}
\date{Settembre 2022}

\maketitle
\tableofcontents

\chapter{Linguaggi formali}
Preso un alfabeto $\Sigma=\{a,b,c\}$ posso formare delle stringhe come
\\ $\{bab, aca, b\}$, questo insieme è un linguaggio sull'alfabeto $\Sigma$.
\\ Con un linguaggio posso fare principalmente 2 cose
\begin{itemize}
    \item Riconoscitori - detti anche \textbf{automi}, sono degli automi che prendono in input
    una stringa e mi dicono se appartiene o no al linguaggio, \textbf{riconoscono} quindi il linguaggio
    \item Generatori - sono le Grammatiche, dalla grammatica posso generare il linguaggio,
    esse mi danno diverse regole attraverso le quali posso generare un linguaggio.
\end{itemize}
\paragraph*{Alfabeto} Un alfabeto è un insieme non vuoto e finito di simboli
\definizione{Un alfabeto è un insieme non vuoto e finito di simboli
definito con la lettera $\Sigma$}
\esempio{$\Sigma = \{0,1\}$}
\definizione{Una \textbf{stringa} è una sequenza finita di simboli dell'alfabeto.
    \\ In questo caso è ammesso l'insieme vuoto, si parla di stringa vuota e viene definita
    come $\epsilon$
    \begin{center}
        $L \subseteq \Sigma^*$
    \end{center}
}
La lunghezza di una stringa viene denotata dai seguenti simboli $|\text{stringa}|$.
\esempio{$|0111|=4$}
$\Sigma^n$ è una stringa di lunghezza.
\definizione{Un \textbf{Linguaggio} (su un alfabeto $\Sigma$ è un sottoinsieme di $\Sigma^*$)}
\paragraph*{Notazioni particolari}
\begin{itemize}
    \item $\Sigma^0 = \{\epsilon\}$
    \item $\Sigma^* = \Sigma^0 \cup \Sigma^1 \cup ...$ oppure $\Sigma^*=\Sigma^+ \cup \Sigma^0 = \Sigma^+ \cup \{\epsilon\}$
    \item $\Sigma^+ = \Sigma^1 \cup \Sigma^2 \cup \Sigma^3 \cup ...$ oppure $\Sigma^+ = \Sigma^* - \{\epsilon\}$
\end{itemize}
\paragraph*{Notazioni stringhe} Per definire una stringa utiizzo una lettera come per esempio
$w = abb$ dove in w sono contenuti dei simboli dell'alfabeto $\Sigma=\{a,b,c\}$.
\\ Definita un'altra stringa $y=cca$ definisco la concatenazione di due stringhe come $wy = abbcca$.
La concatenazione \textbf{NON} è commutativa.
\chapter{Le Grammatiche}
Una grammatica è una quadrupla
\begin{equation*}
    G=(V, T, P,S)
\end{equation*}
Dove
\begin{itemize}
    \item V è l'insieme delle Variabili
    \item T è l'insieme dei simboli Terminali
    \item P è l'insieme delle regole di Produzione
    \item S $\in$ V, è lo Start symbol
\end{itemize}
\definizione{Grammatica: termine che designa una struttura formale per un linguaggio L in grado di generare tutte e sole le stringhe del linguaggio.
    Per questo si parla di grammatica G generativa del linguaggio L o di linguaggio L generato dalla grammatica G, indicandolo con L(G).}
\textit{Definizione Treccani} \href{https://www.treccani.it/enciclopedia/grammatica_%28Enciclopedia-della-Matematica%29/}{Link alla definizione}
\section{Gerarchia di Chomsky}
La gerarchia di Chomsky è un insieme di classi di grammatiche formali che generano linguaggi formali.
Suddivide le grammatiche in 4 tipi:
\begin{itemize}
    \item Tipo 0 - Linguaggi ricorsivamente enumerabili
    \item Tipo 1 - Linguaggi Contestuali
    \item Tipo 2 - Linguaggi Context-Free (Liberi dal contesto)
    \item Tipo 3 - Linguaggi Regolari
\end{itemize}
\subsection{Tipo 0}
Sono definiti nel seguente modo:
\begin{center}
    $\alpha \rightarrow \beta$, con $\alpha$ e $\beta \in (V \cup T)^*$
\end{center}
\esempio{$0S1S0 \rightarrow 00SS1S01$}
Sono detti \textbf{ricorsivamente enumerabili} e vengono accettati dalle 
\textbf{Macchine di Turing deterministiche e non deterministiche}.
\subsection{Tipo 1}
Verranno solo visti un paio di esempi, ma non saranno trattati in questo corso.
\\ Definiti nel seguente modo:
\begin{center}
    $\alpha_1 \, A \, \alpha_2 \rightarrow \alpha_1 \, \beta \, \alpha_2$
    \\ $A\in V \qquad \alpha_1, \alpha_2, \beta \in (V \cup T)^*$
\end{center}
In questo linguaggio sono presenti più vincoli rispetto al tipo 0.
Questi linguaggi sono detti contestuali, questo perchè posso sostituire le variabili solo in un determinato contesto, in questo caso
solo se sono presenti sia $\alpha_1$ e $\alpha_2$ tra $\beta$.
\\ Sono accettati dalle \textbf{Macchine di Turing con nastro "lineare"}.
\subsection{Tipo 2}
$A\rightarrow \gamma \qquad A \in V$, $\gamma \in \, (V \cup T)^*$.
\\ Detti anche \textbf{Context-Free (Liberi dal Contesto)}.
\\ Riconosciuti da \textbf{Automi a Pila non deterministici}
\subsection{Tipo 3}
$A \rightarrow aB$ oppure $A \rightarrow a$ oppure
\\ $A \rightarrow Ba$ oppure $A \rightarrow a$.
\\Sono detti \textbf{Regolari} e sono riconosciuti da \textbf{Automi
a stati finiti deterministici e non deterministici}.
\end{document}