\chapter{DFA e NFA}
I DFA (Deterministic Final Automa) o \textbf{Automi a Stati Finiti Deterministici} un tipo di automa in grado
di accettare grammatiche di tipo 3 (Regolari).
\'E una quintupla definita come segue:
\begin{equation*}
    A={Q, \Sigma, \delta, q_0, F}    
\end{equation*}
Dove delta è una funzione totale.
%Inserire spiegazione lettere
\section{NFA}
I NFA (Non Deterministic Final Automa) o \textbf{Automi a Stati Finiti Non Deterministici} sono dempre definiti
come quintupla definita come segue:
\begin{equation*}
    A={Q, \Sigma, \delta, q_0, F}    
\end{equation*}
In questo caso delta è una funzione parziale
Qua l'automa può assumere più stati in contemporanea, con un simbolo per
esempio posso andare in 2 stati diversi.
\paragraph*{NFA e DFA, hanno la stessa potenza?} Sì, ogni NFA può essere convertito in un DFA, 
c'è da tenere in considerazione che gli NFA sono più compatti dei DFA in termini di rappresentazione
grafica.
\definizione{Eclose(q)
\\ è l'insieme di stati di Q che possono essere raggiunti a partire da q facendo solo 
$\epsilon$-mosse, compreso q stesso.}