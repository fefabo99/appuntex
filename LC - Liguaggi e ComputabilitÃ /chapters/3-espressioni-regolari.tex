\chapter{Espressioni Regolari}
\definizione{Un'espressione regolare (in lingua inglese regular expression o, in forma abbreviata, regexp, regex o RE) 
è una sequenza di simboli (quindi una stringa) che identifica un insieme di stringhe.
Possono definire tutti e soli i linguaggi regolari.}
\section{Operazioni Tra Linguaggi}
\subsection*{Unione}
Dati $L,M \rightarrow L \cup M$
\esempio{L = \{001, 10,111\}
\\ M = \{\eps, 001\} 
\\ L $\cup$ M = \{\eps, 001, 10, 111\}}
Gli elementi in comune non si ripetono
\subsection*{Concatenazione} $L,M \rightarrow LM$
\esempio{L = \{001, 10, 111\}
\\ M = \{\eps, 001\}
\\ LM = \{001, 001001, 10, 10001, 111, 111001\}}
Tutte le possibili concatenazioni
\subsection*{Chiusura di Kleene} $L = \emptyset$
\esempio{$\emptyset^0 = \{\epsilon\}
\\L^0 = \{\epsilon\}\forall L
\\ \emptyset^i = \emptyset \forall_i \geq 1
\\ \emptyset^* = \emptyset^0 \cup \emptyset^1 \cup \emptyset^2 \cup \dots =
\\ \{\epsilon\} \cup \emptyset \cup \emptyset \cup \dots =
\\ \{\epsilon\}$}
\section*{Espressioni Regolari}
\paragraph*{NB} In questo contesto \eps e $\emptyset$ sono ER (Espressioni
Regolari) NON sono stringhe.
\begin{enumerate}
    \item \eps e \empt sono ER \ra L(\eps) = \{\eps\} L(\empt) = \empt
    \item Se $a \in \Sigma$ allora a è una ER (Dove $\Sigma$ è alfabeto)
    \item Variabili che rappresentano linguaggi sono ER
\end{enumerate}
\paragraph*{Induzione}
\begin{enumerate}
    \item \textbf{Unione} \ra se E,F sono ER allora E+F è una ER \\
    $L(E+F) = L(E)\cup L(F)$
    \item \textbf{Concatenazione} \ra se E,F sono ER allora EF è una ER \\
    $L(EF) = L(E)*L(F)$
    \item \textbf{Chiusura} \ra Se E è ima ER allora E* è una ER \\
    $L(E^*) = (L(F))^*$
    \item \textbf{Parentesi} \ra Se E è una ER allora (E) è una ER \\
    $L((E)) = L(E)$
\end{enumerate}
\paragraph*{Proprietà}
\begin{itemize}
    \item \textbf{Unione} \ra Commutativa, Associativa \\
    $L+M = M+L$ \\ $(L+M)+N = L+(M+N) = L+M+N$
    \item \textbf{Concatenazione} \ra Associativa, NON è commutativa \\
    $(LM)N=L(MN)$ \\ $01 \neq 10$
\end{itemize}
%Finito Lezione 6 - Iniziare Lezione 7 (Proprietà Algebriche ER)