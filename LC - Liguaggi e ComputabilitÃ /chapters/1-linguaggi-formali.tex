\chapter{Linguaggi formali}
Preso un alfabeto $\Sigma=\{a,b,c\}$ posso formare delle stringhe come
\\ $\{bab, aca, b\}$, questo insieme è un linguaggio sull'alfabeto $\Sigma$.
\\ Un linguaggio può essere generato o riconosciuto:
\begin{itemize}
    \item Riconoscitori - detti anche \textbf{automi}, sono degli automi che prendono in input
    una stringa e mi dicono se appartiene o no al linguaggio, \textbf{riconoscono} quindi il linguaggio
    \item Generatori - sono le Grammatiche, dalla grammatica posso generare il linguaggio,
    esse mi danno diverse regole attraverso le quali posso generare un linguaggio.
\end{itemize}
\paragraph*{Alfabeto} Un alfabeto è un insieme non vuoto e finito di simboli
\definizione{Un alfabeto è un insieme non vuoto e finito di simboli
definito con la lettera $\Sigma$}
\esempio{$\Sigma = \{0,1\}$}
\definizione{Una \textbf{stringa} è una sequenza finita di simboli dell'alfabeto.
    \\ In questo caso è ammesso l'insieme vuoto, si parla di stringa vuota e viene definita
    come $\epsilon$
    \begin{center}
        $L \subseteq \Sigma^*$
    \end{center}
}
La lunghezza di una stringa viene denotata dai seguenti simboli $|\text{stringa}|$.
\esempio{$|0111|=4$}
$\Sigma^n$ è una stringa di lunghezza.
\definizione{Un \textbf{Linguaggio} (su un alfabeto $\Sigma$ è un sottoinsieme di $\Sigma^*$)}
\paragraph*{Notazioni particolari}
\begin{itemize}
    \item $\Sigma^0 = \{\epsilon\}$
    \item $\Sigma^* = \Sigma^0 \cup \Sigma^1 \cup ...$ oppure $\Sigma^*=\Sigma^+ \cup \Sigma^0 = \Sigma^+ \cup \{\epsilon\}$
    \item $\Sigma^+ = \Sigma^1 \cup \Sigma^2 \cup \Sigma^3 \cup ...$ oppure $\Sigma^+ = \Sigma^* - \{\epsilon\}$
\end{itemize}
\paragraph*{Notazioni stringhe} Per definire una stringa utiizzo una lettera come per esempio
$w = abb$ dove in w sono contenuti dei simboli dell'alfabeto $\Sigma=\{a,b,c\}$.
\\ Definita un'altra stringa $y=cca$ definisco la concatenazione di due stringhe come $wy = abbcca$.
La concatenazione \textbf{NON} è commutativa.