\section{Abbiamo visto ieri: Sintesi delle caratteristiche di un sistema distribuito}
Caratteristiche fondamentali per tutti i sistemi distribuiti:
\subsubsection{Gestione della memoria?}
\begin{itemize}
    \item \textbf{Non c'è memoria condivisa}
    \item Comunicazione via scambio messaggi
    \item Non c'è stato globale: ogni componente (nodo, processo) conosce solo il proprio stato e può sondare lo stato degli altri.
\end{itemize}

\subsubsection{Gestione dell'esecuzione?}
\begin{itemize}
    \item \textbf{Ogni componente è autonomo} => esecuzione concorrente
    \item Il coordinamento delle attività è importante per definire il comportamento di un sistema/applicazione costituita da più componenti
\end{itemize}

\subsubsection{Gestione del tempo (temporizzazione)?}
\begin{itemize}
    \item \textbf{Non c'è un clock globale}
    \item Non c'è possibilità di controllo/scheduling globale
    \item Solo coordinamento via scambio messaggi
\end{itemize}

\subsubsection{Tipi di fallimenti?}
\begin{itemize}
    \item \textbf{Fallimenti indipendenti} dei singoli nodi (independent failures)
    \item Non c'è fallimento globale
\end{itemize}

\section{Il modello Client-Server}
Il modello Client-Server è il modello di interazione tra un processo client e un processo server.
% inserisci immagine modelloCS1
C'è uno strato verticale e uno orizzontale(?)
\subsubsection{Configurazioni client/server}
\begin{itemize}
    \item Accesso a server multipli
    \item Accesso via proxy
\end{itemize}

\chapter{Caratteristiche problematiche di ogni sistema distribuito}
N.B.: (molto) probabile domanda d'esame.
\\Tutti i sistemi distribuiti vanno incontro a 4 problemi fondamentali che devono saper risolvere.
\\Vari step di risoluzione:
\begin{description}
    \item[Identificare ]: fase di \textbf{naming}, dove assegnamo a  un identificativo che deve necessariamente essere \textbf{univoco}; 
    \item[Accedere a ]: fase di \textbf{access point}, una \textit{reference} a cui possiamo fare riferimento;
    \item[Comunicazione 1]: fase di \textbf{protocol}, dove bisogna accordarsi su un formato condiviso di comunicazione;
    \item[Comunicazione 2]: questo è ancora un \textbf{open issue}, dove bisogna accordarsi su .
\end{description}