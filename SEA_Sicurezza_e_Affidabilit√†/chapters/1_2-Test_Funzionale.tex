\chapter{Test Funzionale}
Il test funzionale è un tipo di testing che prende i casi di test dalla spcifica funzionale del programma.
Viene anche chiamato Specification-Based Testing, ovvero dalle specifiche, e Black-Box Testing perchè non si ha visione del codice.

Una \textbf{Specifica Funzionale} è una descrizione di un comportamento atteso del programma.

\section{Test Obligations}
Una \textbf{Test Obbligation}, detta anche test objective, è una specifica parziale del caso di test che richiede alcune proprietà ritenute importanti per un test approfondito

\subsection{Obbligazioni funzionali} le obbligazioni di tipo funzionale sono obbligazioni che testano 
il funzionamento del programma secondo la sua specifica, ad esempio:
\begin{itemize}
    \item Testa almeno una volta con un array già ordinato
    \item Testa almeno una volta l'autenticazione con un utente errato
\end{itemize}
Le obbligazioni Funzionali vengono dalle \textbf{specifiche del software}.

\subsection{Obbligazioni Strutturali}
Le obbligazioni strutturali invece vanno a testare una parte specifica del codice del programma:

\begin{itemize}
    \item Testa almeno una volta il ramo else di uno specifico if-statement
    \item Testa almeno una volta un loop su più di una iterazione
\end{itemize}
Le obbligazioni strutturale derivano direttamente dal codice, senza tenere conto della specifica.

\section{Boundary Testing}
Nel test funzionale vengono testati ogni categoria di input e i \textbf{boundary} tra di esse.
\osservazione{Non c'è garanzia che testare i boundaries trovi più errori, ma per esperienza si trova che spesso gli errori si trovano li.}
Il Boundary Testing è un tipo di test che seleziona obiettivi per tenere conto di comportamenti 
rilevanti nel software, distinguendo tra comportamenti Normali, Eccezzionali/Erronei, e Casi di Boundary.
\\Il test perimetrale suggerisce che gli obiettivi del test devono tenere conto dei comportamenti distinguibili del software e dei confini tra di essi

\subsection*{Catalogi di Testing}
I catalogi di test sono delle guide per identificare obbligazioni per una classe di elementi ben caratterizzati.

Il Boundary Testing può essere assistita e resa più sistematica con cataloghi di test che indichino scelte standard (di casi normali, eccezionali e limite) per tipi tipici di voci di specifica.

Ad esempio, un catolog per testare un Range $L ... U$ può essere
\begin{center}
    L-1, L, Valore tra L e U, U, U+1
\end{center}

%%SLIDE 17


%% Secondo pacco di slides

\section{Test Combinatorio}