\chapter{Algebra Relazionale}
Nell ebasi di dati relazionali esistono 2 tipi di linguaggi di interrogazione
\begin{itemize}
    \item Procedurali - Specificano le modalità di generazione del risultato - Come
    \item Dichiarativi - Specificano le proprietà del risultato - Che cosa
\end{itemize}
L'algebra relazionale è procedurale, mentre SQL è parzialmente dichiarativo.\\
L'algebra relazionale è composta da un insieme di operatori che possono essere
utilizzati su relazioni per produrre relazioni. Possono essere composti dando
luogo a espressioni algebriche di complessità arbitraria.\\
\section{Operatori dell'algebra relazionale}
\paragraph*{Unarie}
\begin{itemize}
    \item Ridenominazione
    \item Selezione
    \item Proiezione
\end{itemize}
\paragraph*{Binarie}
\begin{itemize}
    \item Unione, Intersezione, Differenza (Operatori insiemistici)
    \item Join (Join naturale, Prodotto cartesiano, Theta-join)
\end{itemize}
\section{Operatori insiemistici}
Le relazioni sono insiemi e quindi è possibile applicare gli operatori insiemistici,
è fondamentale sapere che è possibile applicare queste operazioni \textbf{solo a relazioni
    definite sugli stessi attributi}.
\subsection*{Unione}
L'unione di due relazioni $r_1$ e $r_2$ è la relazione che contiene le tuple
che appartengono ad $r_1$ oppure ad $r_2$, oppure ad entrambe.\\
L'unione è commutativa e associativa.
\subsection*{Intersezione}
L'intersezione di due relazioni $r_1$ e $r_2$ è la relazione che contiene le tuple
che appartengono sia a $r_1$ che a $r_2$.\\
L'intersezione è commutativa e associativa ed è inoltre esprimibile per mezzo della
differenza:
\begin{equation*}
    r(X) = r_1(X) \cap r_2(X) = r_1(X) - (r_1(X) - r_2(X))
\end{equation*}
\subsection*{Differenza}
La differenze di due relazioni $r_1(X)$ e $r_2(X)$ definite su un insieme di attributi
X è la relazione $r(X) = r_1(X) - r_2(X)$ che contiene le tuple che appartengono a $r_1(X)$, ma
non a $r_2(X)$.\\
La differenza NON è commutativa.\\
\subsection*{Operatori insiemistici e valori nulli}
Gli operatori insiemistici sono definiti anche per relazioni che contengono valori nulli.\\
\section{Operatorri unari}
\subsection{Operatore di ridenominazione}
Per poter applicare operazioni insiemistiche come unione, intersezione, differenza a relazioni
su attributi in parte diversi è necessario ridenominare attributi, in modo da uniformare
i nomi. Questo viene fatto dall'operatore ridenominazione.\\
Si tratta di un operatore monadico (cioè un solo argomento) che modifica lo schema lasciando
inalterata l'istanza dell'operando. Cambia quindi il nome dell'attributo, ma non il valore.
\paragraph*{Sintassi} Si indica con $\rho_{y \leftarrow x}(r)$ o 
$\text{REN}_{y \leftarrow x}(r)$, dove x è
il nome originale dell'attributo, mentre y è quello nuovo. L'operatore è sempre seguito dal
nome della relazione che stiamo considerando.\\
\'E possibile rinominare più attributi, in questo caso è importante l'ordine degli attributi
dato che la sintassi sarà la seguente:
\begin{equation*}
    \rho_{y1, y_2 \leftarrow x_1, x_2}(r)
\end{equation*}
\paragraph*{Esempio}
\begin{tabular}{|c|c|c|}
    \hline
    \textbf{Padre} & \textbf{Figlio} \\
    \hline
    Adamo          & Abele           \\
    \hline
    Adamo          & Caino           \\
    \hline
\end{tabular}
$\text{REN}_{\text{Genitore}\leftarrow\text{Padre}}(\text{Paternità})$
\begin{tabular}{|c|c|c|}
    \hline
    \textbf{Genitore} & \textbf{Figlio} \\
    \hline
    Adamo          & Abele           \\
    \hline
    Adamo          & Caino           \\
    \hline
\end{tabular}\\
Questa operazione è fondamentale per poter effettuare operazioni insiemistiche tra
relazioni con attributi diversi, in questo modo possiamo uniformare i nomi degli attributi.
\subsection{Selezione}
Permette di selezionare un sottoinsieme delle ennuple, producendo un risultato che:
\begin{itemize}
    \item Ha lo stesso schema dell'operando
    \item Contiene un sottoinsieme delle ennuple dell'operando
    \item Contene le ennuple che soddisfano una condizione espressa dall'operatore
\end{itemize}
\paragraph*{Sintassi} $\sigma_{\text{condizione}}(r) \\$
\paragraph*{Sintassi alternativa}$\text{SEL}_{\text{condizione}}(r)$
Data una relazione r(X) è una formula ottenuta combinando con i connettivi OR, AND e NOT
condizioni atomiche del tipo:
\begin{itemize}
    \item CONFR è un operatore di confronto ($=, <, >, \geq, \leq$)
    \item A e B sono attributi in X sui cui valori CONFR abbia senso
    \item c'è una costante per cui il confronto CONFR sia definito
\end{itemize}
Il risultato contiene le ennuple dell'operando che soddisfano la condizione (cioè su cui
la condizione è vera).
\paragraph*{Esempi} Impiegati che:
\begin{itemize}
    \item Guadagnano più di 50 - STIPENDIO $>$ 50
    \item Guadagnano più di 50 e lavorano a Milano - STIPENDIO $>$ 50 AND FILIALE = 'Milano'
    \item Hanno un cognome uguale al nome della filiale presso cui lavorano - COGNOME = FILIALE
\end{itemize}
Tradotto in Query in Algebra Realzionale: $\text{SEL}_{\text{Stipendio} > 50}(\text{Impiegati})$,
lo stesso per le altre query, la parte scritta andrà sostituita nella parte condizione (sotto il SEL).
\subsection*{Selezione con valori nulli}
La condizione atomica è vera solo per valori non nulli.\\
Se per esempio effettuo una SEL su una tabella con valori nulli, e la condizione seleziona
tutti gli attributi (es. $\text{SEL}_\text{Età}>30 \cup \text{SEL}_\text{Età}\leq30$) 
il risultato sarà una tabella diversa da quella di partenza, perchè le condizioni atomiche
vengono valutate separatamente e i valori nulli non sono valori che possiamo confrontare con un numero
dato che rappresentano un valore di verità intermedio tra vero e falso. Anche inserendo tutto in una
unica SEL il risultato sarebbe il medesimo, quindi senza valori nulli. \\
Per questo esistono gli operatori \textbf{IS NULL e IS NOT NULL}. Per avere la tabella iniziale
per l'esempio Persone basterebbe quindi unire la precedente SEL con la seguente: 
$\text{SEL}_\text{Età IS NULL}(\text{Persone})$. In questo modo otteniamo la
stessa relazione di partenza dato che consideriamo anche i valori NULL.

\section{Proiezione}
Si occupa di selezionare solo alcune delle colonne della tabella presa in considerazione. \\
Per fare un confronto con il SEL:
\begin{itemize}
    \item SEL è un operatore ortogonale di decomposizione orizzontale, infatti
    riduce il numero di righe
    \item PROJ è un operatore ortogonale di decomposizione verticale, infatti
    riduce il numero di colonne
\end{itemize}
Si tratta anche in questo caso di un operatore monadico.
\paragraph*{Sintassi} $\text{PROJ}_{\text{lista di attributi}}(\text{Operando})$, il risultato
conterrà le ennuple dell'perando ristrette ai soli attributi nella ListaAttributi.
\subsection*{Proiezione e Valori Null}
Proiezione, unione e differenza continuano a comportarsi usualmente quindi due tuple sono uguali
anche se ci sono dei NULL. \\
Dato che una relazione è un insieme e un insieme non ha elementi uguali il risultato
della PROJ non conterrà ennuple uguali, esse saranno scartate.
\subsection*{Cardinalità delle proiezioni}La cardinalità di una relazione è il numero delle sue ennuple e
si indica con $|R|$.\\
Una proiezione:
\begin{itemize}
    \item Contiene al più tante ennuple quante l'operando
    \item Può anche contenerne di meno (come spiegato in precedenza)
\end{itemize}
Vale la proprietà che \textbf{se X è una superchiave di R, allora  $\text{PROJ}_X(R)$} contiene
esattamente tante ennuple quante R.\\
Per la definizione di superchiave ogni superchiave compare una sola volta nella relazione.\\
\subsection*{Selezione e Proiezione}
Combinando selezione e proiezione possiamo estrarre interessanti informazioni da una realzione
\paragraph*{Esempio}
\begin{table}[]
    \begin{tabular}{|l|l|l|l|}
    \hline
    Matricola & Cognome & Filiale & Stipendio \\ \hline
    7309      & Neri    & Napoli  & 55        \\ \hline
    5998      & Neri    & Milano  & 64        \\ \hline
    9553      & Rossi   & Roma    & 44        \\ \hline
    5698      & Rossi   & Roma    & 64        \\ \hline
    \end{tabular}
\end{table}
Ci viene richiesto matricola e cognome degli impiegati che guadagnano più di 50:
\paragraph*{Soluzione}
\begin{equation*}
    \text{PROJ}_{\text{Matricola, Cognome}}(\text{SEL}_{\text{Stipendio} > 50}(\text{Impiegati}))
\end{equation*}
Inserisco quindi come argomento della PROJ la SEL delle tuple richieste. Combinando questi due
operatori posso estrarre informazioni da una relazione. Non possiamo però correlare,
mettere insieme informazioni presenti in relazioni diverse, per questo esiste il JOIN.

\section{Join}
Il Join è senz'altro l'operatore più interessante dell'algebra relazionale dato che permette
di correlare, mettere insieme, integrare dati che si trovano in relazioni diverse. Ci sono
diversi tipi di Join, partiamo da quello naturale.
\subsection*{Join naturale}
Operatore binario (generalizzazione), produce un risultato sull'unione degli attributi
degli operandi con ennuple costruite ciascuna a partire da una ennupla di ognuno degli operandi.
\paragraph*{Sintassi} Date due relazioni $R_1(X_1)$ e $R_2(X_2)$, $R_1 \text{JOIN} R_2$ è una
relazione su $X_1, X_2$. Contribuiscono quindi le ennuple che hanno gli stessi valori negli attributi
comuni. Quando ogni ennupla contribuisce al risultato si dice \textbf{Join completo}.\\
Un Join è non completo quando ci sono attributi sulle due relazioni che non corrispondono fra di loro. Se
nessun attributo trova una corrispondenza si ottiene un Join vuoto.\\
Il join si indica anche con $\bowtie$.
\subsection*{Cardinalità del Join}
\begin{enumerate}
    \item Il Join di $R_1$ e $R_2$ contiene un numero di ennuple compreso fra zero e il prodotto di $|R_1|$
    e $|R_2|$
    \item Se il Join coinvolge una chiave di $R_2$ allora il numero di ennuple è compreso fra zero e $|R_1|$.
    \item Se B è chiave in $R_2$ ed esiste vincolo di integrità referenziale fra
    B (in $R_1$) e $R_2$, allora il numero di ennuple è uguale a $|R_1|$\\ $|R_1 \,\text{JOIN}\, R_2| = |R_1|$
\end{enumerate}
Il Join è commutativo e associativo.\\
Il Join naturale non combina due tuple se queste hanno entrambe valore nullo su un attributo in
comune (e valori uguali sugli eventuali altri attributi comuni).
\subsection*{Proprietà del Join}
In assenza di valor nulli l'intersezione di $r_1$ e $r_2$ si può esprimere
\begin{itemize}
    \item mediante il join naturale $r_1 \cap r_2 = r_1 \text{JOIN} r_2$ oppure
    \item sfruttando l'uguaglianza $r_1 \cap r_2 = r_1 - (r_1 - r_2)$
\end{itemize}
In presenza di valori nulli, dalle definizioni date si ha che:
\begin{itemize}
    \item nel primo caso il risultato non contiene tuple con valori nulli 
    \item nel secondo caso, viceversa, tali tuple compaiono nel risultato
\end{itemize}
Nel Join naturale le ennuple che non contribuiscono al risultato vengono tagliate fuori, per
questo viene utilizzato il JOIN esterno.
\subsection*{Join esterno}
Il Join esterno estende, con valori nulli, le ennuple che verrebbero escluse da un join del tipo
precedente (interno). Esiste in tre versioni:
\begin{itemize}
    \item Sinistro $=\bowtie$
    \item Destro $\bowtie=$
    \item Completo $=\bowtie=$
\end{itemize}
\paragraph*{Sinistro $\rightarrow$ LEFT} mantiene tutte le ennuple del primo operando, estenendole
con valori nulli, se necessario.
\paragraph*{Destro $\rightarrow$ RIGHT} mantiene tutte le ennuple del secondo operando, estenendole
con valori nulli se necessario.
\paragraph*{Completo $\rightarrow$ FULL} mantiene tutte le ennuple di entrambi gli operandi
estendendole con valori nulli se necessario.
\subsection*{Prodotto Cartesiano}
Un Join naturale su relazioni che non hanno attributi in comune restituisce un'istanza di 
relazione il cui schema contiene tutti i campi di R (nell'ordine originale) seguiti da tutti
i campi di S (nell'ordine originale). Contiene sempre un numero di ennuple pari al prodotto delle
cardinalità degli operandi (le ennuple sono tutte combinabili).\\
Il prodotto cartesiano ha senso solo se seguito da selezione (dato che produce dati non reali associando
anche tuple tra loro sconnesse a livello semantico). Questa operazione viene chiamata \textbf{theta-join}.
\subsection*{Theta-join}
L'operazione Theta-join viene indicata con $R_1 \,\, \text{JOIN}_{\text{Condizione}}R_2$.\\
La condizione C è spesso una congiunzione AND di atomi di confronto $A_1 \theta A_2$ dove $\theta$ è un
operatore di confronto ($>$, $<$, $=$, $\leq$, $\geq$, $\neq$).\\
Se l'operatore è sempre l'uguaglianza (=) allora si parla di equi-join.\\
\subsection*{Join naturale e Theta-join}
Così come è stato definito, il theta-join richiede in ingresso relazioni con schemi disgiunti,
mentre in diversi libri di testo, lavori scientifici e anche nei DBMS viceversa il theta-join 
accetta relazioni con schemi arbitrati e prende il posto del join naturale, ossia tutti i predicati
del join vengono esplicitati. In questo caso per garantire l'univocità degli attributi nello schema
risultato è necessario ridenominare gli attributi sovrapposti in una delle relazioni o adottare
dei trucchi.\\
Il join naturale utilizza implicitamente i nomi degli attributi per stabilire la condizione,
l'equijoin li indica esplicitamente. I DBMS tipicamente non permettono il join naturale
(solo ultime versioni di SQL lo permettono), però possiamo simularlo per mezzo di altri operatori.
\subsection*{Join e Intersezione}
Quando le due relazioni hanno lo stesso schema ($X_1 = X_2$) allora due tuple fanno match
se e solo se hanno lo stesso valore per tutti gli attributi, ovvero sono identiche, per cui
se $X_1 = X_2$ il join naturale equivale all'intersezione delle due relazioni.
\subsection*{Join e ridenominazione}
Per eseguire il Join di una relazione con se stessa in modo significativo bisogna usare la
ridenominazione.\\ $r(X) \,\, \text{JOIN} \,\,r(X) = r(X)$.\\

\section{Interrogazioni in algebra relazionale}
Dato uno schema R di base di dati, una interrogazione è una funzione che per ogni istanza r di R
produce una relazione su un dato insieme di attributi X.\\
\'E importante definire una metodologia per effettuare le Query per poter trasformare richieste verbali 
in algebra relazionale.
\paragraph*{Metodologia}
\begin{enumerate}
    \item Individua le relazioni coinvolte nella specifica dell'interrogazione, attraverso gli attributi citati e le
    condizioni
    \item Individua i tipi di operazioni necessarie
    \item Individua un possibile ordinamento delle operazioni che porta ad ottenere il risultato
    richiesto
\end{enumerate}
Se si hanno le tabelle sott'occhio è una buona idea effettuare l'interrogazione visivamente
per un solo attributo.\\
L'ordine delle operazioni è importante! Scambiando l'ordine di alcune operazioni potremmo ottenere
espressioni non funzionanti.\\
Esercitarsi rifacendo e capendo le query delle slide è un ottimo allenamento per assimilare
questi concetti.\\
In algebra relazionale non sono previsti i quantificatori universali, manca per esempio l'equivalente di
tutti (simbolo $*$ in SQL), in questi
casi è necessario rivedere la query e riformularla in modo tale da poterla esprimere in algebra relazionale, 
per esempio
esprimendo "tutti guadagnano più di n" come "almeno uno guadagna meno di n, oppure n", esprimendo
così l'operazione tramite differenza insiemistica.
\section{Le viste}
Si tratta di una relazione temporanea, paragonabile a una variabile definita a run time, che alla fine della
query non esiste più. Possono risultare molto comode per salvare risultati intermedi che poi andremo
a riutilizzare anche per altre query, in questo modo non dovremo più ricalcolare qulla query.
\paragraph*{Sintassi} NomeVista = Query
\section{Plus teorici}
\subsection{Rappresentazione delle espressioni tramite alberi}
Ogni espressione dell'algebra relazionale può essere rappresentata tramite un albero, in questo modo
rappresentiamo l'ordine di valutazione degli operatori. Ogni operatore corrisponde
ad un nodo, quindi gli operaetori unari hanno solo un ramo in ingrasso e uno in uscita, mentre
quelli binari hanno 2 rami in entrata e 1 in uscita, la radice è in alto.\\
\subsection{Equivalenza di espressioni}
Due espressioni sono equivalenti se producono lo stesso risultato.
\begin{itemize}
    \item Possono essere assolute se non dipendono dallo schema
    \item Oppure possono dipendere dallo schema
\end{itemize}
I risultati di due Query equivalenti sono sempre equivalenti, ma la scelta non è 
indifferente in termini di risorse necessarie, i risultati più interessanti sono quelli che
permettono una riduzione dei risultati intermedi portando a una semplificazione dell'espressione.
\section{Regole base equivalenza}
\begin{itemize}
    \item Il Join naturale è commutativo e associativo
    \item Selezione e proiezione si possono raggruppare
    \item Selezione e proiezioone commutano
    \item Push down della selezione rispetto al join
\end{itemize}
\section{Riassunto simboli}
\begin{itemize}
    \item Select - $\sigma$
    \item Proiezione - $\pi$ 
    \item Rename - $\rho$
    \item JOIN - $\bowtie$
    \item Prodotto cartesiano - $\times$
\end{itemize}