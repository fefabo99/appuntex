\chapter{Progettazione Logica}
A livello concettuale è la fase intermedia tra la progettazione concettuale e la progettazione fisica.
Essa ha diverse fasi:
\begin{itemize}
    \item Richiede di scegliere il modello dei dati - Relazionale
    \item Obiettivo - Definizione di uno schema logico relazionale corrispondente allo
    schema ER di partenza
    \item Aspetti importanti - Semplificazione dello schema per renderlo rappresentabile
    mediante il modello relazionale, ottimizzandolo per aumentare l'efficienza delle
    interrogazioni
\end{itemize}
L'obiettivo primarioè quello di tradurre lo schema concettuale in uno schema logico
che rappresenti gli stessi dati in maniera corretta ed efficiente.\\
\paragraph*{Dati in ingresso}
\begin{itemize}
    \item Schema concettuale
    \item Informazioni sul carico applicativo
    \item Modello logico
\end{itemize}
\paragraph*{Dati in uscita}
\begin{itemize}
    \item Schema logico
    \item Documentazione associata
\end{itemize}
Non si tratta di una pura e semplice traduzione, dato che alcuni aspetti non
sono direttamente rappresentabili.
\paragraph*{Esempi} 
\begin{itemize}
    \item Entità $\rightarrow$ Relazione del modello relazionale con gli 
    stessi attributi.
    \item Generalizzazione $\rightarrow$ Dipende dalla situazione!
\end{itemize}
Risulta anche necessario considerare le prestazioni.\\
Qui di seguito lo schema riassuntivo conversione ER $\rightarrow$ Modello relazionale.\\
\begin{table}[h]
    \centering
    \vspace{10pt}
    \caption*{Schema riassuntivo conversione ER - Modello relazionale}
    \begin{tabular}{|c|c|}
        \hline
        \textbf{Modello ER} & \textbf{Modello relazionale}\\
        \hline
        Entità & Relazione (tabella)\\
        \hline
        Relazione & Riferimento (chiave esterna)\\
        \hline  
        Attributo semplice & Attributo (campo)\\
        \hline
        Attributo multivalore & Non presente\\
        \hline
        Generalizzazione & Non presente (conversione in diverse modalità)\\
        \hline
    \end{tabular}
\end{table}
\section{Ristrutturazione dello schema ER}
Eliminazione dallo schema E/R di tutti i costrutti che non possono
essere direttamente rappresentati nel modello logico target (relazionale nel
nostro caso).
\begin{itemize}
    \item Eliminazione attributi multivalore
    \item Eliminazione generalizzazioni
\end{itemize}
Inoltre:
\begin{itemize}
    \item Partizionamento/Accorpamento di entità associazioni
    \item Scelta degli identificatori primari
    \item Analisi ridondanze (non dovrebbero esserci)
\end{itemize}
L'obiettivo è quello di semplificare la traduzione e ottimizzare le prestazioni,
teniamo presente che uno schema E-R ristrutturato non è più uno schema concettuale nel senso stretto
del termine.\\
Per ottimizzare il risultato abbiamo bisogno di analizzare le prestazioni a questo livello, ma
le prestazioni non sono valutabili con precisione su uno schema concettuale, dipendono dalle
caratteristiche del DBMS, dal volume dei dati e dalla caratteristiche delle operazioni.\\
\subsection*{Indicatori dei parametri di prestazioni}
Consideriamo indicatori dei parametri che regolano le prestazioni:
\begin{itemize}
    \item spazio: numero di occorrenze previste
    \item tempo: numero di occorrenze (di entità e relationship) viste durante un'operazione
\end{itemize}
\subsection{Carico applicativo}


