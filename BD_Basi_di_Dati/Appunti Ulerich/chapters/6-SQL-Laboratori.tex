\chapter{SQL - Structured Query Language}
SQL è un linguaggio per la definizione e la manipolazione dei dati
in database relazionali adottato da molti DBMS. \\
Ci sono diverse versioni, la prima versione ufficiale risale al 1986.
Poi sono state rilasciate altre versioni come SQL-89, SQL-2, SQL-3\dots\\
Noi faremo riferimento principalmente a SQL-2. Questa versione è ricca e complessa,
tanto che nessun sistema commerciale lo implementa in maniera completa.\\
Esistono 3 livelli di conformità:
\begin{itemize}
  \item Entry level: molto simile a SQL-89
  \item Intermediate level: versione che soddisfa le esigenze di mercato
  \item Full level: versione completa anche delle funzioni avanzate che non sono realizzate
        in alcun DBMS
\end{itemize}
La maggior parte dei database è conforme solo all'entry level.\\
Alcune famose implementazioni di SQL sono:
\begin{itemize}
  \item ORACLE
  \item DB2 (IBM)
  \item Access (Microsoft)
  \item MSSQL server (Microsoft)
  \item MySQL
  \item Firebird
\end{itemize}
\section{Confronto con Algebra Relazionale e Istruzioni principali}
\subsection{SQL e Algebra Relazionale}
SQL è relazionalmente completo: ogni espressione logica può essere tradotta in SQL.
Viene adottata la logica dei 3 valori (T, F, U) dell'Algebra relazionale (U = Unknown).\\
Il modello dati di SQL è basato su tabelle anzichè relazioni (possono essere presenti righe duplicate).\\
SQL è computazionalmente completo, ha istruzioni di controllo.\\
\paragraph*{Logica a 3 valori} Logica che utilizza:
\begin{itemize}
  \item True
  \item False
  \item Unknown
\end{itemize}
Che segue la seguente tabella di verità
\begin{table}[h]
  \centering
  \caption{SQL Three-Value Logic Truth Table}
  \begin{tabular}{|c|c|c|c|}
    \hline
    \textbf{Value 1} & \textbf{Value 2} & \textbf{AND} & \textbf{OR} \\
    \hline
    TRUE             & TRUE             & TRUE         & TRUE        \\
    \hline
    TRUE             & FALSE            & FALSE        & TRUE        \\
    \hline
    TRUE             & UNKNOWN          & UNKNOWN      & TRUE        \\
    \hline
    FALSE            & TRUE             & FALSE        & TRUE        \\
    \hline
    FALSE            & FALSE            & FALSE        & FALSE       \\
    \hline
    FALSE            & UNKNOWN          & FALSE        & UNKNOWN     \\
    \hline
    UNKNOWN          & TRUE             & UNKNOWN      & TRUE        \\
    \hline
    UNKNOWN          & FALSE            & FALSE        & UNKNOWN     \\
    \hline
    UNKNOWN          & UNKNOWN          & UNKNOWN      & UNKNOWN     \\
    \hline
  \end{tabular}
\end{table}

\subsection{Istruzioni principali DDL}
Operazioni di definizione schema e modifica
\begin{itemize}
  \item CREATE - Definisce database, tabelle, domini, viste, vincoli e autorizzazioni
  \item ALTER - Modifica attributi e vincoli
  \item DROP - Elimina database e tabelle
\end{itemize}
\subsection{Istruzioni principali DML}
Operazione di interrogazione: SELECT. Formula Query come nell'AR o anche
richieste più elaborate.\\
\subsection*{Aggiornamento}
\begin{itemize}
  \item INSERT - Inserisce nuove tuple nelle tabelle
  \item DELETE - Elmina tuple nelle tabelle
  \item UPDATE - Modifica tuple
\end{itemize}
Queste ultime possono basarsi sul risultato di una query.
\subsection{SQL è dichiarativo}
Essendo principalmente dichiarativo non possiamo scegliere l'ordine in cui
avvengono le operazioni è necessario attenersi alla struttura sintattica delle
istruzioni.
\subsection{Notazione SQL}
\begin{itemize}
  \item Termini del linguaggio in MAIUSCOLO
  \item Termini variabili (specificati dall'utente) in minuscolo
  \item <x> usate per isolare un termine x
  \item [x] indicano che il parametro è opzionale
  \item | seprara opzioni alternative
\end{itemize}

\subsection{Primo esempio di Query}
\begin{lstlisting}[language=SQL]
  SELECT ListaAttributi
  FROM ListaTabelle
  [WHERE Condizione]
\end{lstlisting}
Dove SELECT e FROM sono obbligatori, mentre WHERE è opzionale (in realtà c'è praticamente
sempre in quanto serve per filtrare i risultati).\\
\section{SQL-DDL}
Uno schema di base di dati è una collezione di oggetti: domini, tabelle,
asserzioni, viste, privilegi. La sintassi è la seguente:
\begin{lstlisting}[language=SQL]
  CREATE SCHEMA 
    [Nome Schema]
    [[AUTHORIZATION] Autorizzazione]
    {DefinizioneElementoSchema}
\end{lstlisting}
Nome schema se omesso indica l'utente che ha lanciato il comando, mentre
autorizzazione è il proprietrio dello schema (utente che lo ha definito).\\
\subsection{Tabelle}
Tramite CREATE TABLE si definisce una tabella. Una tabella non è altro che uno
schema di relazione, con CREATE viene creata un'istanza vuota.
\begin{lstlisting}[language=SQL]
  CREATE TABLE NomeTabella
    (NomeAttributo1 TipoDominio1 [Valore di Default] [Vincolo1],
    NomeAttributo2 TipoDominio2 [Valore di Default] [Vincolo2],
    ...
    [AltriVincoli])
\end{lstlisting}
In questo ambito parleremo di tabelle e non di relazioni, e di righe e non di tuple,
perchè rispetto al modello relazionale possiamo avere righe duplicate.\\
Qui di seguito un esempio reale di definizione di tabella:
\begin{lstlisting}[language=SQL]
  CREATE TABLE Impiegato(
    Matricola CHAR(6) PRIMARY KEY,
    Nome CHAR(20) NOT NULL,
    Cognome CHAR(20) NOT NULL,
    Dipart CHAR(15),
    Stipendio NUMERIC(9) DEFAULT 0,
    FOREIGN KEY(Dipart) REFERENCES
    Dipartimento(NomeDip),
    UNIQUE (Cognome,Nome)
)
\end{lstlisting}
\subsection{Definizione dei Dati: I Domini}
I domini specificano i vlori ammessi da ciascun attributo. SQL ha 
6 domini elementari predefiniti:
\begin{itemize}
  \item Carattere- VARCHAR - Stringa di lunghezza variabile tra 0 e n
  \item Numerico Esatto - INTEGER, SMALLINT, NUMERIC
  \item Numerico Approssimato - FLOAT, REAL, DOUBLE PRECISION
  \item Data/Ora TIMESTAMP, DATE, TIME
  \item Intervallo Temporale
  \item Bit (SQL-2) - BOOLEAN (SQL-3) - con dominio 0,1
\end{itemize}
Bit è stato poi eliminato e sostituito parzialmente da BOOLEAN in SQL-3\\
L'utente pu definire dei domini custom (semplici, ma riutilizzabili).\\
\paragraph*{BLOB, CLOB} sono oggetti di grandi dimensioni, costituiti da binari (BLOB)
o caratteri (CLOB), vengono memorizzati in maniera differente dagli altri dati e sono usati per memorizzare
informazioni non strutturate (immagini, video, testi\dots).\\
\subsection{Il tipo Bit}
Utilizzato spesso per definire se è presente o meno una certa proprietà, dato
che si tratta di un BOOLEAN.
\subsection{Carattere}
Si indica:
\begin{lstlisting}[language=SQL]
  CHAR(n) - Stringa di lunghezza fissa n
  VARCHAR(n) - Stringa di lunghezza variabile tra 0 e n
\end{lstlisting}
\subsection{Numerici Esatti}
Rappresentano numeri interi o numeri decimali in virgola fissa (con un numero prefissato
di decimali, come per i valori monetari). Precisionè il numero di cifre significative,
scala il numero di cifre dopo la virgola.\\
\begin{itemize}
  \item INTEGER/SMALLINT rappresentano valori interi - La precisione varia a seconda della
  specifica implementazione di SQL, SMALLINT richiede meno spazio di memorizzazione
  \item NUMERIC/DECIMAL rappresentano valori decimali - La differenza fra questi due è che
  il primo deve essere implementato esattamente con la precisione richiesta, mentre il secondo
  può avere una precisione maggiore.
\end{itemize}
\subsection{Numerici Approssimati}
Sono utili per rappresentare valori reali approssimati, ad esempio grandezze
fisiche (rappresentazione in virgola mobile, in cui a ciascun numero corrisponde
una coppia di valori: mantissa e esponente).\\
REAL e DOUBLE PRECISION rappresentano valori a singola/doppia precisione in virgola
mobile. FLOAT permette di richiedere la precisione che si desidera.
\subsection{Data e Ora}
Permettono di descrivere informazioni temporali, rappresentando istanti di tempo:
\begin{itemize}
  \item DATE rappresenta le date espresse come anno (4 cifre), mese (2 cifre) e giorno (2 cifre)
  - DATE 'yyyy-mm-dd'
  \item TIME [WITH TIME ZONE] rappresenta l'ora del giorno, espressa come ora (2 cifre), 
  minuti (2 cifre) e secondi (2 cifre)
  \item TIMESTAMP
\end{itemize}
Ciascuno di questi domini è strutturato e decomponibile in un insieme di campi 
(anno, mese, giorno, ora, minuti, secondi).
\subsection{Intervalli temporali}
Permette di rappresentare intervalli di tempo come durate di eventi.\\
INTERVAL rappresenta una durata temporale, esistono interval ani e mesi, oppure
giorni e ore, ma non mesi e giorni poichè i mesi non hanno tutti lo stesso numero di
giorni.
\subsection*{CLOB e BLOB}
Permettono di includere nel database oggetti molto grandi (come dati multimediali).
Sono implementati come valore e non possono essere usati come criterio di selezione per le
query.
\subsection*{Domini definiti dall'utente}