\chapter{Introduzione al corso}
Turno T1 - AL
\\Da cancellare, per Elia, turno T1 AL ma ammettono anche MZ alle lezioni. L'importante è che gli AL sostengono l'esame con il prof di AL e gli MZ con il prof di MZ.
\section{I professori}
\begin{itemize}
    \item prof. Napoletano, di teoria (aula 1014 in U14).
    \\Per eventuali comunicazioni, mail: paolo.napoletano@unimib.it, ma \textbf{tassativamente} bisogna aggiungere la sigla [DB].
    \item prof.ssa Damiani, di esercitazione.
    \item prof. Raganato, di laboratorio.
\end{itemize}

\section{Organizzazione del corso}
\begin{itemize}
    \item lezioni teoriche ed analisi di casi studio (32 ore, pari a quattro crediti) PN 
    \item esercitazioni ed analisi casi studio (20 ore, pari a due crediti). CD 
    \item Laboratorio / esercitazioni (20 ore, pari a crediti). AR
\end{itemize}

\section{Organizzazione degli orari}
\begin{itemize}
    \item mercoledì: comincia a 10:30 spaccate
    \item giovedì: da definire perché c'è una lezione prima (statistica)
    \item venerdì lez: non ho capito io, ma per ora non li fa (marzo)
    \item venerdì lab: boh lo diranno.
\end{itemize}

\section{Organizzazione degli esami}
\subsection{Esoneri}
\begin{itemize}
    \item Ci sono parziali, circa seconda metà di aprile (\textbf{due} es di progettazione concettuale e modello relazionale) e seconda metà di giugno (\textbf{tre} es di SQL, algebra relazionale, progettazione logica).
    \item Per ora, i parziali sono aperti a tutti, ma il prof Napoletano deve sentire il prof Schettini del turno MZ.
\end{itemize}

\subsection{Totale}
\begin{itemize}
    \item Tutto il programma (quindi \textbf{cinque} es) è previsto per l'esame complessivo.
\end{itemize}

\subsection{Laboratorio}
\begin{itemize}
    \item A frequenza facoltativa ma essenziale, prevede una prova unica (sempre facoltativa) alla fine del laboratorio che permettere di avere un punteggio $-1 \leq p \leq 3$ da aggiungere alla media dei parziali o al voto del totale.
    \item Si lavorerà su MySQL, e prevede la progettazione concettuale e logica di una base di dati assegnata utilizzando lo strumento di Data Modeling fornito da MySQL.
    \item Il voto del laboratorio rimane valido tutto l'anno accademico, ovvero fino a Febbraio 2024 compreso, e andrà sempre sommato al voto preso.
\end{itemize}

\subsection{Valutazione}
\begin{itemize}
    \item Per superare l'esame, il voto minimo \textit{per parziale} è di \textbf{15/30}.
    \item Per superare l'esame, la \textit{media dei parziali} deve essere di \textbf{18/30}.
    \item Verrà eventualmente sommato il voto del laboratorio (se sostenuto).
\end{itemize}

\section{Il corso}
Le lezioni non verranno registrate, ma sono disponibili quelle dell'anno precedente.

\subsection{Il programma}
\begin{enumerate}
    \item Introduzione $\rightarrow$ prof. Napoletano
    \item Metodologie e modelli per il progetto delle basi di dati $\rightarrow$ prof. Napoletano
    \item La progettazione concettuale $\rightarrow$ prof. Napoletano
    \item Il modello relazionale \textbf{\textit{(pausa compitini)}} $\rightarrow$ prof. Napoletano
    \item SQL $\rightarrow$ prof.ssa Damiani
    \item Algebra relazionale $\rightarrow$ prof.ssa Damiani
    \item La progettazione logica $\rightarrow$ prof. Napoletano
\end{enumerate}
L'ordine degli argomenti è diverso da quello suggerito dal libro.

\subsection{Cosa vedremo}
Il corso e' dedicato a capire come e' organizzata una base di dati, a cosa serve, come si progetta, come si interroga e si crea.
% tre immagini da inserire

\chapter{Basi di Dati}
Una cosa che impareremo sarà \textbf{organizzare} il lavoro che ci viene presentato, per esempio da un eventuale cliente. Per prima cosa ci servirà un \textit{foglio dei requisiti}, ciò che dobbiamo progettare e sviluppare. Poi dovremo tradurre il passaggio da input ad output in una schematizzazione o mappa.

\section{Introduzione e definizioni}
\subsection{Risorse}
Le risorse di una azienda (o ente, amministrazione):
\begin{itemize}
    \item persone
    \item denaro 
    \item materiali 
    \item informazioni
\end{itemize}

\subsection{Basi di Dati}
Insieme organizzato di dati utilizzati per il supporto allo svolgimento di attività (di un ente, azienda, ufficio, persona).

\section{Sistemi informativi e sistemi informatici - una premessa}
Che cos'è l'informatica? Una definizione:
\begin{itemize}
    \item \textit{Scienza del trattamento razionale, specialmente per mezzo di macchine automatiche, dell'informazione, considerata come supporto alla conoscenza umana e alla comunicazione (Academie Francaise).}
    \item L'informatica ha due anime:
    \\- metodologica: i metodi per la soluzione di problemi e la gestione delle informazioni;
    \\- tecnologica: i calcolatori elettronici e i sistemi che li utilizzano; 
\end{itemize}
\begin{description}
    \item N.B.: \textbf{sistema informativo} $\neq$ \textbf{sistema informatico}
    \item[Sistema informativo] Componente (sottosistema) di una organizzazione che gestisce le informazioni di interesse (cioè utilizzate per il perseguimento degli scopi dell'organizzazione), le cui funzioni sono:
    \\ - acquisizione/memorizzazione
    \\ - aggiornamento
    \\ - interrogazione
    \\ - elaborazione
    \\N.B.: Il concetto di “sistema informativo” è indipendente da qualsiasi automatizzazione!! Anche prima di essere automatizzati, molti sistemi informativi si sono evoluti verso una razionalizzazione e standardizzazione delle procedure e dell'organizzazione delle informazioni.
    \item[Sistema informatico] Porzione automatizzata del sistema informativo: in pratica è la parte del sistema informativo che gestisce le informazioni con tecnologia informatica.
    % inserisci immagine sistemi_informatici_e_informativi1
\end{description}
Ma perché le basi di dati sono così importanti? Proviamo a definirle con degli aggettivi o caratteristiche che spieghino come mai sono così interessanti:
\begin{itemize}
    \item accessibili: le informazioni sono archiviate in modo ordinato;
    \item capienti: sono storate grandi quantità di dati;
    \item (facili da modificare;)
    \item ottimizzate: rapida ricerca delle informazioni; (comune a più sistemi)
    \item possibilità di raggruppare/filtrare le informazioni e schematizzarle/modellizzarle;
    \item sicurezza dei dati:;
    \item facilità di relazione dei dati;
    \item interfaccia per visualizzare in diversi modi;
    \item personalizzabili;
    \item scalabilità (Elia questa me la spieghi);
    \item interoperabilità: lavorabile con più linguaggi e strumenti;
    \item accesso concorrente alle informazioni: più persone possono lavorare allo stesso database o alla stessa sottosezione senza andare incontro ad inconsistenza dei dati;
    \item facilità di gestione delle ridondanze, che aiuta a ridurre al minimo l'inconsistenza dei dati;
    \item limitazione dell'inconsistenza dei dati: devono sempre essere consistenti, ovvero accessibili solo a chi ha diritto di farlo, il ruolo giusto;
\end{itemize}
Ma perché non usare una cosa più semplice come un FileSystem invece di un Database? Il primo mi aiuta con un'organizzazione logica, ma i Database sono dotati  di strumenti (tipo la progettazione modulare) che sono più efficienti.

\subsection{Sistema Informatico}
Gestisce un sistema informativo in modo automatizzato.
\\Garantisce che i dati siano conservati in modo permanente sui dispositivi di memorizzazione.
\\Permette un rapido aggiornamento dei dati per riflettere rapidamente le loro variazioni.
\\Rende i dati accessibili alle interrogazioni degli utenti.
\\Può essere distribuito sul territorio.

\section{Gestione delle informazioni}
Parole chiave:
\begin{itemize}
    \item Raccolta, acquisizione
    \item Archiviazione, conservazione
    \item Elaborazione, trasformazione, produzione
    \item Distribuzione, comunicazione, scambio
\end{itemize}

Nelle attività umane, le informazioni vengono gestite (registrate e scambiate) in forme diverse:
\begin{itemize}
    \item idee informali
    \item linguaggio naturale (scritto o parlato, formale o colloquiale, in una lingua o in un'altra)
    \item disegni, grafici, schemi
    \item numeri e codici
\end{itemize}
e su vari supporti
\begin{itemize}
    \item memoria umana, carta, dispositivi elettronici
\end{itemize}
Nelle attività standardizzate dei sistemi informativi complessi, sono state introdotte col tempo forme di organizzazione e codifica delle informazioni via via più precise (e in un certo senso artificiali).

\section{Informazioni e dati}
\textbf{Informazioni} $\neq$ \textbf{Dati}
\\Nei sistemi informatici (e non solo), le informazioni vengono rappresentate in modo essenziale, spartano: \textbf{attraverso i dati}.
\begin{itemize}
    \item \textbf{informazione}: notizia, dato o elemento che consente di avere conoscenza più o meno esatta di fatti, situazioni, modi di essere.
    \item \textbf{dato}: ciò che è immediatamente presente alla conoscenza, prima di ogni elaborazione; (in informatica) elementi di informazione costituiti da simboli che debbono essere elaborati. 
    \\I dati hanno bisogno di essere interpretati.
\end{itemize}
\textit{Esempio}:
\\'Mario' e '275' su un foglio di carta sono due dati. 
\\Se il foglio di carta viene fornito in risposta alla domanda “A chi mi devo rivolgere per il problema X; qual è il suo numero di telefono?”, allora i dati possono essere interpretati per fornire informazione e arricchire la conoscenza.
\\
\\\textbf{Ma perché i dati?}
\\La rappresentazione precisa di forme più ricche di informazione e conoscenza è difficile.
\\I dati costituiscono spesso una risorsa strategica, perché più stabili nel tempo di altre componenti (processi, tecnologie, ruoli umani)
\\I dati rimangono gli stessi nella \textit{migrazione} da un sistema al successivo.

\section{Basi di Dati}
\begin{description}
    \item[DB:] \textbf{Data Base} Collezione di dati utilizzati per rappresentare le informazioni di interesse di un sistema informativo.
    \item[DBMS:] \textbf{Data Base Management System}. Sistema software capace di gestire collezioni di dati che siano grandi, condivise e persistenti, assicurando la loro affidabilità e privatezza.
\end{description}
Accezione generica, \textbf{metodologica}: insieme organizzato di dati utilizzati per il supporto allo svolgimento delle attività di un ente (azienda, ufficio, persona).
\\Accezione specifica, \textbf{metodologica} e \textbf{tecnologica}: insieme di dati gestito da un DBMS.

\subsection{Altra definizione}
    Possiamo definire una BdD anche come: insieme di archivi in cui ogni dato e' rappresentato logicamente una sola volta e puo' essere utilizzato da un insieme di applicazioni da diversi utenti secondo opportuni criteri di riservatezza.

\subsection{caratteristiche}
\begin{itemize}
    \item i dati sono molti
    \item i dati hanno un formato definito
    \item i dati sono permanenti
    \item i dati sono raggruppati per insiemi omogenei di dati
    \item esistono relazioni specifiche tra gli insiemi di dati
    \item la ridondanza è minima e controllata: è assicurata la consistenza delle informazioni
    \item i dati sono disponibili per utenze diverse e concorrenti (anche contemporanee)
    \item i dati sono controllati: protetti da malfunzionamenti hardware e software 
    \item indipendenza dei dati dal programma    
\end{itemize}

\subsection{Perché studiare le basi di dati?}
Copia.

\subsection{Basi di dati multimediali}
Mi sono persa tutto.

\subsection{Data Base Management System - DBMS}
Un DBMS è un insieme di programmi che permettono di creare, usare e gestire una base di dati.
\\Quindi un DBMS è un sistema software general purpose che facilita il processo di definizione, costruzione e manipolazione del database per varie applicazioni.

\subsection{Creazione di un database}
\textbf{Tre fasi:}
\begin{itemize}
    \item definizione
    \item creazione/popolazione
    \item manipolazione
\end{itemize}

\subsection{Interrogazione di un database}
ziofrass
\begin{lstlisting}[mathescape=true]
    SELECT [Nome], [Cognome], [Indirizzo],
        [Citta]
    FROM Studenti
    WHERE [Cognome]="Rossi";
\end{lstlisting}

L'efficacia della query dipende da:
\begin{itemize}
    \item conoscenza del contenuto del db
    \item esperienza del linguaggio di interrogazione
\end{itemize}
oppure
\begin{itemize}
    \item semplicità ed efficacia dell'interfaccia di interrogazione
\end{itemize}

DataBase Management System
(DBMS)
• Sistema che gestisce collezioni di dati:
- grandi
- persistenti
- condivise
garantendo
- privatezza
- affidabilità
- efficienza
- efficacia

Hanno grandi dimensioni: dimensioni (molto) maggiori della memoria centrale dei sistemi di calcolo utilizzati; il limite deve essere solo quello fisico dei dispositivi.
Sono persistenti: hanno un tempo di vita indipendente dalle singole esecuzioni dei programmi che le utilizzano.
Sono condivise: ogni organizzazione (specie se grande) è divisa in settori o comunque svolge diverse attività; ciascun settore/attività ha un (sotto)sistema informativo (non necessariamente disgiunto).

\textbf{MI SONO PERSA UNA QUINDICINA DI SLIDES}

\section{Schemi e istanze}
In ogni base di dati esistono:
- lo schema, sostanzialmente invariante nel
tempo, che ne descrive la struttura, il
significato (aspetto intensionale).
• nell'esempio, le intestazioni delle tabelle
- l'istanza, i valori attuali, che possono
cambiare anche molto rapidamente (aspetto
estensionale)
• nell'esempio, il “corpo” di ciascuna tabella

lo schema costituisce l'aspetto intensionale,
ovvero la descrizione "astratta" delle
proprietà, ed è invariante nel tempo.
• L'istanza (i valori degli attributi) costituiscono
invece l'aspetto estensionale "concreto", che
varia nel tempo al variare della situazione di
ciò che stiamo descrivendo

Perso altre cinque SLIDES
ho rinunciato
ma cerca il sito DB engines rankings


\chapter{Argomento 2}

