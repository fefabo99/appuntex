% 10/10/2024

\chapter{Esercitazione 1}
\section{Passi su Android Studio}
\begin{itemize}
    \item New Project
    \item Empty Views Activity
    \item Scegliamo linguaggio Java
\end{itemize}
\par Da terminale (seconda icona dal basso a sinistra) possiamo scrivere:
\begin{enumerate}
    \item \texttt{git init} per inizializzare una repository (fatta dentro perché ci lavoro dentro Android Studio e non da VSCode come faccio con tutte le altre repository)
    \item \texttt{levels} per vedere i livelli di git
    \item \texttt{git commit} per fare un commit
    \item \texttt{git branch nomebranch} per creare un nuovo branch
    \item \texttt{git checkout} si aspetta il nome del branch
    \item \texttt{git checkout nomebranch1} per passare al branch nomebranch1
    \item \texttt{git checkout nomebranch2} per passare al branch nomebranch2\\
    Se faccio commit su modifiche non pushate, rischio di andare incontro a conflitti.\\
    L'asterisco (sul sito \url{https://learngitbranching.js.org/}) mi indica su quale branch mi trovo.
    \item \texttt{git merge nomebranch} per fare il merge in nomebranch del branch in cui mi trovo (quello attivo in quello più stabile)
    \item \texttt{git rebase nomebranch} per fare il rebase in nomebranch del branch in cui mi trovo: è un'alternativa a merge, "riscrive la storia", collassa tutti i branch in uno cancellando la storia di quello in cui mi trovo e mettendola in quello in cui voglio fare il rebase.
    \item \texttt{git revert nomedelcommit} per fare il revert di un commit, tornare al commit (es. c6) e cancellare le modifiche da quel commit in poi, magari per eliminare branch che ho fatto inutilmente.
\end{enumerate}

\subsection{Sito per simulare funzionamento di git}
\url{https://learngitbranching.js.org/}

\subsection{Comandi git da terminale}
\par Le commit, partendo dalla cartella del progetto:
\begin{itemize}
    \item \texttt{git pull}
    \item \texttt{git add .}: mette tutto ciò che c'è nella working area nella staging area, il punto mi dice "tutti i file"
    \item \texttt{git commit -m "messaggio in cui dico cosa ho fatto in breve"}: mette tutto ciò che c'è nella staging area nella commit area
    \item \texttt{git push}
\end{itemize}

\subsection{I conflitti}
\par In caso di conflitti, per arrivare alla pagina di risoluzione dei conflitti (al centro versione locale, a sinistra un branch e a destra l'altro), quando si verifica un conflitto esce un pulsante "risolvi conflitti" magari in inglese. Tendenzialmente il main è la versione più stabile, quindi si fa il merge del branch in cui si è lavorato in main.
\par Consiglia di fare branch per funzionalità e non per persona, però è un approccio personale e comunque l'importante è essere consistenti.


\par Git Ignore è un file che va ad appuntare estensioni di file o file o cartelle da ignorare. Al momento dell'inizializzazione del progetto avremo già una lista di file di base ignorati. Di solito i file compilati (es su Java i \texttt{.class}) sono da ignorare.
\par Il sito \texttt{gitignore.io} permette di generare un file \texttt{.gitignore} in base al linguaggio di programmazione che si sta utilizzando con i file da ignorare. Metto Android, Windows. Consiglia di farlo all'inizio anche per alleggerire la compilazione.

\par GitHub e GitLab sono istanze di git, mentre git è il programma generico.

\subsection{Il progetto}
\par Prima cosa da fare: avviare la repository, il gitignore, il README. Decidere come gestire i branch (non valutato, ma l'approccio deve essere consistente) se per team o funzionalità o cosa, vedi online. Dice che ci dovrebbero essere consigli. Meglio tenere il main branch come quello che funziona di più.
\par Per creare una nuova repository su GitHub:
\begin{itemize}
    \item scelgo nome e descrizione
    \item scelgo se pubblica o privata (per il progetto può anche essere privata perché tanto posso aggiungere dopo chi può accedervi)
    \item posso scegliere se aggiungere un README (che è un file markdown che appare nella home della repository) e un \texttt{.gitignore} (che posso scegliere in base al linguaggio di programmazione che sto usando), ma sono cose che posso aggiungere anche dopo
    \item posso scegliere se installare la repository tramite set up desktop, da linea di comando (sono i comandi che abbiamo visto prima) o pushare da linea di comando una repo già esistente
    \begin{itemize}
        \item l'unica riga che non abbiamo visto è \texttt{git remote add origin url}, che serve per collegare la repository locale a quella remota
    \end{itemize}
\end{itemize}