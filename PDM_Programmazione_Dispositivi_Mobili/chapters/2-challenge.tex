% 03/10/2025

\section{Challenge nello sviluppo di app mobili}
\subsection{Caratteristiche che un'app deve avere}
\par Un'app deve essere:
\begin{itemize}
    \item semplice
    \item economica
    \item addicting, chi la scarica deve volerla tenere
\end{itemize}

\subsection{Sfide per un programmatore}
% \begin{itemize}
%     \item 
%     \item 
%     \item 
% \end{itemize}

\subsection{Idea \& business}
\par \textbf{Da dove nasce un'app di successo?}
\begin{itemize}
    \item Un \textbf{bisogno} reale da soddisfare (ma che non sia solo una cosa utile a me, ma a più persone)
    \item Un \textbf{problema} quotidiano da semplificare (es. app che esistono già ma non)
    \item Un'\textbf{esperienza utente} migliore rispetto a soluzioni già esistenti
\end{itemize}
\par Non basta "avere un'idea": serve capire \textbf{perché} qualcuno la userebbe.
\par Suggerimenti:
\begin{itemize}
    \item Effettuare \textbf{analisi di mercato}: deve essere un'app che l'utente vuole.
    \item Renderla \textbf{attrattiva}: in modo che l'utente decida di tenerla.
    \item Renderla \textbf{semplice} da usare.
\end{itemize}

\par \textbf{Quanto costa sviluppare un'app?}
\par Chiaramente non è solo "scrivere codice", è un investimento continuo.
\begin{itemize}
    \item \textbf{Costi diretti} (fondamentali): sviluppo vero e proprio (Android/iOS), backend, infrastruttura cloud.
    \begin{itemize}
        \item Quale piattaforme l'app intende supportare?
        \begin{itemize}
            \item Più piattaforme $\rightarrow$ costi più alti 
            \item Fortunatamente non sempre proporzionali, grazie al \textit{riutilizzo di codice}
            \item Prima si realizza l'architettura, poi si pensa ad una piattaforma, poi si passa ad altre eventualmente
        \end{itemize}
    \end{itemize}
    \item \textbf{Costi indiretti} (non incidono direttamente sullo sviluppo ma devo tenerne conto comunque): design UI/UX, testing, aggiornamenti, manutenzione.
    \begin{itemize}
        \item \textbf{Account dello sviluppatore} (developer account)
        \begin{itemize}
            \item \textit{Apple} abbiamo detto costa 99\$ all'anno per account individuali mentre 299\$ all'anno per gli Enterprise accounts (aziendali)
            \item \textit{Google} costa 25\$ una tantun per un account sviluppatore
        \end{itemize}
        \item Poi abbiamo \textbf{componenti server-side e servizi Cloud}
        \begin{itemize}
            \item Se l'app richiede di archiviare e recuperare dati, allora occorre aggiungere costi per:
            \begin{itemize}
                \item archiviazione
                \item hosting
                \item nome di dominio
                \item pagine di destinazione
                \item backup dei dati
                \item etc\dots
            \end{itemize}
            \item Se l'app utilizza anche altri servizi di cloud a pagamento allora occorre aggiungere
        \end{itemize}
        \item Altri costi indiretti sono legati alla \textbf{manutenzione}
        \begin{itemize}
            \item Evitare che diventi un'app zombie, perciò l'app è da mantenere:
            \begin{itemize}
                \item M
            \end{itemize}
        \end{itemize}
    \end{itemize}
    \item \textbf{Range molto variabile}: da poche migliaia a centinaia di migliaia di euro.
\end{itemize}
\par Alcuni suggerimenti:
\begin{itemize}
    \item Trovare 
\end{itemize}

\par \textbf{Come guadagnano le app?}
\par Monetizzare un'app significa implementare strategie che permettano di generare


\par Scelta cruciale da definire \textbf{\textit{prima}} di sviluppare: influenza design e architettura.
\par Processi strategico che deve tenere conto del comportamento degli utenti, delle tendenze di mercato e delle specificità dell'app.
\par Due modelli principali:
% immagine

\begin{itemize}
    \item In-app purchase
    \begin{itemize}
        \item Paga l'utente
        \item Se si diverte, può essere propenso ad \textbf{acquistare} al suo interno (tipicamente giochi)
        \item Cosa si può acquistare?
        \begin{itemize}
            \item Sbloccare nuovi livelli o contenuti
            \item Scambiare beni e servizi virtuali
            \item Ottenere capacità più avanzate
            \item Più tempo di gioco o più vite
        \end{itemize}
        \par Secondo Forbes (una rivista autorevole), le app con acquisti in-app generano il maggior fatturato fra tutte le app.
    \end{itemize}
    \item Advertising
    \begin{itemize}
        \item Sono gli inserzionisti a pagare
        \item 3 tipi:
        \begin{itemize}
            \item Advertising - Banner
            \begin{itemize}
                \item Occupano poco spazio
                \item Meno invasivi
                \item Gli utenti possono visualizzarli senza interrompere l'attività nell'app
            \end{itemize}
            \item Advertising - Annunci interstiziali
            \begin{itemize}
                \item Occupano tutto lo schermo, a schermo intero
                \item Vengono visualizzati in momenti specifici
                \item Gli utenti possono chiudere questa pubblicità tramite un pulsante di chiusura (che di solito appare dopo un certo intervallo di tempo o dopo almeno un'interazione) posizionato in un angolo (sx o dx)
                \item Devono essere visualizzati al momento giusto per evitare di disturbare troppo l'utente (ad es.: in un gioco al termine di un livello)
            \end{itemize}
            \item Advertising - Video rewarded
            \begin{itemize}
                \item Breve video, di solito 15-30 secondi, che l'utente \textbf{sceglie} di vedere in cambio di una ricompensa
                \item Usati ad esempio nei giochi per premiare con crediti in-app, vite, \dots, in cambio di un breve video
                \item Collocato in un punto in cui l'utente non riesce a progredire nel gioco ed essere più suscettibile ad optare per la visione del video
                \item Video pbblicitari sono più efficaci quando
            \end{itemize}
            \item Advertising - Native
            \begin{itemize}
                \item La pubblicità nativa si presenta come una naturale continuazione dei contenuti e non come una rottura, sia da un punto di vista visivo che tematico
                \item "Come camaleonti, si adattano alla forma dell'app che li contiene" circa
                \item Gli utenti prestano così la propria attenzione online in modo più spontaneo
                \item Meno invasivi
                \item Profitto maggiore: quando l'utente va ad approfondire il contenuto
            \end{itemize}
            
        \end{itemize}
    \end{itemize}
\end{itemize}
\par Esistono poi le cosiddette \textbf{\textit{Freemium app}}:
\begin{itemize}
    \item prevede due o più varianti del prodotto da distribuire a prezzi diversi
    \item di solito due varianti:
    \begin{itemize}
        \item \textit{versione base gratuita}
        \item \textit{versione premium} a 
    \end{itemize}
\end{itemize}
\par Esistono poi le cosiddette \textbf{\textit{Subscription app}}:
\begin{itemize}
    \item Gli utenti \textit{pagano un canone periodico} per l'uso dell'app
    \item Gli utenti si aspettano quindi di ricevere più di quanto offra una (once for all) paid app: le app in abbonamento devono fornire continuamente ?
    \item Es. tipici: Corriere della Sera, servizi streaming tipo Netflix, Disney+, Prime Video...
\end{itemize}
\par Esistono poi le cosiddette \textbf{\textit{Purchase-app-once}} (anche dette \textbf{\textit{paid app}}):
\begin{itemize}
    \item Quasi il 95\% delle app che si possono scaricare dagli store sono \textit{gratuite}...
    \item Gli utenti si aspettano quindi di ricevere più di quanto offra una (once for all) paid app: le app in abbonamento devono fornire continuamente ?
    \item Es. tipici: Corriere della Sera, servizi streaming tipo Netflix, Disney+, Prime Video...
\end{itemize}

\subsection{Sfide tecniche}


cose che ho perso


\subsection{UI/UX cme fattore critico}
\par L'esperienza utente è il primo



\par \textbf{Navigabilità}
\begin{itemize}
    \item chiarezza dei percorsi
    \item accessibilità per utenti con diverse abilità 
    \item ergonomia: ridurre tap superflui, gesti naturali
\end{itemize}

\par \textbf{Testing: le sfide}
\begin{itemize}
    \item Frammentazione OS e dispositivi
    \begin{itemize}
        \item Esistono differenti S.O. con differenti versioni
        \item Senza verifiche cross-platform, si ferdono grossi bacini di utenze
        \item Suggerimenti:
        \begin{itemize}
            \item Occorre verificare che la nuova app sia supportata da diversi S.O. e le due versioni precedenti 
            \item Infrastrutture
        \end{itemize}
    \end{itemize}
    \item Rilascio frequente del S.O.
    \begin{itemize}
        \item I S.O. continuano a evolvere
        \begin{itemize}
            \item Sia Android che iOS hanno più di 10 versioni dei loro s.o.
            \item Continuano a migliorare e aggiornare le loro versioni per migliorare le prestazioni e l'esperienza utente
            \item Sen
        \end{itemize}
    \end{itemize}
    \item ergonomia: ridurre tap superflui, gesti naturali
\end{itemize}