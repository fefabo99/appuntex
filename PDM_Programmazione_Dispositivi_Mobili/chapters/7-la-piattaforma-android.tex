% 16/10/2024

\chapter{La piattaforma Android - prime info}
\section{AOSP - Android Open System Platform}
\par L'Android Open Source Project (AOSP) è un'iniziativa open source (non tutto Android lo è, ma mette delle parti e librerie e SDK a disposizione), vedi slide.

\subsection{Architettura}
% Slide con immagine
\begin{itemize}
    \item basato su un kernel Linux
    \item architettura a stack
    \begin{itemize}
        \item un livello sopra all'altro
        \item ognuno dotato di proprie funzionalità
    \end{itemize}
\end{itemize}

\subsubsection{Kernel Linux}
\par Fornisce supporto per funzionalità di basso livello (threading, gestione della memoria, l'accesso all'hardware via driver, etc.).
\par Fornisce modello di sicurezza (UID e sandbox, vedremo alla prossima lezione).
\par Permette ai produttori di dispositivi di sviluppare driver hardware per un \textit{kernel ben noto}.\\
Componenti che vanno ad interfacciarsi direttamente con l'hardware relativo.
\par Devo basare il driver per interfacciarmi con le funzionalità di quel modello specifico che sto implementando: ogni hardware ha API specifiche, quando scrivo un driver devo interfacciarmi con quelle API.

\subsubsection{HAL - Hardware Abstraction Layer}
\par Fornisce un insieme di \textbf{\underline{interfacce} \underline{standard}} che espongono le funzionalità hardware al \textbf{Java API Framework}.
\par Ciascuna interfaccia (ciò che ci si aspetta dal sensore) fornisce un insieme di servizi offerti dall'hardware corrispondente.
\par Quando una \textbf{Java API Framework} effettua una chiamata per accedere all'hardware del dispositivo, il sistema Android carica il modulo che gestisce quel componente hardware specifico.
\par Quindi quello che devo andare ad individuare è il set di funzioni minime che il sensore deve avere e lo mando poi all'HAL. Io so che devo implementare una interfaccia con queste funzioni specifiche? So quale è l'insieme di funzioni che devo implementare perché l'interfaccia sia usabile.

\subsubsection{ART - Android Runtime}
\par \`E il motore di esecuzione di Android, introdotto dalla versione 5.0 (Lollipop) per sostituire il vecchio Dalvik:
\begin{itemize}
    \item Dalvik: compilazione JIT (Just In Time)
    \item ART: compilazione \textbf{AOT} (\textbf{Ahead Of Time}) e \textbf{JIT} (compilazione Just In Time)
\end{itemize}
\par Dato il bytecode che mi arriva, serve qualcuno che lo compili \textbf{e} lo esegua.
\par Vantaggi ART:
\begin{itemize}
    \item miglioramento efficienza complessiva dell'esecuzione
    \item riduzione consumo di energia
    \item rapido start-up delle applicazioni
    \item miglioramento meccanismi di allocazione della memoria
    \item miglioramento garbage collector (GC)
    \item introduzione di nuove funzionalità di debug delle applicazioni
\end{itemize}
\par \textbf{Profile guided compilation} (introdotto da Android N (Nougat)) praticamente scarica un po' all'installazione un po' quando vedo che l'utente usa particolari funzionalità non inizialmente scaricate all'installazione. Per questo è profile guided, perché devo guardare il profilo dell'utente.
\par Da Android P (Pie) è stato introdotto Profiles in the Cloud: permette di scaricare il profilo dell'applicazione da un server e non dal dispositivo stesso, in modo da avere un'applicazione più leggera e più veloce.

\subsubsection{Librerie Native (C/C++)}
\par Android permette di scrivere librerie native in C/C++ che:
\begin{itemize}
    \item permettono prestazioni elevate
    \item es.: OpenGL, \dots vedi slide
    \item Usate per sfruttare direttamente le capacità hardware
    \item accesso tramite JNI o NDK
\end{itemize}

\subsubsection{Java API Framework}
\par Quello che andremo noi ad usare nel progetto.
\par Espone le API Java che consentono lo sviluppo delle app. Forniscono un insieme ricco di funzionalità:
\begin{itemize}
    \item \textbf{Activity Manager} per la gestione del ciclo di vita delle applicazioni
    \item \textbf{View System} per la costruzione di UI
    \item \textbf{Resource Manager} per accesso a risorse quali immagini
    \item \textbf{Notification Manager} per le notifiche di sistema
    \item \textbf{Content Provider} per l'accesso/condivisione di dati tra app
    \item \textbf{\dots}
\end{itemize}

\subsubsection{System Apps}
\par Applicazioni preinstallate nel sistema Android, che all'occorrenza posso integrare nello sviluppo della mia applicazione.
\par Le app di sistema costituiscono assieme al framework l'ambiente base di Android e offrono 
\par Es.: se devo aprire un URL, non vado mica a costruire a mano un browser, ma uso quello preinstallato.

% fai "precisazioni sugli ultimi due livelli" dalla slide
\par Ci sono due tipi di API:
\begin{itemize}
    \item \textbf{Android API}: sono quelle che possiamo usare per la nostra app, sono pubbliche e liberamente usabili.
    \item \textbf{System API}: sono quelle interne, non pubbliche, che sono destinate solo a uso interno del s.o. o a OEM/Google
\end{itemize}
\par Poi ho anche:
\begin{itemize}
    \item \textbf{Android Apps}: 
    \item \textbf{Priviliged App}: 
    \item \textbf{Device Manifacturer Apps}: 
\end{itemize}

versioni android

Tre concetti importanti:
\begin{itemize}
    \item minimum SDK\\
    Es.: la 8 è la minima versione di Android che supporto, perciò se uno ha la 7 non può installare la mia app
    \item target SDK\\
    Quella per cui andiamo a sviluppare l'app, però siamo noi che garantiamo che la nostra app (per esempio nel caso di minima 8 e target a 10), andiamo a garantire che se funziona per la 10 funzionerà anche per la 8
    \item compile SDK
\end{itemize}

\section{Componenti minime}
\subsection{Android SDK (Software Development Kit)}
\par Insieme di strumenti per sviluppare app per Android: strumenti e librerie necessari per sviluppare app Android e rendere l'implementazione del codice più gestibile.
\par L'SDK contiene strumenti principali:
\begin{itemize}
    \item \textbf{Android Studio:} è l'ambiente di sviluppo integrato (IDE) ufficiale per Android
    \begin{itemize}
        \item basato su IntelliJ IDEA
        \item offre strumenti come l'editor di codice, debugger e strumenti di testing
    \end{itemize}
    \item \textbf{Emulatore Android:} permette di simulare diversi dispositivi Android sul computer
    \begin{itemize}
        \item Utile per testare l'applicazione senza aver bisogno di un dispositivo fisico
    \end{itemize}
    \item \textbf{SDK Tools:} strumenti per compilare, debuggare e testare le app, oltre al profilo delle prestazioni
    \item \textbf{ADB (Android Debug Bridge):} connessione a
    \item \textbf{API:} 
\end{itemize}

\section{Google Play Services}
\par Servizi che possiamo usare, ma serve una \textit{key} da memorizzare sulla nostra applicazione, possibilmente in un posto sicuro quindi non codice sorgente che mando in giro. Senza questa chiave non posso usare i servizi di Google. Di solito si è monitorati perché oltre ad un certo numero di utilizzi si inizia a pagare.

distribuzione delle app

dal download all'esecuzione di un'app

jit e aot: al primo avvio ovviamente JIT con compilazione "al volo" per avere dal mio bytecode il machine code. poi viene memorizzato quale codice viene usato "più spesso" e questo frequently run code viene caricato con l'AOT.

da android P


\chapter{Android - la prima applicazione e le risorse}
\section{Strumenti}
\par Essenziale per iniziare a sviluppare un'applicazione.
\par L'IDE che noi abbiamo usato è Android Studio, che è l'IDE ufficiale per lo sviluppo di app Android. L'ultima versione rilasciata in versione stabile è Ladybug (io avevo scaricato Koala, che va bene uguale, il grosso dei cambiamenti è avvenuto fra Giraffe e Koala).
\par Gli emulatori sono sempre stati molto lenti o addirittura non funzionavano. Genymotion è un buon escamotage (\url{https:}) vedi slide.

\section{Setup}
\par All'installazione, c'è da configurare AVD e .

\subsection{AVD - Android Virtual Device}
\par Emula un dispositivo fisico Android, configurabile anche come meglio credo. Posso scegliere la versione di Android, la dimensione dello schermo, la memoria, \dots c'è la lista sulle slide.
\par Non basta chiaramente vedere se funziona con l'AVD per andare sullo store: devo sempre ricordarmi della frammentazione di Android, ovvero quante diverse versioni di hardware e produttori ci si affidino, quindi devo testare su più dispositivi possibili.

\subsubsection{Set Up di un AVD}
\par Lista sulle slide. Ne ho saltate alcune con screen dei passaggi. Comunque ad usare l'app e smanettarci poi uno ci prende la mano. In ogni caso, è \textbf{fondamentale} consultare sempre la documentazione ufficiale di Android che spiega nel dettaglio come fare qualsiasi cosa. Anche perché noi vediamo le best practice, ma è tutto in continua evoluzione quindi la documentazione costantemente aggiornata è la fonte più affidabile.

\section{Prima di cominciare con la prima app}
\par Sia che usiamo l'\texttt{xml} (senti lezione qua)\\
Es.: la home, la ricerca, la visualizzazione dei risultati della ricerca, \dots, sono tutte activities diverse. Mi sono persa cosa sono i fragment, riascolta la lezione.
\par Ho due componenti:
\begin{itemize}
    \item \textit{Activity}: componente che gestisce l'interfaccia utente (UI) e che l'utente può usare per interagire con l'applicazione
    \begin{itemize}
        \item due cose
    \end{itemize}
    \item \textit{Layout}: elenco degli elementi grafici, definisce un insieme di \textbf{elementi} della UI e la loro \textbf{posizione} sullo schermo
    \begin{itemize}
        \item 
    \end{itemize}
    \par Noi avremo Activity () e Layout (xml, quindi xml è "descrittivo") in due file separati.
\end{itemize}

\subsection{Es.}
\par Slide con esempio di quello che vogliamo costruire.

\subsection{Cominciamo}
\par Sezione "Projects"
\begin{itemize}
    \item New Project
    \item Ho quattro dispositivi target (phone \& tablet, wear OS, tv, automobili), ciascuno con una lista di template
    \item Es. empty activity non permette di scegliere il linguaggio di programmazione, fisso su Kotlin, \textbf{non lo useremo}
    \item Noi ci basiamo sul concetto di View, quindi andrò a prendere Empty View Activity e faccio Via/Next
    \item In package name, dovrei mettere l'inverso dell'azienda (es. com.azienda.nomeapp) e nomeapp lo prende dal nome che ho dato al progetto, mentre azienda è l'identificativo dell'azienda
    \begin{itemize}
        \item il nome del progetto è il nome dell'applicazione e non è modificabile una volta iniziato a sviluppare
        \item chiedono se per fare il progetto dobbiamo mettere "it.unimib.nomedelprogetto" e fa "mh sì penso possa valere la pena" 
    \end{itemize}
    \item Language: Java
    \item Minimum API level: la versione minima che voglio supportare, dalla Android 7 in su \textbf{dovrebbe} funzionare, la 6 sa per certo che non funziona e anzi non viene neanche mostrata nello store ed esce scritto che non è compatibile
    \item Build configuration level: consiglia quello "recommended" 
    \item faccio "Finish"
    \item Mi trovo davanti due finestre: 1 a sinistra del progetto con la sua struttura, 2 a destra con il file sorgente
    \item mi escono due file, un \texttt{activity\_main.xml} (file di Layout in questo caso specifico, anche lui in Java) e un \texttt{MainActivity.java} (che ha un nome inverso di xml, è \textbf{una convenzione}, compreso l'underscore)
    \item con il tasto in alto a destra (sulla riga di .xml e .java, ma riguarda solo l'xml perché riguarda il design) "\underline{Code}" (ALT + Maiusc + Destra) mi cambia visualizzazione xml
    \item "\texttt{<androidx.constraintlayout.widget.ConstraintLayout...}" sono constrain, forzati lì dove sono (c'è una slide "widget" che spiega cosa sono)
    \begin{itemize}
        \item una widget può mostrare testo o grafica, interagire con l'utente, organizzare altri widget sullo schermo
        \item l'SDK Android include molti widget che è possibile configurare
        \item e
    \end{itemize}
    \item il "device manager" (il terzo a destra in verticale) mi fa vedere i dispositivi che ho configurato e mi permette di aggiungerne di nuovi con tutte le cose che posso impostare
    \item slide che mostrano cosa ConstraintLayout e TextView controllano
    \item quando vado tipo a toccare "hello world" mi si aprono una serie di proprietà che posso impostare e modificare che sono quelle che trovo nel codice xml
    \item due a destra di "code" c'è "\underline{design}" che mi fa vedere come viene visualizzato il layout con una serie di strumenti e comandi
    \item c'è una slide con dei numeri verdi che spiega un po' tutta la finestra, inserisci
    \item "component tree" mi mostra la gerarchia dei componenti
    \item tornando su "code", ho una serie di proprietà con ruoli ben precisi
    \begin{itemize}
        \item nella slide, quelli evidenziati in verde
        \item quelli blu sono i constraint
        \item a me interessano quelli che nella slide credo 20 sono sulla sinistra
    \end{itemize}
    \item Concetti di:
    \begin{itemize}
        \item \textbf{Screen Size:}
        \begin{itemize}
            \item 
        \end{itemize}
        \item \textbf{Screen Density:}
        \begin{itemize}
            \item 
        \end{itemize}
        \item \textbf{Screen Resolution:}
        \begin{itemize}
            \item 
        \end{itemize}
    \end{itemize}
    \item Esempio: "si immagini un'app in cui uno scroll è riconosciuto dopo che l'utente si è mosso sullo schermo per almeno 16 pixel". Il gesto viene riconosciuto dopo:
    \begin{itemize}
        \item 2,5 mm (25,4mm*)
    \end{itemize}
    \item Io ragiono in termini di DP e SP, pixel virtuali che sono indipendenti dalla densità:
    \begin{itemize}
        \item slide
    \end{itemize}
    \item In questo modo ho una migliore esperienza utente, "guai a voi" se usiamo i pixel invece dei dp. Questa è una delle cose che \textbf{va a valutare} nel progetto. Ha mostrato un'immagine di minion, screen size uguale ma diversa densità.
    \item "ems" è un'altra unità, prende la dimensione della lettera "M" maiuscola e su questa basa la dimensione del testo, mentre "sp" è una unità di misura che tiene conto della dimensione del testo dell'utente
    \item tornando ad A.S., con la visualizzazione "Project" a sinistra vedo come effettivamente sono organizzate le mie cartelle, ma "Android" è più comoda
    \begin{itemize}
        \item dentro "Android" c'è una cartella "res" \textgreater "values" che contiene xml con diverse risorse (es. aggiunge dentro "strings" "bottone")
        \begin{verbatim}
            <resources>
                <string name="app_name">My Application</string>
                <string name="bottone">Saluta</string>
                <string name="campo">Ciao</string>
            </resources>
        \end{verbatim}
        \item contiene le mie risorse, stringhe, colori, font, layout, immagini, icone \dots In Android c'è una netta distinzione fra codice e risorse, quindi le risorse vanno messe in cartelle apposite (le risorse sono tutto ciò che si distingue dal codice, stringhe non sono codice, colori nemmeno, font nemmeno \dots)
        \item A questo punto tornando su xml, dentro "Button", invece di 
        \begin{verbatim}
            Button> android:text="Saluta"
            TextView> android:text="Ciao!"
        \end{verbatim}
        farò
        \begin{verbatim}
            Button> android:text="@string/bottone"
            TextView> android:text="@string/campo"
        \end{verbatim}
        dove "@" vuol dire "presso", quindi "presso il file delle stringhe prendi quella chiamata bottone".\\
        
    \end{itemize}
    \item Una cosa utile sono le traduzioni: 
    \begin{itemize}
        \item da file "strings.xml" \textgreater "Open editor" \textgreater "Add locale" \textgreater scelgo la lingua \textgreater "OK"
        \item poi metto le traduzioni delle risorse che ho inserito
    \end{itemize}
    \item In generale la sintassi è "@ NomeRisorsa / nomeElemento" senza spazi.
    \item Slide con concetto di risorsa. Nelle slide "raggruppare le risorse" ho una lista delle cartelle che devo avere. Slide "risorsa e id risorsa" interessante ma me la sono persa.
\end{itemize} 
