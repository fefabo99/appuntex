% 27/11/2024

\chapter{Services}
\par Un Service è un componente (uno dei 4 fondamentali di Android, un insieme di API dette "work-manager") che esegue operazioni in background senza fornire un'interfaccia utente. Possono persistere se programmati bene. Altri componenti possono avviare un service e interagire con esso anche se l'applicazione è in background. Un service può essere avviato da un componente (come un'attività) o può essere avviato in risposta a un'evento (come un broadcast receiver).
% slide, es sincronizzazione con firebase, che va in background senza disturbare l'utente
\par Un Service fornisce funzionalità ad altre app (ha fatto esempio di backup a fine giornata).
\par Posso posticiparne l'esecuzione.

\section{Cos'è un Service}
\begin{itemize}
    \item Un service \textbf{\textit{non}} è un \textbf{processo} separato.
    \begin{itemize}
        \item A meno che non sia specificato diversamente, viene eseguito nello stesso processo dell'applicazione di cui fa parte.
    \end{itemize}
    \item Un service \textbf{\textit{non}} è un \textbf{thread}.
    \begin{itemize}
        \item Boh
    \end{itemize}
    \item Un service fornisce \textbf{due} caratteristiche principali:
    \begin{itemize}
        \item guarda slide
    \end{itemize}
\end{itemize}

% quando spengo il telefono i services sono persi, non sono persistenti. Il work manager infatti (come vedremo) va a salvare su un DB.
% i services sono meno onerosi ma i work manager mettono a disposizione API che permettono di fare cose più complesse e renderle persistenti. Es.: chiamate ogni tot è un esempio di una cosa che quelle API possono fare.

\subsection{Platform e Custom Services}
\par \texttt{Context} fa da tramite con hardware e boh risenti.

\section{Tipi di services}
\subsubsection{Foreground}
\subsubsection{Background}
\subsubsection{Bound}

\subsection{Avviare un service:}
\subsubsection{Started}
\subsubsection{Bound}
\subsubsection{Started e Bound}

\subsection{La vita del processo}
\par Un servizio viene killato da Android solo quando la memoria è bassa e risenti.

\subsection{Le basi per la realizzazione}
\subsubsection{Creazione}
\par onBind() obbligatorio implementarlo anche se è un servizio started.

\subsection{Il file Manifest}
\par Services da attivare con intent esplicito (cosa voglio che venga fatto, es.: voglio visualizzare una pagina web, chi c'è che me lo può fare?).

\subsection{Foreground}
\par \textbf{Un Foreground Service è un servizio di cui l'utente è attivamente consapevole.}

\subsubsection{Notifiche}
\par Forniscono informazioni brevi e tempestive sugli eventi di un'app mentre non è in uso.
\par N.B.: è una classe a sé (\texttt), non è un'Activity.

\subsubsection{Anatomia di una notifica}

\subsubsection{Azioni}

\subsubsection{Notification channel: creazione}
% pendingintent: rimane nel tempo più di un intent, gli passo l'intent che mi specifica l'intent dell'activity che

\subsection{Started Foreground Service}
startForegroundService() mi chiama un servizio foreground, di default avrei un started.
startForeground() chiede 2 parametri in ingresso:
\begin{itemize}
    \item boh
\end{itemize}
% co routine su kotlin


