% 02/10/2024

\chapter{Introduzione al corso}
\section{Obiettivi del corso}
Il corso ha come obiettivo acquisire:
\begin{itemize}
    \item \textbf{\underline{Conoscenze}} (principi di buona programmazione) relative al mondo dello sviluppo mobile
    \item \textbf{\underline{Competenze}} sullo sviluppo Android 
\end{itemize}
Ma perché è stato scelto proprio Android? Comporta diversi vantaggi rispetto ad altri sistemi operativi come iOs:
\begin{itemize}
    \item più open source
    \item essendo più open source conosciuto meglio dai docenti che sono quindi più in grado di insegnare e correggere
    \item Android, nel caso uno voglia poi accedere allo Store e pubblicare un'app, prevede una tassa di iscrizione di \texttildelow 25\$ \textbf{una tantum}, mentre per iOs è di 100\$ ma penso sia \textit{annuale}.
\end{itemize}
Ci aspettiamo che alla fine del corso siate in grado di:
\begin{itemize}
    \item Sviluppare un'applicazione ``from scratch'' che segua l'\textbf{\underline{architettura}} di riferimento Android
    \begin{itemize}
        \item alla fine se abbiamo rispettato o no l'architettura presentata a lezione è quello che guardano di più della nostra app, se non è bellissima o funzionante al 101\% importa meno
    \end{itemize}
    \item Comprendere il funzionamento di applicazioni Android
\end{itemize} 

\section{Il corso in pillole}
\begin{enumerate}
    \item Introduzione alla progettazione e allo sviluppo di applicazioni mobili
    \item Linee guida sull'architettura dell'app
    \item Sviluppo di un'app in Java\\
    Per il nostro progetto possiamo usare Java o Kotlin, la teoria rimane la stessa, ma a lezione useremo solo Java. Questo perché è già stato presentato ed usato in altri due corsi e quindi conosciuto meglio da docenti e studenti. Inoltre, è previsto (penso in entrambi i linguaggi) l'uso di lambda functions, più elegante e funzionale in Kotlin, però buono anche in Java. Infine, Java risulta più conveniente per l'uso di librerie esterne di Kotlin, che è più giovane e meno conosciuto e quindi ha meno librerie disponibili.\\
    Eventualmente, sul sito Google ci sono disponibili diversi tutorial gratuiti (video) per imparare Kotlin e per la migrazione del mio progetto da Kotlin a Java.\\
    L'app deve essere robusta. La robustezza si basa sull'autonomia dalla connessione di rete. Il concetto ``\underline{offline-first}'' è molto importante, prima di tutto l'app deve funzionare senza connessione. Inoltre, deve essere anche \underline{evolvibile}. Oltre ai suoi componenti funzionali (ovvero le sue \underline{funzionalità}) di base, ci sono determinate funzionalità che devono essere mantenute ed evolute/ampliate nel tempo.\\
    La nostra app deve essere:
    \begin{enumerate}
        \item Compliant con l'architettura di riferimento\\
        La cosa \underline{importante} della nostra app non è l'estetica o se funziona bene, ma \textbf{\underline{come}} l'architettura presentata a lezione viene sviluppata.
        \item Con UI
        \item Che accede alla rete per i dati (API esterne)
        \item Che fa persistenza (\underline{locale + remoto})\\
        Ovvero deve funzionare \textit{localmente} e salvare localmente i dati (importante per il concetto di ``\textit{offline-first}''). Però deve anche salvare \textit{in remoto} i dati, ma deve rispettare la \textit{cross-device synchronization}.\\
        Importante per quanto riguarda la \underline{cross-device synchronization}, ovvero deve funzionare su diversi dispositivi con stato sincronizzato.
        \item Che usa Firebase\\
        Firebase è un framework di Google che offre una serie di servizi per lo sviluppo di applicazioni mobili.
    \end{enumerate}
\end{enumerate}