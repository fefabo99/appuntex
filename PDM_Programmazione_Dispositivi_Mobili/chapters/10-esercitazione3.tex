% 23/10/2025

\chapter{Esercitazione 3}
\section{Il progetto}
Cominciamo il progetto \texttt{WordNews} con un template \textit{Empty Activity}. Sarà un'app che mostra le notizie.

\begin{itemize}
    \item \texttt{androidx.constraintlayout.widget.ConstraintLayout} diventa \texttt{LinearLayout}, mette gli elementi in lista semplicemente, in verticale o in orizzontale.
    \item dentro ci aggiungo \texttt{android:orientation="vertical"}
    \item \texttt{android:gravity} sposta il testo all'interno dell'oggetto
    \item \texttt{android:layout\_weight} fa in modo che l'oggetto si adatti alla grandezza dello schermo (no valore di default)
    \item ora proviamo il Layout \texttt{RelativeLayout} (lo useremo per poco o nulla) in cui mettiamo una TextView con \texttt{android:layout\_below="@id/testo1"} che mette le view in fila verticalmente, comoda per le view ma non tanto per i singoli oggettini
    \item su moodle ha caricato una risorsa con buoni consigli per la programmazione, su questo sito Material Design Guidelines possiamo anche trovare le palette da inserire nel nostro progetto
    \item ogni schermata è un'activity: per farne una nuova "New -> Activity -> Empty View Activity". Diamo il nome "PickCountryActivity" e mettiamo il layout "activity\_pick\_country" (fa di default veramente)
    \item dentro "AndroidManifest.xml" andiamo a mettere \texttt{<activity android:name=".PickCountryActivity" />} che sarà il nuovo punto di ingresso dell'app. Di seguito:
    \begin{verbatim}        
        <intent-filter>
        <action android:name="android.intent.action.MAIN" />
        <category android:name="android.intent.category.LAUNCHER" />
    </intent-filter> 
    \end{verbatim}
    Questa parte che è dentro MainActivity la mettiamo dentro PickCountryActivity.
    \item dentro "activity\_pick\_country.xml" utile sono le "guidelines" che metto verticale
    \item comunque ha fatto roba tra cui inserire 1-2 cards che mi sono persa, ma quasi tutto mi sembra possa essere fatto dal sito di material design e importato
    \item comunque poi anche il login mail password e "did you forget your password?" è un bottone anche se sembra testo, sul sito material design si vedono i vari stili di bottoni 
\end{itemize}