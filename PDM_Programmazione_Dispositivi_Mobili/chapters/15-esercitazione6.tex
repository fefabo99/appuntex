% 20/11/2025

\chapter{Esercitazione 6}
\par Oggi vedremo una cosa essenziale per la nostra app: la gestione delle viste in maniera dinamica. Quello che dovremo andare ad avere sarà una lista di carte con collegamenti a servizi API.
\begin{itemize}
    \item Grid Layout contiene una lista di carte.\\
    Vogliamo andare a renderlo più dinamico.
    \begin{itemize}
        \item file PickCountryFragment.java
        \item creiamo nella cartella Java il package \texttt{util} in cui creo il file java \texttt{Constants.java}
        \item sull'API delle news (https://newsapi.org/) troviamo la lista dei paesi e delle categorie (in codici che sono costanti e che posso prendere e sbatter nel mio codice): dentro il codice ho delle righe con i codici che avrò per gli stati, poi uguale per le categorie, che andrò ad usare. Sotto avrò delle liste (categorie, drawables delle categorie, codici nazioni, nomi nazioni...)
        \item dentro "\texttt{strings.xml}" andiamo a creare delle stringhe per i paesi e le categorie: \texttt{<string name="countries\_name">"..."</string>} per ogni nazione
        \item Su \url{developers.android.com} andiamo a cercare "List View". Ne useremo la logica per andare a creare la nostra Grid View (che ha la stessa logica).
        \item L'oggetto che stiamo modellando (tipo la categoria) avrà un codice un nome e un'immagine. Fra la lista e l'oggetto ho l'adapter: ho un oggetto, faccio in modo di metterlo dentro una view.
        \item definiamo un package \texttt{model} (classi container): dentro la dir \texttt{java} vado a creare la nuova classe \texttt{Country.java} che sarà l'oggetto che voglio andare a modellare. Avrà un costruttore, getter e setter per nome, codice e immagini.
        \item La nostra dir \texttt{java} avrà:
        \begin{itemize}
            \item \texttt{model} con \texttt{Country.java}
            \item \texttt{ui.welcome} con \texttt{fragment} con \texttt{PickCountryFragment} e \texttt{PickCategoryFragment} e \texttt{LoginFragment}.\\
            Dentro \texttt{PickCountryFragment} vado a mettere l'adapter: \texttt{gridview.setAdapter(new CountryAdapter(getActivity(), countries));}
            \item \texttt{adapter} con \texttt{CountryAdapter.java} che importa:
            \begin{itemize}
                \item android.content.Context
                \item android.view.View
                \item android.view.ViewGroup
                \item android.widget.ArrayAdapter
            \end{itemize}
            e altre tre righe. Avrà poi un costruttore (class CountryAdapter extends ArrayAdapter<Country>) con dentro Context context non nullo e un layout (che fa riferimento al card\_country in \texttt{layout}) in ingresso e un metodo \texttt{getView()} che andrà a restituire la view e in ingresso avrà: int position, View convertView, ViewGroup parent.
        \end{itemize}
        \item La gridview c'è ma va riempita, poi per modificare semplicemente un'immagine vado in CountryAdapter.java e modifico il metodo \texttt{getView()} facendo:\\
        \texttt{imageView.setImageResource(countriesList.get(Position).getImage);}
        \item Dentro Constants.java vado a mettere una lista di oggetti Country:\\
        \texttt{public static final ArrayList<Country> countriesList = new ArrayList<>();}\\
        Faccio un \texttt{for} per la dimensione di COUNTRIES\\
        \texttt{countriesList.add(new Country(context.[i]));}

        \item Dentro PickCountryFragment.java vado a mettere l'adapter: \texttt{gridview.setAdapter();}
        \item L'Adapter fa da ponte fra l'oggetto grafico che voglio mostrare e la logica del suo funzionamento. Noi vediamo l'ArrayAdapter ma ce ne sono di diversi tipi.
        \item MaterialCardView è un componente di Material Design che permette di creare delle card, faccio:\\
        \begin{verbatim}
            MaterialCardView cardView = (MaterialCardView) convertView;\\
            cardView.setOnClickListener(new View.OnClickListener() 
            \{
                @Override
                public void onClick(View view) 
                \{
                    //qui dentro metto il codice per andare a fare il click
                    Navigation findNavController(view).navigate(R.id.action\_pickCountryFragment\_to\_pickCategoryFragment);
                \}
            \});
        \end{verbatim}
        \item In \texttt{util} vado a creare una classe \texttt{SharedPreferences} che vado a prendere da \url{developer.android.com} e che mi permette di salvare delle informazioni in maniera persistente.
        \begin{itemize}
            \item Il metodo \texttt{writeStringData} va a scrivere una stringa (ci salveremo Country)
            \item Il metodo \texttt{writeStringSetData} va a scrivere un set di stringhe (ci salveremo Category)
            \item Poi ho i metodi read.
        \end{itemize}
        \item Dentro \texttt{Country.java} metto tre stringhe dove salvo le chiavi di preferenze, nazione e categoria.
        \item Andiamo a creare dentro \texttt{adapter} un nuovo file \texttt{CategoryAdapter.java} che sarà uguale a \texttt{CountryAdapter.java} ma per le categorie.\\
        Credo copia incollato sostiutendo Country con Category.
        \item Dentro \texttt{model} creo \texttt{Category.java} che sarà uguale a \texttt{Country.java} ma per le categorie.\\
        Credo copia incollato sostiutendo Country con Category.
        \item Anche in Constants.java vado a creare una lista di oggetti Category. anche in pickCategoryFragment.java vado a mettere l'adapter, copia incollando ma stando attenti a rinominare correttamente. Tanto hanno la stessa identica logica.
        \item Ora, voglio poter scegliere più categorie assieme e poi poter confermare. Ha aggiunto perciò un bottone. Questo comunque lo vedo in \texttt{card\_category.xml} e \texttt{PickCategoryFragment.java}. La card avrà lo stato selezionato. Dentro \texttt{PickCategoryFragment.java} ho il floatingActionButton che mi permette di andare avanti (= view.findViedById(R.id.floatingActionButton);).
        \item Posso andare ad aggiungere un riferimento alla precisa istanza di un fragment. Lo faccio in \texttt{PickCategoryFragment.java} con \texttt{private PickCategoryFragment fragment;} e aggiungendolo in ingresso al costruttore.
        \item Solo se la lista di categorie selezionate non è vuota si accende il bottone (introdotto in \texttt{fragment\_pick\_categrories.xml} in \texttt{layout} e trovato sul git di material-components-android). Come? Dentro CategoryAdapter.java vado a mettere fragment.tryEnableFloatingActionButton(); e dentro PickCategoryFragment.java creo il metodo \texttt{tryEnableFloatingActionButton()} che abilita il bottone se la lista non è vuota.
        \item Dentro PickCategoryFragment.java vado a creare il metodo \texttt{floatingActionButton.setOnClickListener()} in cui vado a salvare le categorie selezionate tramite i loro codici e a navigare verso la prossima schermata.
        \item aggiungo un intent in PickCategoryFragment per andare alla prossima schermata (HomeActivity).
    \end{itemize}
\end{itemize}

\section{L'activity Home}
Non la mettiamo in ui.welcome ma in ui direttamente.
\par Ricorda che viene sempre valutata molto nel progetto la struttura: la ui nella cartella ui, i layout nella cartella layout etc.
\par IN questa activity avremo una barra sotto con 4 schede: Scelte in base alla selezione, Top Headlines (più recenti), Ricerca (per keyword o argomento) e credo Preferiti. Sono tutti servizi offerti dall'API delle notizie. Useremo di material-components-android il BottomNavigationView.
\par Ora la nostra struttura sarà:
\begin{itemize}
    \item \texttt{ui}
    \begin{itemize}
        \item \texttt{home} con \texttt{HomeActivity.java}
    \end{itemize}
\end{itemize}
\par Dentro \texttt{res} dentro \texttt{menu} (dir che se non esiste creo) creo \texttt{home\_menu.xml}. Dentro avrò una lista di item con id e titolo e icona (l'ultima non essenziale).\\
\par Importa l'icona vettoriale delle notizie.
\par L'idea è che l'activity abbia sotto la barra con i 4 contenitori e sopra il fragment.
\par Dentro \texttt{navigation} (dir sotto a mipmap, dovrebbe avere già \texttt{nav\_graph.xml}) creo \texttt{home\_nav\_graph.xml} che è il file di navigazione, dentro al quale col simbolino che ho in alto (rettangolo con il +) aggiungo i 4 fragment con id "preferenceNewsFragment", "popularNewsFragment", "searchNewsFragment" e "favoriteNewsFragment". Avrò i 4 xml dentro layout.
\par Per fare la toolbar in alto dentro \texttt{activity\_home.xml} metto un \texttt{Toolbar}: \texttt{androidx.appcompat.widget.Toolbar}.
\par Ora posso navigare fra le schede (non si vede perché i fragment sono tutti uguali e comunque ancora vuoti, next time).



% Mi sono un poco rotta guarderei la fine della registrazione.


\par Note in più rispetto all'anno scorso.
\par Le transizioni: selezionando una freccia (un'azione) posso cambiare in ``Attributes'': enter/exit della nuova schermata che arriva, mentre popEnter e popExit per quella vecchia che se ne va.
\par GridLayout è come LinearLayout ma invece che lineare permette di fare tabelle: sono comunque entrambi statici, perciò vanno dentro una ScrollView.
\par Adesso per la barra in alto con la freccia indietro e il nome del fragment in cui mi trovo (che sincronizzeremo in seguito), usa una TopAppBar (o ToolBar):\\
\url{https://github.com/material-components/material-components-android/blob/master/docs/components/TopAppBar.md}.
\par Nell'esempio su GitHub linkato c'è anche la barra di ricerca.
\par La classe che faremo SharedPreferences è perfetta per salvare preferenze dell'utente quali tema (light/dark mode), lingua ig?, e altre cose. A me per Bookly sarà comoda. \`E un oggetto che già esiste non lo facciamo noi, va a creare coppie chiave-valore e le usa per salvare le preferenze dell'utente. Queste preferenze sono salvate in locale ma possono essere salvate da remoto anche per portabilità dell'account e delle preferenze dell'utente su altri dispositivi.