% 30/10/2024

% onStop praticamente è quando ho la lista delle app aperte
\par Slide ciclo di vita delle activity:
% img
\begin{itemize}
    \item onStart è quando l'applicazione è in primo piano
    \item onPause è quando l'applicazione è ?
    \item onStop è quando l'applicazione è in background (può essere killata qui e tornare a onStart, devo mettere quali risorse voglio ripristinare)
\end{itemize}

\par Un'activity va in uno stato di Destroyed per diversi motivi:
\begin{itemize}
    \item se l'utente vuole eliminare dalla memoria l'activity
    \item se il sistema deve aggiornare la configurazione (voluto dall'utente, es cambio lingua o orientamento verticale/orizzontale o cambio tema giorno/notte\dots praticamente qualsiasi cosa mi faccia cambiare la schermata, l'interfaccia che mi trovo davanti)
\end{itemize}
E l'utente si aspetta di ritrovare l'activity come l'aveva impostata. Per questo devo salvare lo stato dell'activity (cosa che Android di per sé non fa).
\par Ma un'activity può essere cancellata anche dal sistema proprio e non dall'utente:
\begin{itemize}
    \item il sistema deve liberare una RAM limitata
\end{itemize}

\section{Salvare lo stato volatile di un'activity}
\par Quando un utente si aspetta di ritrovare un'activity come l'aveva lasciata, devo salvare lo stato \textbf{volatile} dell'activity (i dati persistenti ci si aspetta vengano sempre salvati). 
\begin{itemize}
    \item classe \texttt{ViewModel} (elemento architetturale)\\
    Usato \textbf{\underline{insieme a \texttt{onSaveInstanceState()}}} perché quest'ultimo comporta costi di serializzazione/deserializzazione, nei casi di dati più complessi.
    \item metodo \texttt{onSaveInstanceState}: salva lo stato volatile dell'activity \textit{in quel momento}\\
    Unico che posso usare per salvare dati della UI \textbf{semplici} e \textbf{leggeri} (come un tipo di dato primitivo o un oggetto semplice come \texttt{String})
    \item memorizzazione locale
\end{itemize}
La prossima lezione IMPORTANTISSIMA vedremo \textbf{architetture}.
\par N.B.: distruggere un processo vuol dire distruggere tutte le activity che ci sono dentro, perdere la memoria.

\subsection{Salvare lo stato volatile di un'activity con \texttt{onSaveInstanceState()}}
\begin{itemize}
    \item Instance state e \texttt{Bundle}
    \begin{itemize}
        \item boh
    \end{itemize}
    \item Il metodo \texttt{onSaveInstanceState()}
    \begin{itemize}
        \item Invocato dopo onStop()
    \end{itemize}
\end{itemize}

\chapter{Task e Back Stack}
\section{Cancellazione della memoria: Task e Back Stack}
\begin{itemize}
    \item Task: insieme di activity con cui l'utente interagisce per svolgere un compito
    \item Le activity vengono messe in uno stack (Back Stack) nell'ordine in cui sono state aperte
    \begin{itemize}
        \item la nuova activity viene messa (push) sopra a quella vecchia
        \item quando viene premuto back, la top task viene istrutta (pop) e si torna alla precedente appena sotto (non visibile) che torna di nuovo visibile
    \end{itemize}
    \item Seguiamo la politica LIFO (Last In First Out) come ha detto la prof, ovvero FILO (First In Last Out)
\end{itemize}

\section{Attivare i componenti}
\par Un'app realizza le funzionalità per cui è stata implementata avvalendosi di componenti sia proprie sia di altre app.
\par Activity, Service e Broadcast Receiver vengono attivati da messaggi asincroni chiamati \textbf{\texttt{Intent}}.
\par L'app demanda il compito al sistema Android inviandogli un messaggio che specifica l'\textbf{intent} di avviare un particolare componente.
\par \textbf{Gli Intent sono quindi usati per attivare comoponenti e condividere dati}.
\par Un Intent wrappa un Bundle.
\par I Content Provider (gestiscono insiemi di componenti utili a più applicazioni, es. i contatti della rubrica) non vengono direttamente attivati attraverso Intent. Ma abbiamo \textbf{Content Resolver} che fa da tramite/intermediario (non li vedremo, si possono trovare nella documentazione).

\subsection{Tipi di Intent}
\par 
\begin{itemize}
    \item \textbf{Espliciti}: specificano l'azione da eseguire e il tipo di dati su cui eseguirla
    \item \textbf{Impliciti}: chi lancia questo Intent non ha idea
\end{itemize}

\subsubsection{Intent esplicito}
\par \textbf{Messaggio per attivare un componente specifico}.
\begin{itemize}
    \item Da usare quando
    \item 
    \item Da utilizzare con i Service (ovvero, i Service possono essere attivati \textbf{solo} con Intent espliciti)
\end{itemize}

\subsubsection{Intent implicito}
\par \textbf{Messaggio per attivare uno specifico \underline{tipo} di componente}.
\begin{itemize}
    \item Da usare quando
    \item Si specifica il tipo di componente attraverso una \textbf{action} che deve essere eseguita e il sistema sceglier
    \item Android
\end{itemize}
% boh senti ci rinuncio copia le slide.

\subsection{Intent Filter}
Dichiara il file Manifest
se conosco il suo nome posso attivarlo altrimenti no
% diagramma dalla slide

\subsubsection{La costruzione di un Intent}
% tabella, tanti costruttori nella terza colonna, scelgo quello che mi serve

Le action hanno semantica ben precisa, se non c'è una action che fa quello che voglio fare, la implemento io.
Gli intent hanno una serie di metodi per andare a mettere tutte le informazioni che mi servono per attivare un componente.

L'invio cambia in base al componente: ci sono dei metodi appositi.
% tabella nella slide, nella terza colonna cosa succede sul componente che riceve.

registri permettono di registrare intent di ritorno di attività

\subsection{Intent impliciti: problemi}
non è detto che l'app esista
ho più app che lo fanno, lascio la scelta all'utente (es condividere un'immagine)

