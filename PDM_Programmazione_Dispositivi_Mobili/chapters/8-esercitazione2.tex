% 17/10/2024

\chapter{Esercitazione 2}
\par Avviamo un nuovo progetto Android Studio, scegliendo il template \textit{Empty Activity}.
\par Manifest (prima directory nella visualizzazione Android) è un xml che contiene le informazioni dell'applicazione, come il nome, l'icona, le activity, i permessi, \dots Dentro, definiamo \texttt{MainActivity} come activity principale, con un target ideale. \texttt{MainActivity} è il nostro punto di partenza. Estende la classe \texttt{AppCompactActivity}, che è un'activity standard che supporta le funzionalità più recenti di Android.

\subsection{Directory \texttt{Java}}
\par Dentro la directory Java e la prima sottodirectory, vado a prendere il \texttt{MainActivity.java}. Qui sostituisco l'override che vedo con:
\begin{verbatim}
    @Override
    protected void onCreate(Bundle savedInstanceState) {
        super.onCreate(savedInstanceState);
        setContentView(R.layout.activity_main);
    }
\end{verbatim}
\par Punto ai layout con il .
\par Best practice:
\begin{itemize}
    \item nomi delle classi: camelcase ma con maiuscola
    \item nomi dei metodi : camelcase ma con minuscola
    \item nomi dei layout: stesso del Java ma invertito e con underscore\\
    Es.: \texttt{MainActivity.java} e \texttt{activity\_main.xml}
\end{itemize}
\par Le altre due directory dentro Java posso ignorarle per ora.

\subsection{La cartella \texttt{res}}
\par Contiene le risorse layout dell'applicazione, come stringhe, colori, font, layout, immagini, icone, \dots

\subsection{Mipmap}
\par Simili a \texttt{drawable} ma di solito usato per le icone. Contiene icone di diverse dimensioni per adattarsi a diverse risoluzioni. 
\par Di solito meglio vettoriali perché png si sgranano.

\subsection{Values}
\par Riservato a colori e stringhe. 
\par Dentro \texttt{colors.xml} posso definire i colori che uso nell'applicazione.
\par Dentro \texttt{strings.xml} posso definire le stringhe che uso nell'applicazione, ovvero tutti i testi che compaiono dentro l'applicazione. Utile perché Android Studio nella sezione "Open Editor" dà uno strumento utile: la traduzione. Posso fare una traduzione automatica delle stringhe in altre lingue, e posso anche fare una traduzione manuale.

\subsection{Themes}
\par Due file: \texttt{themes.xml} e \texttt{themes.xml (night)}. Il secondo è per la modalità notturna.

\subsection{xml}
\par Non andremo a vedere queste cose, ignorare completamente.

\section{Gradle Scripts}
\subsection{build.gradle.kts (:app)}
\par Ho dentro tipo:
\begin{itemize}
    \item namespace = "com.example.esercitazione2": identificativo univoco dell'applicazione
    \item compileSdk = 34: SDK con cui compilo
    \item \texttt{defaultConfig}: 
    \begin{itemize}
        \item \texttt{applicationId =}: identificativo univoco dell'applicazione
        \item \texttt{minSdk =}: SDK minimo con cui funziona
        \item \texttt{targetSdk =}: SDK con cui è stato testato (min SDK $\leq$ target SDK $\leq$ compile SDK)
        \item \texttt{versionCode =}: 
        \item \texttt{versionName =}:
    \end{itemize}
\end{itemize}
\par Dentro \texttt{dependencies} posso aggiungere le dipendenze che voglio. Lui ha inserito \texttt{implementation('com.squareup.retrofit2:retrofit:(insert latest version)')}. Ma dà errore, faccio tasto destro e "Show context actions" e "replace with new library".
\par In breve, a sinistra c'è il tasto "Resource Manager" che mi permette di vedere in sintesi tutto quello che abbiamo visto finora.

\par Torno su "activity\_main.xml", "code" (tastino in alto a destra) e va ad aggiungere 

\par Apriamo l'emulatore. C'è già un emulatore acceso, lo togliamo e mettiamo Pixel 8a (preso a casissimo).
\par Se vogliamo usare il nostro telefono, dall'emulatore andiamo su impostazioni,"developer options", "about emulated device", "build number" e clicchiamo 7 volte. Torniamo indietro e andiamo su "developer options", "usb debugging" e lo attiviamo. Collego il telefono al computer e mi chiede se voglio usare il telefono per debuggare, accetto. Se non mi chiede, vado su "developer options" e attivo "usb debugging".

\par Dentro activity\_main.xml, andiamo a vedere le variabili
\begin{verbatim}
    android:id="@+id/main"
    android:layout_width="match_parent"
    android:layout_height="match_parent"
\end{verbatim}
\par Dentro "TextView" vado a mettere
\begin{verbatim}
    android:id="@+id/textView1"
\end{verbatim}
\par Ho due tipi di misure spaziali:
\begin{verbatim}
    android:layout_width="wrap_content"
    android:layout_height="match_parent"
\end{verbatim}
\par Il primo sta attorno al contenuto, il secondo si adatta all'altezza dello schermo.
\par Se faccio linear layout mi affianca gli elementi, da dentro il primo blocco (prima di TextView) vado a mettere \texttt{android:orientation="vertical"} li mette uno sotto l'altro.