% 07/11/2024

\chapter{Esercitazione 5}
C'è un field chiamato "inputText". è bene comune controllare che i campi abbiano sempre il tipo di input corretto (es. mail per le email \dots).
\par "apache validation" è una libreria che permette di fare la validazione dei campi di input.
\par Abbiamo detto che Gradle è un sistema di build automation, ma è anche un sistema di dependency management. Dentro \texttt{build.gradle} c'è una sezione \texttt{dependencies} in cui si possono mettere le dipendenze del progetto. Se esce il suggerimento di uprgrade alle nuove linee guida, si può fare con un click e fa tutto in automatico.
\par Dentro LoginActivity.java c'è un metodo \texttt{onCreate} che viene chiamato quando l'activity viene creata. Si può fare l'override di questo metodo e mettere il codice che si vuole eseguire quando l'activity viene creata.
\par Dentro LoginActivity.java c'è un metodo \texttt{isMailOk} che controlla se la mail è corretta. EmailValidator.getInstance().isValid(email) è un metodo che controlla se la mail è corretta. Se la mail è corretta, ritorna true, altrimenti ritorna false.
\par Le Snackbar sono delle notifiche che appaiono in basso. Su Material3 \texttt{"material-components-android/docs/components/Snackbar.md"} c'è la documentazione. Su "activity-login.xml" faccio una nuova Log e metto un id. Dentro LoginActivity.java metodo loginButton.setOnClickListener > "Snackbar.make(findViewBy(android.R.id.content), text: "", Snackbar.LENGTH_SHORT).show(); così fa uscire la Snackbar.
\par Introduciamo il concetto di intent. Sono oggetti che permettono alle activities e non solo di dialogare fra loro. Possiamo generarlo o senza parametri (intent implicito) o con come parametro l'activity verso cui voglio andare (intent esplicito).
\par Per fare un intent esplicito, si fa \texttt{Intent intent = new Intent(this, PickACountryActivity.class); startActivity(intent);}. 
\par Sono anche contenitori di informazioni. Si possono mettere informazioni dentro l'intent e passarle all'altra activity. Si fa \texttt{intent.putExtra("key", "value");} (noi facciamo \texttt{(EMAIL\_KEY)}). Per recuperare l'informazione si fa \texttt{String value = getIntent().getStringExtra("key");}.
\par Per fare un intent implicito, si fa \texttt{Intent intent = new Intent(Intent.ACTION\_VIEW, Uri.parse("http://www.google.com")); startActivity(intent);}.

\par In activity\_login.xml per immagine scelta dall'utente, vedi lezione.

\par Introduciamo il concetto di fragment. Un'activity corrisponde ad una schermata. Fragment funziona un po' come un activity, ma è un pezzo di schermata. Si possono mettere più fragment in una activity. 
\par Un esempio. Google foto. Ha sotto una barra con "Photos" e "Search". Ma se clicchiamo su una delle due non cambia l'activity, ma cambia il contenuto. Ovvero, cambia il fragment. Perché l'activity è sempre la stessa, ovvero la barra sotto (e logo di Google Foto in alto e il burger menu in alto a sinistra).
\par A differenza dell'activity, il fragment è ideato a runtime. Costruttore vuoto. Dentro LoginFragment.java c'è un metodo LoginFragment newInstance() {\dots}.
\par Dentro onViewCreated() si fa il binding dei campi. Si fa \texttt{binding = FragmentLoginBinding.inflate(inflater, container, false); return binding.getRoot();}.
\par Dentro activity\_login.xml metto "FragmentContainerView" che è un contenitore per i fragment. Si mette un id. 
\par navGraph è un file xml che contiene la navigazione dell'applicazione. 
\par Sulla modalità Design di nav\_graph.xml si può fare il drag and drop dei fragment manualmente, collegandoli con delle frecce. La cosa si rifletterà in automatico sul xml.
\par Ricapitolando, abbiamo un activity con il login. Dentro c'è un fragment FragmentContainerView. Dentro il fragment c'è un bottone. Quando clicco il bottone, voglio che mi porti ad un'altra schermata. Per fare questo, devo creare un altro fragment e collegarlo al primo. Guidelines sono: se sono schermate "piccole" meglio usare fragment, tenere activities per cose più importanti tipo passare da pagina login a pagina principale.
\par Navigation.findNavController(view).navigate(R.id.action\_loginFragment\_to\_pickACountryFragment); è il codice per navigare da un fragment all'altro.
\par Da un fragment posso passare a due diverse activity. Per fare questo, si fa \texttt{navGraph.xml} e si collega il fragment a due activity diverse. A livello di codice gestirò come fare a passare da un'activity all'altra. Per esempio il findNavController di prima era dentro un if.

\par Per salvare in locale (quindi su file dati persistenti), c'è la funzione \texttt{getSharedPreferences}. Guarda la documentazione.