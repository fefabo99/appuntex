% 02/10/2025

\chapter{Introduzione ai dispositivi mobili}
\section{Breve storia}
\par 3 aprile 1973, NYC, di fronte a giornalisti: telefonata fra Joel (Motorola) e Marty Cooper (AT\&T), per una "guerra" su chi avrebbe progettato il primo cellulare. Venne chiamato DynaTac.
\par Fu la puntata del 29 dicembre 1967 di Star Trek con l'Enterprise e il capitano Kirk ad ispirare Marty Cooper per lo sviluppo del primo telefono mobile.

\section{Perché Android?}
\par Un motivo si riscontra nel mercato dei SO disponibili (al 2025): iOS circa 28\%, Android circa 71\% (KaiOS e altri $<$1\%). Perciò Android è il target su cui concentrarsi (almeno per quanto riguarda il nostro corso).
\par Cos'è KaiOS? è un sistema operativo (anche lui basato su Linux) per dispositivi specifici, per feature phones, destinati al mercato per fasce di popolazione con meno disponibilità economica. Per questo motivo sono nati i \textit{feature phones}, una via di mezzo fra i \textit{dumb phones} (telefoni a conchiglia e con tastiera fissa esterna tipo Nokia 3310, precedenti agli smartphones) e gli \textit{smart phones} come li conosciamo noi, di cui hanno le principali funzionalità.

\section{A cosa servono i telefoni cellulari?}
\par Non si parla più di fare solamente telefonate, ora le tecnologie mettono a disposizione sempre più servizi utili. 

\section{Ma quante sono le app disponibili?}
% \par Grafico dalle slide
\par Passando da Agosto 2024 a Giugno 2025, si nota un decremento: gli stores fanno un po' di pulizia e eliminano quelle note come \textit{app zombie}, presenti ma che non fanno niente quindi passibili di cancellazione.
\par Ma il calo nel \textbf{Google Store} si deve anche ad altri motivi:
\begin{itemize}
    \item politiche di rimozione dello store
    \item sviluppatori più inclini a migrare a iOS (c'è l'impressione che gli utenti spendano di più che sulle app Android)
    \item duplicazione delle app su Android (tipo quando l'aggiornamento/improving viene rilasciato come una seconda app distinta invece di un aggiornamento)
    \item 
    \item app web e alternative nel play store (es. Google Store non è l'unico store disponibile per Android, inoltre è possibile scaricare app anche senza passare dallo store)
\end{itemize}

\par Comunque quando vado a sviluppare un'app mi conviene vedere dove, in quali ambiti ci sia più \textbf{presenza} (non download) che significa più richiesta.

\section{Mentre quelle scaricate?}
% \par Grafici slides

\section{I guadagni}
\par Su Android la distribuzione è: 96.95\% free apps, 3.05\% paid apps.
\par Su iOS la distribuzione è: 95.41\% free apps, 4.59\% paid apps.

\section{Sintesi}
% Guarda le slides