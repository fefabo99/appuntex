% 20/11/2024

\chapter{Architettura dell'applicazione}
\par L'architettura dell'applicazione è un aspetto molto importante per la nostra applicazione, per garantire che sia robusta, testabile e manutenibile. Un'architettura (pezzi di codice con precise responsabilità) ben fatta permette di scrivere codice più pulito, mantenibile e testabile. Inoltre, permette di dividere il lavoro tra più persone in modo più efficiente.
\par Android ci facilita nel nostro lavoro perché mette a disposizione un insieme di librerie e componenti per creare un'architettura solida. 

\section{Cos'è un'architettura SW?}
\par Definisce come il sistema (che sarà sviluppato secondo questa architettura) è strutturato, in che modo i suoi componenti e connettori interagiscono tra loro (rendendoli \textbf{compatti/coesi}, ovvero che di base si preoccupa di una sua sola responsabilità e solo di quello, e \textbf{lascamente connessi}, ovvero che ho poca interazione fra i componenti o che le loro dipendenze siano al minimo per non impattare sugli altri) attraverso \textit{interfacce} e come i dati vengono scambiati.

\section{Principi di base della programmazione}
\par Parliamo di \textbf{S.O.L.I.D.}:
\begin{itemize}
    \item \textbf{S}ingle Responsibility Principle: ogni classe dovrebbe avere una sola responsabilità.
    \item \textbf{O}pen/Closed Principle: le classi dovrebbero essere aperte all'estensione, ma chiuse alla modifica.
    \item \textbf{L}iskov Substitution Principle: gli oggetti di una superclasse devono essere sostituibili con gli oggetti delle sue sottoclassi senza interrompere il funzionamento del programma.
    \item \textbf{I}nterface Segregation Principle: un'interfaccia dovrebbe essere specifica per i suoi clienti.
    \item \textbf{D}ependency Inversion Principle: le classi dovrebbero dipendere da interfacce e non da classi concrete.
\end{itemize}

\subsection{Single Responsibility Principle}
\par Ogni classe dovrebbe avere una e una sola ragione per cambiare (una classe deve essere responsabile solo di un unico aspetto o funzionalità del sistema). Se una classe fa troppo, è difficile da mantenere e testare. Se una classe fa troppo poco, è difficile da riutilizzare.
\par Garantisce:
\begin{itemize}
    \item \textbf{Testing facilitato}: una componente con una sola responsabilità richiederà meno casi di test.
    \item \textbf{Loose coupling}: meno funzionalità in un singolo componente, meno dipendenze.
    \item 
\end{itemize}
% Slide successiva, esempio.

\subsection{Open/Closed Principle}
\par Le classi, i componenti, dovrebbero essere aperte all'estensione, ma chiuse alla modifica. Questo significa che dovremmo essere in grado di estendere una classe senza modificarla. \textbf{Deve essere closed, non posso permettere modifiche}.
\par Garantisce:
\begin{itemize}
    \item \textbf{Non modificabilità del codice}: il rischio di introdurre bug è limitato.
\end{itemize}

\subsection{Liskov Substitution Principle}
\par A parità di \textbf{contratto}, un componente dovrebbe poter essere sostituito senza compromettere il sistema.
\par Ma cos'è il contratto? Quando abbiamo introdotto a Prog2 le interfacce abbiamo detto che si \textit{istituisce un contratto}.
\par Garantisce:
\begin{itemize}
    \item \textbf{Supporto all'evoluzione} del software.
    \item \textbf{Supporto allo sviluppo incrementale} (sub).
\end{itemize}
\par Si basa sui concetti di ereditarietà e polimorfismo visti a Prog2.
\par Vedremo fra poco il concetto di \textbf{dependency injection}.

\subsection{Interface Segregation Principle}
\par Un'interfaccia troppo ampia dovrebbe essere suddivisa in più interfacce più specifiche e piccole.
\par C'è l'esempio dell'interfaccia BearKeeper nelle slide: viola il primo principio di separazione delle responsabilità: viene sostituita da una classe BearCarer che implementa BearFeeder, BearCleaner e BearPetter, ciascuna con le proprie responsabilità.

% Perso un paio di slides.

\subsection{Dependency Inversion Principle}
\par Si riferisce a 
\par Es. Firebase: non posso usare la formulazione a sinistra perché mi lego \textbf{troppo} a Firebase. La formulazione a destra è più corretta perché mi lego solo all'interfaccia ProfileSaver, quando voglio sfruttare Firebase mi affido all'istanza più specifica FirebaseProfileSaver.

\section{Clean Architecture}
\par Clean Architecture è un'architettura software proposta da Robert C. Martin ("Uncle Bob". Non si sa perché questo soprannome, o perché si firmava così nei primi blog o per il suo carattere affabile) nel 2012. È un'architettura che permette di scrivere codice pulito, mantenibile e testabile. È basata sui principi SOLID e su altri principi di progettazione software.
% Diciamo che copierei la slide
% Inserisci slide dei cerchi concentrici
\par La Clean Architecture è un insieme di linee guida per progettare l'architettura di un software.
\par Definisce come partizionare in livelli il software definendo in maniera chiara i confini fra questi.
\begin{itemize}
    \item Al centro: codice di alto livello (logica pura).
    \item All'esterno, codice di basso livello. 
\end{itemize}
\par Codice di base che non dipende dalle architetture progettative.
\par Quando io strutturo quello che è Entities, me ne frego del rosso che è Use Cases. Questo perché dipende dalla dependency ?:
\begin{itemize}
    \item Il codice di basso livello (più esterno) può dipendere da quello di livello superiore (più interno), ma mai il contrario.
    \item Ogni cerchio interno può sapere nulla di qualcosa di un cerchio esterno, cioè il cerchio interno non deve dipendere dal cerchio esterno.
\end{itemize}

\subsection{Entities}
\par Questo è il livello centrale dell'architettura e contiene le entità principali del sistema.


\subsection{Use Cases}
\par Questo livello contiene i casi d'uso che rappresentano i comportamenti specifici dell'applicazione.
\par Siamo ancora nel \textbf{cosa}, non nel \textit{come}.
\par I casi d'uso coordinano le operazioni tra le entità e gestiscono la logica di business.
\par Si tratta di logica

\subsection{Interface Adapters}
\par SI occupa di adattare l'applicazione a elementi esterni, come database, UI, servizi web\dots
\par Questo livello contiene componenti come \textbf{Repositories}, \textbf{Presenters}, \textbf{Controllers}, \dots
\begin{itemize}
    \item Repositories è il design pattern che si occupa di traduzione da database a 
    \item Presenters: si occupa di tradurre i dati in una forma che può essere visualizzata dall'utente.
    \item Controllers: si occupa di tradurre le azioni dell'utente in azioni che l'applicazione può eseguire.
\end{itemize}

\subsection{Frameworks and Drivers}
\par Questo livello esterno contiene i dettagli tecnici e le librerie esterne che l'applicazione utilizza.
% Guarda slide


\section{Architettura moderna delle app Android}
\par Come anticipato a inizio corso, se l'app che andiamo a progettare non rispetta i principi linee huida che stiamo per andare a vedere e che sono universalmente riconosciuti come buone pratiche, l'applicazione sarà valutata non sufficiente.

\subsection{Principi alla base da seguire}
\begin{itemize}
    \item \textbf{Separazione delle responsabilità}.
    \item \textbf{Drive UI from data model}: UI dovrebbe essere generata e aggiornata in base ai dati contenuti nei modelli di dati separati dalla UI.\\
    Dato che c'è stretta correlazione tra dati e app, per non legare la presentazione (elementi grafici) ai dati, si utilizzano 
    \item \textbf{SIngle source of truth (SSOT)}: i dati possono arrivare sia dalla rete che altre fonti. Io devo sempre fare fluire i dati dal mio daatbase che ho istituito come fonte autorevole e affidabile (trustworthy) di dati. Questo garantisce anche di evitare inconsistenze.
    \item \textbf{Unidirectional Data Flow}: il dato fluisce in una sola direzione. Lo stato fluisce in una sola direzione ( $\rightarrow$ UI), mentre gli eventi che modificano i dati fluiscono in direzione opposta.
\end{itemize}

\subsection{Tre livelli}
\par L'architettura moderna delle app Android si basa su tre livelli:
\begin{itemize}
    \item \textbf{UI layer}
    \item \textbf{Data layer}
    \item \textbf{Domain layer}
\end{itemize}

\subsubsection{UI layer}
\par \textbf{Overview}
\begin{itemize}
    \item Il ruolo del livello UI (o livello di presentazione) è:
    \begin{itemize}
        \item la \textbf{visualizzazione} dei dati dell'app sullo schermo
        \item l'\textbf{aggiornamento} dei dati quando cambiano
    \end{itemize}
    \item Il livello UI è composto da due elementi:
    \begin{itemize}
        \item \textbf{UI elements}:
        \item \textbf{State Holders}:
    \end{itemize}
\end{itemize}

\subsubsection{Data layer}
\par \textbf{Overview}
\begin{itemize}
    \item Definisce le regole (business logic) che determinano il modo in cui l'app crea archivia e modifica i dati.
    \item Il livello dati è composto da due elementi:
    \begin{itemize}
        \item \textbf{Repository} (con cui noi ci interfacciamo):
        \begin{itemize}
            \item contengono uno o più data source (\textbf{un repository per tipo di dato})
            \item espongono i dati al resto dell'app
            \item centralizzano la modifica dei dati
            \item risolvono i conflitti quando esistono più data source
            \item nascondono la data source (parliamo di astrazione)
        \end{itemize}
        \item \textbf{Data Sources}:
        \begin{itemize}
            \item fornisce una sola fonte di dati
            \item può essere ..., rete, database locale\dots
        \end{itemize}
    \end{itemize}
\end{itemize}

\subsubsection{Domain layer}
\par \textbf{Overview}
\begin{itemize}
    \item Incapsula la logica di business complessa oppure quella più semplice ma usata da più State Holders.
    % \item Ha skippato la slide di prepotenza, non ho capito perché.
\end{itemize}
% domanda in aula, per tipo password etc dove mettere, ha detto ciò che riguarda i dati data layer

\subsection{Confronto}
\subsection{Gestione delle dipendenze fra componenti}
\subsection{Dependency Injection}
\par Le classi hanno bisogno di riferimenti ad altre classi. Una classe costruisce la dipendenza di cui ha bisogno.

\subsubsection{Soluzione Manuale}
\par La classe riceve le istanze da cui dipende dall'esterno. Ho due modi:
\begin{itemize}
    \item constructor injection: %slide
    \item field (get/set) injection: %slide
\end{itemize}

\subsubsection{Service Locator}
\par Sono classi che forniscono le dipendenze di cui una classe ha bisogno. %slide 
\par Si chiamano davvero così. Creano e memorizzano le dipendenze e le forniscono quando richiesto.
% Slide

\subsubsection{General Best Practice}
\par Documento importante da guardare: \url{https://developer.android.com/topic/architecture#best-practices}