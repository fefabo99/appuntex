% 07/11/2025

\chapter{Architettura dell'applicazione}
\par L'architettura dell'applicazione è un aspetto molto importante per la nostra applicazione, è fondamentale per garantire che sia robusta, testabile e manutenibile. Un'architettura (pezzi di codice con precise responsabilità) ben fatta permette di scrivere codice più pulito, mantenibile e testabile. Inoltre, permette di dividere il lavoro tra più persone in modo più efficiente.
\par Android ci facilita nel nostro lavoro perché mette a disposizione un insieme di librerie e componenti per creare un'architettura solida. 

\section{Cos'è un'architettura SW?}
\par L'architettura di un sistema software definisce come il sistema (che sarà sviluppato secondo questa architettura) è strutturato, in che modo i suoi componenti e connettori interagiscono tra loro (rendendoli \textbf{compatti/coesi}, ovvero che di base si preoccupa di una sua sola responsabilità e solo di quello, e \textbf{lascamente connessi}, ovvero che ho poca interazione fra i componenti o che le loro dipendenze siano al minimo per non impattare sugli altri) attraverso \textit{interfacce} e come i dati vengono scambiati.

\section{Principi di base della programmazione}
\par Parliamo di \textbf{S.O.L.I.D.}:
\begin{itemize}
    \item \textbf{S}ingle Responsibility Principle: ogni classe dovrebbe avere una sola responsabilità.
    \item \textbf{O}pen/Closed Principle: le classi dovrebbero essere aperte all'estensione, ma chiuse alla modifica.
    \item \textbf{L}iskov Substitution Principle: gli oggetti di una superclasse devono essere sostituibili con gli oggetti delle sue sottoclassi senza interrompere il funzionamento del programma.
    \item \textbf{I}nterface Segregation Principle: un'interfaccia dovrebbe essere specifica per i suoi clienti.
    \item \textbf{D}ependency Inversion Principle: le classi dovrebbero dipendere da interfacce e non da classi concrete.
\end{itemize}

\subsection{Single Responsibility Principle}
\par Ogni classe dovrebbe avere \textbf{una e una sola ragione} per cambiare (una classe deve essere responsabile \textit{solo di un unico aspetto} o funzionalità del sistema). Se una classe fa troppo, è difficile da mantenere e testare. Se una classe fa troppo poco, è difficile da riutilizzare.
\par Garantisce:
\begin{itemize}
    \item \textbf{Testing facilitato}: una componente con una sola responsabilità richiederà meno casi di test.
    \item \textbf{Loose coupling}: meno funzionalità in un singolo componente, meno dipendenze, se hanno meno dipendenze e sono quindi più isolate, eventuali cambiamenti su questo componente non impattano su molta roba.
    \item \textbf{Modificabilità semplificata}: gestisco componenti più semplici da usare e modificare.
\end{itemize}
% Slide successiva, esempio.

\subsection{Open/Closed Principle}
\par Le classi, i componenti, dovrebbero essere \textbf{aperte} all'\textit{estensione}, ma \textbf{chiuse} alla \textit{modifica}. Questo significa che dovremmo essere in grado di estendere una classe senza modificarla. \textbf{Deve essere closed, non posso permettere modifiche}.
\par Garantisce:
\begin{itemize}
    \item \textbf{Non modificabilità del codice}: il rischio di introdurre bug è limitato; inoltre non cambio l'implementazione delle altre componenti che si appoggiano al sistema.
\end{itemize}
\par Es.: l'interfaccia della macchina del caffè: non modifico più il comportamento ma posso estenderlo aggiungendo a fianco della moka per esempio quella automatica, entrambe continueranno a fare il caffè ognuno nella loro maniera con la loro interpretazione.

\subsection{Liskov Substitution Principle}
\par A parità di \textbf{contratto} (definito dalle interfaces), un componente dovrebbe poter essere sostituito senza compromettere il sistema.
\par Ma cos'è il contratto? Quando abbiamo introdotto a Prog2 le interfacce abbiamo detto che si \textit{istituisce un contratto}.
\par Garantisce:
\begin{itemize}
    \item \textbf{Supporto all'evoluzione} del software.
    \item \textbf{Supporto allo sviluppo incrementale} (sub).
\end{itemize}
\par Si basa sui concetti di ereditarietà e polimorfismo visti a Prog2.
\par Vedremo fra poco il concetto di \textbf{dependency injection}.
\par Es.: 
\begin{verbatim}
    MacchinaDelCaffè mc = new Moka();
    mc.faiIlCaffè();
\end{verbatim}
\par Posso assegnare l'implementazione che mi interessa.

\subsection{Interface Segregation Principle}
\par Un'interfaccia troppo ampia dovrebbe essere suddivisa in più interfacce più specifiche e piccole.
\par C'è l'esempio dell'interfaccia BearKeeper nelle slide: vìola il primo principio di separazione delle responsabilità: viene sostituita da una classe BearCarer che implementa BearFeeder, BearCleaner e BearPetter, ciascuna con le proprie responsabilità. Cioè: avevo un'unica interfaccia con tre metodi, non va bene, troppa roba: allora faccio tre "sotto" interfacce, ciascuno con un proprio unico comportamento (es.: BearCleaner ha washTheBear(), etc). Poi BearKeeper implementerà le tre interfacce.

\subsection{Dependency Inversion Principle}
\par Si riferisce al disaccoppiamento fra i componenti: dovrebbero dipendere da astrazioni, non da classi concrete.
\par Es. Firebase: non posso usare la formulazione a sinistra perché mi lego a FirebaseProfileSaver, che è un'implementazione concreta. La formulazione a destra è più corretta perché mi lego solo all'interfaccia ProfileSaver, astratta; quando voglio sfruttare Firebase non mi affido all'istanza più specifica FirebaseProfileSaver.
\par Solo in un secondo momento (tramite injection o component service (?))
% immagine

\section{Clean Architecture}
\par Clean Architecture è un'architettura software proposta da Robert C. Martin ("Uncle Bob". Non si sa perché questo soprannome, o perché si firmava così nei primi blog o per il suo carattere affabile) nel 2012. \`E un'architettura che permette di scrivere codice pulito, mantenibile e testabile. \`E basata sui principi SOLID e su altri principi di progettazione software.
% Inserisci slide dei cerchi concentrici
\par La Clean Architecture è un insieme di linee guida per progettare l'architettura di un software, progettata per rendere il sistema indipendente da dettagli implementativi, altamente modulare e facilmente testabile.
\par Definisce come partizionare in livelli il software definendo in maniera chiara i confini fra questi.
\begin{itemize}
    \item Al centro: codice di alto livello (logica pura).
    \item All'esterno, codice di basso livello. 
\end{itemize}
% cerca slide
\par Codice di base che non dipende dalle architetture progettative.
\par Quando io strutturo quello che è Entities, me ne frego del rosso che è Use Cases. Questo perché dipende dalla Dependency Rule:
\begin{itemize}
    \item Il codice di basso livello (più esterno) può dipendere da quello di livello superiore (più interno), ma mai il contrario.
    \item Ogni \textit{cerchio interno} può sapere \textbf{nulla} di qualcosa di un \textit{cerchio esterno}, cioè il \textit{cerchio interno} \textbf{non deve dipendere} dal \textit{cerchio esterno}.
\end{itemize}

\subsection{Entities}
\par Questo è il livello centrale dell'architettura e contiene le entità principali del sistema.
% cerca slide

\subsection{Use Cases}
\par Questo livello contiene i casi d'uso che rappresentano i comportamenti specifici dell'applicazione.
\par Siamo ancora nel \textbf{cosa}, non nel \textit{come}.
\par I casi d'uso coordinano le operazioni tra le entità e gestiscono la logica di business.
\par Si tratta di logica di business perché riguarda \textbf{cosa} viene fatto, non \textbf{come} viene fatto.

\subsection{Interface Adapters}
\par Si occupa di adattare l'applicazione e gli use cases a elementi esterni, come database, UI, servizi web\dots
\par Questo livello contiene componenti come \textbf{Repositories} (mette a disposizione interfacce), \textbf{Presenters}, \textbf{Controllers}, \dots
\begin{itemize}
    \item Repositories: design pattern che si occupa di traduzione da database a 
    \item Presenters: si occupa di tradurre i dati in una forma che può essere visualizzata dall'utente.
    \item Controllers: si occupa di tradurre le azioni dell'utente in azioni che l'applicazione può eseguire.
\end{itemize}

\subsection{Frameworks and Drivers}
\par Questo livello esterno contiene i dettagli tecnici e le librerie esterne che l'applicazione utilizza.
% Guarda slide

\subsection{Es.: aggiungere un nuovo utente}
% Guarda slide

\section{Architettura moderna delle app Android}
\subsection{Clean Architecture e Android}
% Guarda slide


% ----- è arrivata qua -----


% 19/11/2025

\par Come anticipato a inizio corso, se l'app che andiamo a progettare non rispetta i principi linee guida che stiamo per andare a vedere e che sono universalmente riconosciuti come buone pratiche, l'applicazione sarà valutata non sufficiente.

\subsection{Principi alla base da seguire}
\begin{itemize}
    \item \textbf{Separazione delle responsabilità}.
    \item \textbf{Drive UI from data model}: UI dovrebbe essere generata e aggiornata in base ai dati contenuti nei modelli di dati separati dalla UI.\\
    Dato che c'è stretta correlazione tra dati e app, per non legare la presentazione (elementi grafici) ai dati, si utilizzano 
    \item \textbf{SIngle source of truth (SSOT)}: i dati possono arrivare sia dalla rete che altre fonti. Io devo sempre fare fluire i dati dal mio daatbase che ho istituito come fonte autorevole e affidabile (trustworthy) di dati. Questo garantisce anche di evitare inconsistenze.
    \item \textbf{Unidirectional Data Flow}: il dato fluisce in una sola direzione. Lo stato fluisce in una sola direzione ( $\rightarrow$ UI), mentre gli eventi che modificano i dati fluiscono in direzione opposta. Per intenderci, es.: un utente clicca un bottone 
\end{itemize}

\subsection{Tre livelli}
\par L'architettura moderna delle app Android si basa su tre livelli:
\begin{itemize}
    \item \textbf{UI layer} (gestisce in/output degli utenti e l'aggiornamento della visualizzazione)
    \item \textbf{Data layer} (contiene la logica di business dell'app ed espone i dati dell'app)
    \item \textbf{Domain layer} (per semplificare e riutilizzare le interazioni fra gli altri due layer, è l'unico livello facoltativo)
\end{itemize}

\subsubsection{UI layer}
\par \textbf{Overview}
\begin{itemize}
    \item Il ruolo del livello UI (o livello di presentazione) è:
    \begin{itemize}
        \item la \textbf{visualizzazione} dei dati (lo stato) dell'app sullo schermo (in caso di non aggiornamento)
        \item l'\textbf{aggiornamento} dei dati (quando cambiano)
        \begin{itemize}
            \item a causa dell'interazione dell'utente (es. pressione di un pulsante) 
            \item o di input esterni (es.: risposta di rete)
        \end{itemize}
    \end{itemize}
    \item Il livello UI è composto da due elementi:
    \begin{itemize}
        \item \textbf{UI elements}:
        \begin{itemize}
            \item consentono di visualizzare i dati sullo schermo (per creare questi elementi, View o Jetpack Compose)
        \end{itemize}
        \item \textbf{State Holders}:
        \begin{itemize}
            \item slide
        \end{itemize}
    \end{itemize}
\end{itemize}

\subsubsection{Data layer}
\par \textbf{Overview}
\begin{itemize}
    \item Definisce le regole (business logic) che determinano il \textbf{modo} in cui l'app \textit{crea, archivia} e \textit{modifica} i dati.
    \item Il livello dati è composto da due elementi:
    \begin{itemize}
        \item \textbf{Repository} (con cui noi ci interfacciamo):
        \begin{itemize}
            \item contengono uno o più data source (\textbf{un repository per tipo di dato})
            \item espongono i dati al resto dell'app
            \item centralizzano la modifica dei dati
            \item risolvono i conflitti quando esistono più data source
            \item nascondono la data source (parliamo di astrazione)
        \end{itemize}
        \item \textbf{Data Sources}:
        \begin{itemize}
            \item gestisce una sola fonte di dati (è la porta d'ingresso)
            \item può essere un file, la rete, database locale\dots
        \end{itemize}
    \end{itemize}
\end{itemize}

\subsubsection{Domain layer}
\par \textbf{Overview}
\begin{itemize}
    \item Incapsula la logica di business complessa oppure quella più semplice ma usata da più State Holders.
    % \item Ha skippato la slide di prepotenza, perché è il livello potenzialmente facoltativo.
\end{itemize}
% domanda in aula, per tipo password etc dove mettere, ha detto ciò che riguarda i dati data layer

\subsection{Confronto Architettura Moderna vs Clean Architecture}
% cit "reputo non essenziale che la sappiate"
\subsection{Gestione delle dipendenze fra componenti}
% slide

\subsection{Dependency Injection}
\par Le classi hanno bisogno di riferimenti ad altre classi. Una classe costruisce la dipendenza di cui ha bisogno.

\subsubsection{Soluzione Manuale}
\par La classe riceve le istanze da cui dipende dall'esterno. Ho due modi:
\begin{itemize}
    \item constructor injection: %slide
    \item field (get/set) injection: %slide
\end{itemize}

\subsubsection{Service Locator}
\par Sono classi che forniscono le dipendenze di cui una classe ha bisogno. %slide 
\par Si chiamano davvero così. Creano e memorizzano le dipendenze e le forniscono quando richiesto.
% Slide

\subsubsection{General Best Practice}
\par Documento importante da guardare:\\
\url{https://developer.android.com/topic/architecture#best-practices}

\section{UI Layer}
\subsection{Introduzione}
\subsubsection{Sintesi}
\par Il ruolo di un'interfaccia utente (UI) è:
\begin{itemize}
    \item quello di mostrare i dati dell'app sullo schermo
    \item servire come principale punto di interazione dell'utente
\end{itemize}
\par Ogniqualvolta i dati cambiano a causa 


%slide

\subsection{Stato UI}
\subsubsection{Definizione}
\par Lo \textbf{stato della UI} è quell'informazione che l'applicazione stabilisce che l'utente finale dovrebbe vedere.
\par Gli \textbf{elementi UI} (widget grafici) sono un mezzo per mostrare lo stato.
\par Possiamo perciò dire che \textbf{UI elements + UI states = UI}

\subsubsection{Evoluzione dello stato: gestione attraverso UDF}
\par Lo \textbf{stato della UI} può evolvere nel tempo.
\par L'evoluzione dello stato UI è gestita attraverso \textit{unidirectional data flow} (UDF).

\subsubsection{Gestione dell'evoluzione attraverso UDF: state holders}
\par Gli \textbf{state holders} sono classi responsabili del mantenimento dello stato della UI e della logica necessaria al suo aggiornamento.
\par Le classi \texttt{ViewModel} sono un esempio di state holders.

\subsubsection{Esposizione dello stato: Live Data (o StateFlow)}
\par Lo stato (dati) è mantenuto dagli State Holders (ViewModel).
% Slide

\subsection{Eventi UI}
\par Sono azioni che devono essere gestite nel livello UI, sia da UI che da ViewModel.
\par Il tipo più comune di eventi è quello degli eventi utente:
\begin{itemize}
    \item L'\textbf{utente \textit{produce}} eventi interagendo con l'app, ad es. toccando lo schermo o generando gesti
    \item La \textbf{UI \textit{consuma}} questi eventi usando callback come gli ascoltatori \texttt{onCLick()}.
\end{itemize}
%slide
\par Due tipi di eventi UI:
\begin{itemize}
    \item \textit{business logic}: \textbf{\underline{cosa}} occorre fare a fronte di un cambiamento di stato\\%manca roba
    Ha inserito una slide di esempio in cui nel caso di business logic, ci interfacciamo direttamente con il \texttt{ViewModel}.
    \item \textit{UI behaviour logic}: \textbf{\underline{come}} mostrare il cambiamento di stato
\end{itemize}

\section{Domain Layer}
% slide

% Nel caso di domain layer, ha detto che lei di solito rispetta la convenzione di chiamare le classi che si occupano di business logic con il nome terminante per UseCase.

% Nell'esempio delle notizie, data layer ha due repository da cui il domain layer prende i dati e li manipola (classe intermedia) per poi passarli al UI layer.

\section{Data Layer}
\par A differenza di ViewModel e UseCase, i repository non sono elementi architetturali, non sono classi astratte, ma concrete.

% Slide Introduzione Architettura: Acquisizione Data Source e APIs ha un pezzo di codice in cui il costruttore non ha il nome della classe.
\par Slide Architettura: naming conventions. molto importante.
\par Slide Architettura: molteplici livelli di repository. molto importante.
\par Slide Architettura: source of truth. molto importante ocio.

% \section{Riassumendo:}
% \begin{itemize}
%     \item S.O.L.I.D.: la nostra app li deve sempre mostrare
%     \item Separazione flusso dati ed eventi (seguendo pattern di dipendenze ben precisi)\\
%     es. ui dipende da model, model dipende da repository, repository dipende da data source etc
%     \item Unidirectional data flow (meccanismi di osservo osservo)
%     \item NON FARE ROBA A CIRCOLO: entrano in gioco le notifications
%     \item nella mia app devo avere TUTTI i livelli: activity, viewmodel, live data etc
%     \item importante usare service locator
% \end{itemize}
% Prossima volta: ciclo di vita 
