\documentclass[12pt, a4paper, openany]{book}
\usepackage{../generalStyle}

\begin{document}

\title{GAL - Domande d'Esame}

\author{
	Fabio Ferrario\\
	\small{\href{https://t.me/fefabo}{@fefabo}}
}

\date{2023/2024}

\maketitle

\tableofcontents

\chapter{Domande Aperte}
\domanda[Sottospazi Vettoriali]{1}{
	\begin{enumerate}
		\item Determinare la dimensione e trovare una base del sottospazio:
		      \\ $R= \{(x,y,z) \in \R^3 : y-x=2z\}$
		\item Completare la base del punto precedente ad una base di $\R^3$ con un vettore $v$ ortogonale a $T$.
	\end{enumerate}
}
\rispostaaperta{ %Copiata dal file
	\begin{enumerate}
		\item Prendiamo l'equazione che ci da: $y-x=2z$. è chiaramente l'equazione di un piano (quindi con dimensione =2).
		      In ogni caso, la parametrizziamo:

	\end{enumerate}
}

\chapter{Domande Chiuse}
\section{Algebra Lineare}
\domanda[Indipendenza di Vettori]{1}{Se devo verificare che $n$ vettori $v_i \in \R^m$ siano linearmente indipendenti, cosa posso fare?}
\rispostechiuse{Creo una matrice con $v_i$ come vettori riga che abbia determinante non nullo}{Creo una matrice con $v_i$ come vettori riga e cerco una sottomatrice quadrata di ordine $n$ Invertibile}
{Cerco una combinazione lineare dei vettori $v_i$ che mi dia il vettore nullo}{Creo una matrice con $v_i$ come vettori colonna e verifico che il rango di questa matrice sia $m$}
\rispostaaperta{\textbf{b}, perchè se ho una sottomatrice di ordine $n$ invertibile allora il suo determinante è zero. Per il teorema dei minimi, significa che il rango della matrice
è \emph{almeno} $n$, quindi è massimo e tutti i suoi vettori sono linearmente indipendenti.
}

\domanda[Rouchè-Capelli]{2}{Sia $Ax=b$ un sistema di equazioni lineari con più incognite che equazioni. Allora:}
\rispostechiuse{Agendo con operazioni elementari su righe e colonne della matrice completa $A|b$ ottengo una matrice compoleta il cui sistema associato possiede le stesse soluzioni di quello di partenza}
{Scegliendo $b$ opportunamente, il sistema ha un'unica soluzione}
{Dato un $b$ qualsiasi, mi posso scegliere $A$ in modo che il sistema abbia soluzioni e eche la somma di due di esse sia ancora una soluzione}
{Se il rango di $A$ è massimo, allora il sistema ha soluzione}
\rispostaaperta{\textbf{d}, Abbiamo che $n>m$, di conseguenza il rango di $A$ è al massimo $m$. Aggiungendo la colonna $b$, il rango massimo di $(A|b)$ è ancora $m$.
Quindi se il rango di $A$ è $m$, ovvero è massimo, allora il sistema ammette soluzioni (per R-C). 


inoltre per il teorema di Rouchè-Capelli, sappiamo che se il numero delle incognite $>$ rango($A$), allora il sistema ammette $\infty^{m-rank(A)}$ soluzioni.}


\domanda[Rouchè-Capelli]{3}{Sia $Ax = b$ un sistema che non ammette soluzione. Scegliendo un vettore $c$ è possibile ottenere che $Ax=b+c$ abbia infinite soluzioni?}
\rispostechiuse{Si, ma solo se $A$ non è di rango massimo}{Si, per un qualsiasi $A$}{No, mai}{Si, ma solo se $A$ è quadrata e di determinante non nullo.}
\rispostaaperta{\textbf{a (da capire)}. 
Se $Ax = b$ non ammette soluzioni, allora $rg(A)\neq rg(A|b)$.
Per ottenere un sistema con infinite soluzioni, dobbiamo avere $rg(A) = rg(A|b) < n$ con $n$ numero di incognite.
}

\domanda{4}{Se la somma di tre numeri positivi è 120, qual'è il massimo valore possibile tra il loro prodotto?}
\rispostechiuse{$30^2\cdot 80$}{$240^2 \cdot 30$}{$30^4$}{$1600\cdot 40$}
\rispostaaperta{ %Risposta di Chatgpt
	La somma dei tre numeri positivi è $120$, e supponiamo che i tre numeri siano $x$, $y$, e $z$. L'equazione della somma è espressa come:
	\[ x + y + z = 120 \]
	Per massimizzare il prodotto, distribuiremo i numeri in modo che siano il più possibile vicini, il che si verifica quando sono tutti uguali. Quindi, possiamo assegnare a ciascun numero il valore di \( \frac{120}{3} = 40 \). Il prodotto massimo sarà quindi:
	\[ P = x \cdot y \cdot z = 40 \cdot 40 \cdot 40 = 64000 \]
	Pertanto, il massimo valore possibile del prodotto è $64000$, ovvero la risposta $d$.
}

\domanda[Determinante]{5}{Sia $A$ una matrice quadrata e $v,w$ due suoi vettori colonna. Se $b$ è la matrice ottenuta da $A$ rimpiazzando il vettore $v$ con il vettore $v+\alpha \cdot w$ per un numero reale $\alpha$, che informazione abbiamo sul determinante di $B$?}
\rispostechiuse{$Det(B) = -Det(A)$}{$Det(B) = Det(A)$}{$Det(B) = \alpha\cdot Det(A)$}{$Det(B) = 0$}
\rispostaaperta{\textbf{b}. Nelle trasfomazioni elementari, rimpiazzare una riga/colonna $r_i$ con $r_i + \alpha r_j$ non cambia il determinante.}

\domanda[Rango]{6}{Sia $Ax = b$ un sistema di equazioni lineari con pi`u equazioni che incognite. Allora (si scelga l’affermazione
corretta):}
\rispostechiuse
{Se ha soluzione, il rango della matrice completa A|b non pu`o essere massimo}
{La soluzione, se esiste, necessariamente non `e unic}
{ Se possiede soluzione, e non `e unica, allora la somma di due soluzioni (PROSEGUE)}
{ Non ha soluzione}
\rispostaaperta{Se $Ax=b$ ha più equazioni $m$ che incognite $n$, allora il massimo $rg(A) = n > m$.
Supponendo di aggiungere una colonna $b$, adesso il massimo $rg(A|b) = n+1$, ma se $rg(A|b) = n+1$ il sistema non ammette soluzioni perchè è il rango di $A$ è minore.
Quindi, sicuramente se esiste soluzione il rango di $(A|b)$ non può essere massimo.

}

\domanda[Rango]{7}{Sia $A$ una matrice $n\times  m$ di rango $r > 0$. Quali delle seguenti affermazioni è CORRETTA:}
\rispostechiuse{$r$ può essere strettamente maggiore di $m$}{Non esistono $r-1$ vettori riga di $A$ linearmente indipendenti.}
{il determinante di $A$ è uguale a $r$}{Esiste una sottomatrice quadrata $B$ di $A$ di ordine $r-1$ con determinante non nullo (se $r\geq 2$)}
\rispostaaperta{ Andando per esclusione:
\begin{itemize}
	\item[a] No, perchè il rango non può essere maggiore del numero di righe o del numero di colonne.
	\item[b] No, perchè il rango è il massimo numero di vettori riga/colonna linearmente indipendenti.
	\item[c] No, Il rango non da informazioni sul valore del determinante.
	\item[d] Dal criterio dei minori sappiamo che il determinante di $A$ è il massimo ordine dei minori \emph{non nulli} di essa, quindi se il rango è $r$ sicuramente $\exists$ sottomatrice $B$ di ordine $r$ (e quindi $r-1$) con determinante non nullo.    
\end{itemize}
}
\domanda[Determinante]{8}{ Sia $A$ una matrice quadrata e $v,w$ due suoi vettori colonna diversi. Se $B$ è la matrice ottenuta da $A$ rimpiazzando il vettore $v$ con il vettore $\alpha\cdot v + \beta\cdot w$ per $\alpha,\beta\in \R$, che informazioni abbiamo sul determinante di $B$?}
\rispostaaperta{ Possiamo considerare questa operazione come due trasformazioni elementari: Prima moltiplichiamo la colonna $v$ per $\alpha$, quindi anche il determinante viene moltiplicato per $\alpha$, poi sostituiamo $v$ con $v+\beta w$, lasciando il determinante invariato.
Quindi la risposta è $c: det(B) = \alpha \cdot det(A)$.
}

\domanda{9}{Sia $A(t)$ una famiglia di matrici quadrate dipendenti da un parametro $t \in \R$.
Supponiamo che $Det(A(1)) = 5$ e $Det(A(-1)) = -5$. Quali delle seguenti affermazioni è possibile
concludere?}
\rispostechiuse{Tutti i vettori riga $A(1)$ sono indipendenti e il rango di $A(1)$ è massimo.}
{$rg(A(1)) = 5$}{$det(A(0))=0$}{Il rango di $A(1)$ è massimo, e $det(A(1) + A(-1))=0$}
\rispostaaperta{\textbf{a} 
\\In generale, sappiamo che $det(A)\neq 0 \Leftrightarrow rg(A) = n$, quindi il rango è massimo. Se il rango è massimo, tutti i vettori riga (e colonna, essendo quadrata) sono linearmente indipendenti.
\\Inoltre escludiamo la risposta $d$ perchè in generale non possiamo determinare $det(A+B)$ partendo dai determinanti di $A$ e $B$.
}

\domanda{10}{Sia $A$ una matrice quadrata $n\times n$ tale che la somma delle righe è uguale ad una colonna $c$ di $A$. Cosa posso concludere su $A$?}
\rispostechiuse{$rg(A)<n$}{$det(A)\neq 0$}{Esiste un minore di $A$ di ordine $n=1$ invertibile se $c\neq 0$}{Se la colonna $c$ è uguale ad una riga di $A$ non è invertibile.}
\rispostaaperta{
\textbf{c}
Si può ragionare con una matrice $1	times 1$, se l'unica colonna è diversa da 0 allora anche il determinante lo è e quindi è invertibile.
In ogni caso, anche se non lo fosse un minore di ordine $1$ è sempre invertibile se è diverso da 0.
}

\domanda[Minori - Rango]{11}{Supponiamo che una matrice $A$ di dimensioni $4\times 6$ (cioè 4 righe) abbia i determinanti di tutti i minori di ordine 3. Quale delle seguenti affermazioni è falsa?}
\rispostechiuse{Non esistono 4 colonne linearmente indipendenti in $A$}{Il rango massimo che potrebbe avere $A$ è 4}{Potrebbe esistere una sottomatrice $2\times 2$ di $A$ invertibile.}{Le righe di $A$ sono linermente indipendenti.}
\rispostaaperta{
	Dal teorema dei minori sappiamo che, avendo i determinanti di almeno un minore di ordine $3$ diverso da $0$ (almeno così pare dal testo), sicuramente il Rango di $A$ è $\geq 3$.
	\\Quindi non possiamo dire che tutte le righe di $A$ sono linearmente indipendenti, ma che almeno $3$ lo sono.
}

\domanda[Cramer]{12}{Sia Ax = b un sistema di equazioni lineari con un numero di equazioni uguale al numero di incognite.
Allora(si scelga l’affermazione corretta):}
\rispostechiuse{Se ha soluzione, il rango è massimo}{Se $Ax=0$ ha più di una soluzione, $Ax=b$ potrebbe avere una soluzione}{Se non ha soluzione, $A$ non è invertibile}{Se $A|b$ ha rango massimo, allora il sistema ha un'unica soluzione}
\rispostaaperta{\textbf{c}
\\Se non ha soluzione, per il teorema di Cramer $det(A)=0$, una matrie è invertibile \emph{sse} il suo determinante è non nullo. Quindi se il sistema non ha soluzione $\implies det(A)=0 \implies A$ non è invertibile.
}

\domanda{13}{Siano A, B due matrici $5 \times 5$ tali che rank(A) = 3 e rank(B) = 2. Allora}
\rispostaaperta{
	In questa domanda ci sono $j$ punti, riporto qui quelle vere: \textbf{Da capire}
	\begin{enumerate}
		\item Non sono invertibili perchè non hanno rango massimo, quindi $det=0$.
		\item Il rango indica il massimo numero di vettori riga/colonna linearmente indipendenti.
		\item Esistono due minori di ordine 2, $A'$ in A e $B'$ in B tali che $A'\cdot  B'$ è una matrice invertibile.
	\end{enumerate}
}
\domanda{14}{Calcolare il rango di una matrice $3\times 4$ al variare di un parametro $a$}

\domanda{15}{Nel sistema composto dalle equazioni $3x-2y+z=0$, $\alpha x + y + z = 0$ e $x+\alpha y - z=0$, per quali valori di $\alpha$ posso avere soluzioni non banali\footnote{La soluzione banale è $(0,0,0)$}?}

\subsection{Diagonalizzabilità di Matrici}

\domanda[Diagonalizzabilità]{38}{Sia $k$ reale. Si consideri la matrice 
\[ 
A_k:
\begin{pmatrix}
	3 & 0 & -7 \\ k & 3 & 4 \\ 0 & 0 & 2
\end{pmatrix}
\]
Quale delle seguenti affermazioni è corretta?
}
\rispostechiuse{$A_k$ è diagonalizzabile per ogni scelta di $k\neq 0$}{$A_k$ è diagonalizzabile se e solo se $k=0$}{$A_k$ è diagonalizzabile se e solo se $k$ è intero non negativo}{Per qualunque scelta di $k$, $A_k$ non è diagonalizzabile}
\rispostaaperta{\textbf{B}
% Soluzione dal file da guardare
Ricordiamo le condizioni di diagonalizzabilità:

}


\end{document}