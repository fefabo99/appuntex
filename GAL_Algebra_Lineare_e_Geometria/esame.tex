\documentclass[12pt, a4paper, openany]{book}
\usepackage{../generalStyle}

\begin{document}

\title{GAL - Domande d'Esame}

\author{
	Fabio Ferrario\\
	\small{\href{https://t.me/fefabo}{@fefabo}}
}

\date{2023/2024}

\maketitle

\tableofcontents

\chapter{Domande Aperte}
\domanda[Sottospazi Vettoriali]{1}{
	\begin{enumerate}
		\item Determinare la dimensione e trovare una base del sottospazio:
		      \\ $R= \{(x,y,z) \in \R^3 : y-x=2z\}$
		\item Completare la base del punto precedente ad una base di $\R^3$ con un vettore $v$ ortogonale a $T$.
	\end{enumerate}
}
\rispostaaperta{ %Copiata dal file
	\begin{enumerate}
		\item Prendiamo l'equazione che ci da: $y-x=2z$. è chiaramente l'equazione di un piano (quindi con dimensione =2).
		      In ogni caso, la parametrizziamo:

	\end{enumerate}
}

\chapter{Domande Chiuse}
\section{Algebra Lineare}
\domanda{1}{Se devo verificare che $n$ vettori $v_i \in \R^m$ siano linearmente indipendenti, cosa posso fare?}
\rispostechiuse{Creo una matrice con $v_i$ come vettori riga che abbia determinante non nullo}{Creo una matrice con $v_i$ come vettori riga e cerco una sottomatrice quadrata di ordine $n$ Invertibile}
{Cerco una combinazione lineare dei vettori $v_i$ che mi dia il vettore nullo}{Creo una matrice con $v_i$ come vettori colonna e verifico che il rango di questa matrice sia $m$}
\rispostaaperta{La soluzione è la $b$}


\domanda{}{Se la somma di tre numeri positivi è 120, qual'è il massimo valore possibile tra il loro prodotto?}
\rispostechiuse{$30^2\cdot 80$}{$240^2 \cdot 30$}{$30^4$}{$1600\cdot 40$}
\rispostaaperta{ %Risposta di Chatgpt
	La somma dei tre numeri positivi è $120$, e supponiamo che i tre numeri siano $x$, $y$, e $z$. L'equazione della somma è espressa come:
	\[ x + y + z = 120 \]
	Per massimizzare il prodotto, distribuiremo i numeri in modo che siano il più possibile vicini, il che si verifica quando sono tutti uguali. Quindi, possiamo assegnare a ciascun numero il valore di \( \frac{120}{3} = 40 \). Il prodotto massimo sarà quindi:
	\[ P = x \cdot y \cdot z = 40 \cdot 40 \cdot 40 = 64000 \]
	Pertanto, il massimo valore possibile del prodotto è $64000$, ovvero la risposta $d$.
}


\end{document}