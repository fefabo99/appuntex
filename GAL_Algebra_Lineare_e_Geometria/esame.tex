\documentclass[12pt, a4paper, openany]{book}
\usepackage{../generalStyle}

\begin{document}

\title{GAL - Domande d'Esame}

\author{
    Fabio Ferrario\\
    \small{\href{https://t.me/fefabo}{@fefabo}}
}

\date{2023/2024}

\maketitle

\tableofcontents

\chapter{Domande Aperte}
\domanda[Sottospazi Vettoriali]{1}{
    \begin{enumerate}
        \item Determinare la dimensione e trovare una base del sottospazio:
              \\ $R= \{(x,y,z) \in \R^3 : y-x=2z\}$
        \item Completare la base del punto precedente ad una base di $\R^3$ con un vettore $v$ ortogonale a $T$. 
    \end{enumerate}
}
\rispostaaperta{ %Copiata dal file
    \begin{enumerate}
        \item Prendiamo l'equazione che ci da: $y-x=2z$. è chiaramente l'equazione di un piano (quindi con dimensione =2).
        In ogni caso, la parametrizziamo:
        
    \end{enumerate}
}


\end{document}