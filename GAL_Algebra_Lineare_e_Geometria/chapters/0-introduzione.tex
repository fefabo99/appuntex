\chapter*{Introduzione}

Questi appunti di Algebra Lineare e Geometria sono stati fatti con l'obiettivo di riassumere tutti (o quasi) gli argomenti utili per 
l'esame di  Algebra Lineare e Geometria del corso di Informatica dell'Università degli Studi di Milano Bicocca.


\section*{Il Corso}
Gli appunti fanno riferimento alle lezioni di GAL erogate nel secondo semestre dell'anno accademico 22/23.

\subsection*{Programma del corso}
Il programma si sviluppa come segue:
\begin{enumerate}
	\item \textbf{Algebra Lineare}
	\begin{itemize}
		\item Spazi Vettoriali
		\item Dipendenza Lineare
		\item Basi
		\item Prodotto scalare euclideo
		\item Prodotto vettoriale
	\end{itemize}
	\item \textbf{Matrici}
	\begin{itemize}	
		\item Operazioni
		\item Rango
		\item Invertibilità
		\item Determinante
		\item Trasformazioni elementari e riduzione a scala 
	\end{itemize}
	\item \textbf{Sistemi di equazioni lineari}
	\begin{itemize}
		\item Risultati di base
		\item Teoremi di Rouché-Capelli e Cramer
		\item Cenni alla regressione lineare semplice
	\end{itemize}
	\item \textbf{Applicazioni lineari}
	\begin{itemize}
		\item Matrice associata
		\item Proprietà
	\end{itemize}
	\item \textbf{Diagonalizzabilità di Matrici}
	\begin{itemize}
		\item Autovalori
		\item Autovettori
		\item Molteplicità algebrica e geometrica
		\item Teorema Spettrale
	\end{itemize}
	\item \textbf{Geometria Analitica nel Piano}
	\begin{itemize}
		\item Sottospazi lineari affini
		\item Classificazione delle coniche
	\end{itemize}
	\item \textbf{Geometria Analitica nello spazio}
	\begin{itemize}
		\item Sottospazi lineari Affini
	\end{itemize}
	
\end{enumerate}

\subsection*{Prerequisiti}
I prerequisiti per questo corso sono:
Teoria di insiemi di base. Insiemi con strutture (monoidi e gruppi). Dimostrazioni per assurdo e per induzione.
