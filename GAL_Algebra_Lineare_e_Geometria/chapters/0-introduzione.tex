\chapter*{Introduzione}

Questi appunti di Algebra Lineare e Geometria sono stati fatti con l'obiettivo di riassumere tutti (o quasi) gli argomenti utili per 
l'esame di  Algebra Lineare e Geometria del corso di Informatica dell'Università degli Studi di Milano Bicocca.


\section*{Il Corso}
Gli appunti fanno riferimento alle lezioni di GAL erogate nel secondo semestre dell'anno accademico 22/23.

\subsection*{Programma del corso}
Il programma si sviluppa come segue:
\begin{enumerate}
	\item \textbf{Algebra Lineare}
	\begin{itemize}
		\item Spazi Vettoriali
		\item Dipendenza Lineare
		\item Basi
		\item Prodotto scalare euclideo
		\item Prodotto vettoriale
	\end{itemize}
	\item \textbf{Matrici}
	\begin{itemize}	
		\item Operazioni
		\item Rango
		\item Invertibilità
		\item Determinante
		\item Trasformazioni elementari e riduzione a scala 
	\end{itemize}
	\item \textbf{Sistemi di equazioni lineari}
	\begin{itemize}
		\item Risultati di base
		\item Teoremi di Rouché-Capelli e Cramer
		\item Cenni alla regressione lineare semplice
	\end{itemize}
	\item \textbf{Applicazioni lineari}
	\begin{itemize}
		\item Matrice associata
		\item Proprietà
	\end{itemize}
	\item \textbf{Diagonalizzabilità di Matrici}
	\begin{itemize}
		\item Autovalori
		\item Autovettori
		\item Molteplicità algebrica e geometrica
		\item Teorema Spettrale
	\end{itemize}
	\item \textbf{Geometria Analitica nel Piano}
	\begin{itemize}
		\item Sottospazi lineari affini
		\item Classificazione delle coniche
	\end{itemize}
	\item \textbf{Geometria Analitica nello spazio}
	\begin{itemize}
		\item Sottospazi lineari Affini
	\end{itemize}
	
\end{enumerate}

\subsection*{Prerequisiti}
I prerequisiti per questo corso sono:
Teoria di insiemi di base. Insiemi con strutture (monoidi e gruppi). Dimostrazioni per assurdo e per induzione.

\section{Ripasso concetti base}
\begin{itemize}
	\item Insieme
	\item Sottoinsieme
\end{itemize}
La definizione matematica di insieme è complessa, verrà quindi data una definizione
intuitiva. Si tratta di un gruppo di elementi distinti (l'ordine non conta).
\paragraph*{Esempio} $A = \{1,2,3\}$ è un insieme, mentre $B = \{1, 1, 2\}$, NON è
un insieme.
\subsection{Sottoinsieme}
Dato $A = {1,2,3,4}$ $B={2,3}$ è un sottoinsieme di A e si indica con $A \subset  B$.
Si tratta quindi di un insieme contenuto all'interno dell'insieme di partenza (definizione 
assolutamente non formale).
\subsection{Operazioni su insiemi}
\begin{itemize}
	\item Unione - Siano A e B due insiemi, $A \cup B$ è definito come l'insieme che contiene
	gli elementi di A e B.
	\item Intersezione - Siano A e B due insiemi, $A \cap $ è l'insieme degli elementi
	comuni tra A e B.
	\item Complemento - Siano $A \subset B$ due insiemi. L'insieme complemento
	$B \backslash A$ oppure $B - A = \{ x \in B : x \notin A\}$ 
	\item Prodotto Cartesiano - $A, B$ insiemi. $A \times B$:\\
	$A \times B = \{ (x,y): x \in A, y \in B\}$\\
	$B \times A= \{ (x, y): x \in B, y \in A\}$
\end{itemize}
\paragraph*{Osservazione notazione} Scrivere $(x, y)$ è diverso che scrivere $\{x, y\}$,
perchè nel primo caso sto considerando la coppia di elementi $x, y$, mentre nel 
secondo caso sto considerando l'insieme contenente gli elementi $x, y$.\\
Quindi $(x, y) \neq (y, x)$, mentre $\{x, y\} = \{y, x\}$.
\paragraph*{Osservazioni Prodotto cartesiano} 
\begin{itemize}
	\item Non gode della proprietà commutativa
	\item Gode della proprietà distributiva
\end{itemize}
\subsection{Insiemi di numeri}
\begin{itemize}
	\item $\N$ - Insieme numeri naturali
	\item $\Z$ - Insieme numeri interi - $\N \subset \Z$
	\item $\mathbb{Q} $ - Insieme numeri razionali - $\Z \subset \mathbb{Q} $ - Numeri $\frac{n}{m},
	 \, n \in \Z, \, m \in Z$
	\item $\R$ - Insieme numeri reali - $\mathbb{Q} \subset \R$ - Numeri come $\pi, \sqrt{q}, e$
\end{itemize}
\subsection{Funzioni}
Dati due insiemi A e B, una funzione è una relazione che associa ogni elemento di A a uno e
un solo elemento di B. L'insieme A viene chiamato \textbf{Dominio}, mentre B è il \textbf{Codominio}.
