\chapter{Spazi vettoriali}
Gli spazi vettoriali sono degli insiemi con "sopra" delle struttre algebriche.

\section{Definizione di Spazi Vettoriali}
Sia $V$ un insieme e $K$ un "campo" (ad esempio $\R$).
Allora:

\definizione{
    Diremo che $V$ è uno \textbf{Spazio Vettoriale} su $K$ se esistono
    le operazioni di \textbf{Somma} ($+$) e di \textbf{Prodotto per uno scalare}($\cdot$) su $V$.
}
Nota che campo e spazio vettoriali non coincidono mai! se entrambi sono $\R$, allora sono copie diverse di esso.
\subsection{Le operazioni $+$ e $\cdot$}
Perchè un insieme sia uno spazio vettoriale deve essere dotato delle operazioni di Somma e Prodotto per uno scalare,
ma queste due operazioni devono rispettivamente verificare alcune proprietà.

\paragraph{Somma}
La somma è una funzione così definita:
\["+" : V \times V \to V \]
\begin{center}
    ovvero $(\underline{v_1}, \underline{v_2} ) \to "\underline{v_1} + \underline{v_2}\; \forall \underline{v_i} \in V"$.
\end{center}
    Essa deve godere delle seguenti proprietà:
\begin{enumerate}
    \item Nullo: $\exists \underline{0} \in V : \underline{0} + \underline{v} = v \; \forall \underline{v} \in V$
    \item Opposto: $\forall \underline{v} \in V, \exists "-\underline{v}" : \underline{v} + (-\underline{v}) = \underline{0}$
    \item Associatività: $(\underline{v_1} + \underline{v_2}) + \underline{v_3} = \underline{v_1} + (\underline{v_2} + \underline{v_3})$
    \item Commutatività: $\underline{v_1} + \underline{v_2} = \underline{v_2} + \underline{v_1}$
\end{enumerate}

\paragraph{Prodotto per uno Scalare}
Il Prodotto per uno Scalare è una funzione così definita:
\["\cdot" : K \times V \to V \]
\begin{center}
    ovvero $(\underline{\alpha}, \underline{v} ) \to "\alpha\underline{v}\;"$.
\end{center}
Essa deve godere delle seguenti proprietà:
\begin{enumerate}
    \item $(\lambda_1 + \lambda_2) \cdot \underline{v} = \lambda_1\underline{v} + \lambda_2 \underline{v}$ con $\lambda_i \in K, \underline{v} \in V$
    \item $\lambda \cdot (\underline{v_1} + \underline{v_1}) = \lambda\underline{v_1} + \lambda\underline{v_2}$ con $\lambda \in K, \underline{v_1} \in V$
    \item $(\lambda_1 \cdot \lambda_2) \cdot \underline{v} = \lambda_1 \cdot (\lambda_2 \cdot \underline{v})$
\end{enumerate}

\osservazione{Si può dimostrare che:
\begin{itemize}
    \item $0 \cdot \underline{v} = \underline{0} \; \forall \underline{v} \in V$
    \item $\lambda \cdot \underline{0} = \underline{0} \; \forall \lambda \in K$
    \item $-1 \cdot \underline{v} = -\underline{v}$, ovvero l'opposto di $\underline{v} \in V,\; \forall \underline{v} \in V$.
\end{itemize}
}
