\chapter{Diagonalizzabilità di Matrici}
\section{Definizione}
\definizione{
    Sia $A$ una matrice quadrata di ordine $N$ a coefficienti in un campo $\mathbb{K}$.
    Si dice che $A$ è una matrice diagonalizzabile se è simile ad una matrice diagonale $D$ di ordine $n$, ovvero
    è diagonalizzabile se e solo se esiste una matrice invertibile $P$ talche $D=P^{-1}AP$.
}
\subsection{Teorema di Diagonalizzabilità}
Il teorema di diagonalizzabilità fornisce delle condizioni \textbf{necessarie e sufficienti} affinchè una matrice quadrata sia diagonalizzabile:

Una matrice $A$ \textbf{è diagonalizzabile se e solo se} valgono le seguenti condizioni:
\begin{enumerate}
    \item Il numero degli autovalori di $A$ è uguale all'ordine della matrice.
    \item La molteplicità algebrica di ogni autovalore è uguale alla relativa molteplicità geometrica.
\end{enumerate}

\subsection{Casi Particolari}
Esistono alcuni casi particolari:
\begin{itemize}
    \item Se $A$ è una matrice simmetrica, ovvero $a_{ij}=a_{ji} \forall i \neq j$ allora $A$ è sempre diagonalizzabile.
    \item Se $A$ è una matrice quadrata di ordine $n$ che ammette esattamente $n$ autovalori distinti in $\mathbb{K}$, allora $A$ è diagonalizzabile nel campo $K$.
\end{itemize}

\section{Autovalori e Autovettori}
La nozione di Autovalore si riferisce solamente alle matrici quadrate.
\definizione{
    Si dice che lo scalare $\lambda_0 \in \mathbb{K}$ è un autovalore della matrice quadrata $A$ se esiste un vettore colonna non nullo $\underline{v}\in \mathbb{K}^n$ tale che:
    \[ A\underline{v} = \lambda_0 \underline{v}\]
    Il vettore $\underline{v}$ è detto autovettore relativo all'autovalore $\lambda_0$
}
\osservazione{Se $\underline{v}$ è un autovettore di $\lambda_0$ allora lo è anche $\alpha \underline{v}$ con $\alpha \in \mathbb{K}$.}