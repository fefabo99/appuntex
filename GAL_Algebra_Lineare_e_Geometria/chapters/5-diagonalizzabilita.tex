\chapter{Diagonalizzabilità di Matrici}
\section{Definizione}
\definizione{
    Sia $A$ una matrice quadrata di ordine $N$ a coefficienti in un campo $\mathbb{K}$.
    Si dice che $A$ è una matrice diagonalizzabile se è simile ad una matrice diagonale $D$ di ordine $n$, ovvero
    è diagonalizzabile se e solo se esiste una matrice invertibile $P$ talche $D=P^{-1}AP$.
}
\subsection{Teorema di Diagonalizzabilità}
Il teorema di diagonalizzabilità fornisce delle condizioni necessarie e sufficienti affinchè una matrice quadrata sia diagonalizzabile:

Una matrice $A$ è diagonalizzabile se e solo se valgono le seguenti condizioni:
\begin{enumerate}
    \item Il numero degli autovalori di $A$ è uguale all'ordine della matrice.
    \item La molteplicità algebrica di ogni autovalore è uguale alla relativa molteplicità geometrica.
\end{enumerate}

