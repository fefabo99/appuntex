\chapter{Matrici}
In questo capitolo introdurremo le Matrici.
Riporterò le spiegazioni del Prof. Borghesi, e occasionalmente quelle del libro perchè più semplici.

\definizione{ %Definizione di matrice del borghesi
	Una matrice $k\times n$, $k$-righe e $n$-colonne è un elemento di
	$\R^n\times ... \times \R^n$ $k$ volte, oppure $\R^k\times ... \times \R^k$ $n$ volte
	\[ \simeq \R^{k\cdot n}\]
	In entrambi i casi le matrici sono elementi di uno spazio vettoriale.
	\[
		\begin{pmatrix}
			a_{11} & a_{12} & ... & a_{1n} \\
			a_{21} &                       \\
			\vdots                         \\
			a_{k1} & ...    & ... & a_{kn}
		\end{pmatrix}
	\]
}
\paragraph{Spazio Vettoriale}
$M(k,n)$ Denota l'insieme delle matrici a coefficienti reali con $k$ righe e $n$ colonne:
\begin{center}
	$M(k,n)=\{$ Matrici reali $k\times n\}$
\end{center}
$M(k,n)$ è uno \textbf{spazio vettoriale} di \emph{dimensione} $n\cdot m$.
L'elemento neutro è la matrice nulla e la base canonica è $\{E_{ij}\}_{i=1,...,k \;\; j=1,...,n}$:
\[
	E_{ij} =   \begin{pmatrix}
		0     & \vdots     & 0     \\
		\dots & 1          & \dots \\
		0     & \vdots     & 0     \\
		      & \uparrow j
	\end{pmatrix}
	\leftarrow i
\]

\esempio{ La base canonica di $M(2,2)$ è:
	\[ \{
		\begin{pmatrix} 1 & 0 \\ 0 & 0\end{pmatrix},
		\begin{pmatrix} 0 & 1 \\ 0 & 0\end{pmatrix},
		\begin{pmatrix} 0 & 0 \\ 1 & 0\end{pmatrix},
		\begin{pmatrix} 0 & 0 \\ 0 & 1\end{pmatrix}
		\}
	\]
}

\section{Prodotto tra Matrici}
\definizione{
	Siano $A \in M(n,m)$ e $B\in M(p,q)$.
	\\Posso fare $A cdot B$ se e solo se $m=p$, e in tal caso $A\cdot B \in M(n,q)$
}
Ovvero, posso moltiplicare due matrici solo se la prima ha il numero di colonne uguale al numero di righe della seconda.
Nel caso questo sia possibile, la matrice risultante avrà il numero di righe della prima e di colonne della seconda.
\osservazione{
	$A\cdot B$ può essere definito, mentre $B\cdot A$ no.
	\\In $M(n,n)$ cioò potrebbe accadere, ma in generale:
	\[ A\cdot B \neq B\cdot A \]
}
\definizione{
	$A\cdot B = (c_{ij})$ dove $c_{ij} = \sum_{u=1}^n a_{iu} \cdot b_{uj}$
}

\esempio{
	Siano $A=n \text{x} p, B=p \text{x} m$
	allora $\exists A \cdot B$ e $\nexists B \cdot A$

	\[A=\begin{pmatrix} 1 & 0 &  2 \\0 & 3 & -1\end{pmatrix}
		,B= \begin{pmatrix} 4& 1 \\ -2 & 2 \\ 0 & 3 \end{pmatrix}
		\to AB = \begin{pmatrix} 4 & 7 \\ -6 & 3\end{pmatrix}
	\]

	in cui $ab_{1,1} = (a_{1,1} \cdot b_{1,1}) + (a_{1,2} \cdot b_{2,1}) + (a_{1,3} \cdot b_{3,1})$
}

\section{Rango di Matrici}
Il rango è la "misura di quante informazioni contiene una matrice"
\definizione{
	\begin{center}
		rg($A$)= dim $<$Vettori colonna di A $>$
	\end{center}
	O, equivalentemente, il rango è il massimo numero di vettori colonna linearmente indipendenti di A.
}
La stessa cosa è equivalente per i vettori riga.
\osservazione{Dalla definizione si può osservare che vale sempre $0 \leq rg(A) \leq min\{$ Numero di righe di A, Numero di colonne di A$\}$}

\subsubsection*{Ranghi di matrici elementari}
Riportiamo qui alcuni ranghi di matrici elementari, utili per calcolare il rango di altre matrici.
\paragraph{Matrice Nulla}
\[ rg
	\begin{pmatrix}
		0      & ... & 0      \\
		\vdots & 0   & \vdots \\
		0      & ... & 0
	\end{pmatrix}
	= 0
\]
Il rango di una matrice che ha tutti $0$ come scalari è sempre $0$.
Si può osservare che se una matrice ha almeno uno scalare $\neq 0$, allora avrà almeno rango $1$.

\paragraph{Matrice Identità}
\[ rg
	\begin{pmatrix}
		1 & 0 & 0 \\
		0 & 1 & 0 \\
		0 & 0 & 1
	\end{pmatrix}
	= \text{Numero di Righe/Colonne}
\]
Il rango di una matrice identità (che è per forza quadrata) è uguale al numero di righe o colonne della matrice.

\paragraph{Matrice Diagonale}
\[ rg
	\begin{pmatrix}
		a_{11} & 0      & 0      \\
		0      & a_{22} & 0      \\
		0      & 0      & a_{nn}
	\end{pmatrix}
	= \text{Numero di } a_{ii}\neq 0
\]
Il rango di una matrice diagonale è il numero di elementi sulla diagonale diversi da 0

\paragraph{Matrice a Scala}
Il rango di una matrice a scala è il numero di righe della matrice diverse da 0.
\\Il rango delle matrici a scala è molto importante perchè viene utilizzato spesso per trovare il rango di matrici più complesse.

\section{Trasformazioni Elementari}
Le trasformazioni elementari sono operazioni sulle matrici, che possono essere applicate su righe e/o colonne che \textbf{lasciano invariato il rango di una matrice}.
Sono infatti spesso utilizzate per semplificare una matrice in modo da trovarne il rango più facilmente.

\paragraph{Le tre trasformazioni elementari}
Le trasformazioni elementari sono tre, e possono essere effettuate sia su righe che su colonne. Noi riporteremo le operazioni sulle righe, ma sono uguali anche per le colonne.
\begin{enumerate}
	\item Scambiare due righe.
	\item Moltiplicare una riga per $\lambda \neq 0  \in R$
	\item Rimpiazzare una riga $r_i$ della matrice con $r_i + \lambda r_j$, $\lambda \in R$ e $r_j$ un'altra riga della matrice.
\end{enumerate}

\paragraph{Corollario}: Sia $T$ una traformazione elementare, allora $rg(T(A)) = rg(A)$.
\paragraph{Proposizione}: $T(B) = T(Id) \cdot B$


\esempio{
	vogliamo trovare il rango della matrice:
	\[ rg
		\begin{pmatrix}
			1  & -1 & 0 & 2 \\
			2  & 1  & 1 & 1 \\
			-1 & 2  & 1 & 3
		\end{pmatrix}
	\]
	Utilizzo le trasformazioni elementari per ridurre questa matrice a Scala, ovvero devo annullare il primo elemento di $r_2$ e i primi due elementi di $r_3$,
	sostituendo una riga $r_i con r_i + \lambda r_j$:

	\begin{enumerate}
		\item $r_2 := r_2 - 2r_1$
		      \[
			      \begin{pmatrix}
				      1  & -1 & 0 & 2  \\
				      0  & 3  & 1 & -3 \\
				      -1 & 2  & 1 & 3
			      \end{pmatrix}
		      \]
		\item $r_3 := r_3 +r_1$
		      \[
			      \begin{pmatrix}
				      1 & -1 & 0 & 2  \\
				      0 & 3  & 1 & -3 \\
				      0 & 1  & 1 & 5
			      \end{pmatrix}
		      \]
		\item $r_3 := r_3 - \frac{1}{3}r_2$
		      \[
			      \begin{pmatrix}
				      1 & -1 & 0           & 2  \\
				      0 & 3  & 1           & -3 \\
				      0 & 0  & \frac{2}{3} & 6
			      \end{pmatrix}
		      \]
	\end{enumerate}
	Avendo ridotto la matrice a scala, possiamo contare le righe non nulle e trovare che il rango di questa matrice è 3.
}
\osservazione{
	Siccome nell'esempio precedente abbiamo operato sulle righe, i rapporti di linearità sulle colonne vengono mantenuti.
	\\Più precisamente, siano $A_i$ le colonne di una matrice $A$. Allora:

	\[
		\sum_{i=1}^{n} \lambda_i A_i = \underline{0} \longleftrightarrow \sum_{i=1}^{n} \lambda_i T(A_i) = \underline{0}
	\]
	Dove $T(A)$ è la trasformata di A per T \underline{sulle righe}.
}

\subsection{Matrici a Scala}
Le matrici a scala sono un tipo di matrice molto utile perchè è facile trovarne il rango.

\definizione{Una matrice $A$ è \textbf{a scala} se il numero di zeri a sinistra della $i-$esima riga $\underline{r_i}$ è strettamente maggiore al numero di zeri di $r_{i-1}$.}
Sia $B$ una matrice qualunque, allora $\exists T_1,T_2,...,T_h$\footnote{Una serie di trasformazioni} tali che
$T_h(T_{h-1}(...(T(B))))$ è a scala. Ovvero, da qualunque matrice è possibile eseguire delle trasformazioni in modo da trasformarla in una matrice a scala.

\section{Sistemi di Equazioni Lineari}
\definizione{
	Un'equazione lineare è un insieme di simboli:
	\[ a_1 x_1,...,a_nx_n = b \]
	Dove $b \in \R$ e $a_i\in\R$ sono valori fissati, e $x_i$ sono variabili.
}
Un sistema di Equazioni lineari quindi ha questa forma:
\[
	\begin{cases}
		a_{1,1} x_1 + a_{1,2}x_2 ... a_{1,n} x_n \\
		a_{2,1} x_1 + ...
	\end{cases}
\]

Ad un sistema di equazioni lineari possiamo associare due matrici:
\begin{itemize}
	\item Matrice Incompleta: $A=(a_{ij})$
	\item Matrice Completa: $A|\underline{b}$, ovvero $A$ con aggiunta la colonna $\underline{b}$
\end{itemize}
Possiamo quindi riscrivere il sitema in forma matriciale $A \cdot \underline{x} = \underline{b}$, dove
\[x=\begin{pmatrix}x_1\\\vdots\\x_n\end{pmatrix}\]
In un sistema di equazioni lineari $A\underline{x} = \underline{b}$ la soluzione $\underline{c}$ è tale che $A\underline{c}=\underline{b}$.

\proposizione{
	Se $A$ è \textbf{invertibile}, il sistema ha una \textbf{unica soluzione}.
	\\Dato $A\underline{x}=\underline{b}$, tale soluzione è:
	\[A^{-1} A \underline{x} = A^{-1} \underline{b} \implies \underline{x} = A^{-1}\underline{b}\]
}
Nota che $A^{-1} A = Id_n$. Se $\underline{b}=0$, diremo che il sistema è omogeneo.

\subsection{Teorema di Rouche-Capelli}
Introduciamo ora uno dei teoremi più importanti per i sistemi di equazioni lineari.
\\Questo teorema è diviso in due parti:

\begin{enumerate}
	\item Il sistema $A\underline{x}=\underline{b}$ ha soluzione \emph{sse}:
	      \begin{center}
		      rg($A$) = rg($A|\underline{b}$)
	      \end{center}
	      Ovvero se il rango della matrice Incompleta $A$ è uguale a quello della matrice Completa $A|\underline{b}$.
	\item Nel caso in cui ci sia soluzione, l'insieme $V$ di tutte le soluzioni è scrivibile come:
	      \[ \underline{c} + W = \{ \underline{c} + \underline{w} : \underline{w} \in W\} \]
	      Dove $\underline{c}$ è una soluzione qualsiasi del sistema, e $W= \{$ soluzione di $A\underline{x} = \underline{0} \}$.
	      $W$ è un sottospazio vettoriale di $\R^n$.
\end{enumerate}

%Sezione nella lezione 8, fatta dopo il rango.
\paragraph{Soluzione di Sistemi Lineari}
Il teorema di Rouche Capelli ci permette quindi di stabilire se
il sistema $A\underline{x} =\underline{b}$ ammette soluzioni
facendo delle valutazioni sui ranghi di $A$ e di $A|b$:
\begin{center}
	$A\underline{x} =\underline{b}$ ammette soluzioni sse $rg(A) = rg(A|b)$.
\end{center}

\osservazione{ Se il rango di $A$ e di $A|b$ è uguale, significa che $\underline{b}$ è combinazione lineare delle colonne di $A$ e di conseguenza il sistema è risovibile.}

\paragraph{Negli Esercizi}
Dobbiamo usare quindi le Trasformazioni Elementari per stabilire se $\underline{b}$ è combinazione lineare delle colonne di $A$.
Opero quindi sulle righe di $A|b$ per per ridurla a scala.

Perchè opero direttamente su $A|b$? Perchè sia $S(A|b)$ una riduzione a scala di $A|b$ avendo operato sulle righe.
Abbiamo che $S(A|b) = (S(A) | S(b))$, con $S= T_h \circ T_{h-1} \circ ... \circ T_1$, e $S(b)$ è combinazione lineare di $S(A)$ sse
$b$ è combinazione lineare delle colonne di $A$.


\esempio{
	Prendiamo il sistema lineare:
	\[
		\begin{cases}
			x-z+w=1 \\
			y+z-w=0 \\
			x-y+w = -1
		\end{cases}
	\]
	Vogliamo stabilire se esistono edlle soluzioni ed eventualmente trovarle.
	\begin{enumerate}
		\item Stabilisco se esistono delle soluzioni usando il teorema di Rouche-Capelli, verificando se: $rg(A) = rg(A|b)$.
		      costruisco quindi la matrice completa e la riduco a scala:
		      \[
			      A|b =
			      \begin{pmatrix}
				      1 & 0  & -1 & 1  & 1  \\
				      0 & 1  & 1  & -1 & 0  \\
				      1 & -1 & 0  & 1  & -1
			      \end{pmatrix}
		      \]
		      \[
			      r_3 := r_3 - r_1 \to
			      \begin{pmatrix}
				      1 & 0  & -1 & 1  & 1  \\
				      0 & 1  & 1  & -1 & 0  \\
				      0 & -1 & 1  & 0  & -2
			      \end{pmatrix}
		      \]
		      \[
			      r_3 := r_3 + r_2 \to
			      \begin{pmatrix}
				      1 & 0 & -1 & 1  & 1  \\
				      0 & 1 & 1  & -1 & 0  \\
				      0 & 0 & 2  & -1 & -2
			      \end{pmatrix}
		      \]
		      Ora che la matrice $A|b$ è a scala conto le righe non nulle (sia di $A$ che di $A|b$)
		      e trovo il rango di entrambe:
		      \[ rg(A|b) = rg(A) = 3 \]
		      Di conseguenza \textbf{Il sitema ha soluzione}.
		\item Ora devo trovare la soluzione del sistema.
		      Avendo operato solo sulel righe so che la soluzione della matrice trasformata a scala è la stessa della matrice originale, opero quindi su quella in modo da semplificare i conti.
		      \[
			      \begin{cases}
				      x-z+w=1 \\
				      y+z-w=0 \\
				      2z-w = -2
			      \end{cases}
			      \to
			      \begin{cases}
				      x=1-y \\
				      z=y-2 \\
				      w = 2y-2
			      \end{cases}
		      \]
		      Siccome il sistema ha 3 equazioni in 4 variabili, allora ha infinite soluzioni che variano con $y$.
		      La solzione $S$ del sistema può essere scritta in 3 modi diversi:
		      \begin{itemize}
			      \item $S=\{(x,y,z,w)\in \R^4 : x=1-y,z=y-2,w=2y-2, y\in \R \}$ Forma Standard.
			      \item $S=\{(1-y,y,y-2,2y-2):y\in\R\}$ Forma Parametrica.
			      \item $S=\{(1,0,-2-2)+<-1,1,1,2>\}$ Forma dal teorema $\underline{c} + W$.
		      \end{itemize}
	\end{enumerate}
}

\section{Determinante}
Il Determinante é un oggetto matematico definito solo per matrici quadrate.
Come oggetto, esso é una funzione:
\[
	det_n : \{\text{ \small{Matrici quadrate di ordine $n$} }	\} \to \R
\]
Il determinante é legato al rango ma contiene "meno informazioni".

Come dicitura utilizzeremo anche il rango per i vettori colonna di una matrice $n\times n$:
\[ (c_1,c_2, ... ,c_n) \to det_n(c_1,c_2,...,c_n)\]
In alcuni casi il determinante puó dirci se il rango é massimo, quindi se tutte le righe/colonne sono linearmente indipendenti.

\subsection{Proprietá caratterizzanti del Determinante}
Per definire il determinante useremo le 4 proprietá che lo caratterizzano come una funzione unica:
\begin{enumerate}
	\item $det(c_1,...,\underline{a}+ \underline{b},...,c_n) = det(c_1,...,\underline{a},...,c_n) + det(c_1,...,\underline{b},...,c_n)$
	\item $det(c_1,...,\lambda \underline{c},...,c_n) = \lambda \cdot det(c_1,...,\underline{c},...,c_n)$
	\item $det(c_1,...,\underline{c}, underline{c},...,c_n) = 0$ \\
	      Ovvero se due vettori colonna sono uguali il determinante é 0, anche se non sono adiacenti.
	\item $det(\underline{e}_1, \underline{e}_2,...,\underline{e}_n) = 1$, dove $\{e_j\}$ é la base canonica di $\R^n$.

\end{enumerate}
Le proprietá 1 e 2 sono dette di Multilinearitá, mentre la 3 é detta di Alternanza.

\teorema{Esiste un'unica funzione che soddisfi le proprietá da 1 a 4, ed essa é il determinante.}
\subsection{Il calolo del determinante di matrici semplici}
Il calcolo del determinante di una matrice quadrata dipende dalla dimensione della matrice.
\subsubsection{Determinante di Matrici $1\times 1$}
Una matrice quadrata di ordine 1, quindi una matrice formata da un solo scalare, ha come determinante lo scalare stesso:
\[
	det
	\begin{pmatrix}
		a_{11}
	\end{pmatrix}
	= a_{11}
\]
\subsubsection{Determinante di Matrici $2\times2$}
Il determinante di una matrice quadra di ordine 2 è dato dal prodotto degli elementi della diagonale principale meno il prodotto degli elementi dell'antidiagonale.
\[
	det
	\begin{pmatrix}
		a & b \\
		c & d
	\end{pmatrix}
	= (a \cdot d) - (b\cdot c)
\]

\subsubsection{Determinante di Matrici Diagonali}
In una matrice diagonale, il determinante diventa il prodotto degli elementi lungo la diagonale.
Ne segue che la matrice identità avrà sempre determinante 1, a prescindere dalla sua dimensione.
\subsection{Formula di Laplace}
La formula di Laplace ci permette di calcolare il determinante di matrici quadrate di ordine superiore a 2.

Sia $A = (a_{ij})$ una matrice $n\times n$.
\[
	A = \begin{pmatrix}
		a_{11} & a_{12} & ...    & a_{1j} & ...    & a_{1n} \\
		a_{21} & a_{22} & ...    & a_{2j} & ...    & a_{2n} \\
		\vdots & \vdots & \ddots & \vdots & \ddots & \vdots \\
		a_{i1} & a_{i2} & ...    & a_{ij} & ...    & a_{in} \\
		\vdots & \vdots & \ddots & \vdots & \ddots & \vdots \\
		a_{n1} & a_{n2} & ...    & a_{nj} & ...    & a_{nn} \\
	\end{pmatrix}
\]
Denotiamo con $A_{ij}$ la sottomatric ottenuta eliminando la $i-$esima riga e la $j-$esima colonna di $A$.

\paragraph{Il teorema}
Il teorema ci permette di sviluppare sia per riga che per colonna, nel seguente modo, riportato prima per riga:
\[det_n(A) = \sum_{j=1}^{n} (1)^{i+j} \cdot det_{n-1}(A_{ij})\]
oppure, sviluppando lungo la j-esima colonna:
\[det_n(A) = \sum_{i=1}^{n} (1)^{i+j} \cdot det_{n-1}(A_{ij})\]

\paragraph{In parole povere}
Scelgo una riga o una colonna su cui lavorare, possibilmente una con tanti 0 per ridurre il numero di calcoli,
poi scorrendo per ogni elemento della riga/colonna calcolo il suo complemento algebrico e li sommo.

\paragraph{Corollario}
Sia $^tA$ la matrice trasposta di $A$, ovvero la matrice avente come righe le colonne di $A$ e come colonne le righe di $A$. allora:
\[ det(^tA) = det(A)\]

\subsection{Il determinante e le Trasformazioni Elementari}
Quando effettuiamo delle trasformazioni elementari queste possono alterare il determinante, ma lo fanno in maniera prevedibile:
\begin{enumerate}
	\item Permutazione di due righe $\implies$ Il determinante cambia di segno.
	\item Moltiplicazione di una riga per $\lambda \neq 0$  $\implies$ il determinante viene moltiplicato per $\lambda$
	\item Rimpiazzare $r_i := r_i + \alpha r_j$ $\implies$ il determinante non cambia.
\end{enumerate}
L'uso delle trasformazioni elementari può quindi semplificare il calcolo del determinante di una matrice a patto di tracciare tutte le trasformazioni elementari effettuate.

\subsection{Altre proprietà del determinante}
Altre proprietà importanti del determinante sono:
\begin{itemize}
	\item In generale, $det(A+B)$ non è esprimibile in funzione di $det(A)$ e $det(B)$.
	\item Teorema di Binet: $det(A\cdot B) = det(A)\cdot det(B)$.
\end{itemize}

\paragraph{Corollario del Teorema di Binet}
Sia A invertibile, allora:
\[det(A^{-1 \frac{1}{det(A)}})\]
In particolare, $det(A)\neq0$ se $A$ è invertibile.

\subsection{Relazioni tra determinante e Sistemi di Equazioni Lineari}
Sia $A$ una matrice $n\times n$.

\teorema[Formula di Kramer]{
	\begin{enumerate}
		\item Il sistema $A \underline{x} = \underline{b}$ ammette un'unica soluzione sse $det(A)\neq 0$.
		\item In tal caso, la soluzione $^t(c_1,c_2,...,c_n)$ è data da:
		      \[	c_i = \frac{det(A_1|...|A_{i-1}|B|A_{i+1}|...|A_n)}{det(A)} \]
		      Dove $\underline{A}_j$ è la $j-$esima colonna di $A$.
	\end{enumerate}
}

\osservazione{$underline{b}$ è un vettore di $n$ scalari dello spazio $\R^{n\times 1}$, quindi la soluzione $A\underline{x}$ deve essere la stessa cosa.}

\subsection{Relazioni Determinante-Rango}
Come sappiamo, il determinante di matrici non quadrate non esiste, ma esiste il determinante di sottomatrici quadrate che possiamo considerare.
Definiamo quindi cos'è una sottomatrice e il concetto di Minore:
\definizione{
	Una sottomatrice di $A$ è una matrice ottenuta rimuovendo delle righe e/o colonne di $A$.

	Un minore di $A$ è il determinante di una sottomatrice quadrata di $A$. L'ordine di un minore è l'ordine della sua sottomatrice.
}
Da qui, possiamo enunciare il teorema che definisce la relazione tra determinante e rango:
\teorema{
	Sia $A$ una matrice. Il rango di $A$ è uguale al massimo ordine dei minori non nulli di $A$.
}
\esempio{
	Sia $A$ una matrice $3\times 5$:
	\[\begin{pmatrix}
			1 & 0 & -1 & 2  & 0 \\
			0 & 1 & 1  & -1 & 1 \\
			2 & 1 & -1 & 0  & 1
		\end{pmatrix}\]
	Il rango di questa matrice può essere al massimo $3$, proviamo a trovarlo tramite il determinante delle sottomatrici:
	\begin{enumerate}
		\item Il rango non può essere 0, perchè $\exists$ minore di ordine 1 non nullo:
		      \[ det(1) = 1\]
		\item Il rango non può essere 1, perchè  $\exists$ minore di ordine 2 non nullo:
		      \[ det\begin{pmatrix} 1 & 0 \\ 0 & 1 \end{pmatrix} = 1 \]
	\end{enumerate}
	Dal teorema precedente se esibisco una sottomatrice di ordine 3 con $det\neq 0$ ho dimostrato che $rg(A) = 3$.
	\\Se però trovo $B=$ sottomatrice di $A$ di ordine $3\times 3$ e $det(B)=0$ non posso concludere nulla sul determinante di $A$,
	perchè potrebbe essereci un'altra sottomatrice $3\times 3$ con determinante non nullo.
}
% Posso però arrivare concludere che: %Questa non l'ho capita.
% \begin{center}
% 	$det(B)\neq 0 \implies $
% \end{center}