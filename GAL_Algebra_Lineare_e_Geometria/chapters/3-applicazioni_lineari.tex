\chapter{Applicazioni Lineari}

\section{Introduzione}
Le funzioni insiemistiche non sono adatte a studiare gli spazi vettoriali.
Per esempio, riesco a trocare funzioni insiemistiche biunivoche tra $\R^n$ e $\R^m$ $\forall n,m$.

Quindi, per studiare gli spazi vettoriali tramite funzioni è meglio imporre condizioni a tali funzioni.

\definizione{ 
    Siano $V$ e $W$ due spazi vettoriali su $K$me  $f:V\to W$ una funzione.
    Diremo che $f$ è \textbf{Lineare}, o un \textbf{Omomorfismo} se:
    \begin{enumerate}
        \item $f(\underline{v}_1 + \underline{v}_2) = f(\underline{v}_1) + f(\underline{v}_2)$
        \item $f(\lambda \underline{v}) = \lambda f(\underline{v})$.
    \end{enumerate}
    Ed equivalentemente l'immagine di una combinazione lineare è la combinazione lineare delle immagini:
    $f(\lambda\underline{v}_1 + \beta\underline{v}_2) = \lambda f(\underline{v}_1) + \beta f(\underline{v}_2)$
}
\paragraph{Corollario}
\begin{itemize}
    \item Se $f$ è lineare, allora $f(\underline{0}_v) = \underline{0}_w$ e $f(-\underline{v}) = -f\underline{v}$.
    \item Se $U<V$, $f(U) < W$
    %Da completare
\end{itemize}

\osservazione{}