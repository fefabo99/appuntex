\documentclass[12pt, a4paper, openany]{book}
\usepackage{../generalStyle}
\usepackage{enumitem}

\graphicspath{ {./img/} }
\def\arraystretch{2} %define table vertical spacing
\setlist{nolistsep,leftmargin=*} %remove list spacing
\newcolumntype{Y}{>{\centering\arraybackslash}X} %new tabularx centered X column  

\begin{document}
\title{CheatSheet di Analisi Matematica}

\author{
	Fabio Ferrario\\
	\small{\href{https://t.me/fefabo}{@fefabo}}
}
\date{2022/2023}

\maketitle

\tableofcontents



\chapter{Funzioni}
\section*{Insiemistica}

Dati un elemento $m$ e un insieme $A$:
\begin{itemize}
	\item \textbf{Massimo/Minimo}: $m$ si dice massimo/minimo di $A$ se esso \emph{Appartiene ad $A$} ed é il piú grande/piccolo elemento di $A$. 
	\item \textbf{Maggiorante/Minorante}: $m$ si dice maggiorante/minorante di $A$ se é \emph{Maggiore/Minore o uguale} di ogni elemento di $A$.
\end{itemize}

\section*{Studio di Funzione}
Per lo studio di una funzione bisogna trovare:
\paragraph*{Dominio} della funzione, poni:\\

\begin{tabular}{ l|l }
	Denominatore    & $\neq 0$                                 \\
	Logaritmo       & Argomento $>0$                           \\
	Radice$^n$      & Argomento $\geq 0$ (\emph{sse $n$ pari}) \\
	$[f(x)]^{g(x)}$ & $f(x)>0$
\end{tabular}
\paragraph*{Limiti ai punti di frontiera del dominio}
Trovato il dominio, trova \emph{i limiti ai punti di frontiera},
quindi porre i limiti ad ogni punto in cui il dominio si interrompe (sia da destra che da sinistra) e eventualmente a $\pm \infty$.
\paragraph*{Asintoti}
Trovati tutti i limiti, se trovi:
\begin{itemize}
	\item $\limite{x}{\alpha^\pm} f(x) = \pm \infty \implies$ Asintoto \emph{Verticale}.
	\item $\limite{x}{\pm \infty} f(x) = l \implies$ Asintoto \emph{Orizzontale} (di equazione $y=l$)
\end{itemize}
Bisogna anche controllare la presenza di \textbf{Asintoti Obliqui}:
\begin{itemize}
	\item $m = \limite{x}{\pm \infty} \frac{f(x)}{x} \implies$ se $m$ \emph{esiste e non è nullo} trovo $q$:
	\item $q = \limite{x}{\pm \infty} [f(x) - mx]\implies$  se $q$ esiste allora $y=mx+q$ è \emph{asintoto obliquo}
\end{itemize}
\paragraph*{Monotonia}
La monotonia di una funzione si calcola \emph{ponendo $f'(x)>0$.}
Nei punti in cui la derivata è positiva, la funzione è \textbf{Crescente}, nei punti in cui è negativa la funzione è \textbf{Decrescente}
\subparagraph*{Punti di estremo} I punti in cui la derivata cambia direzione sono punti di estremo (max/min).
Se il punto di estremo è il più grande/piccolo di tutta la funzione, allora sono Assoluti.

\paragraph*{Convessità/Concavità}
\begin{itemize}
	\item $-$ conc$\mathcal{A}$va $\cap \implies f''(x)$ positiva
	\item $+$ con$\mathcal{V}$essa $\cup \implies f''(x)$ negativa
\end{itemize}


\paragraph*{Retta Tangente} al grafico in $x_0$:\\
trova $y=mx + q$ ponendo:
\begin{itemize}
	\item $m=f'(x_0)$
	\item $q=f(x_0)-f'(x_0)\cdot x_0$
\end{itemize}

\paragraph*{\textbf{Punti di Discontinuità}}
\begin{enumerate}
	\item Prima specie (Salto): i limiti dx e sx di $x_0$ esistono finiti ma sono diversi.
	\item Seconda spece (Essenziale): Almeno uno dei limiti è inifinito o non esiste.
	\item Terza Spece (Eliminabile): il limite di $x_0$ esiste finito ma è diverso da $f(x_0)$ o non esiste.
\end{enumerate}

\subsection*{Funzioni Pari/Dispari}
\small{(serve solo per le crocette)}
\begin{itemize}
	\item $-$ Dispari $\implies f(-x)=-f(x)$
	\item $+$ Pari $\implies f(-x)=f(x)$
\end{itemize}
\paragraph*{Paritá e disparitá di funzioni note}

$\sin(x)$ è \emph{Pari}, \emph{Decrescente} in $[0,\pi]$ e \emph{Crescente} in $[\pi,2\pi]$.
\\$\cos(x)$ è \emph{Pari}, \emph{Crescente} in $[0,\pi]$ e \emph{Decrescente} in $[\pi,2\pi]$.

\chapter{Serie}

\definizione{
	Condizione \textbf{Necessaria non Sufficiente} per la convergenza:\\
	Il Limite della successione del termine generale $a_n$ deve essere \emph{Inifinitesimo}.
	\[
		\sum_{n=1}^{+\infty} a_n \text{ converge} \implies \lim_{n\to +\infty} a_n = 0
	\]
}

\section{Serie Note}

\paragraph*{Serie Telescopica}
\begin{equation*}
	\serie{1}{+\infty} (a_n - a_{n+k})
\end{equation*}
oppure
\begin{equation*}
	\serie{1}{+\infty} (a_{n+k} - a_n)
\end{equation*}
\subparagraph{Come si risolve una serie telescopica}
É necessario applicare la definzione di serie, cioè la successione delle somme
parziali. Devo quindi manualmente sostituire n=1, n=2, n=3, ...
fino a quando non riconosco il pattern della serie.
\\ \emph{Ricordati di non semplificare Numeratore e Denominatore!},
mantenendo i numeri sostituiti sarà più facile scrivere il carattere della serie.


\paragraph*{Serie Geometrica}
\begin{equation*}
	\serie{0}{+\infty} q^n \begin{cases}
		\text{Diverge}    & q\geq 1  \\
		\text{Converge}   & -1<q<1   \\
		\text{Irregolare} & q\leq -1
	\end{cases}
\end{equation*}
\subparagraphmark{Somma di una Serie Geometrica Convergente} Se una serie geometrica converge ($-1<q<1$),
la somma si calcola:
\[
	\serie{0}{+\infty} q^n = \frac{1}{1-q}
\]

\paragraph*{Serie Armonica Generalizzata}
\begin{equation*}
	\sum \frac{1}{n^\alpha} \begin{cases}
		\text{Diverge}  & \alpha\leq 1 \\
		\text{Converge} & \alpha> 1
	\end{cases}
\end{equation*}

\paragraph*{Serie Armonica Logaritmica}
\begin{equation*}
	\sum \frac{1}{n^\alpha \log^\beta(n)}
	\begin{cases}
		\text{Converge} & \alpha > 1 \wedge \forall \beta \\
		\text{Converge} & \alpha = 1 \wedge \beta > 1     \\
		\text{Diverge}  & \alpha = 1 \wedge \beta \leq 1  \\
		\text{Diverge}  & \alpha < 1 \wedge \forall \beta
	\end{cases}
\end{equation*}


\section{Criteri di Convergenza}
  

\subsection*{Serie Positive def$^{\text{\underline{nte}}}$}
Se $a_n$ è def$^{\text{\underline{nte}}} \geq 0$ uso:
\paragraph*{Criterio del Rapporto} 
\[
	\limite{n}{+\infty} \frac{a_{n+1}}{a_n} = l
	\begin{cases}
		\text{Converge} &  l<1\\
		\text{Diverge} & l>1\\
		\text{Criterio inconclusivo} & l=1
	\end{cases}
\]

\paragraph*{Criterio della Radice} 
\[
	\limite{n}{+\infty} \sqrt[n]{a_n} = l
	\begin{cases}
		\text{Converge} &  l<1\\
		\text{Diverge} & l>1\\
		\text{Criterio inconclusivo} & l=1
	\end{cases}
\]

\paragraph*{Criterio del Confronto}
$a_n\leq b_n$ definitivamente $\implies$
\begin{itemize}
	\item se $b_n$ Converge $\implies$  $a_n$ Converge
	\item se $a_n$ Diverge $+\infty$ $\implies$ $b_n$ Diverge $+\infty$
\end{itemize}

\subsection*{Serie con Segno Alterno}
Se $a_n$ è a segno \textbf{Alterno}: 

\paragraph{Criterio della Assoluta Convergenza}.\\
$\sum a_n$ \textbf{converge assolutamente} se converge $\sum |a_n|$.\\
Se una serie converge assolutamente, allora converge.

\paragraph{Criterio di Leibniz}
DA SCRIVERE
\section{Limiti}

\paragraph*{\underline{Equivalenza asintotica tra funzioni}} Se il limite ($\to x_0$) di $\frac{f(x)}{g(x)} = 1$ allora $f$ e $g$ sono asintoticamente equivalenti

\paragraph*{\underline{$o$-piccolo}} Se il limite ($\to x_0$) del rapporto di $f(x)$ su $g(x)$ è uguale a $0$ allora $f(x)$ è $o$-piccolo di $g(x)$.
Nota, che per $x_0$ si intende un valore arbitrario che può essere anche 0 o $\pm \infty$
$$lim_{x\to x_0} \frac{f(x)}{g(x)} = 0 \implies f(x) = o g(x)\text{ per } x\to x_0 $$

\paragraph*{\underline{Limiti Notevoli}}
\begin{tabularx}{0.8\textwidth}{ |X|X| }
	\hline
	Logaritmo naturale        & $\limite{x}{0} \frac{\ln(1+x)}{x} = 1 $                   \\
	\hline
	Logaritmo con base $a$    & $\limite{x}{0} \frac{\log_a(1+x)}{x} = \frac{1}{\ln(a)} $ \\
	\hline
	$f$ Esponenziale          & $\limite{x}{0} \frac{e^x-1}{x} = 1$                       \\
	\hline
	$f$ Esponenziale base $a$ & $\limite{x}{0} \frac{a^x-1}{x} = \ln(a)$                  \\
	\hline
	Costante e Frazione       & $\limite{x}{0}\frac{ax -1}{x} = \ln(a)$                   \\
	\hline
	Seno                      & $\limite{x}{0}\frac{\sin(x)}{x} = 1$                      \\
	\hline
	Coseno                    & $\limite{x}{0} \frac{1-\cos(x)}{x^2} = \frac{1}{2} $      \\
	\hline
	\hline
	$\ln(x)$                  & $\limite{x}{0} ln(x) = -\infty $                          \\ % 0 %perchè avevo messo 0? $\\
	\hline
\end{tabularx}


\paragraph*{\underline{Equivalenze Asintotiche}}

\begin{tabularx}{0.67\textwidth}{|XYX|}
	\hline
	\multicolumn{3}{|c|}{\textbf{con \emph{x} $\to$ 0}} \\
	\hline
	\hline
	$\sin x$           & $\sim$ & $x$                   \\
	\hline
	$1-\cos x$         & $\sim$ & $\frac{1}{2}x^2$      \\
	\hline
	$\tan x$           & $\sim$ & $x$                   \\
	\hline
	$\ln(1+x)$         & $\sim$ & $x$                   \\
	\hline
	$(1+x)^\alpha -1 $ & $\sim$ & $\alpha x$            \\
	\hline
\end{tabularx}

\paragraph*{\underline{Ordine degli infiniti $\infty$}} In generale:
\begin{center}
	\[ \log_ax\ll x^b\ll x^c\ll d^x\ll g^x\ll x^x \]
\end{center}
\nb{
	\begin{itemize}
		\item 	$\sqrt{x} \gg \ln(x) $
		\item $x \ln (x) \gg \sqrt{x}$
	\end{itemize}
}

\paragraph*{\underline{Forme di indecisione}}.\\
\begin{tabularx}{\textwidth}{|l|X|}
	\hline
	\multicolumn{2}{|c|}{
		$[\frac{0}{0}]$ $[\frac{\infty}{\infty}]$ $[1^\infty]$ $[\infty - \infty]$ $[\infty \cdot 0]$ $[0^0]$ $[\infty^0]$
	}                                                                                       \\
	\hline
	\multicolumn{2}{|X|}{
		\small{Tutte le forme possono essere risolte usando \textbf{Limiti Notevoli} e \textbf{Trucchi algebrici} per ricondursi ad essi.
			In particolare però, questi si risolvono usando anche:}
	}                                                                                       \\
	\hline
	$[\frac{0}{0}]$           & Conf. infinitesmi | Scomp/Racc/Semp | De l'Hopital          \\
	\hline
	$[\frac{\infty}{\infty}]$ & Conf. infinti | Scomp/Racc/Semp | De l'Hopital              \\
	\hline
	$[1^\infty]$              & Identità Logaritmo-Esponenziale                             \\
	\hline
	$[\infty - \infty]$       & Riconduzione a $\frac{0}{0}$ o $\frac{\infty}{\infty}$      \\
	\hline
	$[\infty \cdot 0]$        & Razionalizzazione inversa | Prodotti notevoli al contrario  \\
	\hline
	$[0^0]$ / $[\infty^0]$    & Conf. infiniti/infinitesimi|Identità Logaritmo-Esponenziale \\
	\hline
\end{tabularx}

\subsection*{Teoremi Limiti utili per esercizi}
\paragraph*{Teorema del Confronto} Se ho $x \rightarrow +\infty$ e ho $\sin$ o  $\cos$ potrei dover usare
il teorema del confronto dato che $\sin$ e $\cos$ (NB solo per $x \rightarrow +\infty$)
sono delle costanti che oscillano tra $-1$ e $1$.


\section*{Calcolo Differenziale}

\paragraph*{\underline{Derivate "note"}}
\begin{tabularx}{0.775\textwidth}{|Y|c|c|}
	\hline
	\textbf{Nome}           & \textbf{Funzione} & \textbf{Derivata}     \\
	\hline
	\hline
	Seno                    & $\sin x$          & $\cos x$              \\
	\hline
	Coseno                  & $\cos x$          & $-\sin x$             \\
	\hline
	Arcotangente            & $\arctan$         & $\frac{1}{1+x^2}$     \\
	\hline
	Logaritmo               & $\ln(x) $         & $\frac{1}{x}$         \\
	\hline
	Radice                  &                   &                       \\
	\hline
	Esponenziale            & $e^x$             & $e^x$                 \\
	\hline
	Esponenziale (negativo) & $e^{-x}$          & $ -e^{-x}$            \\
	\hline
	1 su $x^2$              & $\frac{1}{x^2}$   & $-\frac{2}{x^3}$      \\
	\hline
	$x$ alla $\alpha$       & $x^\alpha$        & $\alpha x^{\alpha-1}$ \\
	\hline
\end{tabularx}

\paragraph*{\underline{Derivate Composte}}.\\

\begin{tabularx}{\textwidth}{|l|Y|Y|}
	\hline
	Composizione & $f(g(x))$           & $ f'(g(x))\cdot g'(x)$                                \\
	\hline
	Prodotto     & $f(x)\cdot g(x)$    & $f'(x)\cdot g(x) + g'(x)\cdot f(x)$                   \\
	\hline
	Divisione    & $\frac{f(x)}{g(x)}$ & $\frac{f'(x)\cdot g(x) - g'(x) \cdot f(x)}{[g(x)]^2}$ \\
	\hline
\end{tabularx}

\paragraph*{\underline{Derivata dell'inversa di una funzione}}
Dati:\\
$y_o$ e $f(x)$, avendo $g(x) = f^{-1}(x)$ allora:
Per calcolare $g'(y_0)$
\begin{enumerate}
	\item trovo $x_0$ ponendo $y_0=f(x)$
	\item trovo $g'(y_0)=\frac{1}{f'(x_0)}$
\end{enumerate}

\paragraph*{\underline{Formula di Taylor}} di grado $k$ e centrato in $x_0$:
$$P_k(x)=f(x_0)+f'(x_0)(x-x_0) + \frac{1}{2}f''(x_0)(x-x_0)^2 +... + \frac{1}{k!}f^{(k)}(x_0)(x-x_0)^k$$

\paragraph*{\underline{Formula di Mclaurin}} di grado $k$:
$$P_k(x)=f(0)+f'(0)x+\frac{1}{2}f''(0)x^2+...+\frac{1}{k!}f^{(k)}(0)x^k$$
\small{Mclaurin = Taylor con $x_0=0$}

\paragraph*{\underline{Rapporto incrementale}}
$$\frac{\Delta y}{ \Delta x}\frac{f(x_0+h)-f(x_0)}{h} $$


\section*{Calcolo Integrale}
\paragraph*{Condizione di integrabilità} Per l'integrabilità di una funzione su un intervallo la condizione
che essa sia \emph{continua è sufficiente ma non necessaria}
\begin{multicols}{3}
	\paragraph*{\underline{Primitive elementari}}
	Funzioni il cui integrale è immediatamente calcolabile.
	\columnbreak
	\begin{tabularx}{0.7\textwidth}{ |Y|Y| }
		\hline
		\textbf{Funzione} & \textbf{Primitiva}    \\
		\hline
		$k$               & $kx$                  \\
		$x^a$,$a\neq-1$   & $\frac{x^{a+1}}{a+1}$ \\
		$\frac{1}{x}$     & $\log|x|$             \\
		$\sin x$          & $-\cos x$             \\
		$\cos x $         & $\sin x$              \\
		$a^x$             & $\frac{a^x}{\log a}$  \\
		\hline
		$e^{-x}$          & $-e^{-x}$             \\
		$\frac{1}{x^2+1}$ & $\arctan (c)$         \\
		\hline
	\end{tabularx}
\end{multicols}

\paragraph*{Proprietà degli integrali}
\begin{itemize}
	\item Somma di integrali: $\int f(x)+g(x) dx = \int f(x) dx + \int g(x) dx$
	\item Costante moltiplicativa $\int k \cdot f(x) = k \int f(x)$
\end{itemize}
\subsection*{I metodi di risoluzione}
\paragraph*{Integrazione per Parti}
$$\int f(x)g'(x)dx = f(x)g(x)-\int f'(x)g(x)dx$$
\begin{center}
	\includegraphics[width=0.4\textwidth]{integrazione-per-parti.png}
\end{center}

\paragraph*{\underline{Integrazione per Sostituzione}}
\subparagraph*{\emph{Metodo Generale e Semplificato}} per itnegrali generali $f(x)$:
\\Trovo una funzione $g(x)$ \emph{Derivabile e Invertibile} da sostituire ad $x$.
\begin{enumerate}
	\item decido che $y=g(x)$
	\item Inverto $g(x)$ per isolare la x, ottenendo $x=g^{-1}(y)$
	\item Derivo entrambi i membri e aggiungo $dx$ e $dy$: $\to dx=(g^{-1})'(y)dy$
	\item all'interno di $f(x)$ sosdtituisco $g(x) \to y$ e $dx \to (g^{-1})'(y)dy$
	\item Risolvo l'integrale
	\item Sostiuisco $y \to g(x)$
\end{enumerate}
\subparagraph*{Metodo dalla definizione}: Abbiamo un integrale nella forma
$$\int f(g(x))g'(x) dx$$
\begin{enumerate}
	\item $y=g(x)\to dy=g'(y)dx$
	\item Sostituiamo per ottenere $\int f(y)dy$
	\item Calcolo l'integrale nella nuova variabile
	\item Sostituisco $y\to g(x)$
\end{enumerate}

\paragraph*{Formula Media Integrale}
Considerata f limitata e integrabile su un intervallo $[a,b]$
\begin{equation*}
	M(f,[a,b])=\frac{1}{b-a}\int_a^b f(x)dx
\end{equation*}

\section*{Dimostrazioni per induzione}
Le due casistiche principali sono:
\begin{itemize}
	\item Dimostrazioni con la sommatoria $\sum$
	\item Dimostrazioni con disequazioni
\end{itemize}
\paragraph*{Ricorda} Devi sempre dimostrare che la formula è vera per $n+1$, quindi devi ricondurti a
ciò che hai a destra dell'equazione.

\subsection*{Dimostrazioni con la sommatoria}
In questo caso devo ricordarmi di ricondurmi al caso base estrando dalla sommatoria (n+1)
per ricondurmi alla sommatoria $\sum^n$ e poi sostituendo l'ipotesi induttiva
(la sommatoria che supponiamo vera). Così facendo posso ottenere ciò che ho a sinistra della
formula $\sum^{n+1}$.
\subsection*{Dimostrazioni con le disequazioni}
In questo caso devo ricordarmi che oltre a dover sostituire l'ipotesi induttiva nella disequazione
possono aggiungere numeri che mi possono servire a patto che abbia la certezza che questi numeri non
vadano in contraddizione con il segno della disequazione, quindi se ho $a>b$, aggiungendo numeri non deve succedere
che b diventi maggiore di a.
\paragraph*{Ricorda} Nell'ipotesi avrai una condizione (per esempio per $n>1$), ricordati che puoi e spesso
devi usarla per poter aggiungere numeri utili alla dimostrazione.


\end{document}