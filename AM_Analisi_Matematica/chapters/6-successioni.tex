
\chapter{Successioni}

\section{Introduzione}
\definizione{
	Una \emph{Successione di numeri reali} è una funzione $a: \N \to \R$.\\
	La variabile Indipendente viene di solito indicata con $n$ mentre per la funzione alla notazione $a(n)$ si preferisce $a_n$
}
\esempio{
	$a_n = 2n-3$ è una successione.
	$a_0 = -3$, $a_1 = -1$, $a_2 = 1$, $a_3 = 3$ , ... sono i suoi valori
}

\paragraph*{Limitazioni} Una successione $\{a_n\}$ si dice:
\begin{itemize}
	\item \textbf{Inferiormente limitata} se esiste $m \in \R$ tale che $a_n \geq m \quad \forall n \in \N$
	\item \textbf{Superiormente limitata} se esiste $M \in \R$ tale che $a_n \leq M \quad \forall n \in \N$
	\item \textbf{Limitata} se esistono $m \in \R$ e $M \in \R$ tali che $m \leq a_n \leq M \quad \forall n \in \N$
\end{itemize}
\esempio{
	La successione $\{\frac{1}{n+1}\}$ i cui primi termini sono $1, \frac{1}{2}, \frac{1}{3},$...\\
	è \emph{limitata}: si ha infatti che $0 < a_n \leq 1 \forall n \in \N$
}

\paragraph*{Monotonia} Una successione $\{a_n\}$ si dice:
\begin{itemize}
	\item \textbf{Monotona crescente} se $a_n\leq a_{n+1} \forall n \in \N$
	\item \textbf{Monotona strettamente crescente} se $a_n < a_{n+1} \forall n \in \N$
	\item \textbf{Monotona decrescente} se $a_n \geq a_{n+1} \forall n\in\N$
	\item \textbf{Monotona strettamente decrescente} se $a_n > a_{n+1} \forall n \in \N$
\end{itemize}

\paragraph*{Definizione} Si dice che una successione {$a_n$} possiede (o acquista) una certa proprietà \textbf{definitivamente} se esiste $N \in \N$ tale che
$a_n$ soddisfa quella proprietà per ogni $n \geq N$, quindi da un certo $n$ in poi.

\esempio{Queste serie:\\
	$a_n = 2n-3$ $[-3,-1,1,3,5,...]$  è \emph{Definitivamente positiva}\\
	$a_n =\frac{1}{n+1}$ $[1, \frac{1}{2}, \frac{1}{3},\frac{1}{4}...] $ è \emph{Definitivamente minore di $\frac{1}{3}$}
}

\section{Limiti di Successioni}
\paragraph*{}Calcolare il limite di una successione (che si indica con $\limite{n}{+\infty}a_n$ o $\lim a_n$)
è equivalente a chiedersi che tipo di comportamento ha la successione quando $n$ tende a diventare molto molto grande.

\paragraph*{Definizione} Sia $a_n$ una successione, e sia $l \in \R$.
Si dice che $a_n$ ha per limite $l$ per $n$ tendente all'infinito (o che $a_n$ tende a $l$)
se la successione $a_n$ si trova \emph{definitivamente} in ogni intorno di $l \in \R$
e si scrive $\limite{n}{+\infty} a_n = l$

Quando si calcola il limite bisogna però distinguere 4 casistiche:
\begin{enumerate}
	\item \textbf{Convergenza}, se $l\in\R$, si ha $\lim a_n = l$ sse $\forall \epsilon > 0$ esiste un indice $n_\epsilon\in\N$ tale che si abbia:
	      \begin{center}
		      $|a_n-l|\leq\epsilon$ per ogni $n \geq n_\epsilon$
	      \end{center}
	\item \textbf{Divergenza a $+\infty$}, si ha $\lim a_n = +\infty$ sse $\forall a\in \R$ esiste un indice $n_a\in\N$ tale che si abbia:
	      \begin{center}
		      $a_n\geq a$ per ogni $n \geq n_a$
	      \end{center}
	\item \textbf{Divergenza a $-\infty$}, si ha $\lim a_n = -\infty$ sse $\forall a\in \R$ esiste un indice $n_a\in\N$ tale che si abbia:
	      \begin{center}
		      $a_n\leq a$ per ogni $n \geq n_a$
	      \end{center}
	\item \textbf{Indeterminazione} se non si verifica nessuno dei casi precedenti, quindi il $\lim a_n$ non esiste.
\end{enumerate}
\paragraph*{In parole povere} La successione converge se la successione al crescere di n va ad avvicinarsi sempre di più a $l$,
diverge a $+\infty$ se esiste un indice che permette alla successione di essere più grande di un qualsiasi numero Reale
e infine diverge a $-\infty$ se esiste un indice che permette alla successione di essere più piccolo di una qualsiasi numero Reale.
\subsection*{Teorema di esistenza del limite}
\definizione{
	Se una successione è monotona crescente allora il limite esiste e coincide con l'estremo superiore della successione.
	Se invece è monotona decrescente allora il limite esiste e coincide con l'estremo inferiore della successione.
}
\subsection*{Teorema di permanenza del segno - 1 forma}
\definizione{
	Se $a_n \rightarrow a$ e $a > 0$ allora $a_n > 0$ definitivamente.
	Se $a_n \rightarrow a$ e $a < 0$ allora $a_n < 0$ definitivamente.
}
\subsection*{Teorema di permanenza del segno - 2 forma}
\definizione{
	Se $a_n \rightarrow a \in \mathbb{R}$, e $a_n \geq 0$ definitivamente, allora risulta $a \geq 0$
	più in generale:
	\\ Se $a_n \rightarrow a$, $b_n \rightarrow b$ e $a_n \geq b_n$ definitivamente, allora $a \geq b$.
}
\subsection*{Teorema del confronto}
\definizione{
	Se $a_n \leq b_n \leq c_n$ definitivamente e $a_n \rightarrow l$, $c_n \rightarrow l \in \mathbb{R}$
	\\ allora anche $b_n \rightarrow l$.
}
\section{Principio di Induzione}
\subsection*{Introduzione}
Il principio di Induzione è un teorema noto, che è spesso utile per discutere la validità di una successione \emph{infinita} di proposizioni in un \emph{numero finito di passi}.

\definizione{
	Sia data una proposizione $P(n)$, $\forall n \in \N$ con $n > n_0$ .
	\\Se sono soddisfatte le seguenti condizioni:
	\begin{enumerate}
		\item $\exists n_0$ tale che $P(n_0)$ è VERA
		\item $\forall n \geq n_0$ vale l'implicazione $P(n-1) \implies P(n)$
	\end{enumerate}
	\emph{Allora $P(n)$ è vera per ogni $n \geq n_0$}
}
\paragraph*{}
Una \emph{propisizione} è un qualunque enunciato il cui valore dipende da $n$, per esempio "$n$ è pari", "$n$ è primo".

\esempio{
	Consideriamo l'enunciato: $P(n): 2^n \geq n^3$,
	e proviamo che "$P(n)$ vera $\implies P(n+1)$ vera" per ogni $n\geq 7$.
	Infatti per $n \geq 7$ risulta:
	\begin{equation*} %TODO vai a capire cosa vuol dire sta roba
		2^{n+1} = 2\cdot 2^n \geq 2 \cdot n^3 = n^3 + n \cdot n^2 \geq n^3 + 7n^2 \geq n^3 + 3n^2 +3n+1 = (n+1)^3
	\end{equation*}
	e quindi per ogni $n \geq 7$, se $2^n \geq n^3$ allora $2^{n+1} \geq (n+1)^3$.
	D'altra parte, semplici calcoli mostrano che $P(n)$ è falsa per $n = 7,8,9$, mentre è vera per $n=10$.
	Sintetizzando si può dire che $P(n)$ è \emph{induttiva}, cioè verifica l'implicazione, per ogni $n \geq 7$, mentre è vera per ogni $n \geq 10$, dal momento che (1) è verificata con $n_0=10$.
}

\esempio{ \emph{Somma di Gauss}\\
	Proviamo per induzione che $\forall n\in\N$ risulta:
	\begin{equation*}
		P_n: \sum_{k=0}^n k= \frac{n(n+1)}{2}
	\end{equation*}
	ovvero, la somma dei primi $n \in\N$ numeri equivale a $\frac{n(n+1)}{2}$
	\paragraph*{}Questa formula è banalmente vera per $n=0$.
	Mentre per il passo induttivo si può provare come segue:
	\begin{equation*}
		\sum_{k=0}^{n+1} k = \sum_{k=0}^{n} k + (n+1) = \frac{n(n+1)}{2} + (n+1) = \frac{(n+1)(n+2)}{2}
	\end{equation*}
	\paragraph*{Spiegazione}
	se guardi $\frac{(n+1)(n+2)}{2}$ si può notare che equivale a $\frac{n(n+1)}{2}$ se ad $n$ sostituisci $n+1$
	%TODO trovare un modo migliore per spiegare sta roba mannaggia a te
}

\esempio{
Un esempio sostanzialmente analogo è il seguente:\\
\begin{equation*}
	P_n: \sum_{k=0}^n x^k = \frac{1-x^{n+1}}{1-x}
\end{equation*}
\paragraph*{}Questa formula è banalmente vera per $n=0$.
Mentre per il passo induttivo si può provare come segue:
\begin{equation*}
	\sum_{k=0}^{n+1} x^k = \sum_{k=0}^n x^k \cdot x^{n+1} = \frac{1-x^{n+1}}{1-x}\cdot x^{n+1} = \frac{1-x^{n+2}}{1-x}
\end{equation*}
\paragraph*{Spiegazione} Osserviamo che $\frac{1-x^{n+2}}{1-x}$ equivale a $\frac{1-x^{n+1}}{1-x}$ sostituendo $n$ con $n+1$
}
