\chapter{Calcolo Integrale}

\section{Gli integrali}

\definizione{
	L'integrale di una funzione su $a$ e $b$ ($\int_{a}^{b} f(x)dx$) è \emph{l'area sottesa della funzione nell'intervallo [$x=a, x=b$]}
}

\subsection{Classi di Funzioni Integrabili}
Le classi di funzioni integrabili sono famiglie di funzioni dotate di particolari proprietà che ne garantisconto l'integrabilità \emph{su un intervallo chiuso e limitato}.
Riporto qui le 3 principali condizioni \emph{Sufficienti (ma non necessarie) per l'integrabilità}:
\\Sia $[a,b]$ un intervallo chiuso e limitato.
\begin{enumerate}
	\item Se la funzione $f:[a.b] \to \R$ è \textbf{Continua}, allora essa è integrabile su $[a,b]$.
	\item Se la funzione $f:[a.b] \to \R$ è \textbf{Limitata} e con un \textbf{numero finito di discontinuità}, allora essa è integrabile su $[a,b]$
	\item Se una funzione $f$ è \textbf{Monotona} e definita in $[a,b]$, allora essa è integrabile su $[a,b]$ 
\end{enumerate}

\subsection{Il calcolo degli integrali}
Per calcolare $\int_{a}^{b} f(x) dx$ devo:
\begin{enumerate}
	\item Trovare una funzione che nell'intervallo $[a,b]$ abbia $f(x)$ come derivata, ovvero una \textbf{primitiva} di $f(x)$
	\item Calcolare il valore della primitiva negli estremi di integrazione ($F(a)$ e $F(b)$)
	\item Sottrarre i due valori ($F(b)-F(a)$)
\end{enumerate}
Per cui data una funzione $f(x)$ e la sua primitiva $F(x)$: $$\int_{1}^{b} f(x)dx = F(b)- F(a)$$
\esempio{
	$\it_{0}^{5} 3x^2 dx = [x^3]_{0}^{5} = 5^3 - 0^3 = 125$
}

\subsection{Proprietà degli integrali}
\begin{itemize}
	\item $\int[f(x) \pm g(x)] dx = \int g(x) dx \pm \int f(x) dx$
	\item $\int K f(x) dx = K\int f(x) dx$
	\item $\int_{a}^{b} f(x) dx = \int_{a}^{c} f(x) dx + \int_{c}^{b} f(x) dx$
\end{itemize}

\section{Le primitive}

Si dice primitiva (o antiderivata) di una funzione $f$ una funzione \emph{derivabile} $F$ la cui derivata è uguale alla funzione di partenza.
\\Quindi:
\definizione{
 Si dice che $F(x)$ è una \textbf{primitiva} di $f(x)$ se:
 \begin{itemize}
	\item $F(x)$ è \textbf{derivabile}
	\item La derivata di $F(X)$ è $f(x)$ : $F'(x) = f(x) \forall x \in (a,b)$
 \end{itemize} 
}
Dalla definizione si può notare che:
\begin{itemize}
	\item Ogni funzione $f:[a,b] \to \R$ \emph{continua} ammette primitive
	\item Le primitive non sono \emph{mai uniche}
\end{itemize}
Per indicare una generica primitica si utilizza la notazione $\int f(x) dx$ (integrale indefinito).

\subsection{Primitive Elementari}
Alcune primitive sono immediate da trovare, si chiamano \emph{primitive elementari}\\
\begin{center}
	\begin{tabularx}{0.5\textwidth}{|X|X|}
		\hline
		$f(x)$            & $F(x)$                                  \\
		\hline
		\hline
		$cos(x)$          & $sin(x)$                                \\
		\hline
		$sin(x)$          & $-cos(x)$                               \\
		\hline
		$e^x$             & $e^x$                                   \\
		\hline
		$a^x$             & $\frac{a^x}{\ln a} $                    \\
		\hline
		$x^n$             & $\frac{x^{n+1}}{n+1} \forall n\neq -1 $ \\
		\hline
		$\frac{1}{x}$     & $\ln |x| $                              \\
		\hline
		$\frac{1}{1+x^2}$ & $arctg(x) $                             \\
		\hline
	\end{tabularx}
\end{center}


\subsection{Integrali Quasi Immediati}

Anche dette \emph{primitive "elementari" generalizzate}, ci sono dei casi in cui le funzioni composte sono facilmente derivabili.

\definizione{
	Sia $F(x)$ una primitiva di $f(x)$ e sia $g(x)$ una funzione derivabile e tale che \emph{sia possibile costruire la funzione composta $F(g(x))$}.
	In questo caso:
	$$
	[F(g(x))]' = f(g(x)) \cdot g'(x) \implies \int f(g(x)) \cdot g'(x) dx = F(g(x))+c
	$$
}
\paragraph*{Ovvero} se abbiamo una funzione del tipo $f(g(x)) \cdot g'(x)$ il suo integrale sarà semplicemente $F(g(x)) + c$, quindi bisogna trovare \emph{la primitiva solo della funzione esterna}.
\\\'E comunque ovvio che si possono effettuare "giochetti" per smontare e rimontare la nostra funzione in modo da ottenere una formad i questo tipo.
\esempio{ 
	$\int 3x^2 \cdot\sin(x^3) dx = -\cos(x^3) + c$
}
\esempio{
	$\int x (x^2)^3 dx = \frac{1}{2}\int 2x (x^2)^3 dx = \frac{1}{2}\frac{(x^2)^4}{4} + c = ...$
}
Da quest'ultimo esempio si può notare sia che abbiamo moltiplicato e diviso per due in modo da ottenere la derivata della funzione interna,
ma si può anche dedurre che:
$$
\int f'(x) \cdot [ f(x)]^\alpha dx = \frac{[f(x)]^{\alpha+1}}{\alpha + 1} + c
$$

\paragraph*{Tips}
Questo tipo di integrazione la si può sempre "inventare" quando si hanno delle funzioni composte la cui funzione interna ha come derivata una costante,
oppure la sua derivata è facilmente riconducibile tramite trucchetti algebrici.

\section{Integrazione per Parti}
Se le funzioni da integrare non sono immediate, si può usare l'integrazione per parti, che dice:
\definizione{
	Siano $f(x)$ e $g(x)$ due funzioni, allora
	$$\int f(x) g'(x) dx = f(x)g(x) - \int f'(x)g(x) dx$$
}
In pratica, l'integrale di una funzione derivabile moltiplicata per una funzione integrabile è integrabile per parti.
\\Questa tecnica può essere usata per risolvere integrali complessi.
\esempio{
	$$\int x \cos(x) dx = x \sin(x) - \int \sin(x) dx = x \sin(x) + \cos(x) + c$$
	\\In questo esempio, $f(x) = x$, quindi $f'(x) = 1$ e $g'(x)= \cos(x)$, quindi $g(x) = \sin(x)$.
}

\subsection*{Tecnica della moltiplicazione per 1}
In alcuni casi ci potremmo trovare degli integrali che appaiono molto semplici, ma in realtà non sono per niente \emph{banali}.
Ad esempio alcuni integrali con un solo termine non sono risolvibili da soli, quindi si può utlizzare la tecnica della \emph{moltiplicazione per 1}.
\\Dato un integrale, lo si può moltiplicare (esplicitamente) per 1 in modo da permetterci di usare l'integrazione per parti, dato che il risultato non è modificato e 1 è facilmente integrabile.
\esempio{
	$$\int \ln(x) dx = \int \ln(x) \cdot 1 dx$$
	Questo ci permette di considerare:
	\\ $f(x) = \ln(x) \implies f'(x) = \frac{1}{x}$ e $g'(x) = 1 \implies g(x) = x$.
	\\Di conseguenza, la nostra integrazione prosegue così:
	$$ = \ln(x) \cdot x - \int \frac{1}{x} \cdot x dx = x\ln(x) - \int 1 dx = x \ln(x) - x + c$$
}

\section{Integrazione per Sostituzione}
L'integrazione per sostituzione è uno dei metodi di integrazione più utili, ma anche uno dei più complicati da capire.
In questa sezione introdurremo sia la definizione generale di questo metodo, che le sue effettive implementazioni.
\\Nota che \textbf{ai fini dell'esame è utlie soltanto la seconda implementazione} nella versione semplificata.

\paragraph{Quando si usa?} 
In generale mi accorgo di dover usare la sostituzione quando ho una particolare espressione della $x$ ripetuta che "mi da fastidio", 
oppure quando usare le altre tecniche mi risulta impossibile.

\paragraph*{Definizione} lo scopo dell'integrazione per sostituzione è quello di \emph{trasformare l'integrale in modo da renderlo più semplice e riconducibile a formule immediate}.

\definizione{
	Siano $f:I\to \R$ una funzione continua, e $g:J\to \R$ una funzione derivabile con derivata continua.
	Si ha che:
	$$\int_{a}^{b} f(g(x)) g'(x) dx = \int_{g(a)}^{g(b)} f(y) dy$$
	Nel caso in cui la funzione $g(y)$ sia \textbf{invertibile}, allora vale la seguente formula di integrazione:
	$$\int f(x)dx = \int f(g(y))g'(y)dy$$
	Queste due formule sono equivalenti e bisogna scegliere quale usare in base a come si presenta la funzione integranda.
}
Si può notare che una definizione è l'inverso dell'altra, ed entrambe sono valide (ovviamente).
La seconda definizione è quella più utile nella pratica, però nota che si può usare solo quando $g$ è \textbf{invertibile!}  

\subsubsection*{Primo Uso}
Questa implementazione segue la \emph{prima definizione}, quindi richiede che l'integrale da calcolare si trovi nella forma:
$$\int f(g(x)) g'(x) dx$$
\begin{enumerate}
	\item Poniamo $y=g(x)$ e lo deriviamo, cos' da ottenere il nuovo differenziale $dy = g'(y)dx$
	\item Sostituiamo i nuovi valori nell'integrale, cos' da ottenere la forma $\int f(y) dy$
	\item Calcoliamo l'integrale nella nuova variabile (sperando che sia più semplice) ottendendo così $F(y) + c$
	\item Sostiuiamo a $y$ l'espressione della funzione $g(x)$.
\end{enumerate}
\esempio{Calcola l'integrale di $2xe^{x^2}$.
	\\Si può ben notare che $2x$ è la derivata di $x^2$, quindi questa funzione si presenta nella forma $g'(x)f(g(x))$.
	Poniamo quindi $y=g(x)=x^2$ e $dy=g'(x)dx= 2x dx$.
	\\Sostituiamo, $\int 2xe^{x^2} = \int e^y dy = e^y +c$.
	\\Ri sostiuiamo $y$ con $x^2$ e troviamo $e^{x^2} + c$.
	\\\small{Nota che questo era comunque un integrale quasi immediato}
}

\subsubsection*{Secondo Uso}
Se una generica funzione $\int f(x) dx$ ha un'espressione della $x$ "scomoda" che si ripete, la possiamo sostituire.
Troviamo quindi, dentro $f(x)$, una funzione $g(y)$ \textbf{INVERTIBILE} (che sarà l'espressione scomoda di $x$) e effettuo i seguenti passi:

%Supponiamo di voler calcolare l'integrale di una generica funzione $\int f(x)dx$, in cui però troviamo un'espressione "complicata" della $x$ che si ripete.
%In questo caso andremo alla ricerca di una funzione $g(y)$ da sostiuire alla variabile $x$.
%Qui però dobbiamo richiedere che la funzione $g(y)$ sia non solto derivabile, ma anche invertibile!

\begin{enumerate}
	\item Poniamo $x=g(y)$. lo derivo in modo da ottenere $dx=g'(y)dy$
	\item Sostiuiamo il tutto nell'integrale in modo da ricavare $f(g(y))g'(y)$
	\item Calcoliamo l'integrale ottenuto (nella speranza che sia di più facile risoluzione)
	\item Sostiuiamo di nuovo per tornare nella variabile $x$. Per farlo però è necessario esprimere la variabile $y$ in funzione di $x$ determinando la funzione inversa $y=g^{-1}(x)$
\end{enumerate}

\subsection{Metodo pratico dell'integrazione per parti}
Esiste un processo "semplificato" per utilizzare la seconda definizione.
\\Avendo $\int f(x) dx$, vado a cercare un'espressione di $x$ che vado a rimuovere e sostituire con $y$ per semplificare l'integrale.
Questa "espressione di $x$" la chiameremo $g(x)$ ed essa deve \emph{necessariamente essre invertibile}.

Quindi, avendo $\int f(x)$ che contiene una $g(x)$ che voglio togliere per semplificare il tutto
\begin{enumerate}
	\item Decido che $y=g(x)$
	\item Inverto $g(x)$ in modo da isolare la $x$, ottenendo $x=g^{-1}(y)$
	\item Derivo entrambi i membri e "moltiplico" per $dx$ e $dy$ in modo da ottenere: $dx=(g^{-1})'(y)dy$
	\item All'interno della funzione sostiuisco $g(x)$ con $y$ e $dx$ con $(g^{-1})'(y)dy$
	\item A questo punto mi (dovrei) trovare un integrale più facile, che posso risolvere
	\item Una volta risolto sostiuisco $y$ con $g(x)$.
\end{enumerate}

% Quindi, se abbiamo una funzione da integrare che è formata da una funzione composta $f(g(x))$ moltiplicata alla derivata della funzione interna $g'(x)$,
% si può sostituire con una ipotetica $y$ per sepmplificare.
% \esempio{
% 	$\int sin(e^x) e^x dx$. consideriamo:
% 	\\$y= e^x$, e $dy = e^x dx$. Nota che $dy$ indica la derivata di $y$.
% 		\\Quindi $\int sin(y) dy = -\cos(y) + c = -\cos(e^x) + c$.
% }
\subsection{Integrazione di Funzioni Razionali}
Sono funzioni razionali tutte quelle funzioni del tipo:
$$f(x)=\frac{P(x)}{Q(x)} \text{ con } P(x),Q(x) \text{ Polinomi} $$
Questo tipo di integrale si risolve in modi diversi in base al grado/tipo di numeratore e denominatore:
\paragraph*{Numeratore è la Derivata del Denominatore}
$$\int \frac{f'(x)}{f(x)} = \ln|f(x)|+ c$$
\esempio{
	\\$\int \frac{2x}{x^2-5} = \ln|x^2-5|+c$
	\\Oppure:
	\\$\int \frac{x^3}{x^4+1} = \frac{1}{4}\int \frac{4x^3}{x^4+1} = \frac{1}{4} \ln|x^4+1|+c$
}
\paragraph*{Numeratore Costante e Denominatore di I° grado}
La risoluzione è sostanzialmente uguale al caso precedente, varrebbe ricordarsi però la formula:
$$\int\frac{k}{ax+b} dx = \frac{k}{a} \ln|ax+b|$$
\esempio{$\int 	\frac{3}{2x-1} dx = \frac{3}{2} \int \frac{2}{2x-1} = \frac{3}{2} \ln|2x-1|+c$}

\paragraph*{Numeratore costante e Denominatore al Quadrato}
Bisogna ricordarsi che:
$\frac{1}{[f(x)]^2} = [f(x)]^{-2}$ e che 
$\int f'(x)[f(x)]^n dx = \frac{[f(x)^{n+1}]}{n+1}$ (primitiva di funzione elementare generalizzata)

\esempio{
	$\int \frac{5}{(2x-1)^2} = 5\int (2x-1)^{-2} = \frac{5}{2} \int 2(2x-1)^{-2} = \frac{5}{2} \cdot \frac{(2x-1)^{-1}}{-1} + c = -\frac{5}{4x-2} + c$
}
\paragraph*{Funzioni razionali del tipo $\frac{dx-c}{ax^2+bx+c}$ con $\Delta < 0$}
In questo caso, bisogna smontare la funzione e ricondursi a integrali del tipo:
\begin{itemize}
	\item $\int\frac{f'(x)}{1+[f(x)]^2} dx = \arctan[f(x)]+c$
	\item $\int\frac{f'(x)}{f(x)} = \ln|f(x)| +c $ 
\end{itemize}
\esempio{
	$\int \frac{3x+1}{x^2+1} dx$.
	\\Spacco in due la funzione e la gestisco come due integrali separati
	$\int \frac{3x}{x^2+1} + \int \frac{1}{x^2+1} = \frac{3}{2}\int \frac{2x}{x^2+1} +  \int \frac{1}{x^2+1} = \frac{3}{2}\ln|x^2+1| + \arctan(x) +c$
}
\esempio{Più complicato:
$$\int \frac{18x+3}{9x^2+6x+2}$$
\\Bisogna ragionare:
\\Voglio ottenere una funzione del tipo $\frac{f'(x)}{f(x)}$, quidni calcolo la derivata del denominatore:
$D'=18x+6$ e cerco di trasformare il nuemratore in esso.
Per fare ciò, posso vedere quel "$+3$" al numeratore come un "$+6-3$"
$$ = \int \frac{18x+6-3}{9x^2+6x+2} = \int \frac{18x+6}{9x^2+6x+2} -\int \frac{-3}{9x^2+6x+2} = ln|9x^2+6x+2| - \int\frac{3}{9x^2+6x+2} +c$$
Adesso il secondo integrale lo cerco di trasformare in una funzione del tipo $\frac{f'(x)}{1+[f(x)]^2}$.
Guardo il denominatore e cerco di trasformarlo in $\frac{f'(x)}{f(x)}$:
$9x^2 = (3x)^2$, $6x = 2\cdot 3x \cdot 1$ (come (2ab)) e $+2 = +1+1$. 
Quindi il denominatore diventa: $(3x+1)^2+1$.
$$= \ln|9x^2+6x+2| - \arctan(3x+1) +c$$
}
\section{Altre Proprietà importanti}
\subsection*{Integrale di una funzione dispari}
Sia $f : \R \to \R$ una funzione continua e dispari,  $\int_{-a}^{a} f(x) dx = 0$.\\
Perchè essendo la funzione simmetrica rispetto all'origine, $\int_{-a}^{0} f(x) dx = - \int_{0}^{a} f(x)$, quindi
$\int_{-a}^{-a} f(x) dx = \int_{-a}^{0} f(x) dx + \int_{0}^{-a} f(x) dx = 0$.

% Serie Numeriche
