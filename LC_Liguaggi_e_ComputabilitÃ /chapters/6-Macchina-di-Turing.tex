\chapter{Macchina di Turing}
La Macchina di Turing (in seguito abbreviata con MdT), è definita come
una macchina astratta costituita da un nastro infinito e una testina che
legge ogni carattere del nastro e a seconda della funzione di transizione lo
modifica oppure lo lascia invariato e sposta la testina a destra o sinistra.
\\\'E definita come una settupla:
\begin{equation}
    M=(Q,\Sigma,\Gamma,\delta,q_0,B,F)
\end{equation} 
\begin{itemize}
    \item Q - insieme di stati
    \item $\Sigma$ - alfabeto di input
    \item $\Gamma$ - alfabeto del nastro
    \item $\delta$ - funzione di transizione
    \item $q_0$ - stato iniziale
    \item B - simbolo di blank
    \item F - insieme di stati finali
\end{itemize}
