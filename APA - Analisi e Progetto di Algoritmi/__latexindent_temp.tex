\documentclass[12pt, a4paper, openany, twoside]{book}

\begin{document}
\author{Fabio Ferrario}
\title{Analisi e Progetto di Algoritmi}
\maketitle

\tableofcontents

\chapter*{Cammini minimi}

\section{Definizioni e proprietà}
Sia G=(V,E) un grafo orientato con costi w sugli archi, 
il costo di un cammino p=\textless$ v_1,v_2,...v_k$\textgreater è dato
dalla somma del peso di tutti i vertici di quel cammino

\paragraph*{Cammino minimo} tra una coppia di vertici $x$ e $y$ è un cammino di costo
minore o uguale a quello di ogni altro cammino tra gli stessi vertici

\paragraph*{Sottostruttura ottima}: ogni sottocammino di un cammino minimo è anch'esso minimo

\paragraph*{Albero dei cammini minimi}: I cammini minimi da un vertice $s$ a tutti gli altri vertici del grafo possono
essere rappresentati tramite un albero radicato in $s$, detto albero dei cammini minimi

\section{Algoritmo di Dijkstra}

L’algoritmo di Dijkstra risolve il problema dei cammini minimi da sorgente unica.\\
Vogliamo trovare un camminimo minimo che va da un dato vertice sorgente s $\epsilon$ V a ciascun vertice v $\epsilon$ V in un grafo orientato
pesato G = (V, E)

\paragraph{Attenzione}
Dijksta funziona solo se w $\geq$  0 $\forall$ w $\epsilon$ W


\end{document}