%Questo documento serve a creare dei PDF temporanei
%Che posso condividere


\documentclass[12pt, a4paper, openany]{book}
\usepackage{fstyle}

\begin{document}

\section*{18 Gennaio 2022}
\subsection*{E1}
\paragraph*{Testo}
Si consideri la sequenza $X$ sull'alfabeto $A$ di lunghezza $n$ ed una funzione $\phi:A\to Z$ che associa ad ogni carattere un numero intero.
Si vuole calcolare, mediante la tecnica della programmazione dinamica, la lunghezza di una più lunga sottosequenza di $X$ in cui i numeri associati ai caratteri appaiono alternando un pari ad un dispari.
RISPONDERE PER PUNTI alle seguenti richieste
\begin{enumerate}
	\item Esplicitare e definire le variabili che servono per risolvere il problema
	\item Scrivere l'equazione di ricorrenza per il CASO BASE, giustificando perchè è fatta in quel modo
	\item Scrivere la/le equazione/i di ricorrenza per il PASSO RICORSIVO, giiustificando perchè è/sono fatte in quel modo
	\item Scrivere qual'è la soluzione del problema, espressa rispetto alle variabili introdotte.
	\item Scrivere quindi, in pseudocodice, l'algoritmo relativo.
\end{enumerate}

\paragraph*{Sottoproblema e Variabili}
Per il sottoproblema di taglia $i$ introduco la variabile $S$ di dimensione $n$ il cui generico elemento
$S[i]$ con $i\in {1,...,n}$ è la lunghezza della più lunga sottosequenza di $X$ che termina con l'$i$-esimo elemento di $X$.

\paragraph*{Caso Base} $i=1$.
\\Con una sottosequenza di dimensione 1 le condizioni sono sempre rispettate, quindi:
$$
S[1]= \begin{cases}
	1
\end{cases}
$$
\paragraph*{Passo Ricorsivo} $i>1$
Nel caso in cui $i>1$ si introduce $0<j<1$. 
$S[i]$ sarà il massimo di tutti gli $S[j]$ in cui $S[j]\%2 \neq S[i]\%2$ sia rispettata, aumentata di 1.
Quindi:
$$
S[i]= \begin{cases}
	1 + max(S[j]) | 0 < j < i, S[j]\%2 \neq S[i]\%2
\end{cases}
$$
\paragraph*{Soluzione}
La soluzione del problema è il massimo di tutti gli $S[i]$.

\paragraph*{Pseudocodice}
\begin{lstlisting}[style=small]
def E1(X):
	S[1] = 1
	for i = 2 to n
		max = 0
		for j = 0 to i
			if S[j] > max AND X[j]%2 $\neq$ X[i]%2
				max = S[j]
		S[i] = max + 1
	return max(S) 
	
\end{lstlisting}
\end{document}