\documentclass[12pt, a4paper, openany]{book}
\usepackage{fstyle}

\begin{document}
\title{Preparazione all'esame di Analisi e Progetto di Algoritmi}
\author{Fabio Ferrario}
\date{2022}
\maketitle
\tableofcontents

\chapter{Come si risolvono gli esercizi}
\section{Introduzione}
All'esame di Analisi e Progetto di Algoritmi vengono (generalmente) proposti degli esercizi che ti richiedono di scrivere delle varianti di alcuni algoritmi.
Per risolvere questi esercizi bisogna innanzitutto \emph{sapere bene gli algoritmi di cui bisogna fare le varianti}.
Questi generalmente sono:
\begin{itemize}
	\item Algoritmo di Floyd-Warshall
	\item Algoritmi BFS e DFS
	\item Algoritmo LCS
	\item Raramente Algorimto di Dijkstra
\end{itemize}
Se conoscete bene questi algoritmi vi basterà conoscere i procedimenti che riporto qua sotto per poter risolvere (bene o male) ogni tipo di esercizio.

\section{Varianti di FW}
Un possibile tipo di esercizio d'esame è problema in cui is chiede di creare un algoritmo di programmazione dinamica
che, dato un grafo pesato e una funzione che associa un numero o un colore a ogni arco o vertice, stabilisce {il peso di}/{se seiste} un cammino minimo da $i$ a $j$
che rispetti alcuni vincoli.

\paragraph{Il procedimento}è sostanzialmente sempre lo stesso, bisogna trovare un sottoproblema $k$-esimo e capire se le informazioni date da i vertici $(i,k)$ e $(k,j)$
sono sufficienti a risolvere il problema. in caso contrario bisogna definire un problema ausiliario.

\subparagraph{Variabili associate} del sottoproblema (quindi quelle introdotto) sono sempre: il matricione $D^k$ in cui $d^k_{ij}$ è il peso/l'esistenza del cammino da i a j
i cui vertici intermedi appartengono a $\{1,...,k\}$

\subparagraph{soluzione del sottoproblema}
Se le informazioni sono sufficienti, si formano le due equazioni di ricorrenza in cui si calcola $d^k_{ij}$, per il caso base e per il Passo Ricorsivo.
\\Il passo ricorsivo (k>0) è sempre diviso in due casi ipotetici (sibilla cumana), il caso in cui k NON fa parte del cammino minimo, in cui il valore di $d^k_{ij}$ diventerà
il valore del cammino ij escludendo k come cammino intermedio, quindi k-1.\\Oppure il caso in cui k fa parte del cammino minimo, in cui il valore di $d^k_{ij}$ diventerà la somma (oppure l'AND) dei valori dei cammini ik e kj (tutti e due escludendo k come cammino intermedio)
\\L'equazione di ricorrenza diventerà quindi il minimo dei due casi (oppure un OR nel caso in cui si voglia verificare l'esistenza).\\
Il caso base invece è dove i calcoli veri vengono fatti. in genere il valore di $d^k_{ij}$ ha tre possibilità: se i=j assumerà 0 o TRUE, perchè se i e j coincidono vuol dire che non ci sono archi per passare da uno all'altro.
Se i e j non coincidono, il loro cammino è compreso in E e sono rispettate le condizioni del problema allora $d^k_{ij}$ diventerà il peso del cammino (oppure TRUE). Diventa sempre $\infty$ o FALSE altrimenti.


\subparagraph{Problema ausiliario} Nel caso in cui non sia possibile calcolare il risultato con le informazioni date è necessario introdurre un Problema ausiliario.
Questo problema generalmente è lo stesso iniziale ma aggiunge un controllo aggiuntivo, per esempio sul primo arco uscente e l'ultimo entrante del cammino.
Una volta introdotto il problema ausiliario si procede come se fosse un problema normale, quindi si introduce il sottoproblema, le variabili e le equazioni di ricorrenza.
Con le equazioni di ricorrenza però vengono aggiunte delle condizioni particolari che dipendono caso per caso (guarda esercizi).

\subparagraph{Solzuone del problema} La soluzone del problema nel caso in cui non ci sia un problema ausiliario è semplicemente formata da i valori nel matricione introdotto dal sottoproblema (D),
se invece è presente un problema ausiliario, allora la soluzione diventa il minimo (o OR) di tutte le soluzioni del Problema ausiliario

\section{Varianti DFS}
Questo è un generico esercizio del tipo "conta i vertici $\forall$ Componente Connessa di un grafo"

\subsection*{Spiegazione}
è una modifica di DFS in cui DFS-VISIT ritorna il numero di vertici che ha trovato nella componente connessa
del nodo in esame. Quindi dopo ongi chiamata di DFS-VISIT, DFS si ritrova con il numero dei vertici e può farci quello che vuole.
DFS-VISIT funziona come di norma (senza usare i padri che non servono) ma inizializza un contatore (che poi ritornerà) e ogni volta
che fa una chiamata ricorsiva aggiorna il valore.

\subsection*{Pseudocodice}

\begin{lstlisting}[mathescape=true]
DFS_COUNT(G)
    for ogni $u \in V$
        col[u] = WHITE
    for ogni $u \in V$
        if col[u] == WHITE
            num_in_cc = DFS_VISIT_COUNT(u)
            //Qui fai quello che vuoi con num_in_cc 
\end{lstlisting}

\begin{lstlisting}[mathescape=true]
DFS_VISIT_COUNT(u)
    col[u] = GRAY
    count = 0
    for ogni $v \in adj[u]$
        if col[v] == WHITE
            count = count + DFS_VISIT_COUNT(v)
    col[u] = BLACK
    RETURN count + 1
\end{lstlisting}

\chapter{Esercizi d'esame}

\section{FW: Esattamente 3 vertici BLU} Es 2 esame 2017\\
Grafo G dove ad ogni vertice è associato un colore {R,V,B}. Stabilire per ogni coppia di vertici se esiste un cammino con esattamente 3 vertici blu

\paragraph*{Problema ausiliario} 
Questo esercizio non necessita di un problema ausiliario, viene semplicemente aggiunta una specifica al sottoproblema di dimensione $k$ in cui si "limita" il numero di vertici blu.

\paragraph*{Sottoproblema}
Dato un grafo $G(V,E,f)$, $k\in\{0,...,|V|\}$, $b\in \{0,...,3\}$ si vuole stabilire se per ogni $(i,j)$ in $V \times V$ esiste un cammino $p$ da $i$ a $j$ con vertici intermedi
appartenenti a $\{1,...,k\}$ con esattamente $b$ vertici Blu.

\paragraph*{Variabili Introdotte}
$D^{n,b}$ Matrice $|V|\cdot|V|$\\
$d^{(k,b)}_{ij}$ = True se esiste un cammino da $i$ a $j$ con esattamente $b$ vertici BLU con vertici intermedi appartenenti a $\{1,..,k\}$.\\
con $b = \{0,...,3\}$ e $k=\{0,...,|V|\}$

\subsection*{Caso Base} k=0\\
Tre casi, con $b \in \{0,1,2\}$ perchè $b < k+2$ siccome con 0 vertici intermedi il cammino $(i,j)$ ha al più due vertici

\begin{equation*}
	d_{ij}^{(0,0)} = \begin{cases}
		TRUE  & i=j\land col(i)\neq B\land col(j)\neq B                      \\
		TRUE  & i\neq j\land (i,j)\in E \land col(i)\neq B\land col(j)\neq B \\
		FALSE & \text{Altrimenti}
	\end{cases}
\end{equation*}

\begin{equation*}
	d_{ij}^{(0,1)} = \begin{cases}
		TRUE  & i=j\land col(i)= col(j) = B                          \\
		TRUE  & i\neq j\land (i,j)\in E \land col(i)= B XOR col(j)=B \\
		FALSE & \text{Altrimenti}
	\end{cases}
\end{equation*}

\begin{equation*}
	d_{ij}^{(0,2)} = \begin{cases}
		TRUE  & i\neq j\land (i,j)\in E \land col(i)= col(j)= B \\
		FALSE & \text{Altrimenti}
	\end{cases}
\end{equation*}

\subsection*{Passo ricorsivo}
Se $k \notin cammino$ allora
\begin{equation*}
	d_{ij}^{(k,b)} = d_{ij}^{(k-1,b)}
\end{equation*}
che per comodità chiameremo $e1$

Se $k \in cammino$ allora dobbiamo distinguere due casi:
\paragraph{$col(k) \neq B$}
\begin{equation*}
	d_{ij}^{(k,b)} = d_{ik}^{(k-1,b_1)} \land d_{kj}^{(k-1,b_2)}
\end{equation*}
in cui $b_1 + b_2 == b$, che per comodità chiameremo $e2_b$
\paragraph{$col(k) \neq B$}
\begin{equation*}
	d_{ij}^{(k,b)} = d_{ik}^{(k-1,b_1)} \land d_{kj}^{(k-1,b_2)}
\end{equation*}
in cui $b_1 + b_2 == b+1$, che per comodità chiameremo $e2_a$

Quindi l'equazione del passo ricorsivo è:
\begin{equation*}
	d^{(k,b)}_{ij} = \begin{cases}
		e1 \lor e2_b & k=B     \\
		e1 \lor e2_a & k\neq B
	\end{cases}
\end{equation*}

\subsection*{Soluzione del problema}
La soluzione del Problema è contenuta in tutti i valori di $D^{|V|,3}$

\section{FW: Esattamente 3 Archi Blu}
Questa è una variante dell'esercizio di settembre 2017 in cui si richiede:
\\Grafo $G$ dove ad ogni vertice `e associato un colore R,V,B. Stabilire per ogni
coppia di vertici se esiste un cammino con esattamente 3 vertici blu.
\paragraph*{Sottoproblema $\{k,b\}$}
Dato un grafo $G(V,E,f)$, $k\in \{0,...,n\}$, $b\in \{0,...,3\}$, si vuole stabilire se per ogni coppia $(i,j)\in V\times V$
esiste un cammino da $i$ a $j$ con vertici intermedi apaprtenenti a $\{0,...,k\}$ in cui ci sono esattamente $b$ archi blu.

\paragraph*{Variabili Introdotte}
Siano $k\in \{0,...,n\}$, $b\in \{0,...,3\}$ si consideri il sottoproblema di dimensione $(k,b)$.
Per ogni $(i,j)\in V\times V$ introduciamo la variabile $d_{i,j}^{k,b}$ così definita:
\\$d_{(i,j)}^{(k,b)}$ : True sse esiste un cammino dal vertice $i$ al vertice $j$ con esattamente $b$ archi blu
che utilizza vertici intermedi appartenenti a $\{1,...,k\}$.
\\Al sottoproblema $(k,b)$ associamo la variabile $D^{k,b}$ il cui generico elemento $d^{(k,b)}_{(i,j)}$ corrisponde alla coppia $(i,j)$.

\paragraph*{Caso Base} Con $k=0$.
\\Con $k=0$ Le possibilità sono due: o abbiamo un vertice intermedio (se $(i,j)\in E$ e $i\neq j$) oppure non abbiamo vertici intermedi ($i=j$).
Quindi sappiamo che quando $b\in\{2,3\}$ la variabile è automaticamente false. Di conseguenza nel caso base $k=0$ abbiamo $d_{(i,j)}^{0,b}$ con $b\in \{0,...,3\}$:
$$
d_{(i,j)}^{0,0}= \begin{cases}
	TRUE & \text{sse } i\neq j \wedge (i,j)\in E \wedge (i,j) \neq B\\
	TRUE & \text{sse } i=j \\
	FALSE & \text{Altrimenti}
\end{cases}
$$
$$
d_{(i,j)}^{0,1}= \begin{cases}
	TRUE & \text{sse } i\neq j \wedge (i,j)\in E \wedge (i,j) = B\\
	FALSE & \text{Altrimenti}
\end{cases}
$$
$$
d_{(i,j)}^{0,2}= \begin{cases}
	FALSE
\end{cases}
$$
$$
d_{(i,j)}^{0,3}= \begin{cases}
	FALSE
\end{cases}
$$

\paragraph*{Passo Ricorsivo} con $k>0$.
Consideriamo la generica coppia $(i,j)$. Supponiamo che $b\leq k+1$ (altrimenti $b$ supererebbe il numero di archi nel cammino).
Considerando un generico cammino $p$ da $i$ a $j$ che usa vertici intermedi appartenenti a $\{0,...,k\}$ bisogna considerare due casi: $k\in p$ e $k \notin p$.
\\Se $k \notin p$: $d_{(i,j)}^{k,b} = \begin{cases}d_{(i,j)}^{k-1,b} \end{cases}$
\\Per comodità chiameremo questo caso $e_1$.
\\Se invece $k\in p$: $d_{(i,j)}^{k,b} = \begin{cases} d_{(i,k)}^{k-1,b_1} \wedge d_{(k,j)}^{k-1,b_2}\end{cases}$ tale che $b_1 + b_2 = b$.
\\Per comodità chiameremo questo caso $e_2$.
\\L'equazione del passo ricorsivo sarà quindi: $$d_{(i,j)}^{k,b} = e_1 \vee e_2$$

\paragraph*{Soluzione}
Considerando tutti i $d_{(i,j)}^{n,3}$ è possibile trovare la soluzione del problema.

\section{Cammino minimo vertici con segno alternato} Esercizio 1 esame 12/09/2016\\
Sia A l'insieme dei numeri interi escluso 0.\\
Dato un grafo non orientato (V,E,f) in cui ad ogni vertice è associato un numero intero
diverso da zero (mediante la funzione $f: V\rightarrow A$), mediante la tecnica della
programmazione dinamica si vuole stabilire per ogni coppia di vertici $(i,j)$, se esiste un
cammino da i a j avente vertici che danno luogo ad una alternanza del segno dei numeri
ad esso associati. RISPONDERE PER PUNTI alle seguenti richieste:
\begin{enumerate}
	\item Esplicitare e definire le variabili che servono per risolvere il problema
	\item Scrivere l'equazione di ricorrenza per il CASO BASE, giustificando perché è fatta in quel modo
	\item scrivere la/le equazione/i di ricorrenza per il PASSO RICORSIVO, giustificando perché è/sono fatta/e in quel modo
	\item Scrivere qualè la soluzione del problema, espressa rispetto alle variabili introdotte
\end{enumerate}

\subsection*{Variabili introdotte}
$D^k$ matrice $|V|\cdot|V|$
\\$d^k_{i,j}$ True sse esiste un cammino da $i$ a $j$ che da luogo ad una alternanza di segno usando ${1,...,k}$ vertici intermedi

	\subsection*{Caso Base} Non ci sono vertici intermedi $k=0$
	\begin{equation*}
		d^0_{i,j} = \begin{cases}
			TRUE  & \text{se $i=j$}                                                  \\
			TRUE  & \text{se $i\neq j \land (i,j) \in E \land f(i) \cdot f(j) < 0 $} \\
			FALSE & \text{altrimenti}
		\end{cases}
	\end{equation*}

	\subsection*{Passo Ricorsivo} $k>0$\\
	due casi:
	\begin{itemize}
		\item $k\notin$ cammino $\rightarrow d^k_{ij} = d^{k-1}_{ij}$
		\item $k \in$ cammino $\rightarrow d^k_{ij} = d^{k-1}_{ik} \land d^{k-1}_{kj}$
	\end{itemize}
	Quindi, nel passo ricorsivo con $k>0$:
	\begin{equation*}
		d^k_{ij} = d^{k-1}_{ij} \lor  (d^{k-1}_{ik} \land d^{k-1}_{kj})
	\end{equation*}

	\subsection*{Soluzione del problema}
	La soluzione del problema è costituita da tutti i valori contenuti in $D^{|V|}$

	\section{Esercizio Arco RED mai seguito da arco BLUE}
	Dato un grafo G(V,E,W) orientato e senza cappi in cui ad ogni arco è associato un colore tramite la funzione $col: E\rightarrow \{red,blue\}$
	calcolare $\forall (i,j) \in V^2$ il peso di un cammino minimo da i a j nella quale un arco $red$ non è mai seguito da un arco $blue$

	\paragraph{Definiamo il Problema Ausiliario $P'$}
	Dato $G \forall(a,b) \in C^2$ calcolare $\forall(i,j) \in V^2$ il peso di un cammino minimo da i a j con
	colore del $I$ arco pari ad $a$ e dell'iltimo a $b$

	\paragraph{Sottoproblema $k$esimo di $P'$} con $k \in \{0,...,n\}$
$\forall(a,b) \in C^2$ calcolare $\forall(i,j) \in V^2$ il peso di un cammino minimo da i a j con
	colore del $I$ arco pari ad $a$ e dell'iltimo a $b$ e con vertici intermedi $\in \{0,...,k\}$\\
	\subparagraph{Introduco la variabile $D*{(k,a,b)}$}
$\forall(i,j)\in V^2$ , $d^{(k,a,b)_{ij}}$ è il peso di un cammino minimo da i a j  con
	colore del $I$ arco pari ad $a$ e dell'iltimo a $b$ e con vertici intermedi $\in \{0,...,k\}$\\
	Introduco la variabile $D*{(k,a,b)}$
	\subparagraph{Caso Base sottoproblema di $P'$} $k = 0$

	\begin{equation*}
		d^{(0,a,b)}_{ij} \begin{cases}
			\infty & \text{se $i=j$}                                             \\
			w_{ij} & \text{se $i\neq j \land (i,j)\in E \land col(i,j) = a = b$} \\
			\infty & \text{altrimenti}
		\end{cases}
	\end{equation*}

	\subparagraph{Passo ricorsivo sottoproblema di $P'$} $k > 0$\\
	Abbiamo due casi\\
	se $k \notin cammino$ allora $d^{(k,a,b)}_{ij} = d^{(k-1,a,b)}_{ij}$\\
	se $k \in cammino$ allora \\
	grafo: i [a]- - -[c] k [d]- - -[b] j con $(c,d) \neq (red,blue)$
	tradotto: Se k appartiene al cammino minimo, allora il colore dell'ultimo arco entrante in k deve essere diverso da red, oppure il colore del primo arco uscente da k deve essere diverso da blue.
	quindi: $d^{(k,a,b)}_{ij} = d^{(k-1,a,c}_{ik} + d^{(k-1,d,b}_{kj}$ con $c,d \in C^2$ tc $c\neq red \vee b\neq blue$
	Quindi l'equazione di ricorrenza del passo ricorsivo è:\\
$d^{(k,a,b)}_{ij} = min\{d^{(k-1,a,b)}_{ij} ,d^{(k-1,a,c}_{ik} + d^{(k-1,d,b}_{kj}\}$ con $c,d \in C^2$ tc $c\neq red \vee b\neq blue$

	\subparagraph{Soluzione Problema Ausiliario $P'$} è formata dalle matrici:\\
$D^{(n,red,red)}$ $D^{(n,red,blu)}$ $D^{(n,blu,red)}$ $D^{(n,blu,blu)}$ \\
	Ovvero tutte le combinazioni possibili di cammini minimi con gli archi a e b = red or blu

	\paragraph{Soluzione Problema di partenza $PB$}
	\begin{equation*}
		d^{prob}_{ij} = \begin{cases}
			0                                             & \text{se $i=j$}   \\
			min\{d^{(n,a,b)}_{ij} \forall (a,b) \in C^2\} & \text{altrimenti}
		\end{cases}
	\end{equation*}

	\section{Esercizio k alberi e almeno h vertici} Esercizio 2 esame 12/09/2016\\
	Scrivere un algoritmo che, dati un grafo non orientato e due interi positivi $h>0$ e $k>0$, stabilisce se ENTRAMBE le seguenti condizioni sono verificate
	\begin{itemize}
		\item Esattamente $k$ componenti connesse del grafo sono alberi
		\item Ogni componente connessa del grafo ha almeno $h$ vertici
	\end{itemize}

	\subsection*{Spiegazione}
	Per creare questo algoritmo occorre modificare DFS:
	\begin{itemize}
		\item Per capire quanti alberi ci sono, inizializzo un boolean TREE a True. In visit controllo gli archi all'indietro e metto TREE = False se ne trovo. Al ritorno in DFS, se TREE è True incremento un contatore che alla fine di DFS controllerò che equivalga a k
		\item Per capire il numero di Vertici in una CC, in VISIT incremento un contatore ricorsivamente ogni volta che incontro un nodo white. Una volta tornato in DFS controllo che la CC abbia almeno $h$ vertici.
	\end{itemize}

	\subsection*{Pseudocodice}
	\begin{lstlisting}[mathescape=true]
    DFS-CONDITIONS(G,h,k)
        for ogni $u \in V$
            col[u] = WHITE
            $\pi$[u] = NIL
        tree = True     #variabile globale
        n_trees = 0
        for ogni $u \in V$
            if col[u] == White
                n_vertici = DFS-VISIT-CONDITIONS(u)
                if n_vertici < h
                    RETURN False
                if tree == True
                    n_trees = n_trees + 1
                else
                    tree = true
        if n_trees == k
            RETURN True
        else
            RETURN False
\end{lstlisting}

	\begin{lstlisting}[mathescape=true]
    DFS-VISIT-CONDITIONS(u)
        col[u] = Gray
        n_vertici = 0
        for ogni $w \in$ Adj[u]\$\pi$[u]
            if col[w]==White
                $\pi$[w] = u
                n_vertici = n_vertici + DFS-VISIT-CONDITIONS(w)
            else if col[w] == Gray and tree = True
                tree = False
        col[u] = Black
        return n_vertici + 1
\end{lstlisting}

	\section{Esercizio k grafi completi e h vertici}
	Dati due interi h,k>0 e un grafo non orientato  in cui ad ogni vertice è associato un simbolo (\$ e *)
	stabilisci se: ogni CC del grafo ha almeno h vertici a cui è associato il simbolo \$
	e ci sono al più k componenti connesse del grafo che prese singolarmente sono grafi completi

	\subsection*{spiegazione}
	\begin{itemize}
		\item h vertici per cc..., ogni volta che trovo un vertice white aumento un counter se è associato al dollaro, e alla fine controllo se è > h
		\item k cc sono.... un grafo è completo se ci sono $n_{vert} \cdot (n_{vert} -1)/2$ archi. per contare gli archi si incrementa un counter ogni volta che si trova un vertice nella lista di adiacenza.
	\end{itemize}

	\subsection*{Pseudocodice}
	\begin{lstlisting}[mathescape=true]
DFS_CUSTOM(G,h,k)
    for ogni $u \in V$
        col[u] = WHITE
        $\pi$[u] = NIL
    n_completi = 0
    n_archi = 0
    for ogni $u \in V$
        if col[u] == WHITE
            n_dollari = DFS_VISIT_CUSTOM(u)
            if n_dollari < h
                RETURN False
            if n_archi = n_vertici $\cdot$ (n_vertici - 1)/2
                n_completi = n_completi + 1
            n_archi = 0
    if n_completi > k
        RETURN False
    RETURN True
\end{lstlisting}

	\begin{lstlisting}[mathescape=true]
DFS_VISIT_CUSTOM(u)
    col[u] = GRAY
    if f(u) == "dollar"
        n_dollar = 1
    else
        n_dollar = 0
    for ogni $w \in Adj[u]\backslash\pi[u]$
        n_archi = n_archi + 1
        if col[w] == WHITE
            $\pi$[w] = u
            n_dollar = n_dollar + DFS_VISIT_CUSTOM(w)
    RETURN n_dollar
\end{lstlisting}

	\section{Esercizio conta CC di un grafo}
	Scrivere un algoritmo che determina il numero di componenti connesse di un grafo G =
	(V, E) non orientato.

	\subsection*{Spiegazione}
	Ogni volta che DFS invoca DFS-VISIT significa che ha trovato una componente connessa, quindi basta mettere un contatore che incrementa ogni volta che si chiama DFS-VISIT da DFS
	\subsection*{Pseudocodice}
	\begin{lstlisting}[mathescape=true]
    DFS_COUNT_CC(G)
        for ogni $u \in V$
            col[u] = WHITE
        count = 0
        for ogni $u \in V$
            if col[u] == WHITE
                count = count + 1
                DFS_VISIT(u)
        RETURN count
\end{lstlisting}

	\section{Esercizio Controlla se G è Aciclico}
	Modificare l’algoritmo DFS di visita di un grafo orientato G in maniera tale che stabilisca
	se G è acicilico, ossia se non contiene cicli.
	\subsection*{Spiegazione}
	Un grafo è aciclico se non contiene nessun arco all'indietro (btw se un grafo non orientato è aciclico allora è un albero).
	\\Quindi bisogna semplicemente iniziallizzare un Boolean "acyclic" a TRUE e farlo diventare FALSE qual'ora si incontrasse un arco all'indietro

	\subsection*{Pseudocodice}
	\begin{lstlisting}[mathescape=true]
DFS-ACYCLIC(G)
    for ogni $u \in V$
        col[u] = WHITE
        $\pi$[u] = NIL
    time = 0
    acyclic = true
    for ogni $u \in V$
        if col[u] == WHITE
            DFS-VISIT-ACYCLIC(u)
    return acyclic    
\end{lstlisting}

	\begin{lstlisting}[mathescape=true]
DFS-VISIT-ACYCLIC(u)
    col[u] = GRAY
    time++
    d[u] = time
    for ogni $w \in$ Adj[u]
        if color[w] == WHITE
            $\pi$[w] = u
            DFS-VISIT-ACYCLIC(w)
        else if col[w] == GRAY and acyclic == true
            acyclic = FALSE
    col[u] = black
    time ++
    f[u] = time
\end{lstlisting}

	\subsection*{In caso di grafo NON ORIENTATO}
	Se il grafo non è orientato, l'algorimto è lo stesso ma nel controllo degli adiacenti di u in visit bisogna togliere il padre di u

\chapter{Esami}
\section{12 Settembre 2016}
\subsection{Esercizio 1}
\paragraph*{Testo}Sia $A$ l'insieme dei numeri interi escluso 0.
\\Dato un grafo non orientato $(V,E,f)$ in cui ad ogni vertice è associato un numero intero diverso da zero
(mediante la funzione $f: V\to A)$, mediante la tecnica della programmazione dinamica, si vuole stabilire, per ogni coppia di vertici $(i,j)$,
se esiste un cammino da $i$ a $j$ avente vertici che danno luogo ad una alternanza del segno dei numeri ad essi associati.
RISPONDERE PER PUNTI alle seguenti richieste
\begin{enumerate}
	\item Esplicitare e definire le variabili che servono per risolvere il problema
	\item Scrivere l'equazione di ricorrenza per il CASO BASE, giustificando perchè è fatta in quel modo
	\item Scrivere la/le equazione/i di ricorrenza per il PASSO RICORSIVO, giiustificando perchè è/sono fatte in quel modo
	\item scrivere qual'è la soluzione del problema, espressa rispetto alle variabili introdotte.
\end{enumerate}

\paragraph*{1:} 
Introduco la funzione $s: A\to \{+,-\}$.
\subparagraph*{Sottoproblema $k$-esimo}
Dato un grafo $(V,E,f)$ e $k\in \{0,...,n\}, \forall(i,j) \in V\times V$ si vuole stabilire se esiste un cammino $p$
da $i$ a $j$ avente vertici intermedi appartenenti a $\{1,...,k\}$ e in cui non ci siano due vertici consecutivi a cui è associato un numero con lo stesso segno.
\\Ad ogni sottoproblema di dimensione $k$ è associata una variabile:  
\subparagraph*{Variabile}
Sia $k\in\{0,...,n\}$, $\forall (i,j) \in V\times V$, introduciamo la variabile $d^{(k)}_{(i,j)}$ così definita:
\begin{itemize}
	\item TRUE, se esiste un cammino $p$ da $i$ a $j$ avente vertici intermedi appartenenti a $\{\,...,k\}$ e in cui non ci siano due vertici consecutivi a cui è associato un numero con lo stesso segno.
	\item FALSE, altrimenti
\end{itemize}
Pertanto al sottoproblema $k$-esimo è associata la macrovariabile $D^{(k)}$ il cui generico elemento è $d^{(k)}_{(i,j)}$.
\paragraph*{2:} CASO BASE $k=0$.
\\Si consideri la generica coppia di vertici $(i,j)$.
Nel problema si vuole stabilire se esiste un cammino $p$ da $i$ a $j$ avente vertici intermedi appartenenti a $\{1,...,k\}$ e in cui non ci siano due vertici consecutivi a cui è associato lo stesso segno.
Se $k$ è pari a 0, allora non esistono vertici intermedi.
Per cui sono possibili due situazioni:
\begin{itemize}
	\item Può esistere al massimo un arco che collega $i$ e $j$ (se essi sono due vertici distinti), In tal caso la variabile vale TRUE sse $s(f(i))\neq s(f(j))$.
	\item Se non esiste alcun arco tra $i$ e $j$ e sono nel caso degenere per cui $i=j$, allora la variabile vale TRUE.
\end{itemize}
In tutti gli altri casi la variabile vale false.
$$d^{(0)}_{(i,j)}= \begin{cases}
	TRUE & \text{se } i\neq j \wedge (i,j)\in E \wedge s(f(i)) \neq s(f(j)) \vee i=j \\
	FALSE & \text{Altrimenti}
\end{cases}
$$
\paragraph*{3} PASSO RICORSIVO $k\geq 0$
Si consideri la generica coppia $(i,j)$ e sia $k\in\{1,...,n\}$.
\\Si possono presentare due diverse situazioni:
\begin{itemize}
	\item $k\notin p$: Se $k$ non è un vertice intermedio di $p$ (tra $i$ e $j$), allora tutti i vertici intermedi di p appartengono all'insieme $\{1,...,k-1\}$.
	Esiste un cammino $p$ da $i$ a $j$ avente vertici intermedi appartenenti a $\{1,...,k\}$ e in cui non si siano due vertici consecutivi a cui è associato un numero con lo stesso segno se esiste
	un cammino $p$ da $i$ a $j$ avente vertici intermedi appartenenti a $\{1,...,k-1\}$ con le medesime condizioni
	$$\forall(i,j)\in V\times V, d^{(k)}_{(i,j)} = d^{(k-1)}_{(i,j)} \text{ sse } k\notin p$$
	Per semplicità chiamo questo caso $C_1$
	\item$k\in p$: se $k$ è un vertice intermedio di $p$ (tra $i$ e $j$), suddivido $p$ in due sottocammini $p_1$ da $i$ a $k$ e $p_2$ da $k$ a $j$,
	entrambi aventi vertici intermedi appartenenti all'insime $\{1,...,k-1\}$.
	Per la proprietà della sottostruttura ottima si avrà:
	$$\forall(i,j)\in V\times V, d^{(k)}_{(i,j)} = d^{(k-1)}_{(i,k)} \wedge d^{(k-1)}_{(k,j)} \text{ sse } k\in p$$
	Per semplicità chiamo questo caso $C_2$
\end{itemize}
Poichè non posso sapere a priori se $k$ sia o meno vertice intermedio di $p$, avrò il caso passo:
$$\forall(i,j)\in V\times V, d^{(k)}_{(i,j)} = C_1 \vee C_2$$
\paragraph*{4}Soluzione:
La soluzione del problema PB è costituita da tutti i valori contenuti in $D^{(n)}$.

\end{document}