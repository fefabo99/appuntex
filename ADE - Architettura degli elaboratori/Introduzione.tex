\documentclass[12pt, a4paper, openany]{book}
\usepackage[italian]{babel}
\usepackage{listings}
\usepackage{graphicx}
\usepackage{fancyvrb}

\begin{document}
\title{ADE - Archietettura degli elaboratori}
\author{Elia Ronchetti}
\date{Marzo 2022}

\maketitle
\tableofcontents

\chapter{Introduzione e Argomenti}
\section{Rappresentazione dell'informazione}
\begin{itemize}
    \item Sistemi numerici
    \item Rappresentazione dei numeri interi con e senza segno
    \item Rappresentazione dei numeri in virgola fissa e mobile
    \item Rappresentazione dell'informatica non numerica
\end{itemize}

\section{Circuiti logici}
\begin{itemize}
    \item Reti combinatorie
    \item Reti sequenziali e FSM (Finite State Machine)
    \item Rassegna di circuiti notevoli (decoder, multiplexer, register file, ALU, etc.)
\end{itemize}

\section{Instruction Set Architecture (ISA)}
\begin{itemize}
    \item schema di von Neumann
    \item CPU, registri, ALU e memoria
    \item Ciclo fondamentale di esecuzione di una istruzione (fetch/decode/execute)
    \item Tipi e formati di istruzioni MIPS32
    \item Modalità di indirizzamento
\end{itemize}

\section{Linguaggio Assembly}
\begin{itemize}
    \item Formato simbolico delle istruzioni
    \item Catena di programmazione (compilatore, assembler, linker, loader, debugger, etc.)
    \item Pseudo-istruzioni e direttive dell'assemblatore
    \item Scrittura di semplici programmi assembly
    \item Convenzioni programmative (memoria, nomi dei registri, etc.)
\end{itemize}

\section{Datapath}
\begin{itemize}
    \item Percorsi dei dati per le diverse classi di istruzioni
    \item Controllo del percorso dei dati con FSM
\end{itemize}

\section{Gestione delle eccezioni}
\begin{itemize}
    \item Tassonomia di eccezioni in terminologia MIPS32
    \item Modifiche alla FSM di controllo, registro Cause
\end{itemize}

\section{Tecniche di gestione dell'ingresso/uscita}
\begin{itemize}
    \item Controllo di programma
    \item Interruzione di programma
    \item Accesso diretto alla memoria
\end{itemize}

\section{Gerarchie di memoria: cache}
\begin{itemize}
    \item Cache a mappature diretta
    \item Cahce fully associative
    \item Cache n-way set associative
\end{itemize}

\chapter{Sistemi numerici} 

Con il termine bit definiamo l'unità di misura dell'informazione. Un bit può assumero solo il valore di 0 o 1.

Combinando tra loro più bit si ottengono strutture più complesse, per esempio:
\begin{itemize}
    \item byte, 8 bit
    \item nybble, 4 bit
    \item word, 32 bit
\end{itemize}

Una rappresentazione è un modo per descrivere un'entità

Il sistema numerico decimale:

\begin{itemize}
    \item usa 10 cifre
    \item è un sistema posizionale: ogni cifra assume un valore diverso a seconda della posizione che occupa
\end{itemize}

\paragraph{Confronto tra Basi}
Inserire immagine confronto

\section{Conversione tra basi}

\section{Operazioni aritmetiche e Overflow}

\end{document}
