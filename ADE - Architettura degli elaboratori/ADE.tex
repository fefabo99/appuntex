\documentclass[12pt, a4paper, openany]{book}
\usepackage[italian]{babel}
\usepackage{listings}
\usepackage{graphicx}
\usepackage{fancyvrb}

\begin{document}
\title{ADE - Archietettura degli elaboratori}
\author{Elia Ronchetti}
\date{Marzo 2022}

\maketitle
\tableofcontents

\chapter{Introduzione e Argomenti}
\section{Rappresentazione dell'informazione}
\begin{itemize}
    \item Sistemi numerici
    \item Rappresentazione dei numeri interi con e senza segno
    \item Rappresentazione dei numeri in virgola fissa e mobile
    \item Rappresentazione dell'informatica non numerica
\end{itemize}

\section{Circuiti logici}
\begin{itemize}
    \item Reti combinatorie
    \item Reti sequenziali e FSM (Finite State Machine)
    \item Rassegna di circuiti notevoli (decoder, multiplexer, register file, ALU, etc.)
\end{itemize}

\section{Instruction Set Architecture (ISA)}
\begin{itemize}
    \item schema di von Neumann
    \item CPU, registri, ALU e memoria
    \item Ciclo fondamentale di esecuzione di una istruzione (fetch/decode/execute)
    \item Tipi e formati di istruzioni MIPS32
    \item Modalità di indirizzamento
\end{itemize}

\section{Linguaggio Assembly}
\begin{itemize}
    \item Formato simbolico delle istruzioni
    \item Catena di programmazione (compilatore, assembler, linker, loader, debugger, etc.)
    \item Pseudo-istruzioni e direttive dell'assemblatore
    \item Scrittura di semplici programmi assembly
    \item Convenzioni programmative (memoria, nomi dei registri, etc.)
\end{itemize}

\section{Datapath}
\begin{itemize}
    \item Percorsi dei dati per le diverse classi di istruzioni
    \item Controllo del percorso dei dati con FSM
\end{itemize}

\section{Gestione delle eccezioni}
\begin{itemize}
    \item Tassonomia di eccezioni in terminologia MIPS32
    \item Modifiche alla FSM di controllo, registro Cause
\end{itemize}

\section{Tecniche di gestione dell'ingresso/uscita}
\begin{itemize}
    \item Controllo di programma
    \item Interruzione di programma
    \item Accesso diretto alla memoria
\end{itemize}

\section{Gerarchie di memoria: cache}
\begin{itemize}
    \item Cache a mappature diretta
    \item Cahce fully associative
    \item Cache n-way set associative
\end{itemize}

\chapter{Sistemi numerici} 

Con il termine bit definiamo l'unità di misura dell'informazione. Un bit può assumero solo il valore di 0 o 1.

Combinando tra loro più bit si ottengono strutture più complesse, per esempio:
\begin{itemize}
    \item byte, 8 bit
    \item nybble, 4 bit
    \item word, 32 bit
\end{itemize}

Una rappresentazione è un modo per descrivere un'entità

Il sistema numerico decimale:

\begin{itemize}
    \item usa 10 cifre
    \item è un sistema posizionale: ogni cifra assume un valore diverso a seconda della posizione che occupa
\end{itemize}

\paragraph{Confronto tra Basi}
Inserire immagine confronto

\section{Conversione tra basi}
Svolti esercizi di conversione tra basi
\section{Operazioni aritmetiche e Overflow}
\begin{itemize}
    \item Addizzioni e sottrazioni
    \item Svolti esercizi con Overflow, bit di carry
\end{itemize}

\section{Operazioni con segno}
Ci sono diverse modalità per rappresentare il segno in base 2
\subsection{Modulo e segno}
La rappresentazione modulo e segno divide i bit di rappresentazione in 2, nel caso di 8 bit, 7 sono utilizzati per rappresentare il valore assoluto e il bit
più significativo (MSB - Most significant bit), quello a sinistra, rappresenta il segno, 0 positivo, 1 negativo
\begin{equation}
    1 | 0 0 0 0 1 0 0
\end{equation}
Questa rappresentazone è semplice e con n bit totali, si possono rappresentare i numeri interi nell'intervallo, ma ha alcuni problemi
\begin{itemize}
    \item Esistono 2 rappresentazioni diverse per lo 0
    \item Un bit tra tutti i bit disponibili viene speso per il segno e questo è uno spreco, riduce inoltre la capacità di rappresentazione
\end{itemize}

\subsection{Operazioni aritmetiche con MS}
%Copia le slide
Possiamo avere overflow solo quando %completa perchè la prof legge le slide a 5x
\subsection{Complemento a 1 (CA1)}
\'E un'altra modalità di rappresentazione dei numeri interi con segno. Come indica il nome stesso, questo metodo si basa sull'operazione di complemento

\paragraph{Complemento} è l'operazione che associa ad un bit (o ad ogni sequenza di bit) il suo opposto, cioè il valore ottento
sostituendo tutti gli 1 con 0 e uttti gli 0 con 1
\paragraph{Esecuzione} è semplice e diretta
\begin{enumerate}
    \item Se il numero da condificare è positivo lo si converte in binaro con il metodo tradizionale
    \item Sei il numero è negatio basta convertire in binario il suo modulo e quindi eseguire l'operazione di complemento sul numero appena convertito
\end{enumerate} 
\paragraph{Problema} ancora doppia rappresentazione dello 0

\subsection{Complemento a 2 (CA2)}
Anche qui il MSB è 0 se x è positivo e MSB = 1 se x è negativo
\paragraph{Esecuzione}
\begin{enumerate}
    \item Se il numero X è positivo esso rimane invariato
    \item Se il numero X è negativo 
\end{enumerate}
Così elimino la doppia rappresentazione dello zero

\subsubsection{3 Metodi per il calcolo di CA2}
%Sistemare
\begin{enumerate}
    \item Bau
    \item MiaoM
    \item Regola pratica
\end{enumerate}

\title{Distinzione tra Operazione CA2 e Rappresentazone CA2}
\begin{itemize}
    \item La rappresentazione - come sono organizzati i bit
    \item Il calcolo - procedura di trasformazione
\end{itemize}

\subsection{Operazioni aritmetiche con CA2}
\subsubsection{Somma a Sottrazione}
\begin{enumerate}
    \item Si esegue la somma su tutti i bit egli addendi, segno compreso
    \item Un eventuale riporto (carry) oltre il bit di segno (MSB) viene scartato
    \item Nel caso gli operandi siano di segno concorde occorre verificare la presenza o meno di overflow (il segno del risultato non  è concorde
    con quello dei due addendi)
\end{enumerate}
\section{Operazione di Shift}
Consiste nello spostare (shit) verso destra (right) o verso sinistra (left) la posizione delle cifre di un numero, epsresso in una base qualsiasi, inserendo uno zero
nelle posizioni lasciate libere.
\begin{itemize}
    \item Left equivale a moltiplicare il numero per la base
    \item Right equivale a dividere il numero per la base
\end{itemize}
\section{Rappresentazione Eccesso $2^n$}
Balzata alla grande dalla prof, riprendere dalle slide

\section{Rappresentazione con la virgola}
\subsection(Virgola fissa)
\'E il metodo più semplice, scegliamo dove mettere la virgola e la fissiamo
Il problema è che in base a dove posiziono la virgola ho diverse capacità di rappresentazione della parte intera o frazionaria
\begin{itemize}
    \item Più a destra, scarsa rappresentazione intera, alta rappresentazione frazionaria
    \item Più a sinitra, scarsa rappresentazione frazionaria, alta rappresentazione intera
\end{itemize}
Questo porta rigidità

\subsection{Virgola mobile}
\begin{itemize}
    \item Usa un bit per rappresentare il segno s
    \item Usa altri bit per rappresentare la mantissa m
    \item Usa altri bit per codificare l'esponente e
\end{itemize}

Seguendo lo standard IEEE 754 la suddivisione è effettuata nella seguente modalità
\subsubsection{32 bit}
\begin{itemize}
    \item Segno - 1
    \item Esponente - 8
    \item Mantissa - 23
\end{itemize}
\subsubsection{64 bit}
\begin{itemize}
    \item Segno - 1
    \item Esponente - 11
    \item Mantissa - boh
\end{itemize}
Anche qua la prof va a 200 all'ora e se ne sbatte il cazzo che la gente non sta capendo una sega

\subsection{Errore assoluto ed errore relativo}
Rappresentando un numero reale n in virgola mobile si commette un errore di approssimazione
%Aggiungere cose sensate


\end{document}