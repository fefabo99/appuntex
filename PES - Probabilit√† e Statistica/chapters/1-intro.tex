\chapter{Introduzione}
Il corso di probabilità e statistica per l'informatica è diviso in 2 parti
\begin{enumerate}
    \item Stastica Descrittiva - Descrivere e riassumere i dati
    \begin{enumerate}
        \item Probabilità - Descrivere matematicamente i fenomeni casuali
    \end{enumerate}
    \item Statistica inferenziale - Trarre conclusioni dai dati
\end{enumerate}

%Modalità d'esame
\chapter{Esame}
L'esame sarà strutturato nella seguente maniera
\paragraph{Parte 1 - Teoria}
8 Domande a risposta multipla - Punteggio 10/30
\paragraph{Parte 2 - Pratica}
4 Esercizi a risposta aperta - Punteggio 20/30
\paragraph{Progetto (facoltativo)}
Progetto R, da consegnare prima dell'esame, può fornire un massimo di 2/30
