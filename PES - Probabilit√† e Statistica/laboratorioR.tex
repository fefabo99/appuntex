\documentclass[12pt, a4paper, openany]{book}
\usepackage[italian]{babel}
\usepackage{listings}
\usepackage{graphicx}
\usepackage{fancyvrb}
\usepackage{color}

\definecolor{dkgreen}{rgb}{0,0.6,0}
\definecolor{gray}{rgb}{0.5,0.5,0.5}
\definecolor{mauve}{rgb}{0.58,0,0.82}

\lstset{frame=tb,
  language=R,
  aboveskip=3mm,
  belowskip=3mm,
  showstringspaces=false,
  columns=flexible,
  basicstyle={\small\ttfamily},
  numbers=none,
  numberstyle=\tiny\color{gray},
  keywordstyle=\color{blue},
  commentstyle=\color{dkgreen},
  stringstyle=\color{mauve},
  breaklines=true,
  breakatwhitespace=true,
  tabsize=3
}

\begin{document}
\title{Laboratorio R}
\author{Elia Ronchetti}
\date{Marzo 2022}

\maketitle
\tableofcontents
\chapter{Introduzione a R e Analisi Descrittiva}
Nell'interprete è possibile definire variabili, array, in maniera molto simile
a Python

Nell'esempio seguente dichiariamo delle variabili e leggiamo un dataset
\begin{lstlisting}
    a = 5
    b = c(1, 2, 4)
    arr = c(1:10)
    s = c(1, 2, "ciao")
    str(dataset)
\end{lstlisting}


Come notiamo per creare un array è necessario invocare la funzione \textbf{c} e salvarla in una variabile.
\\ \'E possibile dichiarare degli Array misti, con numeri e stringhe.
\\ Per quanto riguarda l'analisi approssimativa del dataset è possibile utilizzare il
comando \textbf{str}

\section{Grafici}
La stampa dei grafici avviene tramite le apposite funzioni
\begin{lstlisting}
    stripchart(rivers, method="overplot")
    stripchart(rivers, method="jitter")
    hist(rivers)
    hist(rivers, breaks=30) #commento
\end{lstlisting}
Nell'penultimo esempio gli istogrammi stampati non hanno un numero di barre definito dall'utente
ma viene definito automaticamente da R studio, posso definire io il numero di barre attraverso
il parametro \textbf{breaks}, come nell'ultimo esempio 

\section{BoxPlot}
Per la stampa di un Boxplot è sufficiente utilizzare la funzione \textbf{boxplot(data)}
In questo modo è possibile stampare il grafico con i relativi outliers.
\paragraph{Statistiche} Per sapere valore minimo, quartili e varie statistiche basta utilizzare
\textbf{boxplot.stats(rivers)}
\paragraph{Outliers} Gli outliers vegono considerati come tali se superano il terzo quartile
+ 1,5 il range interquartile

Per effettuare un confronto la ALU usa l'operatore SLT

\end{document}