\chapter{Analisi Descrittiva}
\section{Descrivere i dati}
Per descrivere una raccolta dati in maniera chiara e immediata è utile utilizzare una \textbf{tabella delle frequenze}
all'interno della quale sono contenuti:
\begin{itemize}
    \item Valori
    \item Frequenze Assolute - Numero di volte in cui compare "i" nell'insieme di dati
    \item Frequenze Relative - Frazione di volte in cui compare i nell'insieme di dati
    \item Percentuali - (Frequenza relativa x 100)
\end{itemize}

Il dato che compare con frequenza più alta è detto \textbf{moda}.

I dati possono essere
\begin{itemize}
    \item Qualitativi
    \item Quantitativi 
\end{itemize}
Noi useremo i dati \textbf{quantitativi}

\subsection{Rappresentazione dei dati}
Per rappresentare le frequenze (assolute o relative) risulta efficace e immediato l'utilizzo di un grafico a barre detto istogramma,
esso rappresenta in graficamente la tabella, chiaramamente da esso è possibile risalire alla tabella stessa.
Capita di avere degli insiemi di dati che assumono un valore elevato di valori distini, per questo conviene suddividerli in classi e
determinare la frequenza di ciascuna classe. In questo modo c'è una perdita d'informazioni (sui valori specifici), ma così facendo possiamo
calcolare le frequenze delle classi e avere un'idea migliore della distribuzione dei dati.

\subsection{Dati Bivariati}
Quando per ciascun individuo vengono misurate due variabili ci troviamo un insieme di N dati a coppie detti \textbf{dati bivariati}.
Anche in questo caso è possibile calcolare le frequenze, in questo caso detto \textbf{frequenze congiunte}.

è possibile, inoltre, misurare la correlazione tra le due variabili attraverso per esempio un diagramma di dispersione (detto anche scatterplot).

\paragraph{Correlazione non significa causalità!} Non è detto che l'aumento di una variabile causi la diminuzione dell'altra o viceversa, potrebbe esserci una causa comune. 

\section{Riassumere i dati}
Dopo aver rappresentato i dati vogliamo ora riassumerli mediante quantità numeriche, dette \textbf{Statistiche Campionarie}, al fine di sintetizzare le proprietà
salienti dei dati.

\subsection{Indici di posizione}
Per definire il centro dell'insieme dei dati definiamo la 
\paragraph{\textbf{Media Campionaria}} \scalebox{1.5}{$\frac{x_1 + x_2 + \dots + x_n}{N}$} %Aggiungere sommatoria

Per misurare il valore in posizione centrale (considerando l'insieme di dati ordinato), utilizziamo la
\paragraph{\textbf{Mediana}}
\begin{itemize}
    \item Se N dispari $\rightarrow$ {$X_\frac{N+1}{2}$}
    \item Se N pari $\rightarrow$ $m = \frac{X_{\frac{N}{2}}+X_{\frac{N}{2}+1}}{2}$
\end{itemize}

La mediana è insesibile alle code, se per esempio quindi aumento anche di molto il valore dell'ultima cifra lasciando invariate
le altre la mediana non cambierà (a differenza della media).

\section{Coefficiente di correlazione lineare}
Posso misurare il grado di correlazione tra una coppia di dati attraverso il coefficiente di correlazione lineare. 

\begin{equation}
    r = \frac{\sum_{k=1}^N (x_i - x)(y_i - y)}{(N -1)S_x S_y}
\end{equation}

Si può mostrare che:
\begin{equation}
    -1<=r<=1
\end{equation}

In generale $r > 0$ indica una correlazione positiva
\\ $r < 0$ indica una correlazione negativa 

\subsection{Correlazioni significative}
$|r| > 0.7$ Correlazione significativa
\\$|r| < 0.3$ Correlazione debole

\section{Percentili e quantili}
Per analizzare la distribuzione dei dati è utile fissare un numero k che rappresenta la posizione all'interno dato all'interno dell'insieme
questo valore percentuale è detto \textbf{k-esimo Percentile Campionario}, valore t per cui
\begin{itemize}
    \item almeno il k\% dei dati è $ <= t$
    \item almeno il $(100 -k)\%$ dei dati è $<= t$
\end{itemize}

I casi più importanti sono per k = 25, 50, 75
\\ Risulta pratico scrivere $k = 100p$ dove $p = \frac{k}{100}\in [0, 1]$, dove i casi importanti sono per:
\begin{itemize} 
    \item $p = \frac{1}{4}: k = 100p$ = 25-esimo percentile = primo quartile $q_1$
    \item $p = \frac{1}{2}: k = 100p$ = 50-esimo percentile = secondo quartile $q_2$ = mediana m
    \item $p = \frac{3}{4}: k = 100p$ = 75-esimo percentile = terzo quartile $q_3$
\end{itemize}

Per calcolare il k-esimo percentile t è necessario:
\begin{enumerate}
    \item Ordinare l'insieme di dati $x_1 <= x_2 <= \dots <= x_n$
    \item Se $N_p$ non è intera $t=x_i$ è il dato la cui posizione i è l'intero successivo a $N_p$
    \item Se $N_p$ è intera $t = \frac{x_(Np) x_(Np+1)}{2}$ è la media aritmetica fra il dato in posizione N e il successivo 
\end{enumerate}

\paragraph{Nota per R} Esistono diverse definizioni di quantile, R per esempio ne utilizza una diversa di default.

\'E possibile utilizzare i \textbf{Boxplot} per la rappresentazione dei quantili
