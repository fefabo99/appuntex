\chapter{Teoremi di convergenza}
Siano $X_1, X_2 \dots $ successione di Variabili aleatorie indipendenti identicamente distribuite (V.A I.I.D.).
\\ Discrete $ \rightarrow p_x(x) = p(x) $
\\ Assolutamente continue $ \rightarrow f_{x_i} (x) = f(x) $

Per osservare $X_1, X_2$ devo raccogliere dei dati, il problema è che $p(x), f(x)$ non sono completamente
note. 
\paragraph*{Statistica inferenziale} Dalle osservazioni posso fare delle deduzioni (Inferenze) sulla distribuzione
comune di $X_1, X_2$ ossia su $p(x)$ e $f(x)$.
\\ Non posso osservare tutte le V.A per questo ne scelgo casualmente n. 
\\ $X_1, \dots ,X_n$ Campione aleatorio di ampiezza n. Lavoreremo con variabili aleatorie che siano funzioni del 
campione aleatorio, cioè del tipo $g(x_1, \dots ,x:n)$
\paragraph*{Statistica campionaria} Dato un campione aleatorio $X_1, \dots, X_n$ una statistica campionaria
è una qualunque funzione del campione:
\begin{equation*}
    g(X_1, \dots, X_n) \text{   } (g:\mathbb{R}^n\rightarrow\mathbb{R})
\end{equation*}

\paragraph*{Esempio di statistica campionaria} Media campionaria
\begin{equation*}
    \bar{X_n}:\frac{1}{n}\sum_{n=1}^n X_i
\end{equation*}
\section{Distribuzione di X}
\begin{equation*}
    \mathbb{P} (a  < \bar{X_nx} \leq b) = \mathbb{P} (\frac{a - \mu}{\frac{\sigma}{\sqrt{n}}} 
    < \frac{X_n - \mu}{\frac{\sigma}{\sqrt{n}}} \leq \frac{b - \mu}{\frac{\sigma}{\sqrt{n}}})   
\end{equation*}
\begin{equation*}
    \text{Var}(\bar{X_n}) = \frac{\sigma ^ 2}{n} 
\end{equation*}
\begin{equation*}
    \frac{\sigma}{\sqrt{n}} = \sqrt{\text{Var}}
\end{equation*}

\section{Teorema del limite centrale}
Siano $X_1, ..., X_n$ v.a i.i.d. (Variabili aleatorie indipendenti identicamente distribuite)
$E(i) = \mu$ e $\text{Var}X_i = \sigma^2$ (con media e varianza finite).
\begin{equation*}
    \mathbb{P} (\frac{X_n-\mu}{\frac{\sigma}{\sqrt{n}}} \leq t) \rightarrow \Phi (t) \quad n \rightarrow + \infty   
\end{equation*}
Dove $\Phi$ è la funzione di ripartizione di una V.A $Z\sim N(0,1)$
\begin{equation*}
    \Phi(t) = \mathbb{P}(Z \leq t)
\end{equation*}
\paragraph*{Condizione per usare l'approssimazione normale} Abbiamo definito 
che solo con $n \geq 30$ si può usare l'approx per semplicità, in realtà non è semplice
dare un criterio univoco dato che dipende dall'asimmetria del campione di partenza.
\paragraph*{Nota per i campioni discreti} Quando ho un campione discreto è sempre necessario
applicare la correzione di continuità 
%https://it.wikipedia.org/wiki/Standardizzazione_(statistica)

\section*{Note per esercizi}
\begin{itemize}
    \item Riconosci la distribuzione notevole
    \item Ricava media e Var
    \item Scrivi la formula che ti serve (es P che una va sia $\leq$ di n)
    \item Effettua la correzione di continuità se si tratta di una VA discreta
    \item Verifica se $n\geq 30$, questo per verificare se si può applicare il TLC (teorema del limite centrale)
    \item Sottrai media e dividi per $\sqrt{\text{Var}}$
    \item Se hai $>$ come segno, anzichè $\mathbb{P}$ dovrai trovare $1-\mathbb{P}$ e invertire il segno in $<$
    \item Se trovi un valore $z$ negativo all'interno di $\Phi(z)$ dovrai calcolare $1 - \Phi(-z)$, quindi rendere positivo
    z, calcolare il suo $\Phi$ e sottrarlo a 1
\end{itemize}