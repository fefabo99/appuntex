\chapter{Verifica di Ipotesi}
\paragraph*{Ipotesi statistica} Un'ipotesi Statistica è un'affermazione sulla
distribuzione della popolazione in esame.
\\ Può essere espressa in termini di un parametro (\textbf{Test Parametrici}), oppure
può riguardare la natura della distribuzione della popolazione o altre
caratteristiche (\textbf{Test Non Parametrici}): es verificare se una popolazione
ha distr. normale, verifica l'indipendenza.
\\ Verificare un'ipotesi statistica significa verificare se è compatibile con i dati
del campione.
\\ Un'ipotesi statistica denotata con $H_0$ è detta \textbf{Ipotesi Nulla}.
\\ La negazione di $H_0$ è denotata con $H_1$ e viene definita \textbf{Ipotesi Alternativa}.
L'ipotesi nulla viene rifiutata se risulta incompatibile con i dati del campione,
altrimenti NON viene rifiutata.
\\ Lo scopo della \textbf{Verifica di Ipotesi} è trovare una regola che sulla base dei
dati campionari permetta di rifiutare o meno. Per questo si utilizzerà un'opportuna
statistica detta \textbf{Statistica del Test}. A seconda del suo valore assunto sui dati
campionari si rifiuterà o meno.
\\ Un test per la verifica dell'ipotesi nulla $H_0$ contro l'ipotesi alternativa 
$H_1$ consiste nel trovare una \textbf{regione C}, detta \textbf{Regione Critica o Regione di Rifiuto}
tale che se $(x_1, ..., x_n) \in C$ si rifiuta $H_0$, quindi si accetta $H_1$,
tale regione sarà calcolata utilizzando la statistica del test.
\section{Possibili errori durante il calcolo della regione critica}
Durante il calcolo della regione critica possiamo incorrere in due tipi di errore
\begin{itemize}
    \item Errore di prima specie (o tipo): si rifiuta $H_0$ e $H_0$ è vera
    \item Errore di seconda specie (o tipo): si accetta $H_0$ e $H_0$ è falsa
\end{itemize}
La regione critica ideale dovrebbe rendere piccola la probabilità di commettere
entrambi gli errori, ma questo in genere è impossibile: restringendo la regione
critica diminuisce la prob. di commettere un errore di prima specie, ma non
può aumentare la prob. di commettere un errore di seconda specie e viceversa.
\\ Di solito si controlla la prob. di errore di prima specie.
\paragraph*{$\alpha =$ livello di significatività del test} Fissare $\alpha$ piccolo
$(\alpha = 0.10, 0.05, 0.01)$ e chiedere che la prob. di rifiutare $H_0$ quando è
vera sia $\leq \alpha$. Ossia un test per la verifica di $H_0$ con regione critica C
ha livello significatività $\alpha$ se:
\begin{equation}
    \mathbb{P}_H((x_1, ..., x_n)\in C) \leq \alpha
\end{equation}
Controllare errore di I specie $\rightarrow$ assimetria tra $H_0$ e $H_1$.
\\ Se $(x_1, ..., x_n) \in C$ ossia si rifiuta $H_0$ i dati sperimentali sono
in contraddizione significativa con $H_0$.
\\ Se $(x_1, ..., x_n) \notin C$ i dati sperimentali non sono
in contraddizione signficativa con $H_0$.
Non è detto che siano in contraddizione con $H_1$, ma solo che non escludono
in modo significativo che $H_0$ sia vera.
\\ La conclusione forte è il rifiuto di $H_0$.
\\paragraph*{Implicazione} Se si vuole dimostrare con dati sperimentali una certa ipotesi
sulla distribuzione della popolazione, ossia di una variabile, si adotterà questa
ipotesi come \textbf{Ipotesi Alternativa}.
\\paragraph*{Osservazione} L'appartenenza di $(x_1, ..., x_n)$ alla regione
critica dipende dalla scelta del livello di significatività di $\alpha$.
\\ Esiste un $\bar{\alpha}$ t.c.
\begin{itemize}
    \item per $\alpha > \bar{\alpha}$ si rifiuta $H_0$
    \item per $\alpha \leq \bar{\alpha}$ si accetta $H_0$
\end{itemize}
$\bar{\alpha}$ viene detto \textbf{p-value} ($\sigma$ p dei dati, $sigma$ valore p
del test.
\\ Più piccolo è il p-value, più i dati sono in contraddizione con $H_0$.
\paragraph*{Definizione p-value} Minimo livello di significatività t.c. i dati
consentono di rifiutare $H_0$.
\\paragraph*{Legame tra Test di Ipotesi e I.C.}
Nel test ipotesi bilatero: rifiuto $H_0 \, (\mu=\mu_0)$ a livello $\alpha$ se $\bar{x_n}$
non appartiene all'intervallo di confidenza di livello $100(1-\alpha)\%$ per $\mu$.
C'è anche un legame tra estermi superiori e inferiori di confidenza e regione critiche dei
test unilateri.
\paragraph*{Efficacia di un test} 1 - prob di errore di II tipo.

\section*{Test Non Parametrici} Non si fanno inferenze nei parametri ma nella
distribuzione di una o più popolazioni.
\subsection{Test Chi quadro di buon adattamento}
Test adattamento: verificare se una certa popolazione abbia o meno una certa distribuzione,
ipotizzato sulla base di dati sperimentali (incluso verificare che una popolazione
abbia distribuzione normale).