\chapter*{Introduzione}
Il corso di probabilità e statistica per l'informatica è diviso in 2 parti
\begin{enumerate}
    \item Stastica Descrittiva - Descrivere e riassumere i dati
    \begin{enumerate}
        \item Probabilità - Descrivere matematicamente i fenomeni casuali
    \end{enumerate}
    \item Statistica inferenziale - Trarre conclusioni dai dati
\end{enumerate}

\section*{Programma Esteso}

\begin{enumerate}
    \item \textbf{Statistica Descrittiva}
    \begin{itemize}
        \item Introduzione all'analisi dei Dati
        \item Statistiche Campionarie (media, mediana, quantili, varianza, correlazione)
        \item Rappresentazioni grafiche
    \end{itemize}
    \item \textbf{Spazi di Probabilità}
    \begin{itemize}
        \item Fenomeni Aleatori, Spazi di probabilità ed eventi
        \item Proprietà di base delle probabilità
        \item Probabilità condizionata
        \item Elementi di calcolo Combinatorio
        \item Indipendenza di Eventi
    \end{itemize}
    \item \textbf{Variabili Aleatorie}
    \begin{itemize}
        \item Variabili aleatorie discrete
        \item Valore medio, momenti, varianza e covarianza
        \item Variabili aleatorie assolutamente continue
        \item Distribuzioni notevoli discrete e assolutamente continue
        \item Variabili aleatorie normali
    \end{itemize}
    \item \textbf{Teoremi di Convergenza}
    \begin{itemize}
        \item Convergenza di variaibli aleatorie e distribuzioni (cenni)
        \item Legge dei grandi numeri
        \item Teorema limite centrale
    \end{itemize}
    \item \textbf{Stima di Parametri}
    \begin{itemize}
        \item Campioni e Statistiche
        \item Stimatori (media e varianza campionarie)
        \item Intervalli di confidenza
    \end{itemize}
    \item \textbf{Verifica di ipotesi}
    \begin{itemize}
        \item Test per la verifica di un'ipotesi, errori di I e II specie
        \item Test parametrici per media e varianza
        \item Test non parametrici di buon adattamento ed indipendenza
    \end{itemize}
    \item \textbf{Regressione Lineare}
    \begin{itemize}
        \item Introduzione alla Regressione
        \item Inferenza statistica sui parametri
        \item analisi dei residui 
    \end{itemize}
\end{enumerate}



\section*{Esame}
L'esame è costituito da una prova scritta e da una eventuale prova orale e riceve un voto in trentesimi.
\\La prova scritta è costituita da due parti:
\begin{itemize}
    \item \textbf{Parte 1 - Teoria}
    8 Domande a risposta multipla - Punteggio 10/30
    \item \textbf{Parte 2 - Pratica}
    4 Esercizi a risposta aperta - Punteggio 20/30
\end{itemize}
La prova orale è facoltativa (su richiesta dello studente e/o docente) e può contribuire sia in maniera positiva che in maniera negativa al voto finale.

\paragraph{Progetto}
è possibile fare un progetto con il software "R", da consegnare prima dell'esame, può fornire un massimo di 2 punti.
