\documentclass[12pt, a4paper, openany]{book}
\usepackage[inline]{enumitem}
\usepackage{../generalStyle}
\usepackage{amsmath}

\graphicspath{ {./img/} }

\begin{document}

\title{Esami di Ricerca Operativa e Pianificazione delle Risorse}
\author{
	Fabio Ferrario\\
	\small{\href{https://t.me/fefabo}{@fefabo}}
}

\date{2023}
\maketitle


\tableofcontents

\chapter{Domande di Teoria dal Mega}
Si faccia riferimento a un problema di Massimizzazione ed, ove richiesto, ad un problema artificiale per la fase 1 del metodo del simplesso.

\domandaaperta{1?}{Se B é la matrice di base associata ad una base ottima, il valore della funzione obiettivo associato alla corrispondende SBA  non negativo}

\domandaaperta{2}{Il valore Nullo di una variabile indica che essa sia fuori base}
\rispostaaperta{Falso, una variabile puó valere 0 anche nel caso sia in base (soluzione degenere)}

\chapter{Domande di Teoria - Bibbia}


\affermazionetrue
{L'analisi di sensitivitá ha l'obiettivo di identificare i Parametri Sensibili.}
{Uno degli obiettivi principali dell'analisi di sensitività è l'identificazione dei PARAMETRI SENSIBILI, vale
a dire quei parametri il cui valore non può essere modificato senza portare ad una nuova soluzione
ottimale.}

\affermazionetrue
{Un parametro é sensibile se la sua variazione porta alla variazione della soluzione ottimale.}
{I Parametri sensibili sono parametri il cui valore non può essere modificato senza portare a una nuova soluzione ottimale}

\affermazionetrue
{Dato un termine noto destro, il metodo del simplesso identifica il corrispondente prezzo ombra tramite il coefficiente
della corrispondente variabile slack nella riga zero del tableau finale.}
{}

\affermazione
{É sempre vero che un termino noto non é un parametro sensibile se il corrispondente prezzo ombra é nullo}

\affermazione
{Un termine noto é un parametro sensibile se il corrispondente prezzo ombra é positivo.}

\paragraph{PL - Metodo del Simplesso}

\affermazionetrue
{I valori delle variabili di base si ricavano risolvendo il sistema di equazioni lineari determinato dai vincoli funzionali in forma aumentata.}
{Avendo il valore 0 delle variabili non in base si può risovlere il sistema di equazioni lineari per trovare il valore delle variabili in base.}

\affermazionefalse
{Una variabile nulla é certamente una variabile di base.}
{Una variabile nulla può essere sia in base, che fuori base. Una variabile fuori base è sicuramente nulla,
ma se una variabile è nulla può anche essere in base}

\affermazione
{il numero delle variabili di base é sempre determinato dal numero di vincoli funzionali.}

\paragraph{Nel caso di un problmea di PL, selezionare le risposte corrette:}
\begin{itemize}
    \item Se le variabili di base soddisfano i vincoli di non negativitá, la soluzione di base é una soluzione ammissibile di base.
\end{itemize}
\paragraph{Si consideri un problema primale in forma di massimo. Se il duale é inammissibile, allora}:
\begin{itemize}
    \item Nessuna risposta corretta
    \item Il primale é sempre illimitato come corollario del teorema della dualitá Debole
    \item Il primale ha almeno una soluzione ottima
    \item Valgono le condizioni degli scarti complementari.
\end{itemize}

\paragraph{Nell'ambito della programmazione lineare, quali risposte sono corrette:}
\begin{itemize}
    \item Un vertice ammissibile é una soluzione ammissibile che non giace lungo un segmento che connette altre due soluzioni ammissibili.
    \item Dato un vertice (soluzione) ammissibile, se non esiste uno spigolo, incidente in esso, cui compete un tasso positivo di miglioramento, allora la soluzione corrente é ottimale.
    \item In ogni problema di PL con n variabili di decisione e m vincoli, ogni vertice ammissibile giace all'intersezione di n+m frontiere di altrettanti vincoli.
    \item Una soluzione che giace lungo un segmento che collega due altre soluzioni ammissibili non é un vertice ammissibile.
\end{itemize}


\chapter{Domande Aperte}
\domanda[Proprietá dei Vertici Ammissibili]{3}{
    Si enuncino le Proprietà dei Vertici Ammissibili di un problema di PL. Si scelga poi una delle proprietà e si mostri un esempio grafico o numerico.
}
\rispostaaperta{
    I vertici ammissibili di un problema di PL hanno le seguenti proprietá:
    \begin{enumerate}
        \item Se esiste una sola soluzione ottima, questa sará un vertice ammissibile.
        Se esistono piú soluzioni con regione ammissibile limitata, allora almeno due di queste sono vertici ammissibili tra loro adiacenti.
        \item Il numero di vertici ammissibili é finito e dipende da n vincoli di non negativitá e m vincoli funzionali.
        il numero di combinazioni di m + n vincoli presi a gruppi di n é pari $\frac{(m+n)!}{m!n!}$. Questa quantitá (finita) rappresenta un limite superiore al numero di vertici ammissibili.
        \item Se un vertice ammissibile non ha vertici adiacenti migliori, allora non ci sono vertici migliori. 
        Quindi se il problema ha una soluzione otttima, questo vertice é la soluzione ottima.
    \end{enumerate}
}

\domanda[Proprietá di una Soluzione di Base]{4}{Si elenchino le proprietá di una Soluzione di Base}
\rispostaaperta{
    \begin{enumerate}
        \item Una variabile puó essere una variabile di base o una variabile non di base.
        \item Il numero delle variabili di base eguaglia il numero dei vincoli funzionali.
        \item La variabili non di base vengono poste a zero.
        \item I valori delle variabili di base sono ottenuti come risoluzione simultanea del sistema di equazioni lineari.
        \item Se le variabili di base soddisfano i vincoli di non negativitá, la soluzione di base é una soluzione ammissibile di base.
    \end{enumerate}
}

\domanda[Dualitá Debole e Forte]{5}{Si dia una defninizione di Dualitá Debole e Forte}
\begin{itemize}
    \item Dualitá Debole: Il valore della funzione obiettivo per una qualsiasi soluzione ammissibile del problema primale (max) non puó eccedere
    il valore della funzione obiettivo per una qualisasi soluzione ammissibile del problema duale.
    l valore del problema duale fornisce quindi un limite superiore del problema primale.
Detto breve: se il primale ha soluzione illimitata allora il duale non ha soluzione.
    \item Dualitá forte: Se esiste una soluzione ottima (finita), il valore ottimo della funzione obiettivo del problema primale è uguale al valore ottimo della funzione obiettivo del problema duale.
\end{itemize}

\domanda[Proprietá di Complementarietá]{9}{Si definisca la Proprietá di Complementarietá in PL. Si diano due esempi reali in cui é utilizzabile e cosa permette di concludere.}
\rispostaaperta{ %Da rivedere
    La complementarietá in un problema di Programmazione Lineare si evince dalla relazione tra problmea primale e duale.
    In particolare, la complementarietá afferma che ogni soluzionee primale ha una soluzione complementare duale tale che $W=Z$
    Se un problema lineare in forma primale ha soluzione ottimale x* allora anche il problema
    }

\end{document}