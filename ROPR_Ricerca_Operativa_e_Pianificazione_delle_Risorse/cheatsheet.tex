\documentclass[12pt, a4paper, openany]{book}
\usepackage{../generalStyle}
\usepackage{enumitem}

\graphicspath{ {./img/} }
\def\arraystretch{2}
\newcolumntype{Y}{>{\centering\arraybackslash}X} %new tabularx centered X column  

\begin{document}
\title{CheatSheet di Ricerca Operativa e Pianificazione delle Risorse}

\author{
	Fabio Ferrario\\
	\small{\href{https://t.me/fefabo}{@fefabo}}
}
\date{2022/2023}

\maketitle

\tableofcontents

\chapter{Ottimizzazione Non Lineare Vincolata}
\section{Condizioni di KKT}
In un problema di ottimizzazione vincolata definito come:
\begin{center}
    opt $f(x_1,...,x_n)$,
    \\
    $g_m(x_1,...,x_n) = 0$ Vincoli di Uguaglianza,
    \\
    $h_l(x_1,...,x_n) \leq 0$ Vincoli di Disguaglianza,
\end{center}
Generiamo la Lagrangiana cosí definita:
\[
    L(V) = f(X) \pm \sum_{i=0}^{m} \lambda_i \cdot g_i(X) \pm \sum_{j=0}^{l} \mu_j \cdot h_j(X) \text{ Per i problemi di MIN}
\]
in cui $\pm$ diventa $+$ per i problemi di MIN e $-$ per i problemi di MAX, 
Abbiamo che $\lambda$ sono i moltiplicatori lagrangiani associati ai vincoli di Uguaglianza, e $\mu$ quelli associati ai vincoli di Disuguaglianza.
\\\small{con $V=\{x_1,...,x_n,\lambda_1,...,\lambda_m, \mu_1,...,\mu_l\}$, ovvero tutte le variabili e $X=\{x_1,...,x_n\}$, ovvero tutte le variabili origniali.}

\paragraph{La tabella e il sistema}
Avendo questo, bisogna quindi generare un sistema che avrá $n+m+l$ incognite utilizzando le KKT,
riportate qui in modo semplificato:\\
\begin{tabularx}{\textwidth}{|Y|}
    \hline
    Stazionarietá Problemi di MIN (-)\\
        $ \nabla f = - \sum \lambda_i \cdot \nabla g_i - \sum \mu_j \cdot \nabla h_j$ \\
    \hline \hline
    Stazionarietá Problemi di MAX (+)\\
        $ \nabla f = + \sum \lambda_i \cdot \nabla g_i + \sum \mu_j \cdot \nabla h_j$ \\
    \hline
\end{tabularx}\\
\begin{tabularx}{\textwidth}{|l|cY|}
    \hline
    Ammissibilitá Vincoli Uguaglianza &  $\forall$ & $ g_i = 0$\\
    \hline
    Ammissibilitá Vincoli Disuguaglianza &  $ \forall$ & $ h_j\leq 0$\\
    \hline
    Condizione di Complementarietá & $\forall$ & $ \mu_j \cdot h_j = 0$ \\ 
    \hline
    Non Negativitá di $\mu$ &$\forall $ & $ \mu_j \geq 0$\\
    \hline   
\end{tabularx}
\\Dove con $\forall$ si intende chiaramente tutti quelli presenti.
\end{document}