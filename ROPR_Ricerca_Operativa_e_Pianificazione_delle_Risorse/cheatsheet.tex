\documentclass[12pt, a4paper, openany]{book}
\usepackage{../generalStyle}
\usepackage{enumitem}

\graphicspath{ {./img/} }
\def\arraystretch{2}
\newcolumntype{Y}{>{\centering\arraybackslash}X} %new tabularx centered X column  

\begin{document}
\title{CheatSheet di Ricerca Operativa e Pianificazione delle Risorse}

\author{
	Fabio Ferrario\\
	\small{\href{https://t.me/fefabo}{@fefabo}}
}
\date{2022/2023}

\maketitle

\tableofcontents

\chapter{Ottimizzazione Non Lineare Vincolata}
\section{Condizioni di KKT}
In un problema di ottimizzazione vincolata definito come:
\begin{center}
    opt $f(x_1,...,x_n)$
    \\
    \[
      \begin{cases}
        g_1((x_1,...,x_n)) = 0\\
        ... & \text{Vincoli di Uguaglianza}\\
        g_m((x_1,...,x_n)) = 0
      \end{cases}  
    \]
    \[
      \begin{cases}
        h_1((x_1,...,x_n)) \leq 0\\
        ... & \text{Vincoli di Disuguaglianza}\\
        h_l((x_1,...,x_n)) \leq 0
      \end{cases}  
    \]
\end{center}
Generiamo la Lagrangiana cosí definita:
\[
    L(V) = f(X) + \sum_{i=0}^{m} \lambda_i \cdot g_i(X) + \sum_{j=0}^{l} \mu_j \cdot h_j(X) \text{ Per i problemi di MIN}
\]
\[
    L(V) = f(X) - \sum_{i=0}^{m} \lambda_i \cdot g_i(X) - \sum_{j=0}^{l} \mu_j \cdot h_j(X) \text{ Per i problemi di MAX}
\]
con $V=\{x_1,...,x_n,\lambda_1,...,\lambda_m, \mu_1,...,\mu_l\}$, ovvero tutte le variabili e $X=\{x_1,...,x_n\}$, ovvero tutte le variabili origniali.

I punti stazionari vengono caratterizzati con le condizioni KKT che generano un sistema di $n+m+l$ incognite cosí definito:\\
\begin{tabularx}{\textwidth}{|Y|}
    \hline
    Stazionarietá Problemi di MIN (-)\\
        $ \nabla f(X) = - \sum_{i=0}^{m} \lambda_i \cdot \nabla g_i(X) - \sum_{j=0}^{l} \mu_j \cdot \nabla h_j(X)$ \\
    \hline \hline
    Stazionarietá Problemi di MAX (+)\\
    $ \nabla f(X) = + \sum_{i=0}^{m} \lambda_i \cdot \nabla g_i(X) + \sum_{j=0}^{l} \mu_j \cdot \nabla h_j(X)$ \\
    \hline
\end{tabularx}\\
\begin{tabularx}{\textwidth}{|l|Y|}
    \hline
    V. Uguaglianza & $g_i(X) = 0$ con $i=1,...,m$ \\
    \hline
    V. Disuguaglianza & $h_j(X) \leq 0$ con $j=1,...,l$ \\
    \hline
    Complementarietá & $\mu_j \cdot h_j = 0$ con $j=1,...,l$ \\ 
    \hline
    Non Negativitá & $\mu_j \geq 0$  con $j=1,...,l$ \\
    \hline   
\end{tabularx}
\end{document}