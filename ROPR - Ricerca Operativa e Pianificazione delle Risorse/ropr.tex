\documentclass[12pt, a4paper, openany]{book}
\usepackage[inline]{enumitem}


\begin{document}

\title{Ricerca Operativa e Pianificazione delle Risorse}
\author{Fabio Ferrario}
\date{2022/2023}
\maketitle

\tableofcontents

\chapter{Introduzione}

\section{Il corso}
Il corso di ROPR verrà svolto da:
\begin{itemize}
    \item Fabio Antonio Stella
    \item Guglielmo Lulli
\end{itemize}

Il testo di riferimento è \emph{Frederick S. Hiller and Gerald J. Lieberman, Ricerca Operativa, McGraw-Hill, nona edizione, 2010.}


\section{Il programma}

\begin{itemize}
    \item Introduzione: storia-motivazione-esempi

    \item Ottimizzazione non lineare.
    \begin{enumerate}
        \item Ottimizzazione di funzioni non lineari ad una variabile: ricerca dicotomia-metodo Bisezione- metoto Newton.
        \item Ottimizzazione di funzioni non lineari multivariate: metodo Gradiente - metodo Newton
        \item Ottimizzazione non lineare vincolate: Condizioni di Karush-Kuhn-Tucker.
    \end{enumerate}
    \item Ottimizzazione lineare e intera.
    \begin{enumerate}
        \item Introduzione alla programmazione linare (PL): Proprietà dei problemi di PL, strategie di modellizzazione.
    \end{enumerate}
    \item Soft Computing per l'ottimizzazione.
\end{itemize}

\end{document}