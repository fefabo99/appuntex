\documentclass[12pt, a4paper, openany]{book}
\usepackage[inline]{enumitem}


\begin{document}

\title{Ricerca Operativa e Pianificazione delle Risorse}
\author{Fabio Ferrario}
\date{2022/2023}
\maketitle

\tableofcontents

\chapter{Introduzione}

\section{Il corso}
Il corso di ROPR verrà svolto da:
\begin{itemize}
    \item Fabio Antonio Stella
    \item Guglielmo Lulli
\end{itemize}

Il testo di riferimento è \emph{Frederick S. Hiller and Gerald J. Lieberman, Ricerca Operativa, McGraw-Hill, nona edizione, 2010.}


\section{Il programma}

\begin{itemize}
    \item Introduzione: storia-motivazione-esempi

    \item Ottimizzazione non lineare.
    \begin{enumerate}
        \item Ottimizzazione di funzioni non lineari ad una variabile: ricerca dicotomia-metodo Bisezione- metoto Newton.
        \item Ottimizzazione di funzioni non lineari multivariate: metodo Gradiente - metodo Newton
        \item Ottimizzazione non lineare vincolate: Condizioni di Karush-Kuhn-Tucker.
    \end{enumerate}
    \item Ottimizzazione lineare e intera.
    \begin{enumerate}
        \item Introduzione alla programmazione linare (PL): Proprietà dei problemi di PL, strategie di modellizzazione.
    \end{enumerate}
    \item Soft Computing per l'ottimizzazione.
\end{itemize}

\section{Modalità d'esame}
Due Modalità:
\begin{itemize}
    \item Modalità con Parziali
    \item Modalità con Scritto unico
\end{itemize}

\subsection*{Parziali}
Primo parziale sarà durante la settimana di sospensione didattica, il secondo a fine del corso a ridosso del primo appello completo.
Entrambe le prove hanno voto massimo 14 e soglia di superamento 6.
\\É possibile \emph{recuperare solo una delle due prove parziali}, a ridosso degli appelli completi.
\paragraph*{Assignments} Durante l'insegnamento verranno assegnati 4 Assignments da risolvere a casa e consegnare entro date stabilite,
Ad ogni assignment può andare un voto minimo di 0 a un massimo di 1.
\paragraph*{Orale} non obbligatorio, consiste in 3 domande (aperte/esercizi) su tutto il programma. 
Si accede all'orale se si hanno già raggiunti i 18 punti. L'orale ha un voto di $\pm 3$.

\subsection*{Scritto unico}
Scritto (voto max 28) con domande/esercizi su tutto il programma. 
Orale faoltativo si accede con un voto minimo di 15.



\chapter{Prerequisiti}
\section{Introduzione}
Per il corso di ROPR sono necessari alcuni prerequisiti, che verranno ripassati a lezione ma è bene conoscere già.
\\I prerequisiti indicati a lezione sono:
\begin{itemize}
    \item Familiarità con le \textbf{Funzioni}.
    \begin{itemize}
        \item Definizione di Funzione, dominio, codominio
        \item Definizione di F Suriettiva, iniettiva e biiettiva
        \item Definizione di F lineare/non lineare
        \item Definizione di F concava/convessa
    \end{itemize}
    \item Familiarità con gli \textbf{Spazi Vettoriali}.
    \begin{itemize}
        \item Definizione di Vettore, Spazio Vettoriale, Base di uno S. Vettoriale
        \item Definizione di vettori linearmente dipendenti/indipendenti
    \end{itemize}
    \item Familiarità con le \textbf{Matrici e Sistemi Lineari}.
    \begin{itemize}
        \item Definizione di matrice, rango e determinante
        \item Calcolo di una matrice inversa
        \item Definizione e risoluzione di un sistema lineare
    \end{itemize}
    \item Basi di \textbf{Topologia}.
    \item Basi di \textbf{Teoria dei Grafi}.
    \item Familiarità con il \textbf{Calcolo Differenziale}.
\end{itemize}
Un ripasso sui principali requisiti può essere trovato su E-Learning

\end{document}