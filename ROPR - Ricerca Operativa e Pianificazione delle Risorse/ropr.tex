\documentclass[12pt, a4paper, openany]{book}
\usepackage[inline]{enumitem}


\begin{document}

\title{Ricerca Operativa e Pianificazione delle Risorse}
\author{Fabio Ferrario}
\date{2022/2023}
\maketitle

\tableofcontents

\chapter{Introduzione}

\section{Il corso}
Il corso di ROPR verrà svolto da:
\begin{itemize}
    \item Fabio Antonio Stella
    \item Guglielmo Lulli
\end{itemize}

Il testo di riferimento è \emph{Frederick S. Hiller and Gerald J. Lieberman, Ricerca Operativa, McGraw-Hill, nona edizione, 2010.}


\section{Il programma}

\begin{itemize}
    \item Introduzione: storia-motivazione-esempi

    \item Ottimizzazione non lineare.
    \begin{enumerate}
        \item Ottimizzazione di funzioni non lineari ad una variabile: ricerca dicotomia-metodo Bisezione- metoto Newton.
        \item Ottimizzazione di funzioni non lineari multivariate: metodo Gradiente - metodo Newton
        \item Ottimizzazione non lineare vincolate: Condizioni di Karush-Kuhn-Tucker.
    \end{enumerate}
    \item Ottimizzazione lineare e intera.
    \begin{enumerate}
        \item Introduzione alla programmazione linare (PL): Proprietà dei problemi di PL, strategie di modellizzazione.
    \end{enumerate}
    \item Soft Computing per l'ottimizzazione.
\end{itemize}

\section{Modalità d'esame}
Due Modalità:
\begin{itemize}
    \item Modalità con Parziali
    \item Modalità con Scritto unico
\end{itemize}

\subsection*{Parziali}
Primo parziale sarà durante la settimana di sospensione didattica, il secondo a fine del corso a ridosso del primo appello completo.
Entrambe le prove hanno voto massimo 14 e soglia di superamento 6.
\\É possibile \emph{recuperare solo una delle due prove parziali}, a ridosso degli appelli completi.
\paragraph*{Assignments} Durante l'insegnamento verranno assegnati 4 Assignments da risolvere a casa e consegnare entro date stabilite,
Ad ogni assignment può andare un voto minimo di 0 a un massimo di 1.
\paragraph*{Orale} non obbligatorio, consiste in 3 domande (aperte/esercizi) su tutto il programma. 
Si accede all'orale se si hanno già raggiunti i 18 punti. L'orale ha un voto di $\pm 3$.

\subsection*{Scritto unico}
Scritto (voto max 28) con domande/esercizi su tutto il programma. 
Orale faoltativo si accede con un voto minimo di 15.



\chapter{Prerequisiti}
\section{Introduzione}
Per il corso di ROPR sono necessari alcuni prerequisiti, che verranno ripassati a lezione ma è bene conoscere già.
\\I prerequisiti indicati a lezione sono:
\begin{itemize}
    \item Familiarità con le \textbf{Funzioni}.
    \begin{itemize}
        \item Definizione di Funzione, dominio, codominio
        \item Definizione di F Suriettiva, iniettiva e biiettiva
        \item Definizione di F lineare/non lineare
        \item Definizione di F concava/convessa
    \end{itemize}
    \item Familiarità con gli \textbf{Spazi Vettoriali}.
    \begin{itemize}
        \item Definizione di Vettore, Spazio Vettoriale, Base di uno S. Vettoriale
        \item Definizione di vettori linearmente dipendenti/indipendenti
    \end{itemize}
    \item Familiarità con le \textbf{Matrici e Sistemi Lineari}.
    \begin{itemize}
        \item Definizione di matrice, rango e determinante
        \item Calcolo di una matrice inversa
        \item Definizione e risoluzione di un sistema lineare
    \end{itemize}
    \item Basi di \textbf{Topologia}.
    \item Basi di \textbf{Teoria dei Grafi}.
    \item Familiarità con il \textbf{Calcolo Differenziale}.
\end{itemize}
Un ripasso sui principali requisiti può essere trovato su E-Learning

\chapter*{Lezioni}
\section*{Lezione 1 04/10}
\paragraph*{Organizzazione}Il corso è erogato in maniera verticale, quindi la prima parte con il professor Stella e la seconda con la professoressa Messina.
Di conseguenza i due turni sono uguali. Non ci saranno streaming ma ci sono delle videolezioni registrate.

Libro di testo, gran parte delle lezioni \underline{seguono} il libro di testo, quindi tanto vale prenderlo. 

\paragraph*{Che cos'è la ricerca operativa}
è una disciplina trasversale a molte altre discipline (informatica, economia, ingegneria, \dots).

Utilizza strumenti matematici per risolvere problemi complessi dove è necessario prendere delle decisioni
\begin{itemize}
    \item Come instradare il flusso dei dati in una rete
    \item come gestire un portafoglio di asset finanziari
    \item come apprendere un modello di ML
    \item \dots
\end{itemize}

\paragraph*{Prerequisiti} Sono importanti, alcuni ripassi si trovano su elearning.
\paragraph*{Modalità d'esame} come sopra. Chi è \emph{fuori corso PUÒ FARE I PARZIALI}.

\subsection*{Stella}
\paragraph*{Regola del contratto \small{lol}} si può usare il pc, il telefono o dormire (emoji con la faccia piatta) ma non si può parlare, disturbare e lanciare aeroplanini (ha messo hulk)

\paragraph*{Modelli nella ricerca operativa} La RO si occupa di Problemi di Ottimizzazione:
Data la funzione $f:R^n \to R$, un problema di ottimizzazione è formulabile come segue:
$$opt f(x) s.a. x\in X, X\subseteq R^n$$ 
(opt = ottimizzazione, s.a. = soggetto a), 
dove $opt \in \{min,max\}$, 
ci sono problemi di minimizzazione (min f(x)) e massimizzazione (max f(x)).
\\La funzione $f:R^n...$ è detta funzione obiettivo, $X$ è la regione ammissibile e $x\in X$ sono le variabili decisionali.

Quindi un problema di ottimizzazione consinste nel determinare (se esistono) uno o più punti di min max $x^*$(x star è una particolare assegniazione che gode di una certa proprietà),
{DISEGNO 1 slide 2}
\paragraph*{Ottimizzazione Vincolata/non Vincolata} Ci possiamo accontentare di avere come spazio \emph{tutto $R^n$}, però possiamo anche definire delle regioni di vincolo:
\\$X=R^n$ è ottimizzazione NON VINCOLATA, la ricerca viene effettuata in tutto $R^n$.
\\$X\subset R^n$ è ottimizzazione VINCOLATA, la ricerca è soggetta al muoversi all'interno di una certa regione ($x \in [a, +\infty)$ per esempio)
Nel caso dell'ottimizzazione vincolata semplicemente si considera la regione dello spazio definita da un determinato intervallo, quindi le soluzione valide di solito cambiano.
\paragraph*{Ottimizzazione Intera e BInaria}, non esiste solo l'ottimizzazione in cui le variabili assumono valori nello spazio reale, ma alcune hanno delle caratteristiche particolari.
$X\in Z^n$ richiede che le mie soluzioni \emph{siano valori interi}, quindi ottimizzazione intera.
$x\in \{0,1\}^n$ è un caso molto "reale", si chiama ottimizzazione binaria (vero falso, spento acceso,...). Entrambe queste ottimizzazioni formano le ottimizzazioni a numeri intere.
Esiste anche l'ottimizzazione mista, in cui alcune variabili sono intere e altre binarie.
\paragraph*{Come si specigicano i vincoli} Quando andiamo a specificare come è fatto lo spazio con delle equaz o disequaz allora parliamo di \emph{programmazione matematica}.
In questo caso i vincoli sono espressi con delle speressioni del tipo {SLIDE 6}, quindi intendiamo che $g_i(x)$ può essere maggiore, minore o uguale a Zero. NB i vincoli dipendono dalle stesse variabili da cui dipende la funzone obiettivo 
$g_i :X\to R$ è una funzione generica che lega tra loro variabili decisionali. In generale possimao avere più vincoli che definiscono la \emph{regione ammissibile}.
{SLIDE 7 equazione }Osserviamo quindi che abbiamo m vincoli e n variabili, se x è in X allora è una soluzione ammissibili
\paragraph*{esempio} consideriamo il seguente porblema: $min (x^2+y^2) s.a x+y\leq 3, x\geq 0, y\geq0$. 
x e y sono variabili di decisione, $(x^2+y^2)$ è la funzone obiettivo e ci sono tre vincoli. la regione ammissibile è tutto R limitato dai vincoli (combinati).
\paragraph*{Possibilita PM} Quindi abbiamo le seguenti Possibilita:
\begin{itemize}
    \item Problema non ammissibile, $X = \emptyset$ (problema mal posto con regione ammissibile vuota)
    \item Problema illimitato, ovvero per ogni soluzione che trovo ne posso trovare un'altra che è MIGLIORE di quella che ho trovato, (può essere illim superiormente o infer)
    \item Problema con soluzione ottima unica
    \item Problema con più soluzioni ottime (anche infinite), tutte le ottime hanno lo stesso valore della funzione obiettivo.
\end{itemize}
\paragraph*{esempio di problema illimitato } max($x^2+y^2$) s.a. $x\geq 0,y\geq 0$ non ha un "soffitto", quindi il problema è illimitato superiormente. Un problema per essere illimitato dipende sia dalla regione ammissibile che dal tipo di ottimo (max,min).


\paragraph*{Ottimi Globali e Locali} I punti di ottimo possono essere Locali o Globali (se la funzione è convessa Locale=Globale). NB un minimo globale è anche locale, un minimo locale non necessariamente è globale.
Ad oggi sappiamo dire se una soluzione sia Globale, sappiamo solo con certezza che sia locale (a meno di usare brute force).
La risoluzione di un problema di programmazione matematica consiste nel trovare una soluzione ammissibile che sia ottimo GLOBALE.
Alla fine della fiera la definizione di globale e locale corrisponde a quella di Analisi Matematica.
I punti di ottimo globali possono essere multipli.

\paragraph*{Programmazione lineare} Con lo stella, la prog lineare è la PM in cui ogni funzione (obiettivo e vincoli) sono funzioni LINEARI (polinomio di grado 1).
Se una funzione vincolo è di secondo grado ma è riscrivibile (scomponibili) in più vincoli a grado uno allora può essere lineare ($x^2-1 =(x-1) \cdot (x+1)$, quindi $x^2-1 =(x-1) \geq 0 \to $ {SLIDE 30})
{ESEMPIO del cuba libre slide 31}

\section{Lezione: Introduzione alla programmazione lineare} 11/10/22
\paragraph*{La Wyndor Glass Co.} Si introduce il problema della Wyndor Glass Co., che sarà l'esempio per il resto della lezione
Produce vetri, incluso finestre e porte.
Ha tre impianti:
\begin{itemize}
    \item Impianto 1: Produce le cornici in alluminio e le altre componenti metalliche
    \item Implianto 2: Produce le cornici in legno.
    \item Impianto 3: produce i vetri e assembla i vari prodotti.
\end{itemize}
A causa di guadagni in calo si è deciso di modificare la linea di prodotti, dismettendo la produzione di prodotti non economicamente sostenibili per liberare risorse
per incrementare gli altri prodotti.
\begin{itemize}
    \item Prodotto 1: Porta di vetro extra lusso (impianti 1 e 3)
    \item Prodotto 2: finestra con doppia apertura (impianti 2 e 3)
\end{itemize}
Non c'è limite al numero di prodotti che si possono produrre.
Questi prodotti utilizzano gli impianti 1 e 2 in maniera esclusiva, però utilizzano entrambi l'impianto 3.

\textbf{Determinare} quali tassi di produzione devono essere adottati per i due prodotti al fine di \textbf{massimizzare il profitto totale}.
Bisogna ovviamente tenere in considerazione i vincoli (tecnologici) di capacità dei tre impianti.
Ogni prodotto viene realizzato in lotti da 20 unità e i tassi di produzione vengono espressi come lotti prodotti settimanalmente.

\paragraph*{Collezioniamo i dati} per formulare il problema di programmazione matematica:
\begin{itemize}
    \item Numero di ore settimanali disponibili in ogni impianto per la produzione
    \item Numero di ore di lavorazione necessarie per produrre un lotto.
    \item profitto ottenuto dalla produzione di ogni lotto
\end{itemize}
I dati sono contenuti in questa tabella: [TABELLA SLIDE 4]

\paragraph*{Traduciamo in termini di PL}
\textbf{Variabili decisionali}
\begin{itemize}
    \item $x_1$ numero di lotti di prod 1 per settimana
    \item $x_2$ ...
\end{itemize}
\textbf{Funzione obiettivo}: max$Z = 3 \cdot x_1 + 5 \cdot x_2$
\\\textbf{vincoli}: $x_1\leq 4$ --- $2 \cdot x_2 \leq 12$ --- $3\cdot x_1 + 2\cdot x_2 \leq 18$
\\Ovviamente $x_1 \geq 0$ e $x_2 \geq 0$

\subsection*{La soluzione grafica} Essendo questo un problema molto piccolo, con sole due variabili è possibile
utilizzare la \emph{soluzione grafica}:
\begin{itemize}
    \item Disegno la regione ammissibile
    \item Determino l'ottimo
\end{itemize} 
Per disegnare la regione ammissibile, bisogna andare a eliminare una alla volta le zone del grafico che sono escluse dai vincoli.
Dentro la regione ammissibile vivono tutte le possibili soluzioni al nostro problema, bisogna solo determinare la funzione ottima.

\paragraph*{Funzione obiettivo}
Per trovare l'ottimo bisogna rappresentare la funzione obiettivo:
in questo caso, si aggiunge un asse Z e si mettono i piani di \emph{isolivello}.
[Guarda dalle lezioni bene la spiegazione] [In pratica si rappresenta Z come un piano per alcuni valori di x1 e x2 e si proiettano su x e y i segmenti che genera.
si trova quindi l'andazzo del piano di isolivello e si incrementa fino a trovare il massimo.]
\paragraph*{Vincoli funzionali di $\leq$} slide 16
\paragraph*{Regione ammissibile} La regione ammissibile $X$ è data dal soddisfacimento dei vari vincoli (rette e semipiani):
$x= x\in {R}^n : g_i ...$ finire a slide 20.
La regiione ammissibile da un punto di vista geometrico corrisponde a un poliedro convesso e può essere limitata (\textbf{politopo}) o illimitata.
\\Con la soluzione grafica si possono verificare quattro situazioni:
\begin{itemize}
    \item Ammette un'unica soluzione ottima in un \emph{vertice del poligono} che delimita la regione ammissibile.
    \item Ammette Infinite soluzioni ottime in un \emph{lato del poligono} se la direzione di decrescita è perpendicolare a un lato del poligono.
    \item Non ammette soluzione perchè la regione ammissibile è illimitata e la funzione obiettivo è limitata superiormente.
    \item Non ammette soluzione perchè la regione ammissibile è vuota
\end{itemize}
\paragraph*{quindi} un problema di PL è struttrato come segue:
$opt_{x\in X} Z = \sum_{j=1}^{n} c_j\cdot x_j$ è la funzione obiettivo (n numero di variabili decisionali)
\\ $\sum_{j=1}^n a_{ij} \cdot x_j \leq b_i$ con $i=\,...,m$ sono i vincoli (m numero vincoli)
\\$x_j$ è variabile decisionale, $c_j$ coefficiente di costo, aij termine noto sinistro bi termine noto destro (questi ultimi tre sono i parametri).

\subsection*{Programmazione Lineare: Assunzioni}
Un problea di PL si appoggia su 4 assunzioni base:
\begin{itemize}
    \item Proporzionalità: il contributo di ogni variabile decisionale è porporzionale rispetto al valore assunto.
    \item Additività
    \item Continuità
    \item Certezza
\end{itemize}

\subsection*{Esempio Personnel Scheduling}
\dots Questo è un problema di programmazione intera (non posso dividere le persone) e per "fortuna" il risulato è intero, quindi è accettabile.
Se la soluzione non fosse stata intera avremmo potuto fare due cose: arrotondare (non garantisce l'ottimalità) o aggiungere vincoli che garantiscono l'interzza delle variabili di decisione.

\section*{Metodo del simplesso}
Algoritmo generale per la risoluzione di problemi di programmazione lineare (e non solo)
\\Nel caso medio il tempo è \emph{lineare al numero delle variabili}, nel caso peggiore può risultare esponenziale.
Ancora oggi è uno degli algoritmi più efficienti per risolvere un problema di PL.

\paragraph*{cos'è} è una procedura algebrica ma i suoi concetti base sono geometrici.
\textbf{La frontiera di un vincolo} è la retta corrispondente al vincolo.
\textbf{Un vertice} si trova all'intersezione di coppie di frontiere di vincoli.
alcuni vertici sono nella regione ammissibile e si chiamano ammissibili, mentre altri sono non ammissibili.
\textbf{Spigolo} Due vertici adiacenti se sono collegti da un segmento che giace sull'intersezione delle frontiere dei vincoli condivisi.
questo segmento si chiama spigolo.
\paragraph*{Test di ottimalità} Si consideri ogni problema di PL tale da ammettere almeno una soluzione ottimale:
Se una soluzione vertice non ammette soluzioni vertice a lei adiacenti con valore della funzione obiettivo Z migliore, allora la soluzione in questione è ottimale.
\paragraph*{I 6 concetti chiave} [SLIDE 15]
\begin{enumerate}
    \item Primo
        \begin{itemize}
            \item Il metodo del simplesso ispeziona solo
            soluzioni ammissibili corrispondenti a
            vertici.
            \item Per ogni problema di PL che ammetta
            almeno una soluzione ottimale, trovarne
            una, richiede di trovare solamente il vertice
            ammissibile cui compete il miglior valore
            della funzione obiettivo(*)
            .
            \itemDato che il numero di soluzioni ammissibili
            è generalmente infinito, ridurre il numero
            di soluzioni da ispezionare ad un numero
            finito e piccolo è una semplificazione
            notevole.
        \end{itemize}
    \item Secondo: il metodo è un algoritmo iterativo
    \begin{enumerate}
        \item Inizializzazione
        \item Test di ottimalità
    \end{enumerate}
    \item Terzo: 
\end{enumerate}
 

\end{document}
