\documentclass[12pt, a4paper, openany]{book}
\usepackage[inline]{enumitem}
\usepackage{../generalStyle}


\begin{document}

\title{Ricerca Operativa e Pianificazione delle Risorse}
\author{Fabio Ferrario}
\date{2022/2023}
\maketitle

\tableofcontents

\chapter{Introduzione}

\section{Il corso}
Il corso di ROPR verrà svolto da:
\begin{itemize}
    \item Fabio Antonio Stella
    \item Guglielmo Lulli
\end{itemize}

Il testo di riferimento è \emph{Frederick S. Hiller and Gerald J. Lieberman, Ricerca Operativa, McGraw-Hill, nona edizione, 2010.}


\section{Il programma}

\begin{itemize}
    \item Introduzione: storia-motivazione-esempi

    \item Ottimizzazione non lineare.
          \begin{enumerate}
              \item Ottimizzazione di funzioni non lineari ad una variabile: ricerca dicotomia-metodo Bisezione- metoto Newton.
              \item Ottimizzazione di funzioni non lineari multivariate: metodo Gradiente - metodo Newton
              \item Ottimizzazione non lineare vincolate: Condizioni di Karush-Kuhn-Tucker.
          \end{enumerate}
    \item Ottimizzazione lineare e intera.
          \begin{enumerate}
              \item Introduzione alla programmazione linare (PL): Proprietà dei problemi di PL, strategie di modellizzazione.
          \end{enumerate}
    \item Soft Computing per l'ottimizzazione.
\end{itemize}

\section{Modalità d'esame}
Due Modalità:
\begin{itemize}
    \item Modalità con Parziali
    \item Modalità con Scritto unico
\end{itemize}

\subsection*{Parziali}
Primo parziale sarà durante la settimana di sospensione didattica, il secondo a fine del corso a ridosso del primo appello completo.
Entrambe le prove hanno voto massimo 14 e soglia di superamento 6.
\\É possibile \emph{recuperare solo una delle due prove parziali}, a ridosso degli appelli completi.
\paragraph*{Assignments} Durante l'insegnamento verranno assegnati 4 Assignments da risolvere a casa e consegnare entro date stabilite,
Ad ogni assignment può andare un voto minimo di 0 a un massimo di 1.
\paragraph*{Orale} non obbligatorio, consiste in 3 domande (aperte/esercizi) su tutto il programma.
Si accede all'orale se si hanno già raggiunti i 18 punti. L'orale ha un voto di $\pm 3$.

\subsection*{Scritto unico}
Scritto (voto max 28) con domande/esercizi su tutto il programma.
Orale faoltativo si accede con un voto minimo di 15.



\chapter{Prerequisiti}
\section{Introduzione}
Per il corso di ROPR sono necessari alcuni prerequisiti, che verranno ripassati a lezione ma è bene conoscere già.
\\I prerequisiti indicati a lezione sono:
\begin{itemize}
    \item Familiarità con le \textbf{Funzioni}.
          \begin{itemize}
              \item Definizione di Funzione, dominio, codominio
              \item Definizione di F Suriettiva, iniettiva e biiettiva
              \item Definizione di F lineare/non lineare
              \item Definizione di F concava/convessa
          \end{itemize}
    \item Familiarità con gli \textbf{Spazi Vettoriali}.
          \begin{itemize}
              \item Definizione di Vettore, Spazio Vettoriale, Base di uno S. Vettoriale
              \item Definizione di vettori linearmente dipendenti/indipendenti
          \end{itemize}
    \item Familiarità con le \textbf{Matrici e Sistemi Lineari}.
          \begin{itemize}
              \item Definizione di matrice, rango e determinante
              \item Calcolo di una matrice inversa
              \item Definizione e risoluzione di un sistema lineare
          \end{itemize}
    \item Basi di \textbf{Topologia}.
    \item Basi di \textbf{Teoria dei Grafi}.
    \item Familiarità con il \textbf{Calcolo Differenziale}.
\end{itemize}
Un ripasso sui principali requisiti può essere trovato su E-Learning

\chapter{Modelli nella Ricerca Operativa}
\section{Problemi di Ottimizzazione}
La Ricerca Operativa si occupa di Problemi di Ottimizzazione:
\definizione{
    Data la funzione $f:\R^n \to \R$, un \textbf{problema di ottimizzazione} è formulabile come segue:
    $$\text{ opt } f(x) \text{ s.a. } x\in X, X\subseteq \R^n$$
    Dove \emph{opt} $\in \{min,max\}$ e \emph{s.a.} = "soggetto a".

}
I problemi di ottimizzazione possono essere:
\begin{itemize}
    \item Minimizzazione $min f(x)$
    \item Massimizzazione $max f(x)$
\end{itemize}
\definizione{
    In un problema di ottimizzazione la \emph{funzione} $f:R^n...$ è detta \textbf{funzione obiettivo},
    $X$ è la \textbf{regione ammissibile} e $x\in X$ sono le \textbf{variabili decisionali}.

}

Quindi un problema di ottimizzazione consiste nel determinare (se esistono) uno o più punti di \emph{min}/\emph{max} $x^*$ (x star è una particolare assegniazione che gode di una certa proprietà).

\paragraph*{Ottimizzazione Vincolata/non Vincolata}
Esistono due tipi principali di Ottimizzazione, determinate dall'esistenza o meno di delle \emph{regioni di vincolo}:
\begin{itemize}
    \item \textbf{Non Vincolata} $X = \R^n$: la ricerca dell'ottimo viene effettuata in tutto $R^n$.
    \item \textbf{Vincolata} $X \subset \R^n$: La ricerca dell'ottimo è soggetta al muoversi all'interno di una certa regione ($x \in [a, +\infty)$ per esempio).
\end{itemize}

Nel caso dell'ottimizzazione vincolata semplicemente si considera la regione dello spazio definita da un determinato intervallo,
quindi le soluzioni valide di solito cambiano.
\paragraph*{Ottimizzazione Intera e Binaria}
Non esiste solo l'ottimizzazione in cui le variabili assumono valori nello spazio reale, ma esistono anche ottimizzazioni con delle caratteristiche particolari.
\\Il vincolo $X\in Z^n$ richiede che le mie soluzioni \emph{siano valori interi}, quindi \textbf{ottimizzazione intera}.
\\Un altro vincolo molto "reale" è $x\in \{0,1\}^n$, in questo caso si chiama \textbf{ottimizzazione binaria} (vero falso, spento acceso,...).
Entrambe queste ottimizzazioni formano le ottimizzazioni a numeri interi.
Esiste anche l'ottimizzazione mista, in cui alcune variabili sono intere e altre binarie.

\section{Programmazione Matematica}
\definizione{
    Quando l'insieme delle soluzioni ammissibili di un problema di ottimizzazione viene espresso attraverso un sistema di equazioni e disequazioni,
    esso prende il nome di \emph{Programmazione Matematica}.
    }
Ovvero, se i vincoli di un problema di ottimizzazione vengono espressi tramite delle equazioni/disequazioni, allora questo problema diventa un problema di programmazione matematica.

\subsection*{Come si definiscono i vincoli}
I vincoli vengono espressi con delle espressioni del tipo $g_i(x) \begin{cases}
    \geq \\ = \\ \leq %%sistemare questa dicitura
\end{cases} 0$
in cui $g_i: X\to \R$ è una funzione generica che lega tra loro le variabili decisionali.
Nota che i vincoli dipendo dalle \emph{stesse variabili da cui dipende la funzine obiettivo}.
\\L'insieme (somma) di tutti i vincoli definisce la \emph{regione ammissibile}, se $x \in X$ allora è una soluzone ammissibile.
\\In un problema di ottimizzazione abbiamo quindi $m$ vincoli e $n$ variabili.

\paragraph*{Le possibilità} Quando risolviamo un problema di PM abbiamo quindi le seguenti possibilità:
\begin{itemize}
    \item Problema non ammissibile, $X = \emptyset$ (problema mal posto con regione ammissibile vuota)
    \item Problema illimitato, ovvero per ogni soluzione che trovo ne posso trovare un'altra che è MIGLIORE di quella che ho trovato, (può essere illim superiormente o infer)
    \item Problema con soluzione ottima unica
    \item Problema con più soluzioni ottime (anche infinite), tutte le ottime hanno lo stesso valore della funzione obiettivo.
\end{itemize}

\paragraph*{Ottimi Globali e Locali}
I punti di ottimo possono essere Locali o Globali (se la funzione è convessa Locale=Globale). NB un minimo globale è anche locale, un minimo locale non necessariamente è globale.
Ad oggi \emph{non} sappiamo dire se una soluzione sia Globale, sappiamo solo con certezza che sia locale (a meno di usare brute force).
\\La risoluzione di un problema di programmazione matematica consiste nel trovare una soluzione ammissibile che sia ottimo GLOBALE.
Alla fine della fiera la definizione di globale e locale corrisponde a quella di Analisi Matematica.
I punti di ottimo globali possono essere multipli.

\chapter{Metodo del Simplesso}
Il metodo del simplesso è una pocedura algebrica che si basa su dei \emph{concetti geometrici}.
\paragraph*{Alcuni Termini}
Questo metodo si basa su alcune terminologie che ci servono per capire come funziona.
Definiamo quindi:
\begin{itemize}
    \item \textbf{Frontiera del vincolo}: è la "linea" di demarcazione di un vincolo.
    \item \textbf{Vertice}: un vertice è un punto di intersezione di coppie di frontiere di vincoli.
    \begin{itemize}
        \item \textbf{Vertici Adiacenti}: Due vertici si dicono adiacenti se condividono $n-1$ frontiere di vincoli.
        \item \textbf{Spigolo}: segmento che collega due vertici adiacenti
    \end{itemize}
    \item \textbf{Vertice Ammissibile}: Un vertice è detto ammissibile se \emph{fa parte} della regione ammissibile
    \item \textbf{Vertice \emph{non} ammissibile}: Un vertice è detto \emph{non} ammissibile se \emph{non} fa parte della regione ammissibile.
\end{itemize}
\paragraph*{Test di Ottimalità}
L'interesse per i \emph{vertici adiacenti} sta nella seguente proprietà di cui godono:
\definizione{
    Si consideri ogni problema di PL tale da ammettere almeno una soluzione ottimale:
    \\Se una soluzione vertice non ammette soluzioni vertice a egli adiacenti con valore della funzione obiettivo
    migliore, allora la \textbf{souzione} in questione è \textbf{ottimale}.
}

\end{document}
