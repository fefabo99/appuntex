\chapter{Organizzazione del corso}
\section{Informazioni sul corso}
\subsection{Lezioni}
Crediti: 4 crediti lezione (32 ore) + 4 crediti esercitazione (40 ore).\\
Due turni
\begin{itemize}
    \item Turno 1: cognomi dalla A alla L
    \item Turno 2: cognomi dalla M alla Z
\end{itemize}
Corso in blended e-learning:
\begin{itemize}
    \item Lezioni frontali in presenza in aula e da remoto (a causa di ristrutturazione delle aule, tenere sempre controllato il calendario)
    \item Esercitazioni in e-learning asincrono
    \item Materiale didattico attraverso il sito del corso (su elearning.unimib.it)
    \begin{itemize}
        \item Slides e video registrazioni delle lezioni in presenza
        \item Materiale video e testuale erogato in e-learning
        \item Quiz di autovalutazione
        \item Materiale di approfondimento (non oggetto di esame)
        \item Indispensabile l'interazione attraverso i forum con docenti, esercitatori e compagni
    \end{itemize}
\end{itemize}

\subsection{Personale del corso}
Docenti:
\begin{itemize}
    \item Pietro Braione: docente per la parte di sistemi operativi e responsabile del corso
    \item Marco Savi: docente per la parte di reti
\end{itemize}
Esercitatori:
\begin{itemize}
    \item Jacopo Maltagliati: esercitatore per la parte di sistemi operativi
    \item Samuele Redaelli: esercitatore per la parte di reti
\end{itemize}
Tutor:
\begin{itemize}
    \item Samuele Redaelli: tutor e-learning
\end{itemize}

\subsection{Modalità di svolgimento del corso}
Le parti di reti e di sistemi operativi si svolgono in contemporanea.\\
Il calendario del corso, disponibile sul sito, riporta le date delle lezioni in presenza, in remoto a causa di assenza aula, e le date in cui saranno pubblicati i materiali per l'e-learning asincrono.\\
Il programma del corso (anch'esso disponibile sul sito) riporta le esatte sezioni dei libri di testo da studiare per ciascun argomento del corso, e la distribuzione degli argomenti sulle due prove in itinere.

\subsection{Il sito del corso}
Strumento indispensabile, dal momento che il corso è in blended e-learning!\\
Aperte le iscrizioni spontanee (iscrivetevi il prima possibile se non siete già iscritti).\\
Le notizie sul corso verranno comunicate attraverso il forum avvisi.\\
I materiali didattici vengono distribuiti attraverso il sito.\\
Sono a disposizione dei forum anonimi per interagire con docenti, esercitatori e tra di voi, allo scopo di discutere gli argomenti del corso e di chiarirvi i dubbi: usateli!

\subsection{Appelli d'esame}
Cinque (o sei?) appelli:
\begin{itemize}
    \item Due prove in itinere, la prima a novembre 2023 e la seconda a gennaio 2024.
    \item Due appelli nella sessione di gennaio/febbraio 2024.
    \item Due appelli nella sessione di giugno/luglio 2024.
    \item Un appello (o due?) nella sessione di settembre 2024.
    \item Si può recuperare una (una sola) prova in itinere nel secondo appello della sessione di gennaio/febbraio 2024.
\end{itemize}
Modalità d'esame:
\begin{itemize}
    \item Questionario online svolto su computer in laboratorio.
    \item Le domande possono essere sia teoriche che esercizi che richiedono calcoli, a scelta multipla oppure domande aperte, in qualunque combinazione.
    \item Ogni prova d'esame comprende sia domande di reti che domande di sistemi operativi: non è possibile sostenere separatamente le parti di reti e di sistemi operativi!
    \item Regolamento dettagliato di esame disponibile sulla pagina del corso.
\end{itemize}

\subsection{Le tempistiche}
\subsubsection{Parte Sistemi Operativi}
Durata Corso (4 CFU)
\begin{itemize}
    \item 16 ore di didattica frontale
    \item 10 ore verranno erogate in presenza (aula), le restanti 6 ore online (sincrono)
    \item Più due ulteriori lezioni in remoto asincrono
    \item 20 ore di esercitazioni in blended e-learning
    \item Video e quiz di autovalutazione caricati sulla pagina Moodle secondo calendario didattico
    \item Possibile qualche incontro in remoto sincrono e un incontro in presenza fuori orario (per chi è interessato)
    \item Previste inoltre due sessioni di Q\&A prima delle prove in itinere
\end{itemize}
Orario lezioni: vedi calendario sul sito (verranno comunicate di volta in volta così come se in presenza o da remoto).\\
Controllare sempre il calendario sul sito per sapere se c'è lezione e quando verranno pubblicati i video/quiz per le esercitazioni!

\subsubsection{Parte Reti}
Durata Corso (4 CFU)
\begin{itemize}
    \item 16 ore di didattica frontale (in presenza o da remoto a seconda dei giorni)
    \item Previste inoltre due sessioni di Q\&A prima delle prove in itinere
    \item 20 ore di esercitazioni in blended e-learning
    \item Video e quiz di autovalutazione caricati sulla pagina Moodle secondo calendario didattico
\end{itemize}
Orario lezioni: vedi calendario sul sito (verranno comunicate di volta in volta così come se in presenza o da remoto).\\
Controllare sempre il calendario sul sito per sapere se c'è lezione e quando verranno pubblicati i video/quiz per le esercitazioni!

\section{Contatti}
\subsubsection{Parte Sistemi Operativi}
Prof. Pietro Braione.\\
Ufficio: Edificio U14 (DISCo), secondo piano, stanza 2051.\\
Email: pietro.braione@unimib.it
\begin{itemize}
    \item Inserire "[RSO]" prima dell'oggetto dell'email!
\end{itemize}
Telefono: 0264487915\\
Orario di ricevimento: appuntamento via email.\\
Team:
\begin{itemize}
    \item Jacopo Maltagliati - Esercitatore (email: j.maltagliati@campus.unimib.it)
    \item Samuele Redaelli - Tutor (email: samuele.redaelli@unimib.it) 
\end{itemize}

\subsubsection{Parte Reti}
Prof. Marco Savi\\
Ufficio: Edificio U14 (DISCo), Secondo Piano, Stanza 2035\\
Email: marco.savi@unimib.it
\begin{itemize}
    \item Inserire "[RSO]" prima dell'oggetto dell'email!
\end{itemize}
Telefono: 0264487884\\
Orario di ricevimento: appuntamento via email\\
Team:
\begin{itemize}
    \item Samuele Redaelli - Esercitatore (email: samuele.redaelli@unimib.it) 
    \item Samuele Redaelli - Tutor (email: samuele.redaelli@unimib.it) 
\end{itemize}

\section{Obiettivi del corso}
\subsubsection{Parte Sistemi Operativi}
Acquisire le conoscenze fondamentali relativi ai sistemi operativi:
\begin{itemize}
    \item A cosa servono i servizi operativi? Perché sono necessari?
    \item Che servizi offrono ai programmi e agli utenti di un sistema?
    \item Come sono implementati i sistemi operativi e i servizi che offrono?
\end{itemize}
Particolarmente importanti sono le esercitazioni pratiche, nelle quali imparerete ad usare i servizi di un sistema operativo moderno (Linux).
\begin{itemize}
    \item Necessario un computer laptop
    \item Sono argomento di esame!
\end{itemize}

\subsubsection{Parte Reti}
Acquisire le conoscenze fondamentali per comprendere l'architettura e i protocolli principali delle reti di telecomunicazioni basate sullo stack TCP/IP.
\begin{itemize}
    \item Lo stack TCP/IP è alla base della quasi totalità dei servizi di comunicazione.
\end{itemize}
Al termine del corso avrete appreso i principi fondamentali del funzionamento di Internet.

\section{Materiale e strumenti didattici}
\subsubsection{Parte Sistemi Operativi}
Libro di riferimento:
\begin{itemize}
    \item Abraham Silberschatz, Peter Baer Galvin, Greg Gagne, Sistemi Operativi - Concetti e Esempi, Decima edizione, Pearson, 2019.
    \item Versione in inglese: Abraham Silberschatz, Peter Baer Galvin, Greg Gagne, Operating Systems Concepts, 10th Edition, John Wiley and Sons, 2018.
\end{itemize}
Materiale online su Moodle.
\begin{itemize}
    \item Slides e registrazioni video delle lezioni
    \item Quiz di autovalutazione
    \item Video delle esercitazioni
\end{itemize}
Forum tematico anonimo di discussione su Moodle (indispensabile per le esercitazioni!)

\subsubsection{Parte Reti}
Materiale online su Moodle
\begin{itemize}
    \item Slides ufficiali del libro di riferimento (rivisitate, in inglese)
    \item Video sulla parte di esercitazioni
\end{itemize}
Libro di riferimento
\begin{itemize}
    \item Jim Kurose, Keith Ross, Reti di calcolatori ed Internet - Un approccio top-down, Ottava edizione, Pearson, 2021
    \item Versione in inglese: Jim Kurose, Keith Ross, Computer Networking - A Top-Down Approach, 8th Edition, Pearson, 2021
\end{itemize}
Forum su Moodle per la parte di reti (specialmente utile per la parte di esercitazioni\dots)

\section{Programma del corso}
\subsubsection{Parte Sistemi Operativi}
Sistemi operativi: struttura e servizi\\
Servizi:
\begin{itemize}
    \item Processi e thread: i servizi
    \item Gestione della memoria: i servizi
    \item File system: i servizi
\end{itemize}
Struttura:
\begin{itemize}
    \item Interfaccia e struttura del kernel
    \item Processi e thread: la struttura
    \item Gestione della memoria: la struttura
    \item File system: la struttura
\end{itemize}

\subsubsection{Parte Reti}
Parti specifiche del libro (da Capitolo 1 a Capitolo 6)
\begin{itemize}
    \item Capitolo 1: Introduzione alle reti di calcolatori e Internet
    \item Capitolo 2: Livello di applicazione [cenni]
    \item Capitolo 3: Livello di trasporto
    \item Capitolo 4: Livello di rete - Piano dei dati
    \item Capitolo 5: Livello di rete - Piano di trasporto
    \item Capitolo 6: Livello di collegamento e reti locali
\end{itemize}