\documentclass[12pt, a4paper, openany]{book}
\usepackage{../generalStyle}

\graphicspath{ {./img/} }

\begin{document}

\title{Domande Orale APS}

\author{
    Elia Ronchetti\\
	\small{\href{https://t.me/ulerich}{@ulerich}}
}

\date{2022/2023}

\maketitle

\tableofcontents

\chapter{Domande Capitolo 1}
\subsection*{Che cosa sono l'analisi e la Progettazione}
\begin{itemize}
    \item Analisi - Enfatizza l'investigazione di un problema e dei suoi requisiti,
    anzichè di una soluzione
    \item La progettazione enfatizza una soluzione concettuale che soddisfa i
    requisiti del problema
\end{itemize}

\subsection*{Che cosa sono l'analisi e la progettazione Object Oriented}
L'analisi OO (Object Oriented) definisce le classi concettuali (di dominio), servono a descrivere i concetti
o oggetti (del mondo reale) relativi al problema.\\
La progettazione OO definisce le classi software che servono a modellare i concetti del
mondo reale in classi e oggetti software utilizzabili all'interno del codice scritto in un linguaggio
OO.

\subsection*{UML è una metodologia?}
No è un linguaggio visuale

\subsection*{Cos'è un processo software}
\'E un approccio disciplinato per la costruzione, il rilascio e la manutenzione del codice. Definisce chi (ruoli) fa cosa
(attività), quando (organizzazione temporale) e come (metodologie) per raggiungere un certo obiettivo. Vi sono
vari tipi di processi per lo sviluppo, ma la maggior parte hanno (o rielaborano) le attività fondamentali.
\paragraph*{Come ricordarsi la risposta} Parallelo con la costruzione di una casa, costruzione-rilascio(vendita)-manutenzione.\\
Il chi sono le varie aziende che operano (ruoli), il cosa sono le attività (caldaia, pannelli solare, cappotto, ecc.), il quando
è (quando l'azienda arriva, organizzazione tmeporale) e come il metodo che utilizzano per il lavoro.
\subsection*{Quali sono le attività fondamentali di processo?}
Le seguenti attività sono solitamente incluse in ogni processo software (che tuttavia può cambiarle e gestirle come
vuole):
\begin{itemize}
    \item Requisiti
    \item Analisi
    \item Progettazione
    \item Implementazione
    \item Validazione
    \item Rilascio e installazione
    \item Manutenzione ed evoluzione
    \item Gestione del progetto
\end{itemize}
\paragraph*{Come ricordarsi la risposta}Parallelo con richiesta PC assemblato dove, chiedo quali sono le componenti,
analizzo se sono ok, produco un preventivo (progettazione), costruisco la macchina (implementazione)
Testo se è tutto oK (Validazione), installo il sistema operativo (rilascio e installazione), il cliente poi
dovrà mantenere la macchina (manutenzione ed evoluzione) ed eventualmente effettuare upgrade in futuro, quindi gestirla
(gestione del progetto).\\
Altro esempio valido è la costruzione di una casa.
\subsection*{Che cos'è il processo a cascata}
Il processo a cascata è un processo software sequenziale che prevede le seguenti fasi:
\begin{itemize}
    \item Definizione dei requisiti
    \item Design del sistema e del software
    \item Implementazione e testing unitario
    \item Integrazione e testing di sistema
    \item Rilascio e manutenzione (che fa ripartire la cascata da uno degli step precedenti)
\end{itemize}
Si è rivelato essere un approccio fallimentare dato che il suo presupposto è che i requisiti
siano stabili nel tempo, assunzione falsa dato che possono variare e anche di molto nel tempo (a
seconda per esempio delle esigenze del cliente).

\subsection*{Che cos'è UP e quali sono le sue fasi?}
UP (Unified Process) è un processo per lo sviluppo software di tipo iterativo, incrementale
ed evolutivo, dato che si basa su iterazioni time-boxed alla fine delle quali si verifica se i requsiti definiti sono corretti
e se ciò che è stato prodotto durante l'iterazione (software eseguibile) è ok. Incrementale dato che
nella successiva iterazione si riparte dai risultati prodotti dalla precedente ed evolutivo perchè è in grado
di cambiare ed evolersi secondo le esigenze del cliente grazie alle iterazioni. Le sue fasi sono le seguenti
\begin{itemize}
    \item Ideazione - Analisi e stime iniziali per avviare il progetto (stime sia temporali che economiche)
    \item Elaborazione - Realizzazione del nucleo dell'architettura, visione raffinata, identificazione di gran parte
    dei requisiti
    \item Costruzione - Implementazione delle capacità operative iniziali e successiva preparazione al rilascio
    \item Transizione - Completamento del progetto
\end{itemize}

\subsection*{Che cos'è un'iterazione}
Si tratta di una finestra temporale (2-6 settimane) all'interno della quale si produce
un mini progetto (tranne nella fase dell'ideazione), ogni iterazione include:
\begin{itemize}
    \item Pianificazione
    \item Analisi e Progettazione
    \item Costruzione
    \item Implementazone e Test
    \item Release
\end{itemize}
\subsection*{Che cos'è il Metodo Agile}
Lo sviluppo Agile è una forma di sviluppo che incoraggia l'agilità, ovvero una risposta
rapida e flessibile ai cambiamenti, adottabile da qualsiasi processo iterativo. UP
può essere reso agile aggiungendo:
\begin{itemize}
    \item Un piccoli insieme di attività ed elaborati
    \item I requisiti e la progettazione non vengono completati prima dell'implementazione, ma
    emergono in modo adattivo durante una serie di iterazioni, anche sulla base di feedback
    \item Applicazione di UML in stile agile (fare solo ciò che serve)
\end{itemize}

\chapter{Capitolo 2 - Analisi dei requisiti e Casi d'uso}
\subsection*{Che cosa sono i requisiti e di che tipo possono essere?}
Un requisito è una capacità o condizione a cui il software deve essere conforme. Ogni software
deve possedere delle funzionalità o caratteristiche di qualità.\\
I requisiti possono essere:
\begin{itemize}
    \item Funzionali - Definiscono servizi o funzionalità che soddisfano le richieste dal cliente. Sono
    descritti dai \textbf{Casi d'uso}
    \item Non funzionali - Requisiti legati alle proprietà  del sistema (velocità, sicurezza, affidabilità, ecc.)
\end{itemize}
I requisiti funzionali sono spesso più vincolanti dei funzionali, un requisito non funzionale per
essere rispettato può generare la creazione di una serie di requisiti funzionali.\\
I requisiti sono estremamente importanti in un progetto, considerando che la loro
definizione incompleta è una delle cause principali del fallimento di progetti.\\
In UP vengono definiti inizialmente durante l'ideazione (insieme al cliente) e successivamente
nella fase di elaborazione. Non è da escludere che possano essere modificati anche più tardi,
ma con UP tendono a stabilizarsi nel tempo. Consideriamo che il 25 \% dei requisiti cambia dopo
l'ideazione. UP incoraggia un'acquisizione dei requisiti agile, attraverso la scrittura 
dei \textbf{Casi d'uso} con i clienti, anche tramite interviste, workshop dei requisiti con gli sviluppatori
e clienti, e feedback dai clienti dopo ogni iterazione.

\subsection*{Cosa sono i casi d'uso?}
I casi d'uso sono storie scritte in formato testuale (preferibilmente in forma ridotta)
e costituiscono un dialogo fra uno o più attori e un sistema che svolge un compito. Sono
utilizzati per scoprire e registrare i requisiti funzionali (possono anche descrivere requisiti
non funzionali).\\
Se presenti il caso d'uso prevede anche il SSD (System Sequence Diagram - Diagrammi di Sequenza di Sistema).

\subsection*{Perchè scriviamo i casi d'uso}
\begin{itemize}
    \item Perchè ci aiutano a identificare e descrivere i requisiti funzionali
    \item Sono comprensibili dal cliente dato che sono i formato testuale e sono privi di gergo informatico
    \item Mettono in risalto gli obiettivi dell'utente e il loro punto di vista
    \item Sono utili per produrre test e la guida utente
\end{itemize}

\subsection*{Che cos'è un attore e quali sono i vari tipi}
Un attore è qualcosa dotato di un comportamento, anche il Sistema in discussione è considerato un attore
quando ricorre ai servizi di altri sistemi.\\
I vari tipi di attori sono:
\begin{itemize}
    \item Attore primario - Utilizza direttamente i servizi del sistema in discussione affinchè
    vengano raggiunti gli obiettivi dell'utente
    \item Attore finale - Vuole che il sistema venga utilizzato per raggiungere i propri obiettivi 
    (spesso coincide con il primario)
    \item Attore di supporto - Offre un servizio al Sistema (es. Servizi esterni di pagamento)
    \item Attore fuori scena - Attori interessati al comportamento del Caso d'uso, ma non è nessuno dei 3 sopra elencati,
    e non interviene nel caso d'uso (es. Stato per quanto riguarda il rispetto di una normativa).
\end{itemize}

\chapter{Elaborazione}

\subsection*{Cosa succede nella fase dell'elaborazione?}
Si stabilizzano la maggior parte dei requisiti (circa l'80 \%) viene programmato, definito e testato il nucleo
dell'architettura e si attenuano i rischi maggiori. L'intera fase dura alcuni mesi (sicuramente più 
di 1 iterazione). Qua si producono elaborati come il Modello di Dominio, il Modello di Progetto e un Documento
sull'architettura.

\subsection*{Cos'è un Modello di Dominio}
Il modello di Dominio è un diagramma delle classi concettuali che mostrano i concetti
del mondo reale e come sono in relazione fra di loro nel dominio di interesse. \'E il documento
principale dell'analisi a oggetti. In esso troviamo:
\begin{itemize}
    \item Classi concettuali - Rappresentano un'idea un oggetto del mondo reale, è un descrittore di
    oggetti che possiedono le stesse caratteristiche
    \item Attributi - Proprietà elementare degli oggetti (visibilità di default private)
    \item Associazioni - Relazione semantica fra le classi
\end{itemize}
Viene utilizzato per:
\begin{itemize}
    \item Mell'analisi per comprendere il dominio del sistema e per definire un linguaggio
    comune sulle parti interessate
    \item Nella progettazione come fonte di ispirazione per lo strato di dominio
\end{itemize}

\subsection*{Che cos'è un SSD}
Un SSD è un elaborato che descrive eventi di input e output relativi al Sistema
in discussione, illustra inoltre come gli attori (o più sistemi) interagiscono con esso.\\
Si adotta un approccio a scatola nera, quindi tratteremo il sistema come un'unica entità che 
interagisce con gli attori. Il testo dei Casi d'uso sarà la base per costruire questo
diagramma. \'E buona norma produrre un SSD per ogni scenario principale di ciascun caso d'uso.

\subsection*{Cosa sono le operazioni di sistema}
Le operazioni di sistema sono funzionalità pubbliche che il sistema, a scatola nera, 
mette a disposizione tramite la sua interfaccia. Vengono eseguite quando l'utente genera
un evento di sistema. Ricordiamoci che siamo nel livello di dominio
quindi non dobbiamo ragionare come se stessimo creando metodi. Un metodo software è l'implementazione di un'operazione.

\subsection*{Cosa sono i Contratti}
I contratti delle operazioni descrivono nel dettaglio il cambiamento degli oggetti di dominio
dopo aver eseguito un'operazione di sistema. La descrizione del cambiamento
post operazione viene effettuata nelle Post Condizioni.

\subsection*{Che cos'è l'architettura logica}
L'architettura logica si occupa di dividere le classi software in starti/pacakges/sottosistemi
in base alla loro semantica e appartenenza a un determinato strato/package/sottositema a
livello software.

\subsection*{Con quali criteri possiamo dividere l'Architettura Logica}
\begin{itemize}
    \item Livello
    \item Strato
    \item Partizione
\end{itemize}
\paragraph*{Per ricordarlo}LiSP (Livello Strato Partizione) "Anagrammi che evocano forti
emozioni sono utili per ricordare"
\subsection*{Quali sono gli strati dell'architettura logica}
\begin{itemize}
    \item UI (Presentazione o Interfaccia Utente)
    \item Logica Applicativa
    \item Servizi Tecnici
\end{itemize}

\subsection*{Che cos'è la logica applicativa}
Lo stato che riguarda gli oggetti di dominio, utile suddividere questo strato in:
\begin{itemize}
    \item Strato del dominio - Oggetti di dominio
    \item Strato application - Oggetti che gestiscono il workflow degli oggetti di dominio
\end{itemize}
La logica applicativa non va messa direttamente nell'interfaccia UI.

\subsection*{Cosa sono i diagrammi di interazione}
Sono i diagrammi che descrivono il modo in cui interagiscono gli oggetti
attraverso lo scambio di messaggi. Un'interazione è una specifica di come gli oggetti
si scambiano messaggi nel tempo per eseguire un compito.\\
Ci sono 4 tipi di diagrammi di interazione:
\begin{itemize}
    \item Diagrammi di sequenza SD - Mostrano le interazioni tra linee vita verticali
    \item Diagrammi di comunicazione CD - Mostrano le interazioni tra oggetti in un formato a grafo o rete
    \item Diagrammi di interazione generale
    \item Diagrammi di temporizzazione
\end{itemize}
Noi abbiamo visto i primi 2

\subsection*{Cosa sono i Sequence Diagram}
Simili agli SSD per quanto riguarda la struttura, in questo caso però siamo a livello
software e non più di dominio. negli SD infatti le linee vita rappresentano spesso 
gli oggetti

\chapter{Classi}

\subsection*{Che cos'è un diagramma delle Classi Software}
Si tratta di un diagramma che rappresenta le classi, le interfacce e le relative associazioni,
sono utilizzati per la modellazione statica delle classi (già introdotti nel modello di
dominio).

\subsection*{Che cos'è un oggetto}
Un oggetto è un'istanza di una classe che definisce l'insieme comune di
caratteristiche (operazioni e attributi) condivisi da tutte le istanze.\\
Tutti gli oggetti hanno:
\begin{itemize}
    \item Identificativo univoco
    \item Valore degli Attributi - la parte dei dati (detta stato)
    \item Operazioni - la parte del comportamento
\end{itemize}

\subsection*{Che cos'è un'interfaccia}
Un'interfaccia è un insieme di funzionalità pubbliche identificate da un nome,
separa le specifiche di una funzionalità dall'implementazione stessa, definendo un contratto
con la classe che andrà a estendere l'interfaccia stessa, la classe dovrà rispettare questo
contratto.
\end{document}
