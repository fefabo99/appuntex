\chapter{Dal progetto al codice}
Gli elaborati creati fin'ora saranno utilizzati come input per il processo
di generazione del codice. Anche se la creazione di codice non fa parte
dell'OOA/D (Object Oriented Analysis/Design), è comunque l'obiettivo finale
della progettazione di un software. Durante la programmazione ci si devono
aspettare e si devono pianificare numerosi cambiamenti e deviazioni rispetto al
progetto e questo ha diverse implicazioni (per esempio aggiornare documentazione).
\section{Creazione di Classi e Interfacce}
I diagrammi di classe servono proprio a questo, dato che contengono
\begin{itemize}
    \item Nomi delle classi, interfacce e superclassi
    \item attributi
    \item firme delle operazioni
    \item associazioni tra classi
\end{itemize} 
Queste informazioni sono sufficienti per creare un'implementazione di base delle
classi, alcuni strumenti effettuano automaticamente queste operazioni (come RSA).
\subsection{Gestione delle Eccezioni}
Il comportamento del sistema quando sono generate delle eccezioni può essere descritto
con i modelli dinamici, se la gestione delle eccezioni non è opportunamente progettata il
loro effetto diventa preso un fenomeno globale e incontrollato.\\
\paragraph*{Alcune linee guida}
\begin{itemize}
    \item Definire un numero finito di eccezioni che possono essere rilasciate da 
    un componente e che siano interpretabili dal client (re-mapping delle eccezioni)
    \item Nascondere infomrazioni inutili ai client e diminuire il set di eccezioni (es.
    ThreadInterruptedException è scarsamente utile al client)
    \item Attenzione a non nascondere informazioni utili
    \item Le eccezioni sono appropriate quando si hanno fallimenti di risorse (disco, memoria, accesso rete)
    \item Il valore dell'eccezioni può essere utilizzare per fornire info aggiuntive
    \item In UML le eccezioni possono essere indicate nelle stringhe di proprietà dei messaggi
    e delle dichiarazioni delle operazioni
\end{itemize}
\subsection{Ordine di implementazione delle classi}
Esistono 3 stili principali, di cui l'ultimo è un mix dei primi due:
\begin{enumerate}
    \item Guidato dagli scenari descritti nei modelli dinamici (si implementano più parti di diverse
    classi)
    \item Basato sulle classi - Ordine definito in base alle dipendenze
    \item Misto - Si implementano prima le classi che rappresentano le entità
    principali del sistema e si implementano iterativamente le operazioni di sistema
\end{enumerate}
\section{Progettare la visibilità}
La visibilità viene definita come la capacità di un oggetto di "vedere" o di avere un
riferimento ad un altro oggetto. Affinchè un oggetto mittente invii un messaggio a un oggetto
destinatario, il destinatario deve essere visibile al mittente - il mittente
deve avere qualche tipo di riferimento (o puntatore) all'oggetto destinatario.\\
Risulta fondamentale garantire le visibilità necessarie per consentire le interazioni fra gli oggetti.
\subsection{Visibilità tra oggetti}
Come possiamo determinare se una risorsa (ad esempio un'istanza) rientra nella portata
di un'altra?\\
La visibilità può essere ottenuta dall'oggetto A all'oggetto B in quattro modi comuni:
\begin{itemize}
    \item Visibilità per attributo - B è un attributo di A
    \item Visibilità per parametro - B è un parametro di un metodo di A
    \item Visibilità locale - B è un oggetto locale (non parametro) di A
    \item Visibilità globale - B è in qualche modo visibile globalmente
\end{itemize}
\subsection{Visibilità per attributo}
La visibilità per attributo da A a B esiste quando B è un attributo di A.\\
Si tratta di una visibilità relativamente permanente perchè persiste finchè A e B esistono,
è molto diffusa nei sistemi OO.
\subsection{Visibilità per parametro}
La visibilità per parametro da A e B esiste quando B viene passato come parametro ad un
metodo di A. Si tratta di una visibilità relativamente temporanea perchè persiste solo
nell'ambito del metodo.
\subsection{Visibilità locale}
La visibilità dichiarata localmente da A a B esiste quando B è dichiarato come oggetto locale
nell'interno di un metodo di A. Si tratta di una visibilità relativamente temporanea
perchè persiste solo nell'ambito del metodo.
\begin{itemize}
    \item Creare una nuova istanza locale e assegnarla ad una variabile locale
    \item Assegnare l'oggetto di ritorno da un'invocazione di metodo ad una variabile locale
\end{itemize}
\paragraph*{Trasformare la visibilità} è possible trasformare la visibilità locale
in visibilità per parametro oppure la visibilità per parametro in visibilità per attributo.
\subsection{Visibilità globale}
La visibilità globaile da A a B esiste quando B è globale per A. Si tratta di una
visibilità relativamente permanente perchè persiste finchè A e B esistono. la forma 
meno comune, puiò favorire alto accoppiamento. Un modo per ottenere questo risultato è quello
di assegnare un'istanza a una variabile globale (possibile in C++, ma non in Java).\\
Il metodo migliore per ottenere la visibilità globale è quello di utilizzare il pattern
Singleton.
\section{Test Driven Development}
Il TDD (Test Driven Development) è una delle pratiche introdotte da Extreme Programming (XP) 
ed è diventata una pratica molto diffusa anche in metodologie non agili. Applicabile
anche a UP e Scrum, consente lo sviluppo iterativo ed evolutivo. Prevede Refactoring continuo
del codice.
\subsection{Benefici del TDD}
Alla fine dell'attività di sviluppo è stata ottenuta sia l'implementazione della classe che i suoi
casi di test d'unità/integrazione e quindi ci siamo tolti un bel po' di lavoro, dato che
altrimenti andrebbe fatto tutto dopo lo sviluppo delle classe.\\
Interfaccia e comportamenti saranno dettagliati dato che si dovrà ragionare sui
metodi necessari ai test e sui possibili comportamenti. Il testing è oltretutto ripetibile
automaticamente. Ci sarà anche maggior confidenza nel cambiare il sistema e il codice sarà
migliore.
\paragraph*{Tipologie di test}
\begin{itemize}
    \item Test di unità
    \item Test di integrazione
    \item Test di sistema
    \item Test di accettazione
    \item Test di regressione
\end{itemize}
\subsection{Test di unità}
Il test di unità si traduce in:
\begin{itemize}
    \item Scrivere il codice di un test prima della classe da testare
    \item Implmenetare parte della classe in modo che questa superi il test
    \item Scrivere un nuovo caso di test
    \item Reiterare
\end{itemize}
Dopo aver scritto i casi di test procediamo con l'implementazione.
Una volta che i casi di test sono stati superati dall'implementazione attuale procediamo
progettando nuovi casi di test oppure passando ad una nuova classe/metodo.
\paragraph*{Schema dei metodi di test di unità}
\begin{itemize}
    \item Preparazione - Crea l'oggetto da verificare (detto anche fixture)
    \item Esecuzione - Far fare qualcosa alla fixture al fine di eseguire alcune operazioni
    che si desidera testare
    \item Verifica - Verifica che i risultati ottenuti corrispondono a quelli previsti
    \item Rilascio - Opzionalmente rilascia o ripulisce gli oggetti e le risorse utilizzate nel test
\end{itemize}
\paragraph*{Esempi di Framework di test}
xUnit è un Framework per test di unità per molti linguaggi, per esempio NUnit è per .NET,
mentre JUnit è per Java.\\
\subsection{Ciclo del TDD}
\begin{itemize}
    \item Scrivere un caso di test che fallisce 
    \begin{itemize}
        \item Non scrivere codice di produzione fino a quando non hai scritto un
        test unitario
        \item Non scrivere più di un test di quanto è sufficiente per far fallire la
        compilazione o esecuzione del test
    \end{itemize}
    \item Scrivi il codice più semplice possibile per far passare il test
    \item Riscrivi o ristruttura il codice migliorandolo, oppure passa a scrivere il
    prossimo test unitario
\end{itemize}
Il Refactoring può essere applicato dopo aver effettuato ciascun test, oppure dopo diversi
test.
\chapter{Refactoring}
\section{Varie definizioni}
Il refactoring è il processo di modifica di un sistema software che non altera il suo
comportamento esterno e migliora la sua struttura interna (più semplice da capire e modificare).\\
Un'altra definizione è che il refactoring è un cambiamneto al sistema che lascia inalterato
il suo comportamento, ma che migliora alcune qualità funzionali come semplicità, flessibilità,
chiarezza, performance, riusabilità, manutenibilità.\\
\subsection{Quando fare refactoring}
\begin{itemize}
    \item Quando aggiugi una funzione
    \item Quando hai bisogno di fixare un bug
    \item Mentre fai revisione del codice
    \item Quando il codice è difficile da capire
\end{itemize}
\subsection{Quando NON fare refactoring}
\begin{itemize}
    \item Quando il codice non funzione
    \item Quando sei troppo vicino a una deadline
    \item QUando il design è così pessimo che è necessario riscrivere il codice
    \item Prima di effettuare il refactoring, assicurati di avere un buon set di casi di test
\end{itemize}
\subsection{Come effettuare refactoring}
\begin{itemize}
    \item Essere sicuro che i testi passini
    \item Trova del codice che "smells" (puzza)
    \item Determina come semplificare il codice
    \item Effettuare le semplificazioni
    \item Esegui Test per verificare che non siano stati introdotti bug
    \item Ripeti il processo finchè la "smell" (puzza) non se nè andata
\end{itemize}
\subsection{Perchè fare refactoring}
\begin{itemize}
    \item Migliorare il design del test (che decade nel tempo)
    \item Ridurre la quantità di codice richiesta per completare un task (rimuovendo
    codice inutile)
    \item Migliorare la leggibilità del codice
    \item Ti aiuta a trovare Bug (codice più complesso è più soggetto ad avere bug)
\end{itemize}
\subsection{Problemi con il refactoring}
\begin{itemize}
    \item Database (richiede di modificare i dati per adattarsi al nuovo design,
    la migrazione più essere molto lungha e costosa)
    \item Personal Ownership of Code
    \item Cambiare le interfacce (interfacce pubbliche non possono essere cambiate)
\end{itemize}
\subsection{Esempi di refactoring}
\begin{itemize}
    \item Extract Method - Trasforma un metodo lungo in uno più breve, estraendone una parte in
    un metodo di supporto
    \item Extract Class - Crea una nuova classe e vi muove alcuni campi e metodi di un'altra classe
    \item Move Method - Crea un nuovo metodo, con un corpo simile, nella classe che lo
    usa di più
\end{itemize}
\section{Code Smells}
La nozione di code smells è molto importante per il refactoring.\\
Si tratta sostanzialmente di caratteristiche che sono forti indicatori di una struttura
cattiva che dovrebbe essere ristrutturata.\\
\begin{quote}
    If it stinks, change it. (M.Fowler, Refactoring)
\end{quote}
I code smelss sono regole empiriche che indicano che il codice potrebbe essere migliorato.
Risulta sempre importante valutare il caso specifico, perchè non sempre un code smell porta
necessariamente a un refactoring. Comunque un codice che presenta code smells è probabile
che significhi che dovresti cambiare qualcosa.\\
\section{Bad Smells in Code}
\begin{itemize}
    \item Duplicated Code
    \item Long Method
    \item Large Class
    \item Long Parameter List
    \item Divergent Change
    \item Shotgun Surgery
    \item Feature Envy
    \item Data Clumps
    \item Primitive Obsession
    \item Switch Statements
    \item Parallel Inheritance Hierarchies
    \item Lazy Class
    \item Speculative Generality
    \item Temporary Field
    \item Message Chain
    \item Middle Man
    \item Inappropriate Intimacy
    \item Alternative Classes with Different Interfaces
    \item Incomplete Library Class
    \item Data Class
    \item Refused Bequest
    \item Comments
\end{itemize}
Uno smells è un sintonmo, il refactoring è la cura. Usa il relativo refactoring per
il relativo smells. Anche se stessi sintomi possono necessitare di cure diverse, così come
ogni smelss può suggerire più di un refactoring.
\subsection{Duplicated Code}
Due frammenti di codice che sembrano uguali.\\
I duplicati si verificano solitamente quando più programmatori stanno lavorando
su differenti parti di uno stesso programma simultaneamente e dato che stanno 
lavorando su differenti task, potrebbero non accorgersi che stanno creando codice che
è già stato creato da altri colleghi.\\
Si tratta di duplicazione anche quando specifiche parti di codice sembrano diverse,
ma fanno la stessa cosa. Questo tipi di codice duplicato è difficile da trovare e sistemare.\\
\subsection{Long Method}
Un metodo contiene troppe righe di codice. Generalmente qualsiasi metodo lungo più di 10 righe
dovrebbe iniziare a far scattare l'allarme.\\
Questo accade perchè spesso per pigrizia è più difficile scrivere un nuovo metodo, risulta
più facile aggiungere qualche linea di codice in più a un metodo già esistente, così facendo
mano a mano si aggiunge codice e si arrivare ad avere un metodo enorme contenente molte righe.\\
\paragraph*{Risoluzione} Come regola generale, se senti che qualcosa all'interno di un metodo
necessita di un commento per essere spiegato probabilmente dovresti creare un metodo separato.\\
Come Refactoring è buona norma usare Extract Method per ridurre la lunghezza del corpo del
metodo.
\subsection{Feature Envy}
Un metodo in una classe accede dati di un altro oggetti più spesso che ai dati della sua stessa
classe. Questo può accadere dopo aver spostato attributi in una classe dati. Se è questo il caso
potresti voler spostare anche le operazioni da effettuare sui dati in questa classe.\\
\paragraph*{Risoluzione} Come regola base, se le cose cambiano insieme, mantienile insieme nello
stesso posto. Solitamente dati e funzioni che usano quei dati sono modificati insieme.\\
Utilizza qundi Move Method per spostare i metodi nella classe corretta, se una sola parte del
metodo acede ai dati di un'altro oggietto usa Extract Method per spostare le parti in questione.
\paragraph*{Vantaggi}
\begin{itemize}
    \item Riduzione code duplication
    \item Miglior organizzazione del codice
\end{itemize}
\subsection{Large Class}
Una classe che contiene molti metodi, molte variabili istanziate è una clase large class, una
classe che prova a fare troppo.\\
Questo indica un problema di astrazione, probabilmente nel codice c'è più di un problema o 
qualche metodo appartiene ad altre classi.\\
Le classi solitamente vengono create piccole, ma nel tempo vengono gonfiate più il programma
cresce. Questo accade per la stessa motivazione dei Long Methods, i programmatori solitamente trovano
mentalmente più facile aggiungere codice ad una classe già esistente piuttosto che crearne una nuova.\\
\paragraph*{Risoluzione} Dividi la classe in più classi, usando Extract Class per separare
il comportamento della classe in componenti dedicate, Move Method per spostare i metodi nelle nuove
classi e Extract Subclass per creare sottoclassi per un subset di features se sono usate solo da
alcune istanze.
\paragraph*{Vantaggi}
\begin{itemize}
    \item Effettuare il refactoring di queste classi rende la vita degli sviluppatori più semplice,
    dato che non devono ricordarsi un gran numero di attributi per ogni classe
    \item In molti casi dividere le classi in classi più piccole evita duplicazione di codice e rende
    più funzionale lo sviluppo
\end{itemize}
\subsection{Switch Statements}
Classica situazione dove si ha uno switch complesso, più switch identifici in posti diversi
o una cascata di if statements.\\
Fortunatamente nella programmazione object-oriented, questo tipo di codice è raro, dato che gli
switch e case sono poco usati, il problema del loro utilizzo è che quando una condizione viene
aggiunta è necessario andare a ricercare tutti gli switch e case per aggiungere la nuova condizione.\\
Come regola generale se vedi switch dovresti pensare al polimoriismo.\\
\paragraph*{Risoluzione} Per isolare lo switch e metterlo nella classe giusta potresti aver
bisogno di Extract Method e poi Move Method. Dopo aver specificato la struttura di ereditarietà utilizza
Replace Condtional with Polymporphism per sostituire lo switch con polimorfismo.
\subsection{Data Class}
Una classe che contiene solo variabili di clase, getter/setter e nient'altro. Sta
semplicemente operando come data holder.\\
La vera potenza degli oggetti è che possono contenere sia dati che operazioni e operare
sui dati contenuti in essa.
\paragraph*{Risoluzione} Esaminare i metodi che utilizzano dati e usare Move Method per spostarli
nella data class.
\subsection{Long Parameter List}
Un metodo che richiede troppi parametri.\\
\paragraph*{Risoluzione} Rimpiazza i parametri con metodi, introduci Oggetti parametro e
utilizzao Preserve Whole Object, cioè incapsula tutti i valori che passeresti come parametro
in un unico oggetto.
\begin{lstlisting}[language=java]
    //Prima
    int low = daysTempRange.getLow();
    int high = daysTempRange.getHigh();
    boolean withinPlan = plan.withinRange(low, high);
    //Dopo
    boolean withinPlan = plan.withinRange(daysTempRange);
\end{lstlisting}
\subsection{Shotgun Surgery}
Si parla di Shotgun Surgery quando la modifica di una classe è causa di moltri altri cambiamenti ù
in altre classi, il cambiamento di un'operazione implica il tanti piccoli cambiamenti in
diverse operazioni e classi. Viene definita in questo modo perchè è come se si effettuasse un'operazione 
chirurgica sparando con un fucile a pompa, quindi colpendo molte parti del corpo, richiama un approccio
goffo e soggetto a errori, piuttosto che preciso.
\paragraph*{Cause} Questo accade quando si programma copiando e incollando codice, ottenendo quindi
codice duplicato e non utilizzando i corretti Design Pattern.
\paragraph*{Problemi} Aumenta il tempo richiesto per effettuare una modifica, dato che è necessario
effettuare più parti del codice. Aumenta la probabilità di introdurre bug dato che è necessario
ricordarsi di modificare più punti del codice. Oltretutto si produce più codice di quanto richiesto.
\paragraph*{Risoluzione} Move Method e Move Field. Per ridurre ridondanze e sviluppare funzioni e
procedure apposite.
\paragraph*{Esempio} Un metodo che implementa una funzionalità utilizzata da un gran numero 
di classi e metodi. Cambiare questo metodo implicherebbe cambiamenti a cascata su tutti i metodi
in un gran numero di classi.
\subsection{Refused Bequest}
Una classe che non usa ciò che ha ereditato dalla sua superclasse.\\
Se una sottoclasse utilizza solo alcuni dei metodi e proprietà ereditate dai
suoi parenti l'ereditarietà è fuori luogo. I metodi non richiesti potrebbero essere non
utilizzati o essere ridefiniti e restituire eccezioni.\\
Questo accade perchè si è motivati a creare ereditarietà solo per riutilizzare codice, ma
la superclasse e la sottoclasse sono completamente diverse.
\paragraph*{Risoluzione} Replace Inheritance with Delegation per eliminare l'ereditarietà, 
in questo caso inutile e forzata.
\subsection{Comments (called sweet smeel)}
I commenti nel codice dovrebbero essere pochi e brevi, perchè il codice dovrebbe spiegarsi
da solo, se necessita di lunghi commenti significa che non è chiaro. Dovresti effettuare
refactoring per rendere il codice più chiaro e ridurre così i commenti.\\
Quando senti il bisogno di scrivere un commento prima di tutto ristruttura il codice così 
che qualsiasi  commento sia superfluo.
\paragraph*{Refactoring}
\begin{itemize}
    \item Extract Method
    \item Rename Method: rinomina il metodo in modo tale da rendere più chiaro il suo scopo
\end{itemize}
\subsection{Lazy Class}
Capire e mantenere una classe è sempre un costo in termini di tempo e denaro,\\
quindi se una clase
non fa abbastanza per guadagnarsi la tua attenzione dovrebbe essere cancellata.\\
\paragraph*{Cause del problema}
Probabilmente una classe è stata progettatata per essere completamente funzionale, ma dopo alcuni 
refactoring è diventata molto piccola oppure è stata creata per supportare sviluppi futuri che non
sono mai stati effettivamente fatti.
\paragraph*{Risoluzione} Componenti che non sono richiesti dovrebbero essere cancellati, le poche
responsabilità che aveva quella classe vengono spostate in un'altra classe. 
\paragraph*{Vantaggi}
\begin{itemize}
    \item Riduzione delle dimensioni del codice
    \item Mantenimento più semplice
\end{itemize}
\subsection{Middle Man}
Se una classe effettua una sola operazione, delegando il lavoro ad altre classi, perchè esiste?
\paragraph*{Cause del problema}
Questi smells possono essere il risultato di una eliminazione troppo aggressiva dei messaggi a catena.
In altri casi possono essere il risultato del refactoring di una classe che era troppo grande e
quindi dello spostamento dei metodi in altre classi. La classe di partenza resta diventa
quindi un guscio vuoto che non fa altro che delegare il lavoro ad altre classi.\\
In alcuni casi è da ignorare perchè potrebbe essere il risultato dell'utilizzo di Design
Pattern come Proxy o Decorator.
\subsection{Data Clumps}
Qualche volta differenti parti di codice contengono gruppi identici di variabili (come parametri
per connettersi a un database). Questi gruppi dovrebbero essere spostati nelle loro classi di
appartenenza.
\paragraph*{Cause} A volte questi data groups sono il risultato di un'operazione di copia e incolla 
o di una struttura povera del programma.
\paragraph*{Risoluzione} Usa Extract Class per creare una classe che contenga il gruppo di dati ripetuto.
Verificare se sono tutti essenziali.
\subsection{Message Chains}
Una catena di chiamate di metodi su oggetti restituiti da altri metodi. Questa catena
può risultare molto difficile da gestire e mantenere perchè anche un cambiamento minore
della struttura degli oggetti può richiedere modifiche in molti punti del codice.
\paragraph*{Risoluzione} Usare Hide Delegate, tecnica utilizzate per nascondere
la complessità di un oggetto delegato. Meno il client conosce i dettagli delle relazioni
tra oggetti più è facile apportare modifiche al programma. Può causare \textbf{Middle-man}.\\
Un'altra soluzione è usare Extract Method per creare un metodo che prenda in input l'oggetto su
cui vengono chiamati i metodi della atena e che restituisca il valore finale desiderato. In questo
modo si chiama una sola funzione, semplificando lettura e manutenzione del codice.\\
Terzo metodo è utilizzare Move Methodo per spostare il metodo nella classe che viene
richiamata maggiormente dalla catena.
\subsection{Speculative Generality}
Ci sono classi, metodi o parametri che non sono utilizzati.
\paragraph*{Cause} A volte il codice è creato per supportare in anticipo funzionalità future che
non verrano mai implementate, il risultato è che si ottiene un codice difficile da capire e supportare.
\paragraph*{Risoluzione} Per rimuovere classi inutilizzate usare Collapse Hierarchy per
effettuare il merge delle sottoclassi nelle superclassi, oppure Inline Class per eliminare
le classi inutilizzate e spostare i pochi attributi e metodi rimasti in altre classi.
\subsection{Incomplete Library Class}
Una libreria che stai utilizzando non ha le funzionalità che ti servono.\\
\paragraph*{Cause}Questo si verifica perchè l'autore si rifiuta di implementare il metodo perchè
non conviene a livello computazionale o perchè è troppo costoso in temrinidi tempo e denaro.
Può anche essere che l'autore è inconsapevole che il metodo sia necessario.
\paragraph*{Risoluzione} Se ci sono pochi metodi nella libreria possiamo implementarli con riferimenti esterni,
ma questo può causare Feature Envy.\\
Possiamo altrimenti creare sottoclassi della libreria in locale e implementare i metodi mancanti, introducendo
una maschera per nascondere la libreria (wrapper).
\paragraph*{Vantaggi}
Riduce i duplicati, ma se i più progetti usano queste soluzioni risulta più difficile mantenere
la compatibilità.
\subsection{Fonti utili}
Fonti consigliate dai prof (e anche da me dato che sono molto utili e ben fatte) sono i seguenti link:
\begin{itemize}
    \item Per il refactoring - \url{https://sourcemaking.com/refactoring/refactorings}
    \item Per i code smells - \url{https://sourcemaking.com/refactoring/smells}
\end{itemize}
