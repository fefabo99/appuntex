\chapter{NP Completezza} %copiato diretto dagli appunti, va sistemato che non si capisce una sega TODO
\section{Cenni Teorici}
Lo studio della NP-Completezza inizia formalizzando il concetto dei problemi risolvibili in tempo polinomiale.
In primo luogo, sebbene sia ragionevole considerare intrattabile un problema che richiede $\theta(n^{100})$, ci sono pochissimi problemi nella pratica che richiedono un tempo polinomiale così alto.
Una volta che viene scoperto un algoritmo con un tempo polinomiale, spesso seguono altri alogritmi più efficienti.
In secondo luogo, per molti modelli di calcolo, un problema che può essere risolto in tempo polinomiale in un modello può essere riolsto sempre in tempo polinomiale in un altro modello.
In terzo luogo, la classe dei problemi risolvibili in tempo polinomiale ha delle interessanti proprietà in chiusura, in quanto i polinomi sono chiusi rispetto all'addizione, alla moltiplicazione e alla composizione.

\paragraph*{Problema di decisione}
Un problema di decisione è un problema tale per cui la risposta è \emph{si o no}.
\paragraph*{Classe di complessità}
Nella teroia della complessità computazionale, una classe di complessità è un insieme di problemi di una certa complessità.
Un esempio tipico di definizione di classe di complessità a la forma:
L'insieme di problemi che, se esistem la soluzione, possono essere risolti d una macchina astratta $M$ usando $O(f(n))$ della risorsa $R$, con $n$ dimensione dell'input
\section{P}
$P$ è una classe di complessità che rappresenta l'insieme di tutti i problemi di decisione che possono essere risolti in tempo polinomiale, ovvero quei problemi tali che dato un input, riescono a dare un output si o no in tempo polinomiale.
I problemi P sono i problemi che possono essere risolti da una macchina di Turing deterministica in tempo polinomiale.
\section{NP}
NP è una classe di complessità che rappresenta l'insieme di tutti i problemi di decisione per i quali le istanze che danno si come risposta possono essere verificate in tempo polinomiale.
Ovvero: se la Sibilla Cumana mi dice che, dato un input $x$, la risposta è si, posso verificare la correttezza dell'affermazione in tempo polinomiale.
I problemi NP sono i problemi che possono essere risolti da una macchina di Turing non-deterministica in tempo polinomiale.
\section{NP-Completo}
NP-completo è una classe di complessità che rappresenta l'insieme di tutti i problemi $x\in NP$ tali che è possibile una riduzione da un qualsiasi altro problema $y \in NP$ a x in tempo polinomiale.
Intuitivamente, possiamo risolvere $y$ velocemente se sappiamo risolvere $x$ velocemente.
\section{NP-Hard}
Un problema $x$ è NP-Hard se e solo se esiste un problema $y$ NP-completo tale che $y$ è riducibile a x in tempo polinomiale.
Intuitivamente questi sono i problemi che sono almeno difficili quanto i problemi NP-completi.
I problemi NP-hard non necessariamente sono in NP e non necessariamente sono problemi di decisione.
I problemi NP-hard e NP sono i problemi NP-completi.
Non tutti i problemi NP-hard sono NP-completi: affinchè un problema NP-hard $x$ sia NP-complete, x deve essere in NP.
è importante notare che poichè un qualsiasi problema NP-completo può essere ridottoa a qualsiasi altro problema NP-completo in tempo polinomiale e tutti i problemi in NP sono riducibili in tempo polinomiale a problemi NP-completi.
Tutti i problemi NP possono essere ridotti a un qualsiasi problema NP-hard in tempo polinomiale.
La conseguenza di queste osservaizoni è che se esiste una soluzione in tempo polinomiale a un qualsiasi problema NP-hard, allora esiste una soluzione a tutti i problemi NP in tempo polinomiale!
\end{document}