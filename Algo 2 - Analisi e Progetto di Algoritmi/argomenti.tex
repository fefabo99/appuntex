\documentclass[12pt, a4paper, openany]{book}
\usepackage[italian]{babel}
\usepackage{listings}
\usepackage{graphicx}
\usepackage{fancyvrb}

\begin{document}
\title{Argomenti Algo 2}
\author{Elia Ronchetti}
\date{Marzo 2022}

\maketitle
\tableofcontents

\chapter{Programmazione dinamica}
\section{Esempi Introduttivi}
\section{Caratteristiche principali}
\section{Implementazione con matrici}

\chapter{Algoritmi greedy}
\section{Esempio: scheduling di attività}
\section{Elementi della strategia greedy}
\section{Algoritmo di Huffman}
\section{Dimostrazione di correttezza}
\section{Greedy vs Dynamic programming: knapsack}
\section{Definizione di matroide; esempi}
\section{Il teorema di Rado}

\chapter{Algoritmi su grafi}

\section{Rappresentazione in memoria di un grafo}
\section{Visita in ampiezza e profondità}
\section{Componenti connesse di un grafo non orientato}
\section{Ricerca di cammini minimi in un grafo}
\section{Costruzione di alberi di copertura minimi}
\section{Problemi di massimo flusso}

\chapter{NP completezza}

\section{Problemi trattabili e intrattabili}
\section{Riducibilità polinomiale}

\chapter{Esame}
L'esame completo consiste in una prova scritta contenente esercizi e domande teoriche.
Su e-learning c'è un PDF con delle ipotetiche domande di teoria.
%Prendere diversi esami MEGA e valutare la media degli esercizi che escono 
\end{document}