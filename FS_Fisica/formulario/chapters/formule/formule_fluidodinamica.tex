\chapter*{Fluidodinamica}

    \section*{Equazione di Continuità} 
        \begin{equation*}
            R_v = Av = \textsf{cost} \, \Bigg[\frac{m^3}{s}\Bigg]
        \end{equation*}
    $R_v$ prende il nome di \textbf{portata volumica}, con $A$ la sezione del 
    tubo in un qualsiasi punto e $v$ la velocità del fluido.
    Sappiamo che la massa di un fluido è data dalla seguente equazione:
        \begin{equation*}
            m = \rho V
        \end{equation*}
    Quindi moltiplicando la Portata Volumica per la densità del fluido 
    otterremo la \textbf{portata massica}, definita come:
        \begin{equation*}
            R_m = \rho R_v = \rho A v = \textsf{cost}
        \end{equation*}

    \section*{Equazione di Bernoulli} 
        \begin{equation*}
            p_1 + \frac{1}{2}\rho v^2_1 + \rho g y_1 
            =
            p_2 + \frac{1}{2}\rho v^2_2 + \rho g y_2 
            = 
            \textsf{cost}
        \end{equation*}
        \begin{quote}
            In un fluido a flusso laminare, la somma della pressione, 
            dell'Energia Cinetica per unitò di Volume e dell'Energia 
            Potenziale gravitazionale per unità di Volume è 
            \textbf{costante}.
        \end{quote}
