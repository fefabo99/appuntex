\chapter*{Fluidostatica}

    \section*{Pressione} 
        \begin{equation*}
            p = \frac{F}{A} \; \Bigg[Pa \textsf{ Pascal} = \frac{Kg}{m \cdot 
            s^2} \Bigg]
        \end{equation*}
    È uno scalare, proporizionale al modulo della forza.

    \section*{Densità}
        \begin{equation*}
            \rho = \frac{m}{V} \; \Bigg[\frac{Kg}{m^3}\Bigg]
        \end{equation*}
    Se $\rho$ è costante, il liquido preso in considerazione è incompressibile.

    \section*{Legge di Stevino} 
        \begin{equation*}
            p_2 = p_1 + \rho g(y_1-y_2)
        \end{equation*}
    Chiamiamo $h$ la profondità di un campione in un fluido misurata a partire
    da un livello di riferimento a cui la pressione assume un valore $p_0$, la 
    precedente equazione diventa:
        \begin{equation*}
            p = p_0 + \rho gh
        \end{equation*}
    In cui $p$ è la pressione alla profondità $h$.
    La pressione è la stessa per tutti i punti allo stesso livello.

    \section*{Forza di Bouyant}
        \begin{align}
            F_{archimede} &= m_{liquido}g \\
            F_a &= (\rho V) g
        \end{align}
        \subsection*{Oggetto totalmente sommerso} S
            \begin{equation*}
                F_g = M_g = (\rho_0 V_0) g
            \end{equation*}

        \subsection*{Oggetto galleggiante}
            \begin{align}
                F_a = F_g :
                \begin{cases}
                    F_a &= (\rho_f V_f)g \\
                    F_g &= M_g = (\rho_0 V_0)g
                \end{cases}
            \end{align}
        $F_a = F_g$ perché l'oggetto è in equilibrio.

    \section*{Peso appartente di un oggetto}
        \begin{equation*}
            P_{app} = P - F_a
        \end{equation*}
            